\chapter*{\introname}
\addcontentsline{toc}{chapter}{Введение}

\titlespacing{\section}{\theotstup\parindent}{*2}{*1}
\titlespacing{\subsection}{\theotstup\parindent}{*1}{*0.5}
\newcommand{\actuality}{\section*{Актуальность темы исследования}}
\newcommand{\previouswork}{\section*{Степень разработанности темы исследования}}
\newcommand{\aimsandtasks}{\subsection*{Цели и задачи}}
\newcommand{\aim}{\vspace{1em}\textbf{Целью}}
\newcommand{\tasks}{\textbf{задачи}}
\newcommand{\defpositions}{\subsection*{Положения, выносимые на~защиту}}
\newcommand{\novelty}{\subsection*{Научная новизна}}
\newcommand{\influence}{\subsection*{Теоретическая и практическая значимость работы}}
\newcommand{\methods}{\subsection*{Методология и методы исследования}}
\newcommand{\reliability}{\subsection*{Степень достоверности}}
\newcommand{\probation}{\subsection*{Степень достоверности и апробация результатов}}
\newcommand{\contribution}{\subsection*{Личный вклад автора}}
\newcommand{\publications}{\subsection*{Публикации}}

%{\actuality} 
Терагерцовое излучение (0.1--10 ТГц), расположенное на шкале частот между радиоволнами и оптическим диапазоном, сохраняет характерную для радиоволн способность проникать во многие материалы на макроскопическую глубину, обладая при этом более высокой частотой и меньшей длиной волны, что делает его привлекательным для медицинской диагностики, систем безопасности и беспроводной связи. Кроме того, в терагерцовом диапазоне лежат многие характерные частоты в различных материалах, такие как частоты колебательных и вращательных переходов в молекулах, частоты фононов и двумерных плазмонов в твёрдых телах, обратные времена рассеяния носителей в полупроводниках и т. д., из-за чего он находит широкое применение в спектроскопии.

Однако генерация терагерцового (ТГц) излучения оказывается нетривиальной задачей, так как многие источники электромагнитного излучения либо вообще неспособны работать в ТГц диапазоне, либо сильно теряют в эффективности (так называемая <<терагерцовая дыра>>). Так, быстродействие радиоэлектронных приборов, таких как диоды Ганна, резонансно-туннельные диоды, IMPATT- и TUNNET-диоды, схемы на полевых, биполярных и HEMT-транзисторах, а также умножители частоты на диодах Шоттки, ограничено частотами 1--2 ТГц, при этом мощность генерируемого излучения выше 1 ТГц лежит в микроваттном диапазоне. Лазерные диоды на основе полупроводников \AIIIBV{} работают в более коротковолновой области из-за недостаточно узкой запрещённой зоны. Существуют источники ТГц излучения с выходной мощностью в единицы мВт и более, но при этом более громоздкие по сравнению с полупроводниковыми приборами; к таким источникам относятся лампы обратной волны и другие электровакуумные приборы, терагерцовые молекулярные лазеры и схемы, использующие конверсию инфракрасного излучения в терагерцовое. Лазеры на $p$-легированном германии имеют выходную мощность до $\sim$ 10 Вт, но требуют больших электрических и магнитных полей и работают при температурах ниже 40 К.

Одними из наиболее практичных источников ТГц излучения являются квантово-каскадные лазеры (ККЛ), работающие в диапазоне 1--5 ТГц с типичной выходной мощностью в единицы мВт при 77 К. Генерация терагерцовых разностных гармоник в инфракрасных ККЛ позволила поднять рабочую температуру до комнатной при выходной мощности в десятки мкВт. К сожалению, сильное решёточное поглощение вблизи частот оптических фононов не позволяет традиционным GaAs/AlGaAs ККЛ работать в диапазоне 5--10 ТГц. Кроме того, производство ККЛ требует выращивания множества квантовых ям с высокой точностью, из-за чего один лазер стоит сотни тысяч рублей и более (на момент написания диссертации).

Более дешёвой альтернативой ККЛ могли бы стать лазерные диоды на основе материалов с запрещённой зоной в ТГц диапазоне. Однако с уменьшением ширины запрещённой зоны резко возрастает темп оже-рекомбинации, в результате чего уже для InAlSb лазеров, генерирующих на длине волны 4 мкм, рабочая температура снижается до 165 К.

Существует, однако, класс материалов, в котором можно ожидать сильного подавления оже-рекомбинации из-за того, что выполнение законов сохранения затруднено. Это так называемые дираковские материалы, в которых закон дисперсии приближённо описывается формулой $E = \pm\sqrt{v_0^2 p^2 + E_g^2/4}$, аналогичной закону дисперсии релятивистских электронов в вакууме. При таком законе дисперсии оже-процессы низшего порядка оказываются запрещены, так же как электрон в вакууме не может самопроизвольно породить электрон-позитронную пару, поэтому оже-рекомбинация возможна только либо за счёт отклонения реального закона дисперсии от дираковского, либо за счёт процессов с участием более трёх носителей.

Возможность использования лазерных диодов на основе дираковских материалов для генерации ТГц излучения подтверждается существованием лазеров на PbSnSe, генерирующих вплоть до 46.5 мкм (6.5 ТГц)~\cite{lead_salt_record_wavelength}. Однако из-за технологических проблем (высокого остаточного легирования) работа этих лазеров в ТГц диапазоне требует гелиевых температур.

Известны и другие дираковские материалы, для которых уже существуют технологии изготовления высококачественных гетероструктур с низким остаточным легированием. К таким материалам относятся графен и квантовые ямы из теллурида кадмия-ртути. В этих материалах уже наблюдалось вынужденное излучение --- на 5.2 ТГц при 100 К в графене с электрической накачкой~\cite{graphene_lasing} и на 15 ТГц при 20 К в ямах из теллурида кадмия-ртути с оптической накачкой~\cite{HgCdTe-stimulated_emission}, однако ТГц инжекционные лазеры на ямах из теллурида кадмия-ртути пока не были реализованы, а в графене достигнутая мощность генерации составляет всего 0.1 мкВт.

В свете вышесказанного представляется актуальным теоретическое исследование возможностей лазерных диодов на основе графена и теллурида кадмия-ртути и определение их ключевых характеристик, таких как достижимый частотный диапазон генерации, рабочая температура и пороговые токи.

{\previouswork}
Имеются теоретические исследования, демонстрирующие возможность оптического усиления на ТГц частотах в графене; в последующих работах были предложены конструкции лазерных диодов на основе графена. Также для этого материала проводились расчёты темпов излучательной рекомбинации, рекомбинации с испусканием фононов и плазмонов и оже-рекомбинации~\cite{Rana-Auger,Tomadin-theory,Malic-dynamic}.

В работе \cite{Rana-Auger} было показано, что трёхчастичная оже-рекомбинация в графене разрешена только для случая коллинеарных импульсов носителей. В дальнейшем было показано, что динамическое экранирование подавляет коллинеарные процессы, и для получения ненулевого темпа оже-рекомбинации требуется учёт межэлектронного рассеяния \cite{Tomadin-theory}. В отдельных работах <<размытие>> корневой особенности диэлектрической проницаемости для коллинеарных процессов было учтено приближённым образом ~\cite{Malic-dynamic,Tomadin-theory}, однако единого рассмотрения оже-рекомбинации в графене с учётом всех ключевых эффектов межэлектронного взаимодействия, таких как размытие и искривление дираковского конуса, а также наличие плазмонных полюсов в экранированном кулоновском взаимодействии, до сих пор не было в литературе.

В теллуриде кадмия-ртути темп оже-рекомбинации рассчитывался для объёмного материала и сверхрешёток; в квантовых ямах расчёты проводились только в среднем ИК-диапазоне и в приближении параболических зон. Для квантовых ям дальнего ИК и ТГц диапазона рассчитывались только энергетические пороги оже-рекомбинации и темп излучательной рекомбинации. Также теоретически показана возможность ТГц генерации в квантово-каскадных лазерах и на разностных гармониках в ИК лазерных диодах на основе квантовых ям из этого материала; исследование возможности прямой ТГц генерации на межзонных переходах пока ограничивается расчётами оптической проводимости в условиях инверсии населённостей и экспериментами по получению вынужденного излучения на всё больших длинах волн при оптической накачке~\cite{HgCdTe-stimulated_emission}.

\aimsandtasks\ 
Целью данной работы является теоретическое исследование возможности межзонной лазерной генерации в ТГц диапазоне в узкозонных полупроводниках на примере графена и квантовых ям из теллурида кадмия-ртути.

Для~достижения поставленной цели необходимо было решить следующие задачи:
\begin{enumerate}
  \item Разработать метод расчёта темпа оже-рекомбинации в материалах с дираковским законом дисперсии, в которых обычные трёхчастичные оже-процессы запрещены законами сохранения энергии и импульса.
  \item Разработать программу для расчёта темпа безызлучательной рекомбинации в графене с учётом влияния межэлектронного взаимодействия на спектр носителей.
  \item Разработать программу для расчёта темпа оже-рекомбинации и оптической проводимости в квантовых ямах из теллурида кадмия-ртути с учётом непараболичности зон и наличия множества подзон размерного квантования.
  \item Рассчитать закон дисперсии двумерных плазмонов и границу области межзонных переходов в квантовых ямах из теллурида кадмия-ртути.
  \item Определить пороговые концентрации носителей, при которых рекомбинация с испусканием плазмонов в квантовых ямах из теллурида кадмия-ртути становится разрешена законами сохранения.
  \item Разработать программу для расчёта темпа рекомбинации с испусканием плазмонов в квантовых ямах из теллурида кадмия-ртути.
  \item Рассчитать темп рекомбинации в графене, инкапсулированном в различные диэлектрики и при различных температурах.
  \item Рассчитать темп рекомбинации в квантовых ямах из теллурида кадмия-ртути различной толщины при различных температурах.
  \item Сравнить полученные результаты с известными экспериментальными данными.
  \item Оценить пороговые токи, необходимые для достижения ТГц лазерной генерации в лазерных диодах на основе графена и квантовых ям из теллурида кадмия-ртути.
  \item Оценить рабочие температуры и достижимый диапазон частот генерации в лазерных диодах на основе графена и квантовых ям из теллурида кадмия-ртути.
\end{enumerate}

\defpositions
\begin{enumerate}
\item Оже-рекомбинация в материалах с дираковским законом дисперсии, номинально запрещённая законами сохранения энергии и импульса, становится возможной при учёте уширения спектра носителей из-за их рассеяния друг на друге. Расчёты, выполненные методом неравновесных функций Грина в самосогласованном $GW$-приближении, показывают, что время оже-рекомбинации в слабо неравновесном нелегированном графене приблизительно обратно пропорционально температуре и составляет 1--3 пс при 300 К для диэлектрических проницаемостей окружения $\kappa = 5$--25.
\item В узких квантовых ямах HgTe с шириной запрещённой зоны 20--40 мэВ пороговые энергии оже-рекомбинации достигают 20--30 мэВ из-за близких эффективных масс электронов и дырок и непараболичности зон. В результате времена оже-рекомбинации увеличиваются на полтора--два порядка при 77 К по сравнению с гипотетической ситуацией большой электрон-дырочной асимметрии, характерной для полупроводников \AIIIBV{}, и составляют 40--140 пс. Вместе с малыми частотами оптических фононов ($< 5$ ТГц) это создаёт благоприятные условия для межзонной лазерной генерации в диапазоне 6--10 ТГц, недоступном для существующих квантово-каскадных лазеров.
\item Рекомбинация с испусканием плазмонов в квантовых ямах HgTe имеет порог по концентрации неравновесных носителей, возникающий из-за ограничений, накладываемых законами сохранения энергии и импульса. Пороговый процесс рекомбинации для ям толщиной от 4.5 до 7.3 нм при 77 К соответствует переходу электрона из минимума зоны проводимости в побочный максимум валентной зоны с испусканием плазмона. Снижение температуры приводит к экспоненциальному уменьшению количества дырок в побочном максимуме валентной зоны и, соответственно, к подавлению плазмонной рекомбинации.
\end{enumerate}

\novelty
\begin{enumerate}
  \item Впервые исследована роль многочастичных эффектов в оже-рекомбинации в рамках самосогласованного $GW$-приближения.
  \item Впервые теоретически оценены пороговые токи терагерцовых лазерных диодов на основе графена с учётом оже-рекомбинации.
  \item Впервые рассчитан темп безызлучательной рекомбинации в квантовых ямах из теллурида кадмия-ртути ТГц диапазона и теоретически оценены пороговые токи лазерных диодов на их основе.
  \item Впервые определён относительный вклад оже-рекомбинации и рекомбинации с испусканием плазмонов в суммарный темп рекомбинации в квантовых ямах из теллурида кадмия-ртути.
\end{enumerate}

\influence\
Теоретическая значимость работы состоит в том, что разработан метод расчёта темпа оже-рекомбинации в дираковских материалах, учитывающий эффекты межэлектронного взаимодействия, такие как размытие и искривление дираковского конуса, а также наличие плазмонных полюсов в экранированном кулоновском взаимодействии.

Практическая значимость работы состоит в том, что исследована возможность ТГц генерации в лазерных диодах на основе графена и квантовых ям теллурида кадмия-ртути, определены их пороговые токи, рабочие температуры и достижимый диапазон частот генерации.

\methods\
Для расчёта темпа рекомбинации в графене использовался метод неравновесных функций Грина и самосогласованное $GW$-приближение. Вычисление соответствующих интегралов производилось с использованием быстрых преобразований Фурье и Ханкеля.

Законы дисперсии и волновые функций электронов в квантовых ямах из теллурида кадмия-ртути рассчитаны в четырёхзонной модели Кейна с использованием приближения огибающих функций.

Оптическая проводимость в квантовых ямах из теллурида кадмия-ртути рассчитывалась по формуле Кубо-Гринвуда.

Темп оже-рекомбинации в квантовых ямах из теллурида кадмия-ртути рассчитывался по золотому правилу Ферми с использованием усовершенствованного метода Монте-Карло для вычисления интегралов.

Темп рекомбинации с испусканием плазмонов в квантовых ямах из теллурида кадмия-ртути рассчитывался по формуле, выведенной методом неравновесных функций Грина с использованием диэлектрической проницаемости в приближении плазмонного полюса. Численное интегрирование производилось методом Монте-Карло.

\probation\
Достоверность полученных результатов обеспечивается использованием экспериментально проверенных приближений и сравнением рассчитанных времён рекомбинации и пороговых интенсивностей оптической накачки с экспериментальными данными по кинетике фотовозбуждённых носителей в графене и наблюдению вынужденного излучения в квантовых ямах из теллурида кадмия-ртути.

Основные результаты работы докладывались на следующих конференциях:
\begin{itemize}
\item XVIII Всероссийская молодежная конференция по физике полупроводников и наноструктур, полупроводниковой опто- и наноэлектронике, Санкт-Петербург, 2016;
\item 5th Russia-Japan-USA-Europe Symposium on Fundamental \& Applied Problems of Terahertz Devices \& Technologies, Сендай, Япония, 2016;
\item Graphene Week 2017, Афины, Греция;
\item 26th International Symposium ``Nanostructures: Physics and Technology'', Минск, Беларусь, 2018;
\item IV International Conference on Metamaterials and Nanophotonics (METANANO 2019), Санкт-Петербург;
\item XIV Российская конференция по физике полупроводников, Новосибирск, 2019;
\item 59-я научная конференция МФТИ с международным участием, Долгопрудный, 2016;
\item 61-я Всероссийская научная конференция МФТИ, Долгопрудный, 2018.
\end{itemize}

Диссертация состоит из трёх глав, основные результаты которых изложены в трёх статьях~\cite{my_graphene,my_HgCdTe,my_plasmon}. Все статьи опубликованы в рецензируемых международных журналах (Physical Review B, ACS Photonics, Journal of Physics: Condensed Matter), включённых в библиографические базы Scopus и Web of Science.
    
\contribution Общая постановка задачи осуществлялась научным руководителем
автора Свинцовым Д. А. Коллеги автора, участвовавшие в обсуждении методов и результатов исследования, указаны в работах~\cite{my_graphene,my_HgCdTe,my_plasmon} в качестве соавторов. Все результаты, изложенные в настоящей диссертации, получены автором лично.

\section{Introduction}

Thermal reflow can significantly modify the profile of structures obtained in polymer resist and this phenomenon has its advantages and disadvantages.
For example, thermal reflow could be used for smoothing of the relief obtained by grayscale e-beam lithography~\cite{Kirchner_GL_review} which allows to obtain various 3D structures.
On the other hand, thermal reflow leads to relief deformation, which is undesirable in certain cases~\cite{NIL_reflow}.
In the light of the above, the method allowing exact determination of thermal reflow influence on resulting structure profile in any specific processes is highly desirable.

Two common approaches to thermal reflow simulation of polymer structures could be distinguished.
The first include analytical methods based on transfer equations.
For instance, Leveder~\cite{Leveder_2010,Leveder_2011} used an analytical spectral method to simulate the reflow of periodic structures obtained in polystyrene by nanoimprint lithography.
This method is based on two-dimensional Navier-Stokes equation coupled to continuity equation  considering Laplace pressure and Hamaker energy with the assumption of no slip length and no Marangoni effect.
In this method the initial structure profile undergoes Fourier transform and then thermal reflow is being simulated by decay of profile harmonic modes:
\begin{equation} \label{eq:Fourier_1}
	h(x, t) = h_0 + \tilde{h}(x, t),
\end{equation}
\begin{equation} \label{eq:Fourier_2}
	\tilde{h}(x, t) = \sum_{-\infty}^{+\infty} a_n(0) \exp \left(-\frac{t}{\tau_n}+i n \frac{2 \pi}{\lambda} x \right),
\end{equation}
\begin{equation} \label{eq:Fourier_3}
	\tau_n = \frac{3 \eta}{\gamma h_0^3} \times \left( \frac{\lambda}{2 \pi n} \right)^4,
\end{equation}
where $\lambda$ -- profile spatial periodicity, $\eta$, $\gamma$ -- polymer viscosity and surface tension, respectively, $a_n(0)$ -- Fourier coefficients of initial polymer profile, $h_0$ -- polymer layer thickness.
Polymer viscosity depends both on temperature and polymer molecular weight, which should be taken into account.
Temperature dependence of viscosity could be described by Williams–Landel–Ferry (WLF) equation~\cite{Bird_WLF}:
\begin{equation} \label{eq:WLF}
	\log \left( \frac{\eta(T)}{\eta(T_0)} \right) = -\frac{C_1(T-T_0)}{C_2+(T-T_0)},
\end{equation}
which parameters $\eta(T_0)$, $C_1$, $C_2$ and $T_0$ for three different polymers are provided in Table~\ref{table:WLF}~\cite{Aho_WLF}.
The dependence of polymer viscosity on its molecular weight could be described by empirical formula:

\begin{equation} \label{eq:3p4_3p1}
	\eta \propto M_n^\alpha,
\end{equation}
where $M_n$ -- number average polymer molecular weight. For polymethyl methacrylate (PMMA) $\alpha$ comprises 3.4 at $M_n > 48000$ and 1.4 at $M_n < 48000$~\cite{Leveder_2010,Bueche_3p4_1p4}.
Equations~\ref{eq:WLF}, and \ref{eq:3p4_3p1} allow one to calculate polymer viscosity for different temperatures and values of number average molecular weight (Fig.~\ref{fig:eta_vary_T_Mn}).

\begin{figure}
	\begin{center}
		\includegraphics[width=0.6\linewidth]{eta_vary_T_Mn}
	\end{center}
	\vspace{-2em}
	\caption{Temperature viscosity dependencies for PMMA with different number average molecular weights, obtained by equations~(\ref{eq:WLF}, \ref{eq:3p4_3p1}).}
	\label{fig:eta_vary_T_Mn}
\end{figure}

\begin{table}[t]
	\centering
	\caption{Parameters of equation~\ref{eq:WLF}, obtained by Aho et al. for polystyrene 143E by BASF (PS), polymethyl methacrylate Plexiglas 6N by Degussa (PMMA) and polycarbonate Lexan HF1110R by GE Plastics (PC)~\cite{Aho_WLF}.}
	\begin{tabular}{@{}llll}
		\br
		Parameter \hspace{8.9em} & PS \hspace{5em} & PMMA \hspace{5em} & PC \\
		\mr
		$\eta(T_0)$, Pa$\cdot$s \hspace{8.9em} & 7310.4 \hspace{5em} & 13450 \hspace{5em} & 2763 \\
		$C_1$ \hspace{8.9em} & 10.768 \hspace{5em} & 7.6682 \hspace{5em} & 4.7501 \\
		$C_2$, $^\circ$C \hspace{8.9em} & 289.21 \hspace{5em} & 210.76 \hspace{5em} & 110.12 \\
		$T_0$, $^\circ$C \hspace{8.9em} & 190 \hspace{5em} & 200 \hspace{5em} & 200 \\
		\br
	\end{tabular}
	\label{table:WLF}
\end{table}

The second approach, numerical one, is based on search of minimal surface by finite elements method. It can be processed by free software ``Surface Evolver'' (SE) -- the program for the modelling of liquid surfaces shaped by various forces and constraints~\cite{Brakke_SE}. SE allows a wide spectrum of possible energies to be assigned like gravitational energy, surface energy, and further different implementations of mean and Gaussian curvature. For the purpose of polymer reflow simulation only surface energy should be taken into account.

In SE simulation algorithm the structure is only described by its ``outer shell'' (soapfilm modeling)~(Fig.~\ref{fig:SE_basic}a). The resist surface is being divided into triangle facets defined by vertices $v_0$, $v_1$ and $v_2$ and oriented edges $\vec{e_0}$, $\vec{e_1}$ and $\vec{e_2}$, and the polymer reflow is simulated by moving of facet vertices, maintaining the constant volume inside the surface. The force on vertex $v_0$ (the tail of vector $\vec{e_0}$) is

\begin{equation}
	\vec{F}_{v_0}=\frac{\gamma_i}{2} \cdot \frac{\vec{e}_1 \times\left(\vec{e}_0 \times \vec{e}_1\right)}{\left\|\vec{e}_0 \times \vec{e}_1\right\|},
\end{equation}
where $\gamma_i$ -- is surface tension of $i$-th facet~(Fig.~\ref{fig:SE_basic}b). SE could be operated in the area normalization mode to approximate a vertex motion by mean curvature, i.e., a surface tension flow. In this mode, the velocity of a vertex is proportional to force and indirectly proportional to the area of the facets surrounding this vertex. The $i$-th facet has three vertices associated with it, therefore the relative area contribution to the force of one vertex is 1/3 the area of the surrounding facets $A$. The vertex velocity in the area normalization mode is

\begin{equation} \label{eq:SE_v}
	\vec{v} = \frac{\vec{F}}{A/3} \cdot \mu,
\end{equation}
where $\mu$ is so called vertex mobility. The vector of vertex movement $\vec{\delta}$ is then calculated as product of vertex velocity and \textit{scale} factor, the physical representation of simulation step time:

\begin{equation} \label{eq:SE_delta}
	\vec{\delta} = \vec{v} \cdot scale.
\end{equation}
In most cases SE is used for calculation of minimal energy geometries only~\cite{SE_example_1,SE_example_2}, which doesn't imply simulation of liquid or polymer flow dynamics.

\begin{figure}[t]
	\centering
	\includegraphics[width=\linewidth]{Fig_2} \\
	\caption{a) A mound of liquid sitting on a tabletop with gravity acting on it, defined by its surface in SE, b) definition of vertices and oriented edges for $i$-th facet in SE.}
	\label{fig:SE_basic}
\end{figure}

Polymer structures obtained by grayscale e-beam lithography have strongly non-uniform distribution on number average molecular weight, which result in non-uniform viscosity profile.
Analytical spectral approach could not be applicated in this case, but numerical one can still be used.
Kirhner~\cite{Kirchner_SE_1} applied SE for thermal reflow simulation of double-step structures obtained in PMMA by e-beam grayscale patterning and wet development.
The structures consisted of two regions with different values of PMMA number average molecular weight, and the difference in region viscosity was taken into account by setting different vertex mobilities.
Mobility ratios for structure regions were determined empirically by the comparison of simulated profiles to the experimental ones.
In this case, the simulation algorithm is only applicable for the structures obtained with the same exposure doses, and reflow simulation of any other structure requires preliminary measurements.
On the other hand, in case of any structure reflow simulation, carried out using SE, the only question is the distribution of structure vertex mobilities.
Kirchner mentioned that inverse mobility should correlate with PMMA viscosity, but the relation between mobility and viscosity was still unclear.
%Thus, the purpose of this study is to investigate the relation between polymer viscosity and mobility of its surface vertices and to develop the numerical approach for thermal reflow simulation of non-uniform polymer structures obtained by grayscale e-beam lithography using SE as a calculation engine.
Thus, the aim of this study is to develop the numerical approach for thermal reflow simulation of non-uniform PMMA structures obtained by grayscale e-beam lithography using SE as a calculation engine.
For this purpose one should simulate viscosity profile of e-beam exposed PMMA first and then
investigate the relation between PMMA viscosity and mobility of its surface vertices.

% Характеристика работы по структуре во введении и в автореферате не отличается (ГОСТ Р 7.0.11, пункты 5.3.1 и 9.2.1), потому её загружаем из одного и того же внешнего файла, предварительно задав форму выделения некоторым параметрам

%Диссертационная работа была выполнена при поддержке грантов ...

%\underline{\textbf{Объем и структура работы.}} Диссертация состоит из~введения, четырех глав, заключения и~приложения. Полный объем диссертации \textbf{ХХХ}~страниц текста с~\textbf{ХХ}~рисунками и~5~таблицами. Список литературы содержит \textbf{ХХX}~наименование.

%\newpage
%\titlespacing{\section}{\theotstup\parindent}{*\theintvl}{*\theintvl}
%\titlespacing{\subsection}{\theotstup\parindent}{*\theintvl}{*\theintvl}
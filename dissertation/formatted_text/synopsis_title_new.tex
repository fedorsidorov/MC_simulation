\newcommand{\sfs}{\fontsize{14pt}{15pt}\selectfont}
\thispagestyle{empty}

\vspace{0pt plus1fill} %число перед fill = кратность относительно некоторого расстояния fill, кусками которого заполнены пустые места
\begin{flushright}
  \large{На правах рукописи}
\end{flushright}

\vspace{0pt plus3fill} %число перед fill = кратность относительно некоторого расстояния fill, кусками которого заполнены пустые места
\begin{center}
{\large \thesisAuthor}
\end{center}

\vspace{0pt plus3fill} %число перед fill = кратность относительно некоторого расстояния fill, кусками которого заполнены пустые места
\begin{center}
\textbf {\Large \thesisTitle}

\vspace{0pt plus3fill} %число перед fill = кратность относительно некоторого расстояния fill, кусками которого заполнены пустые места
\textbf {\large Специальность:} \\ 
{\large \thesisSpecialtyNumber\ "--- \thesisSpecialtyTitle}

\vspace{0pt plus1.5fill} %число перед fill = кратность относительно некоторого расстояния fill, кусками которого заполнены пустые места
\Large{Автореферат}\par
\large{диссертации на соискание ученой степени\par \thesisDegree}
\end{center}

\vspace{0pt plus4fill} %число перед fill = кратность относительно некоторого расстояния fill, кусками которого заполнены пустые места
\begin{center}
{\large{\thesisCity\ "--- \thesisYear}}
\end{center}

\newpage
% оборотная сторона обложки
\thispagestyle{empty}

%\singlespacing
\setstretch{1.125}

\noindent Работа выполнена в Федеральном государственном бюджетном учреждении \linebreak науки «Физико-технологический институт им. К.А. Валиева Российской \linebreak академии наук»

\vspace{5mm}
    \noindent
    \begin{tabular}{@{}lp{11.3cm}}
        \sfs Научный руководитель:  &  {\sfs \textbf{\supervisorFio,} \par
                                      	кандидат физико-математических наук, \par
                                      	старший научный сотрудник, руководитель \linebreak лаборатории технологий электронной и оптической литографии Физико-технологического института \linebreak им. К.А. Валиева РАН \vspace{3mm}
                                      } \\
        {\sfs Официальные оппоненты:} &
        {\sfs \textbf{\opponentTwoFio,} \par
                  \opponentTwoRegalia, \par
                  главный научный сотрудник лаборатории \linebreak теоретической физики Института проблем \linebreak технологии микроэлектроники и особочистых материалов РАН \par \vspace{3mm}
                  \textbf{\opponentOneFio,} \par
                  \opponentOneRegalia, \par
                  ведущий научный сотрудник, заведующий \linebreak лабораторией фотополимеризации и полимерных \linebreak материалов Института металлоорганической химии им. Г.А. Разуваева РАН \vspace{3mm}
        } \\
        {\sfs Ведущая организация:} & {\sfs Научно-производственный \linebreak комплекс «Технологический центр»}
    \end{tabular}  

\vspace{6mm}
\noindent Защита диссертации состоится <<\underline{\hspace{2em}}>> \underline{\hspace{8em}} 2023 г. в \underline{\hspace{2em}} часов \underline{\hspace{2em}} минут на заседании диссертационного совета 24.2.326.07 в РТУ МИРЭА по адресу: 119454, г. Москва, проспект Вернадского, 78.

\vspace{5mm}
\noindent С диссертацией можно ознакомиться в библиотеке РТУ МИРЭА по \linebreak адресу: 119454, г. Москва, проспект Вернадского, 78. Автореферат диссертации  \linebreak размещен на сайте РТУ МИРЭА www.mirea.ru.

\vspace{5mm}
\noindent{Автореферат разослан \synopsisDate}

\vspace{5mm}
\noindent Ученый секретарь \\
диссертационного совета
\sfs \defenseCouncilNumber \\
\defenseSecretaryRegalia \hspace{50mm} Л.Ю. Фетисов

\onehalfspacing

\newpage

\section{Модель процессов растекания в слое ПММА}

Как было показано выше, распределение среднечисловой молекулярной в слое ПММА в процессе СЭЛТР является неоднородным. Следовательно, согласно формуле~\ref{eq:3p4_3p1}, распределение вязкости в слое ПММА в процессе СЭЛТР также является неоднородным. Разработанный подход моделирования локальной среднечисловой молекулярной массы ПММА вкупе с формулами~\ref{eq:WLF} и \ref{eq:3p4_3p1} позволяет промоделировать локальное значение вязкости ПММА в процессе СЭЛТР. Однако, существующие модели растекания не могут быть использованы в этом случае, так как в исходном виде они применимы только для однородных структур.

В данной работе для моделирования процессов растекания в слое \linebreak ПММА с неоднородным профилем вязкости был разработан численный подход на основе метода конечных элементов. В его основе лежало предположение о существовании связи между вязкостью слоя ПММА и подвижностью вершин его поверхности. Для определения характера этой связи было проведено моделирование термического растекания одной и той же структуры -- прямоугольной решетки -- аналитическим и численным методами. Период решетки и ее глубина составляли 2 мкм и 28 нм соответственно, что по порядку величин согласуется с характерными параметрами структур, получаемых литографическими методами. Вязкость вещества решетки варьировалась в диапазоне 10$^\text{2}$--10$^\text{6}$ Па\:$\cdot$\,с.

Сначала растекание решетки было промоделировано аналитически, что позволило определить профиль решетки в различные моменты времени. Далее растекание решетки было промоделировано численным методом с использованием программы ``Surface Evolver'' с значением подвижности вершин поверхности решетки, равным 1. Это позволило получить значения переменной $s$, которые обеспечивали соответствие между профилями, промоделированными аналитически и численно для различных значений времени растекания при различных значениях вязкости (рисунок~\ref{fig:reflow_1} а)). При этом было установлено, что для значений вязкости вещества решетки в диапазоне 10$^\text{2}$--10$^\text{6}$ Па\:$\cdot$\,с зависимости $s$ от времени растекания $t$ может быть с высокой точностью описана прямой пропорциональностью (рисунок~\ref{fig:reflow_1} б)):
\begin{equation}
	s = \alpha \cdot t.
\end{equation}
Вычисление коэффициента $\alpha$ для каждого значения вязкости позволило получить зависимость $\alpha(\eta)$, которая в логарифмических координатах оказалась практически линейной (рисунок~\ref{fig:eta_alpha}). Аппроксимация зависимости $\alpha(\eta)$ функцией вида
\begin{equation}
	\alpha = C / \eta^\beta
\end{equation}
изображена на рисунке~\ref{fig:eta_alpha}, параметры $C$ и $\beta$ составили 26.14 и 0.989, соответственно (для значения вязкости в единицах СИ).

\begin{figure}[t]
	\begin{minipage}{0.5\textwidth}
		\includegraphics[width=0.91\linewidth]{reflow/grating_eta_100000_14_CORR_200} \\
		\vspace{-28.5ex} \\ \text{\hspace{0em} a}) \\ \vspace{28.5ex}
	\end{minipage}
	\begin{minipage}{0.5\textwidth}
		\hspace{-1em} \includegraphics[width=\linewidth]{reflow/alpha_100000_14_up_200} \\
		\vspace{-28.5ex} \\ \text{\hspace{-0.8em} б}) \\ \vspace{28.5ex}
	\end{minipage}
	\vspace{-3.5em}
	\caption{Моделирование растекания прямоугольной решетки с периодом 2 мкм и глубиной 28 нм аналитическим и численным методами. Вязкость вещества решетки составляет 10$^\text{5}$ Па\:$\cdot$\,с.}
	\label{fig:reflow_1}
\end{figure}

\begin{figure}[h]
	\begin{center}
		\includegraphics[width=0.6\textwidth]{reflow/С_gamma_12_SI_200}
	\end{center}
	\vspace{-1.2em}
	\caption{Рассчитанная зависимость коэффициента $\alpha$ от вязкости вещества решетки.}
	\label{fig:eta_alpha}
\end{figure}

Данная зависимость была использована для определения значений подвижности вершин поверхности решетки для различных значений вязкости вещества решетки. При численном моделировании растекания образца программа ``Surface Evolver'' позволяет отслеживать значение параметра $s$, и логично потребовать, чтобы это значение в точности соответствовало времени растекания. Учитывая уравнения~\ref{eq:SE_v}, \ref{eq:SE_v}, в этом случае подвижность вершин поверхности решетки ($\mu$) может быть выражена следующим образом:
\begin{equation}
	\mu = t / s \equiv \alpha.
\end{equation}
Следовательно, полученная выше зависимость $\alpha(\eta)$ является искомой зависимостью подвижности вершин поверхности резиста от его вязкости, при которой значение переменной $s$ будет в точности соответствовать времени растекания:
\begin{equation}
	\mu(\eta) \approx \frac{26.14}{\eta}.
\end{equation}
Найденная зависимость позволяет промоделировать растекание сплошной структуры в резисте с неоднородным (в плоскости $XY$) профилем вязкости: сначала на основе распределения вязкости рассчитываются подвижности вершин поверхности структуры, затем в программе ``Surface Evolver'' задается поверхность с нужными значениями подвижности вершин и далее производится моделирование эволюции поверхности в заданном промежутке значений переменной $s$.

Однако, слой ПММА в процессе СЭЛТР не только имеет неоднородный профиль вязкости, но является неоднородным в целом.
Как уже было отмечено, в процессе термической деполимеризации в слое ПММА образуется большое количество свободного мономера, который быстро покидает область травления. Это приводит к появлению микрополостей, объем которых может быть вычислен по формуле
\begin{equation}
	V_\mathrm{cav} = N_\mathrm{sci} \cdot 1/\gamma \cdot V_\mathrm{mon},
\end{equation}
где $N_\text{sci}$ -- число разрывов молекул ПММА в некоторой области, $1/\gamma$ -- средняя длина кинетической цепи при деполимеризации, $V_\mathrm{mon}$ -- средний объем одного мономера ($\approx$ 0.14 нм$\ppp$). Процессы растекания в слое ПММА протекают за счет действия сил поверхностного натяжения на границах микрополостей, и для возможности применения разработанного алгоритма для моделирования растекания ПММА в процессе СЭЛТР было использовано следующее приближение. На основе распределения числа разрывов молекул ПММА рассчитывался объем и положение микрополостей, далее в программе ``Surface Evolver'' задавалась сплошная пилообразная структура, объем ``внутренних зубьев'' которой равнялся суммарному объему микрополостей в нужной области резиста (рисунок~\ref{fig:reflow_surface}). Далее на основе распределения молекулярной массы резиста рассчитывалось и задавалось распределение подвижности вершин поверхности пилообразной структуры. После этого растекание данной структуры моделировалось в течение нужного промежутка времени.
\begin{figure}[h]
	\begin{minipage}{0.48\textwidth}
		\includegraphics[width=\linewidth]{reflow/reflow_initial_circles_14_200} \\
		\vspace{-28.5ex} \\ \text{\hspace{0em} a}) \\ \vspace{28.5ex}
	\end{minipage}
	\begin{minipage}{0.48\textwidth}
		\includegraphics[width=\linewidth]{reflow/reflow_fin_14_200} \\
		\vspace{-28.5ex} \\ \text{\hspace{-0.1em} б}) \\ \vspace{28.5ex}
	\end{minipage}
	\vspace{-3.5em}
	\caption{Иллюстрация подхода к моделированию растекания слоя ПММА со внутренними микрополостями.}
	\label{fig:reflow_surface}
\end{figure}


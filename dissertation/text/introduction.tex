\actuality

Формирование трехмерных микро- и наноструктур является востребованным во множестве областей, таких как микроэлектроника, дифракционная оптика и нанофотоника, микро- и нанофлюидика и др. В настоящее время существует множество подходов к решению этой задачи, однако такие преимущества метода, как универсальность, высокая производительность и доступность зачастую оказываются взаимоисключающими. Универсальные методы с высоким разрешением (например, полутоновая литография~\cite{GL_general}, двухфотонная литография~\cite{TPL_castle} или сканирующая зондовая литография~\cite{SPL_mechanical}) предполагают использование сложного высокоточного оборудования и обладают при этом достаточно низкой производительностью. Более производительные и доступные методы позволяют получить только периодические структуры (интерференционная литография~\cite{IL_metamaterials}), либо структуры определенного вида (наноимпринтная литография~\cite{NIL_1}). Таким образом, в настоящее время отсутствует метод получения произвольных микро- и наноструктур, являющийся одновременно высокопроизводительным и простым в реализации.

Ввиду этого внимания заслуживает метод сухого электронно-лучевого травления резиста (СЭЛТР) –- относительно новый одностадийный литографический метод формирования рельефа в слое позитивного резиста, основанный на цепной реакции деполимеризации полимерного резиста и самопроявлении изображения непосредственно в процессе электронно-лучевого экспонирования резиста, проводимого при температурах выше его температуры стеклования~\cite{Bruk_2013, Bruk_2016_mee}. Отличительными особенностями метода являются исключительно высокая чувствительность резиста, высокое разрешение по вертикали и возможность формирование рельефа без этапа проявления, а также скругленные стенки профиля линии. Высокая чувствительность резиста обеспечивает производительность метода в сотни раз превышающую производительность обычной электронно-лучевой литографии. Благодаря этим особенностям метод может быть использован для формирования дифракционных оптических элементов, различных трехмерных микро- и наноструктур или масок. Также возможной областью его применения является формирование каналов для использования в микро- и нанофлюидике, поскольку отсутствие острых углов в сечении канала положительно скажется на его гидравлическом диаметре.

Однако, латеральное разрешение метода ограничено, и в настоящее время при использовании электронно-лучевых систем с диаметром электронного луча около 10-15 нм удается получать линии шириной 300-400 нм. Область применения метода могла бы быть существенно расширена, если бы удалось повысить его латеральное разрешение. В силу одновременного протекания в процессе СЭЛТР множества различных процессов точный механизм формирования конечного профиля линии не был понятен, что не позволяло выявить пути оптимизации данного метода. Таким образом, целесообразным является разработка физической модели метода СЭЛТР, что позволит определить возможности метода и оптимизировать его для применения в различных областях.


\previouswork

Первые шаги в изучении метода микролитографии на основе термической деполимеризации резиста описываются в работе~\cite{Bruk_2000}. В ней приводятся результаты инициированной $\gamma$-излучением деполимеризации ПММА в виде нанометрового слоя, адсорбированного на поверхности пор силохрома. Несмотря на то, что в данной работе термическая деполимеризация не использовалась для формирования структуры в резисте, а исследовалась в общем, результаты работы позволили определить особенности потенциально возможного метода микроструктурирования на основе этого явления. Так, например, были получены оценки для времени диффузии мономера в слое ПММА после разрушения молекулы и длины кинетической цепи деполимеризации, сделаны выводы о масштабах протекания процессов передачи активного центра деполимеризации на мономер и полимер. Также было установлено, что при термической деполимеризации ПММА при температурах 120--180 $^\circ$C влияние процессов реполимеризации пренебрежимо мало.

Впоследствии были проведены эксперименты по изучению термической деполимеризации полиметилметакрилата (ПММА), протекающей при экспонировании электронным лучом, а также впервые были продемонстрированы двумерные и трехмерные структуры, полученные в ПММА в этом процессе~\cite{Bruk_2013}.

Наиболее актуальные на сегодняшний день экспериментальные результаты по исследованию метода сухого электронно-лучевого травления резиста приведены в работах~\cite{Bruk_2015_co, Bruk_2016_mee}. Помимо вышеописанных ступенчатых профилей, в этих работах исследовались периодические профили, полученные при экспонировании резиста электронным лучом вдоль серии параллельных линий. Было продемонстрировано, что при таком экспонировании результирующий профиль приобретает практически синусоидальную форму, что является аргументом в пользу применения метода СЭЛТР для формирования дифракционных оптических элементов~\cite{Mitreska_sin_gratings}. Также была продемонстрирована возможность переноса профиля, полученного в ПММА, в вольфрам или кремний за счет сухого травления в реакторе индуктивно-связанной плазмы. Этот факт теоретически позволяет использовать метод СЭЛТР для формирования, например, штампов для термической наноимпринтной литографии.


\aimsandtasks

Целью данной работы является создание модели процесса электронно-лучевого травления резиста и разработка на ее основе метода, позволяющего оценить параметры процесса для формирования необходимого профиля. Для достижения поставленной цели было необходимо решить следующие задачи:

\begin{enumerate}
	\item Выделить основные процессы, влияющие на профиль линии в методе СЭЛТР;
	\item Разработать модели этих процессов и их совместного протекания;
	\item Провести экспериментальную верификацию разработанной модели;
	\item Используя созданную модель разработать метод определения параметров СЭЛТР (ток, энергия и профиль электронного пучка, температура подложки, скорость охлаждения подложки) для формирования необходимого профиля.
\end{enumerate}


\defpositions

\begin{enumerate}
	\item Создана модель сухого электронно-лучевого травления резиста, позволяющая определить профиль линии получаемый при заданных условиях процесса;
	\item Разработан метод определения параметров СЭЛТР (ток, энергия и профиль электронного пучка, температура подложки, скорость охлаждения подложки) для формирования необходимого профиля, состоящий в оптимизации параметров путем многократного моделирования профиля;
	\item С помощью разработанного метода продемонстрирована возможность формирования методом СЭЛТР синусоидальных дифракционных и голографических элементов, продемонстрированы возможности метода СЭЛТР (разрешение, контраст, чувствительность метода к флуктуации параметров СЭЛТР, возможность формирования произвольных трехмерных структур с профилем, задающимся дифференцируемой функцией);
\end{enumerate}


\novelty

\begin{enumerate}
	\item Установлены основные процессы, определяющие форму линии в методе СЭЛТР: рассеяние электронного пучка в резисте и подложке, электронно-стимулированные разрывы полимерных молекул, термическая деполимеризация резиста, диффузия мономера и растекание резиста, а также разработаны модели перечисленных процессов;
	\item Впервые проведено исследование процесса формирования профиля с помощью электронно-стимулированной термической деполимеризации резиста и показано, как параметры процесса влияют на профиль;
	\item Предложена модель температурной зависимости радиационно-химического выхода разрывов ($G_\mathrm{s}$) -- увеличение $G_\mathrm{s}$ с ростом температуры от 0 до 200 $^\circ$C может быть описано за счет увеличения вероятности разрыва молекулы при электрон-электронном рассеянии от 0.045 до 0.105;
	\item Разработан подход к моделированию растекания резиста с неоднородным профилем вязкости, состоящий в определении подвижности вершин поверхности резиста $\mu$ на основе его вязкости $\eta$ (в Па с) по формуле: $\mu \approx 26.14 / \eta$;
\end{enumerate}


\influence

Теоретическая значимость работы состоит в том, что впервые была создана модель формирования рельефа в резисте за счет совместного воздействия основных процессов, характерных для метода СЭЛТР -- электронно-стимулированных разрывов молекул резиста при повышенной температуре, термической деполимеризации резиста, диффузии мономеров слое резиста и растекания профиля линии за счет пониженной вязкости.
Практическая значимость работы заключается в том, что был разработан подход определения тока и энергии электронного пучка, температуры подложки и скорости охлаждения подложки в методе СЭЛТР для формирования произвольных трехмерных структур с профилем, задающимся дифференцируемой функцией.


\methods

Основным методом исследования основных процессов СЭЛТР являлось математическое моделирование. Для моделирования рассеяния электронного пучка использовался Монте-Карло алгоритм с дискретными потерями энергии. Моделирование слоя ПММА производилось на основе модели идеальной цепи. Моделирование диффузии мономера в слое ПММА проводилось путем численного решение уравнения диффузии. Для моделирования растекания профиля линии применялся аналитический подход, основанный на решении уравнения Навье-Стокса для периодического профиля в резисте с однородной вязкостью и численный подход на основе метода конечных элементов.


\probation

При моделировании рассеяния электронного пучка в системе ПММА/Si сечения упругих и неупругих процессов вычислялись с использованием наиболее современных моделей взаимодействия излучения с веществом (моттовские дифференциальные сечения упругого рассеяния и сечения неупругого рассеяния, рассчитанные с помощью функции потерь энергии). Вероятность электронно-стимулированного разрыва молекулы ПММА определялась на основе моделирования радиационно-химического выхода разрывов, вычисляемого экспериментально из распределения молекулярной массы ПММА. Полученные значения для длины кинетической цепи при деполимеризации ПММА при различных температурах согласуются с опубликованными значениями, рассчитанными на основе констант деполимеризации и обрыва кинетической цепи деполимеризации в кинетических моделях термической деструкции ПММА. Подход, использующийся для моделирования растекания профиля линии в процессе СЭЛТР, эффективно применяется в смежной области -- моделировании растекания структур, полученных методом наноимпринтной литографии, и его точность отмечена в ряде работ. Все вышеперечисленное вкупе с соответствием между экспериментальными и промоделированными профилями обеспечивает достоверность полученных результатов.

Основные результаты работы докладывались на следующих конференциях:
\begin{itemize}
	\item 60-я всероссийская научная конференция МФТИ, Долгопрудный (2016);
	\item International conference on information technology and nanotechnology (ITNT), Самара (2017, 2018, 2020, 2022);
	\item III International Conference on modern problems in physics of surfaces and nanostructures (ICMPSN17), Ярославль (2017);
	\item Micro- and Nanoengineering (MNE), Копенгаген (2018), Родос (2019);
	\item International School and Conference ``Saint-Petersburg OPEN'' on Optoelectronics, Photonics, Engineering and Nanostructures, Санкт-Петербург (2019, 2020).	
\end{itemize}

Диссертация состоит из четырех глав, основные результаты которых изложены в статьях~\cite{my_CO, my_microlenses, my_evidence, my_detailed, my_review_RU, my_MEE, my_Gvalue, my_microscopic, my_Isaev_RU}. Все статьи опубликованы в рецензируемых международных журналах, включённых в библиографические базы (РИНЦ, Scopus, Web of Science).


\contribution

Общая постановка задачи осуществлялась научным руководителем автора \textnohyphenation{Рогожиным} А. Е. Для верификации результатов моделирования были использованы структуры, полученные методом СЭЛТР М. А. Бруком, А. Е. Рогожиным и Е. Н. Жихаревым. Все результаты, изложенные настоящей диссертации, получены автором лично.

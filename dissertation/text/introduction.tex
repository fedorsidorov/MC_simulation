\actuality
Формирование трехмерных микро- и наноструктур является ключевым процессом во множестве областей, таких как микроэлектроника, микро- и наноинженерия, дифракционная оптика и нанофотоника, микро- и нанофлюидика и др. Несмотря на то, что в настоящее время существует множество методов микро- и наноструктурирования, для отдельно взятого метода такие преимущества, как универсальность, высокая производительность и доступность зачастую оказываются взаимоисключающими. Универсальные методы с высоким разрешением (например, полутоновая литография~\cite{GL_general}, двухфотонная литография~\cite{TPL_castle} или сканирующая зондовая литография~\cite{SPL_mechanical}) предполагают использование сложного высокоточного оборудования и обладают при этом крайне низкой производительностью. В свою очередь, более производительные и доступные методы позволяют получить только периодические структуры (интерференционная литография~\cite{IL_metamaterials}) либо структуры определенного вида (наноимпринтная литография~\cite{NIL_1}).

Ввиду вышеописанных особенностей основных существующих методов микроструктурирования внимания заслуживает относительно новый одностадийный литографический метод формирования рельефа в слое позитивного резиста -- сухое электронно-лучевое травление резиста (СЭЛТР). В его основе лежит реакция цепной термической деполимеризации позитивного полимерного резиста, протекающая  в процессе экспонирования резиста электронным лучом при температурах, превышающих температуру стеклования резиста, и обеспечивающая формирование изображения в резисте непосредственно при экспонировании~\cite{Bruk_2013, Bruk_2016_mee}. Отличительными особенностями метода СЭЛТР являются исключительно высокая чувствительность резиста, высокое разрешение по вертикали, возможность формирования рельефа без этапа проявления, а также скругленный профиль сформированного рельефа. Высокая чувствительность резиста обеспечивает производительность метода, в десятки раз превышающую производительность обычной электронно-лучевой литографии. Благодаря этим особенностям метод СЭЛТР может быть использован для формирования различных микро- и наноэлектромеханических систем, оптоэлектронных приборов, дифракционных и голографических оптических элементов, различных трехмерных микро- и наноструктур или масок. Также возможной областью его применения является формирование каналов для микрофлюидных устройств, поскольку сглаженный профиль канала положительно скажется на его гидравлическом диаметре.

Однако, латеральное разрешение метода СЭЛТР и контраст изображения, получаемого этим методом, ограничены. До настоящего времени при использовании электронно-лучевых систем с диаметром луча около 10~нм c помощью метода СЭЛТР удавалось получать канавки c минимальной шириной 300-400~нм и максимальным углом наклона стенок около 20$^\circ$. В силу одновременного протекания при СЭЛТР множества различных процессов точный механизм формирования конечного профиля линии не был понятен, что не позволяло выявить пути оптимизации данного метода. Таким образом, целесообразным являлось создание физической модели метода СЭЛТР, которая позволила бы определить возможности метода и оптимизировать его для применения в различных областях.


\previouswork

Первые шаги в изучении метода микролитографии на основе радиационно-стимулированной термической деполимеризации резиста описываются в работе~\cite{Bruk_2000}. В ней проводилось исследование инициированной $\gamma$-излучением деполимеризации полиметилметакрилата (ПММА), адсорбированного на поверхности пор силохрома. Несмотря на то, что в данной работе термическая деполимеризация не использовалась для формирования рельефа в резисте, а исследовалась в общем, результаты работы позволили определить особенности потенциально возможного метода микроструктурирования на основе этого явления. Так, например, были получены оценки для средней длины кинетической цепи при деполимеризации ПММА и времени диффузии мономера в слое ПММА после разрушения молекулы, а также были сделаны выводы о масштабах протекания процессов передачи активного центра деполимеризации на мономер и полимерную молекулу. Помимо этого было установлено, что при радиационно-стимулированной термической деполимеризации ПММА в области температур 120--180~$^\circ$C влияние процессов реполимеризации пренебрежимо мало.

Впоследствии были проведены эксперименты по изучению термической деполимеризации ПММА, протекающей при его экспонировании электронным лучом, а также впервые были продемонстрированы двумерные и трехмерные структуры, полученные в этом процессе~\cite{Bruk_2013}.

Наиболее актуальные на сегодняшний день результаты экспериментальных исследований процесса сухого электронно-лучевого травления резиста приведены в работах~\cite{Bruk_2015_co, Bruk_2016_mee}. В них исследовались профили, полученные методом СЭЛТР при экспонировании резиста вдоль серии параллельных линий при различных параметрах экспонирования. Было продемонстрировано, что при таком экспонировании может быть получен рельеф с профилем, близким к синусоидальному, что является аргументом в пользу применения метода СЭЛТР для формирования различных дифракционных и голографических оптических элементов~\cite{Mitreska_sin_gratings}. Также была продемонстрирована возможность переноса профиля, полученного в ПММА, в вольфрам или кремний путем сухого травления в реакторе индуктивно-связанной плазмы, что теоретически позволяет использовать метод СЭЛТР для формирования штампов для термической наноимпринтной литографии.


\aimsandtasks
Целью данной работы является создание модели процесса сухого электронно-лучевого травления резиста и разработка на ее основе метода, позволяющего оценить параметры процесса для формирования необходимого профиля. Для достижения поставленной цели было необходимо решить следующие задачи:

\begin{enumerate}
	\item Выделить основные процессы, влияющие на профиль линии в методе СЭЛТР.
	\item Разработать модели этих процессов и модель их совместного протекания.
	\item Провести экспериментальную верификацию разработанной модели СЭЛТР.
	\item Используя созданную модель, разработать метод определения параметров СЭЛТР (тока, энергии и профиля электронного пучка, температуры подложки, скорости охлаждения подложки) для формирования необходимого профиля.	
\end{enumerate}


\defpositions
\begin{enumerate}
	\item Впервые создана модель сухого электронно-лучевого травления резиста, учитывающая рассеяние электронного пучка, электронно-стимулированные разрывы молекул резиста, процессы деполимеризации, диффузии и растекания и позволяющая определить профиль линии, получаемый при заданных условиях процесса.
	\item Определены минимальная ширина и максимальный угол наклона стенок канавки, получаемой методом СЭЛТР при экспонировании в линию -- 300 нм и 70$^\circ$ соответственно.
	\item Определено влияние флуктуаций параметров процесса СЭЛТР на конечную форму профиля, продемонстрирована возможность формирования методом СЭЛТР синусоидальных дифракционных и голографических элементов.
\end{enumerate}


\novelty
\begin{enumerate}
	\item Впервые проведено исследование процесса формирования канавки с помощью электронно-стимулированной термической деполимеризации резиста и показано, как параметры процесса влияют на профиль канавки.
	\item Предложена модель температурной зависимости радиационно-химического выхода разрывов ($G_\mathrm{s}$) молекул ПММА -- увеличение $G_\mathrm{s}$ с ростом температуры от 0 до 200 $^\circ$C может быть описано за счет увеличения вероятности разрыва молекулы при электрон-электронном рассеянии от 0.045 до 0.105.
	\item Разработан подход к моделированию растекания резиста с неоднородным профилем вязкости, состоящий в определении подвижности вершин поверхности резиста $\mu$ на основе его вязкости $\eta$ (в Па с) по формуле: $\mu \approx 26.14 / \eta$.
\end{enumerate}


\influence
Теоретическая значимость работы состоит в том, что впервые была создана модель формирования рельефа в резисте за счет совместного протекания основных процессов, характерных для метода СЭЛТР -- рассеяния электронного пучка, электронно-стимулированных разрывов молекул резиста, термической деполимеризации резиста, диффузии мономера и растекания резиста.


Практическая значимость работы заключается в том, что был разработан метод определения тока, энергии и профиля электронного пучка, температуры подложки и скорости охлаждения подложки в методе СЭЛТР для формирования произвольных трехмерных структур с профилем, задающимся дифференцируемой функцией.


\methods
Основным методом исследования процессов СЭЛТР являлось математическое моделирование. Для моделирования рассеяния электронного пучка использовался алгоритм на основе метода Монте-Карло. Моделирование слоя ПММА проводилось на основе модели идеальной цепи. Для моделирования термической деполимеризации ПММА использовалась кинетическая модель, учитывающая изменение количества молекул различной степени полимеризации за счет основных процессов, протекающих при деполимеризации. Моделирование диффузии мономера в слое ПММА проводилось путем численного решение уравнения диффузии. При моделировании растекания резиста применялся аналитический подход, основанный на решении уравнения Навье-Стокса для периодической структуры в резисте с однородным профилем вязкости, и численный подход на основе метода конечных элементов.


\probation
При моделировании рассеяния электронного пучка в системе ПММА/Si использовались сечения упругих и неупругих процессов, рассчитанные на основе наиболее современных подходов (моттовские сечения упругого рассеяния и сечения неупругого рассеяния, рассчитанные на основе функции потерь энергии). Вероятность разрыва молекулы ПММА при электрон-электронном рассеянии вычислялась путем моделирования значений радиационно-химического выхода разрывов, полученных экспериментально. Для описания цепной реакции термической деполимеризации ПММА использовалась кинетическая модель, учитывающая основные процессы, протекающие при деполимеризации. Константа скорости инициирования кинетической цепи была промоделирована на основе разработанного подхода к описанию электронно-стимулированных разрывов молекул ПММА при различных температурах. При моделировании диффузии мономера в слое ПММА использовались значения коэффициентов диффузии, согласующиеся с экспериментальной зависимостью потока мономера из слоя ПММА от времени при ионно-стимулированной деполимеризации ПММА. Подходы, на основе которых была разработана модель растекания резиста в методе СЭЛТР, эффективно применяются для моделирования растекания структур, полученных методом наноимпринтной литографии и полутоновой электронно-лучевой литографии, и их точность отмечена в ряде работ. Все вышеперечисленное вкупе с соответствием между экспериментальными и промоделированными профилями обеспечивает достоверность полученных результатов.

Основные результаты работы докладывались на следующих конференциях:
\begin{itemize}
	\item 60-я всероссийская научная конференция МФТИ, Долгопрудный (2016);
	\item International conference on information technology and nanotechnology (ITNT), Самара (2017, 2018, 2020, 2022, 2023);
	\item III International Conference on modern problems in physics of surfaces and nanostructures (ICMPSN17), Ярославль (2017);
	\item Micro- and Nanoengineering (MNE), Копенгаген (2018), Родос (2019);
	\item International School and Conference ``Saint-Petersburg OPEN'' on Optoelectronics, Photonics, Engineering and Nanostructures, Санкт-Петербург (2019, 2020).	
\end{itemize}

Диссертация состоит из четырех глав, основные результаты которых изложены в статьях~\cite{my_CO, my_microlenses, my_evidence, my_detailed, my_review_RU, my_MEE, my_Gvalue, my_microscopic, my_Isaev_RU}. Все статьи опубликованы в рецензируемых международных журналах, включённых в библиографические базы (РИНЦ, Scopus, Web of Science).


\contribution
Общая постановка задачи осуществлялась научным руководителем автора \textnohyphenation{Рогожиным} А. Е. Для верификации результатов моделирования были использованы структуры, полученные методом СЭЛТР М. А. Бруком, А. Е. Рогожиным и Е. Н. Жихаревым. Все результаты, изложенные в настоящей диссертации, получены автором лично.

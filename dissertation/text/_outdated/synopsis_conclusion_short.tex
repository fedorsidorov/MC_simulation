\chapter*{Заключение}
\addcontentsline{toc}{chapter}{Заключение}

В данной работе описывается исследование относительно нового метода формирования трехмерных микро- и наноструктур -- сухого электронно-лучевого травления резиста (СЭЛТР). В основе данного метода лежит цепная реакция термической деполимеризации, которая протекает при экспонировании позитивного полимерного резиста электронным лучом при температурах, превышающих температуру стеклования резиста. 
Преимуществами метода СЭЛТР являются высокая производительность и возможность формирования рельефа со сглаженным профилем, а к его недостаткам можно отнести невысокие латеральное разрешение и аспектное отношение получаемых структур. Поскольку в методе СЭЛТР рельеф формируется при одновременном протекании нескольких различных процессов, определение влияния каждого из процессов на результирующий профиль рельефа, разработка методов оптимизации метода СЭЛТР и оценка его возможностей на основе лишь экспериментальных исследований представлялись затруднительными. Таким образом, целесообразным являлось создание физической модели метода СЭЛТР, которая позволила бы определить возможности метода и оптимизировать его для применения в различных областях.

В данной работе разработана физическая модель относительно нового метода формирования трехмерных микро- и наноструктур -- сухого электронно-лучевого травления резиста (СЭЛТР). В основе данного метода лежит цепная реакция термической деполимеризации, которая протекает при экспонировании позитивного полимерного резиста электронным лучом при температурах, превышающих температуру стеклования резиста. В разработанной модели учитываются основные процессы, определяющие конечный профиль линии, получаемой методом СЭЛТР -- рассеяние электронного пучка, электронно-стимулированные разрывы молекул резиста, процессы деполимеризации, диффузии и растекания. Данная модель позволяет промоделировать профиль линии, получаемой методом СЭЛТР, при различных параметрах экспонирования и последующего охлаждения образца, что было использовано для детального изучения метода СЭЛТР. Было установлено, что предельные латеральное разрешение метода СЭЛТР и угол наклона стенок канавки, получаемой этим методом, составляют около 300 нм и 70$^\circ$, соответственно. Было исследовано влияние флуктуаций параметров экспонирования и охлаждения на результирующий профиль линии, что позволило сформулировать требования к стабильности параметров в методе СЭЛТР.  Помимо этого, было продемонстрировано, что при правильном подборе параметров метод СЭЛТР может быть использован для формирования синусоидальных голографических решеток с плотностью штрихов до 2000 1/мм. Также в качестве демонстрации возможностей разработанного алгоритма были промоделированы профили линий, полученных методом СЭЛТР при экспонировании с различными распределениями дозы экспонирования по области линии. Это указывает на возможность использования алгоритма в целях определения параметров СЭЛТР для формирования необходимого профиля.
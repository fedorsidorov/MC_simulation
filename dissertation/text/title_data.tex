%%% Основные сведения %%%
\newcommand{\thesisAuthor}             % Диссертация, ФИО автора
{%
    \texorpdfstring{% \texorpdfstring takes two arguments and uses the first for (La)TeX and the second for pdf
        Сидоров Федор Алексеевич% так будет отображаться на титульном листе или в тексте, где будет использоваться переменная
    }{%
        Сидоров, Федор Алексеевич% эта запись для свойств pdf-файла. В таком виде, если pdf будет обработан программами для сбора библиографических сведений, будет правильно представлена фамилия.
    }%
}
\newcommand{\thesisUdk}                % Диссертация, УДК
{\todo{xxx.xxx}}
\newcommand{\thesisTitle}              % Диссертация, название
{\texorpdfstring{\MakeUppercase{Физические механизмы сухого электронно-лучевого травления}}{Физические механизмы сухого электронно-лучевого травления}}
\newcommand{\thesisSpecialtyNumber}    % Диссертация, специальность, номер
{XX.XX.XX}
\newcommand{\thesisSpecialtyTitle}     % Диссертация, специальность, название
{XXX}
\newcommand{\thesisDegree}             % Диссертация, научная степень
{кандидата физико-математических наук}
\newcommand{\thesisCity}               % Диссертация, город защиты
{Москва}
\newcommand{\thesisYear}               % Диссертация, год защиты
{2022}
\newcommand{\thesisOrganization}       % Диссертация, организация
{Федеральное государственное бюджетное учреждение науки «Физико-технологический институт им. К.А. Валиева Российской академии наук»}

\newcommand{\thesisInOrganization}       % Диссертация, организация в предложном падеже: Работа выполнена в ...
{Федеральном государственном бюджетном учреждении науки «Физико-технологический институт им. К.А. Валиева Российской академии наук»}

\newcommand{\supervisorFio}            % Научный руководитель, ФИО
{Рогожин Александр Евгеньевич}
\newcommand{\supervisorRegalia}        % Научный руководитель, регалии
{кандидат физико-математических наук}

\newcommand{\opponentOneFio}           % Оппонент 1, ФИО
{\todo{XXX}}
\newcommand{\opponentOneRegalia}       % Оппонент 1, регалии
{\todo{регалии}}
\newcommand{\opponentOneJobPlace}      % Оппонент 1, место работы
{\todo{место работы}}
\newcommand{\opponentOneJobPost}       % Оппонент 1, должность
{\todo{должность}}

\newcommand{\opponentTwoFio}           % Оппонент 2, ФИО
{\todo{XXX}}
\newcommand{\opponentTwoRegalia}       % Оппонент 2, регалии
{\todo{регалии}}
\newcommand{\opponentTwoJobPlace}      % Оппонент 2, место работы
{\todo{место работы}}
\newcommand{\opponentTwoJobPost}       % Оппонент 2, должность
{\todo{должность}}

\newcommand{\leadingOrganizationTitle} % Ведущая организация, дополнительные строки
{Ведущая организация}

\newcommand{\defenseDate}              % Защита, дата
{\underline{\hspace{2em}}.\underline{\hspace{2em}}.2020 в \underline{\hspace{2em}}:\underline{\hspace{2em}}}
\newcommand{\defenseCouncilNumber}     % Защита, номер диссертационного совета
{\underline{\hspace{12em}}}
\newcommand{\defenseCouncilTitle}      % Защита, учреждение диссертационного совета
{Московском физико-техническом институте}
\newcommand{\defenseCouncilAddress}    % Защита, адрес учреждение диссертационного совета
{Московская область, г. Долгопрудный,
Институтский переулок, д. 9}

\newcommand{\defenseSecretaryFio}      % Секретарь диссертационного совета, ФИО
{XXX}
\newcommand{\defenseSecretaryRegalia}  % Секретарь диссертационного совета, регалии
{регалии}            % Для сокращений есть ГОСТы, например: ГОСТ Р 7.0.12-2011 + http://base.garant.ru/179724/#block_30000

\newcommand{\synopsisLibrary}          % Автореферат, название библиотеки
{МФТИ и на сайте института: https://mipt.ru/education/post-graduate/soiskateli-fiziko-matematicheskie\\
-nauki.php}
\newcommand{\synopsisDate}             % Автореферат, дата рассылки
{<<\underline{\hspace{2em}}>> \underline{\hspace{8em}} 2020 года}

\newcommand{\keywords}%                 % Ключевые слова для метаданных PDF диссертации и автореферата
{}
%%% Основные сведения %%%
\newcommand{\thesisAuthor}             % Диссертация, ФИО автора
{%
    \texorpdfstring{% \texorpdfstring takes two arguments and uses the first for (La)TeX and the second for pdf
        Сидоров Федор Алексеевич% так будет отображаться на титульном листе или в тексте, где будет использоваться переменная
    }{%
        Сидоров, Федор Алексеевич% эта запись для свойств pdf-файла. В таком виде, если pdf будет обработан программами для сбора библиографических сведений, будет правильно представлена фамилия.
    }%
}
\newcommand{\thesisUdk}                % Диссертация, УДК
{\todo{xxx.xxx}}
\newcommand{\thesisTitle}              % Диссертация, название
{\texorpdfstring{\MakeUppercase{Физические механизмы сухого электронно-лучевого травления}}{Физические механизмы сухого электронно-лучевого травления}}
\newcommand{\thesisSpecialtyNumber}    % Диссертация, специальность, номер
{2.2.2}
\newcommand{\thesisSpecialtyTitle}     % Диссертация, специальность, название
{Электронная компонентная база микро- и наноэлектроники, квантовых устройств}
\newcommand{\thesisDegree}             % Диссертация, научная степень
{кандидата физико-математических наук}
\newcommand{\thesisCity}               % Диссертация, город защиты
{Москва}
\newcommand{\thesisYear}               % Диссертация, год защиты
{2023}
\newcommand{\thesisOrganization}       % Диссертация, организация
{Федеральное государственное бюджетное учреждение науки «Физико-технологический институт им. К.А. Валиева Российской академии наук»}

\newcommand{\thesisInOrganization}       % Диссертация, организация в предложном падеже: Работа выполнена в ...
{Федеральном государственном бюджетном учреждении науки «Физико-технологический институт им. К.А. Валиева Российской академии наук»}

\newcommand{\supervisorFio}            % Научный руководитель, ФИО
{Рогожин Александр Евгеньевич}
\newcommand{\supervisorRegalia}        % Научный руководитель, регалии
{кандидат физико-математических наук}
\newcommand{\supervisorJobPost}        % Научный руководитель, должность
{старший научный сотрудник, руководитель лаборатории технологий электронной и оптической литографии Физико-технологического института им. К.А. Валиева РАН}
%\newcommand{\supervisorJobPlace}       % Научный руководитель, место работы
%{Физико-технологический институт им. К.А. Валиева РАН}

\newcommand{\opponentOneFio}           % Оппонент 1, ФИО
{Чесноков Сергей Артурович}
\newcommand{\opponentOneRegalia}       % Оппонент 1, регалии
{доктор химических наук}
\newcommand{\opponentOneJobPost}      % Оппонент 1, место работы
{ведущий научный сотрудник, заведующий лабораторией фотополимеризации и полимерных материалов Института металлоорганической химии им. Г.А. Разуваева РАН}
%\newcommand{\opponentOneJobPlace}       % Оппонент 1, должность
%{ведущий научный сотрудник, заведующий лабораторией фотополимеризации и полимерных материалов}

\newcommand{\opponentTwoFio}           % Оппонент 2, ФИО
{Зайцев Сергей Иванович}
\newcommand{\opponentTwoRegalia}       % Оппонент 2, регалии
{доктор физико-математических наук}
\newcommand{\opponentTwoJobPost}      % Оппонент 2, место работы
{главный научный сотрудник лаборатории теоретической физики Института проблем технологии микроэлектроники и особочистых материалов РАН}
%\newcommand{\opponentTwoJobPlace}       % Оппонент 2, должность
%{главный научный сотрудник лаборатории теоретической физики}

\newcommand{\leadingOrganizationTitle} % Ведущая организация, дополнительные строки
{Государственный научный центр Научно-производственный комплекс ``Технологический центр''}

\newcommand{\defenseDate}              % Защита, дата
{\underline{\hspace{2em}}.\underline{\hspace{2em}}.2023 в \underline{\hspace{2em}}:\underline{\hspace{2em}}}
\newcommand{\defenseCouncilNumber}     % Защита, номер диссертационного совета
%{\underline{\hspace{12em}}}
{24.2.326.07}
\newcommand{\defenseCouncilTitle}      % Защита, учреждение диссертационного совета
{...}
\newcommand{\defenseCouncilAddress}    % Защита, адрес учреждение диссертационного совета
{...}

\newcommand{\defenseSecretaryFio}      % Секретарь диссертационного совета, ФИО
{Л.Ю. Фетисов}
\newcommand{\defenseSecretaryRegalia}  % Секретарь диссертационного совета, регалии
{доктор физико-математических наук, доцент}            % Для сокращений есть ГОСТы, например: ГОСТ Р 7.0.12-2011 + http://base.garant.ru/179724/#block_30000

\newcommand{\synopsisLibrary}          % Автореферат, название библиотеки
{РТУ МИРЭА по адресу: 119454, г. Москва, проспект Вернадского, 78. Автореферат диссертации размещен на сайте РТУ МИРЭА www.mirea.ru}
\newcommand{\synopsisDate}             % Автореферат, дата рассылки
{<<\underline{\hspace{2em}}>> \underline{\hspace{8em}} 2023 г.}

\newcommand{\keywords}%                 % Ключевые слова для метаданных PDF диссертации и автореферата
{}
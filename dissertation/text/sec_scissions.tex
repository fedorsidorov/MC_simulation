\section{Моделирование электронно-стимулированных разрывов полимерных молекул}
Количественной характеристикой процесса электронно-стимулированной деградации полимера является радиационно-химический выход разрывов $G_s$, определяемый как число разрывов молекул полимера, происходящее при выделении в нем энергии 100 эВ:
\begin{equation}
	G_s = \frac{N_{scissions}}{100 \text{ эВ}}.
\end{equation}

Экспериментально значение $G_s$ определяется на основе среднечисловых значений молекулярной массы полимера до и после экспонирования ($M_n$ и $M_f$, соответственно), определяемых методом гель-проникающей хроматографии. При известных $M_n$ и $M_f$ значение $G_s$ может быть определено из выражения~\cite{Greeneich1979_Mf_Mn}:
\begin{equation}
	{M_f = \frac{\displaystyle M_n}{1 + \frac{\displaystyle G_s E}{\displaystyle 100 \rho N_A}}}
\end{equation}

Исходя из различных экспериментов по измерению $G_s$, его значение для ПММА при экспонировании электронным лучом было принято равным 1.8~\cite{Charlesby_1964_Gs}. Также было установлено, что при экспонировании гамма- и электронным излучения при различных температурах ($T$) зависимость $\ln G_s (1/T)$ близка к линейной.

Считается, что электронно-стимулированные разрывы полимерных молекул происходят в результате взаимодействия налетающего электрона с валентными электронами атомов углерода, образующими C---C связь в главной цепи ПММА~\cite{Stepanova_2006}.

\begin{narrowfig}{G_value_exp}{G_value_exp}
	Зависимость $G_s$ от T для ПММА при экспонировании гамма- и электронным излучением.
\end{narrowfig}
\chapter{Методы моделирования}

В конце предыдущей главы был сделан вывод о целесообразности построение физической модели метода СЭЛТР для определения возможностей этого метода. В процессе формирования профиля линии в методе СЭЛТР одновременно протекают различные процессы, которые могут быть разделены на пять основных групп:
\begin{enumerate}
	\item Рассеяние электронного пучка в резисте
	\item Электронно-стимулированные разрывы молекул резиста
	\item Термическая деполимеризация резиста
	\item Диффузия продуктов деполимеризации в слое резиста
	\item Процессы растекания резиста
\end{enumerate}

Отдельно друг от друга эти процессы уже исследовались, и в данной главе будут описаны существующие подходы к их описанию.


\section{Моделирование рассеяния электронного пучка в веществе}
Поскольку упругие и неупругие процессы, протекающие при рассеянии заряженных частиц в веществе изучаются практически с начала прошлого века, в настоящее время существует множество подходов к их описанию. При выборе конкретных моделей для процессов упругого и неупругого рассеяния на их основе можно реализовать алгоритм моделирования рассеяния электронного пучка в веществе.

\input{text/subsec_el_inel}
\input{text/subsec_Boltzman_MC}

\section{Моделирование электронно-стимулированных разрывов полимерных молекул}
Количественной характеристикой процесса электронно-стимулированной деградации полимера является радиационно-химический выход разрывов $G_s$, определяемый как число разрывов молекул полимера, происходящее при выделении в нем энергии 100 эВ:
\begin{equation}
	G_s = \frac{N_{scissions}}{100 \text{ эВ}}.
\end{equation}

Экспериментально значение $G_s$ определяется на основе среднечисловых значений молекулярной массы полимера до и после экспонирования ($M_n$ и $M_f$, соответственно), определяемых методом гель-проникающей хроматографии. При известных $M_n$ и $M_f$ значение $G_s$ может быть определено из выражения~\cite{Greeneich1979_Mf_Mn}:
\begin{equation}
	{M_f = \frac{\displaystyle M_n}{1 + \frac{\displaystyle G_s E}{\displaystyle 100 \rho N_A}}}
\end{equation}

Исходя из различных экспериментов по измерению $G_s$, его значение для ПММА при экспонировании электронным лучом было принято равным 1.8~\cite{Charlesby_1964_Gs}. Также было установлено, что при экспонировании гамма- и электронным излучения при различных температурах ($T$) зависимость $\ln G_s (1/T)$ близка к линейной.

Считается, что электронно-стимулированные разрывы полимерных молекул происходят в результате взаимодействия налетающего электрона с валентными электронами атомов углерода, образующими C---C связь в главной цепи ПММА~\cite{Stepanova_2006}.

\begin{narrowfig}{G_value_exp}{G_value_exp}
	Зависимость $G_s$ от T для ПММА при экспонировании гамма- и электронным излучением.
\end{narrowfig}
\subsection{Моделирование термической деполимеризации резиста}
Термическая деполимеризация полимеров включается в себя процессы возникновения активного центра деполимеризации, его распространения вдоль молекулы полимера и последующего затухания или переноса на новую молекулу~\cite{Boyd_1}. Кинетические схемы этих процессов может быть представлена в следующем виде (обозначения $P_n$ и $R_n$ относятся к стабильной полимерной молекуле и радикализованной полимерной молекуле, соответственно, $n$ --степень полимеризации, $R_E$ -- концевой радикал):

\begin{center}
	\textbf{Возникновение активного центра деполимеризации}
	\begin{align*}
		P_n \quad \rightarrow \quad & R_r + R_{n-r} \quad & \text{случайное возникновение} \\
		P_n \quad \rightarrow \quad & R_n + R_E \quad & \text{возникновение на конце молекулы}
	\end{align*}
	\textbf{Распространение активного центра деполимеризации}
	\begin{align*}
		R_n \quad \rightarrow \quad  R_{n-1} + P_1 \qquad\qquad\qquad\qquad\quad\; \text{$P_1$ -- летучий мономер}
	\end{align*}
	\textbf{Перенос активного центра деполимеризации}
	\begin{align*}
		P_n + R_s \quad \rightarrow \quad & P_r + R_{n-r} + P_s
	\end{align*}
	\textbf{Затухание активного центра деполимеризации}
	\begin{align*}
		R_n \quad \rightarrow \quad & P_n \quad & \text{реакция 1 порядка} \\
		R_r + R_s \quad \rightarrow \quad & P_r + P_s \quad & \text{диспропорционирование} \\
		R_r + R_s \quad \rightarrow \quad & P_{r + s} \quad & \text{рекомбинация}
	\end{align*}
\end{center}

На основе данных кинетические схем можно ввести константы вышеописанных процессов и выразить скорость изменения числа стабильных полимерных молекул $P_n$ и радикализованных полимерных молекул $R_n$ за счет каждого из процессов:
\begin{center}
	\textbf{Скорость изменения числа стабильных полимерных молекул $P_n$} \\
	\textit{уменьшения $P_n$ за счет разрывов внутри молекулы:} \\
	$k_s (n-1) P_n$ \\
	\textit{уменьшения $P_n$ за счет разрывов на концах молекулы:} \\
	$k_E P_n$ \\
	\textit{уменьшения $P_n$ за счет переноса активного центра деполимеризации:} \\
	$k_I(R / V)(n-1) P_n$ \\
	\textit{увеличение $P_n$ за счет переноса активного центра деполимеризации:} \\
	{$\displaystyle k_I \frac{R_n}{V} \sum_{n=2}^{\infty} n P_n + k_{\mathrm{r}} \frac{R}{V} \sum_{j=n+1}^{\infty} P_j$} \\
	\textit{увеличение $P_n$ за счет уменьшения числа радикалов:} \\
	$k_T \alpha_n$, $\alpha_n = \left\{
	\begin{array}{l}
		R_n \quad\quad\quad\quad\quad\quad\quad\: \text{реакция 1 порядка} \\
		R_n R / V \quad\quad\quad\quad\quad\: \text{диспропорционирование} \\
		{\displaystyle \frac{1}{2} \sum_{i+j=n}^{\infty} R_i R_j / V} \quad\quad \text{рекомбинация}
	\end{array}\right.,$ \\
	где $R$ -- полное число радикалов: {$\displaystyle R = \sum_{i=1}^{\infty} R_i$} \\
	\textbf{Скорость изменения числа радикализованных полимерных молекул $R_n$} \\
	\textit{увеличение $R_n$ за счет разрывов внутри молекулы:} \\
	$2 k_S \sum_{j=n+1}^{\infty} P_j$ \\
	\textit{увеличение $R_n$ за счет разрывов на концах молекулы:} \\
	$k_E P_n$ \\
	\textit{увеличение $R_n$ за счет распространения активного центра деполимеризации:} \\
	$k_P R_{n+1}$ \\
	\textit{уменьшение $R_n$ за счет распространения активного центра деполимеризации:} \\
	$k_P R_n$ \\
	\textit{увеличение $R_n$ за счет переноса активного центра деполимеризации:} \\
	${\displaystyle k_I \frac{R}{V} \sum_{j=n+1}^{\infty} P_j}$ \\
	\textit{уменьшение $R_n$ за счет переноса активного центра деполимеризации:} \\
	${\displaystyle \frac{k_I R_n}{V} \sum_{n=2}^{\infty} n P_n}$ \\
	\textit{уменьшение $R_n$ за счет уменьшения числа радикалов:} \\
	$k_T \beta R_n$, $\beta = \left\{
	\begin{array}{l}
		1 \quad\quad\quad \text{реакция 1 порядка} \\
		R / V \quad\;\: \text{диспропорционирование или рекомбинация}
	\end{array}\right..$ \\
\end{center}

Учет всех процессов, приводящих к изменению $P_n$ и $R_n$, позволяет описать весь полимерный образец системой уравнений, описывающих каждую степень полимеризации:
\begin{equation} \label{eq:kinetic_system_original}
	\left\{
	\begin{aligned}
		&\dots \\
		&\frac{d P_n}{d t}=-(n-1)\left(k_S+k_I R / V\right) P_n-k_E P_n+k_I R / V \sum_{j=n+1}^{\infty} P_j+k_I R_n \frac{d_0}{m_0}+k_T \alpha_n \\
		&\dots \\
		&\frac{d R_n}{d t}=\left(2 k_S+k_I R / V\right) \sum_{j=n+1}^{\infty} P_j+k_E P_n-\left(\frac{k_I d_0}{m_0}+k_P+k_T \beta\right) R_n+R_P R_{n+1} \\
		&\dots \\
		&\frac{d R_1}{d t}=\left(2 k_S+k_I R / V\right) \frac{W}{x m_0}+\frac{k_E}{m_0} \frac{W}{x}-\left(\frac{k_I d_0}{m_0}+k_T \beta\right) R_1+k_P R_2,
	\end{aligned}
	\right.
\end{equation}
где $m_0$ -- масса мономера, $x$ -- среднечисловая степень полимеризации образца. 

В исходном виде система \ref{eq:kinetic_system_original} включает в себя 2$N$ уравнений ($N$ -- максимальная степень полимеризации молекул образца), и ее решение представляет собой трудоемкую задачу -- не в последнюю очередь за счет суммирования в слагаемых, описывающих эффект переноса активного центра деполимеризации на другую молекулу. Это явление является важной частью процесса полимеризации (в этом случае происходит перенос центра полимеризации)~\cite{chain_transfer_polymerization}, однако его проявление в процессе деполимеризации до сих пор находится под вопросом~\cite{Mita_PMMA_zip_lengths_T}.

Исключение процесса переноса активного центра деполимеризации из рассмотрения, а также предположение о постоянной концентрации радикализованных молекул в слое полимера существенно упрощают систему~\ref{eq:kinetic_system_original}. В предположении о инициировании деполимеризации за счет разрывов в произвольном месте молекулы она принимает вид~\cite{Boyd_3}:
\begin{equation} \label{eq:Boyd_system_3}
	\left\{
	\begin{aligned}
		&\dots \\
		&d P_n / d t=-(n-1) k_s P_n+k_r R_n \bar{R}, \\
		&\dots \\
		&\dots \\
		&d R_n / d t=2 k_s \sum_{j=n+1}^{\infty} P_j+k_p\left(R_{n+1}-R_n\right)-k_T \bar{R} R_n=0 \quad(n \geq 2) \\
		&\dots \\
		&d R_1 / d t=2 k_s\left(W / x m_0\right)+k_p R_2-k_T \bar{R} R_1=0
	\end{aligned}
	\right.,
\end{equation}
где $\bar{R} = R/V$.

Решение системы~\ref{eq:Boyd_system_3} упрощается при условии, что распределение молекулярной массы полимера является известным. В этом случае преобразования системы приводят ее к совокупности уравнений вида~\cite{Boyd_3}:
\begin{equation} \label{eq:moment_equation}
	\frac{d M_i}{d t}=k_s\left(\frac{2}{i+1}-1\right) M_{i+1}+\frac{d M_0}{d t}-k_s M_1 - \frac{i}{\gamma}\left(k_s M_i+\frac{d M_{i-1}}{d t}\right) \quad(i \geq 1),
\end{equation}
где $1/\gamma = k_p / (k_T \bar{R})$ -- средняя длина цепи деполимеризации, $M_i$ -- момент функции распределения порядка $i$:
\begin{equation}
	M_i=\sum_{n=2}^{\infty} n^i P_n.
\end{equation}

В качестве распределения молекулярной массы полимера может использоваться распределение Шульца-Цимма~\cite{Boyd_3, Schulz-Zimm_distribution}, корректно описывающее полимеры, полученные методом радикальной полимеризации~\cite{Schulz-Zimm_distribution_proof}:
\begin{equation} \label{eq:Schulz-Zimm_distribution}
	P_n = C_0 n^z \exp (-n/y)
\end{equation}
где $P_n$ -- число молекул степени полимеризации $n$, $C_0$ -- нормировочный множитель. Параметр $z$ характеризует ширину распределения:
\begin{equation}
	M_w / M_N=(z+2) /(z+1),
\end{equation}
где $M_n$ и $M_w$ -- среднечисловая и средневесовая молекулярная масса, соответственно, а параметр $y$ определяется из выражения:
\begin{equation}
	x=y(z+1).
\end{equation}

При этом моменты функции распределения высших порядков могут быть выражены через параметры $y$ и $z$ и момент первого порядка $M_1$:
\begin{equation}
	M_i=M_1 \prod_{n=2}^i(z+n) y^{i-1}.
\end{equation}
Отметим, что $M_0$ выражает полное число полимерных молекул, $M_1$ -- среднечисловую степень полимеризации.

Далее удобно ввести безразмерные переменные:
\begin{equation}
	\begin{aligned}
		\tau & = y^0 k_s t \\
		\tilde{M}_1 & = M_1 / M_1^0 \\
		\tilde{y} & = y / y^0 \\
		\tilde{\gamma} & = \gamma y^0 \\
		\tilde{x} & = x / x^0 = \left[y(z+1) / y^0\left(z^0+1\right)\right],
	\end{aligned}
\end{equation}
которые в дальнейшем используются в уравнениях вида~\ref{eq:moment_equation} для $i$, равного 1, 2 и 3. В конечном счете система этих трех нелинейных дифференциальных уравнений первого порядка принимает вид:

\begin{equation} \label{eq:scary_system}
	\begin{aligned}
		&\frac{\tilde{M_1}^{\prime}}{\tilde{M_1}}=\left[\frac{1}{\tilde{y}} \frac{d \tilde{y}}{d \tau}+\frac{1}{(z+1)} \frac{d z}{d \tau}-\tilde{y}(z+1)\right] /[1+\tilde{\gamma} \tilde{y}(z+1)] \\
		&\tilde{y}^{\prime}=(B F-C E) /(A E-D B) \\
		&z^{\prime}=(C D-A F) /(A E-D B),
	\end{aligned}
\end{equation}
где
\begin{equation}
	\begin{aligned}
		&A=-\left[\frac{1}{\tilde{y}[1+\tilde{\gamma} \tilde{y}(z+1)]}+\frac{(z+2) \tilde{\gamma}}{(z+2) \tilde{\gamma} \tilde{y}+2}\right] \\
		&B=-\left[\frac{1}{(z+1)[1+\tilde{\gamma} \tilde{y}(z+1)]}+\frac{\tilde{\gamma} \tilde{y}}{(z+2) \tilde{\gamma} \tilde{y}+2}\right] \\
		&C=\left[\frac{\tilde{y}(z+1)}{\tilde{\gamma} \tilde{y}(z+1)+1}-\frac{\frac{1}{3}(z+2)(z+3) \tilde{\gamma} \tilde{y}^2+2(z+2) \tilde{y}}{(z+2) \tilde{\gamma} \tilde{y}+2}\right] \\
		&D=\left[\frac{(z+2) \tilde{\gamma}}{(z+2) \tilde{\gamma} \tilde{y}+2}-\frac{2(z+2)(z+3) \tilde{\gamma} \tilde{y}+3(z+2)}{(z+2)(z+3) \tilde{\gamma} \tilde{y}^2+3(z+2) \tilde{y}}\right] \\
		&E=\left[\frac{\tilde{\gamma} \tilde{y}}{(z+2) \tilde{\gamma} y+2}-\frac{\tilde{\gamma} \tilde{y}(2 z+5)+3}{(z+2)(z+3) \tilde{\gamma} \tilde{y}+3(z+2)}\right] \\
		&F=\left[\frac{\frac{1}{3}(z+2)(z+3) \tilde{\gamma} \tilde{y}^2+2(z+2) \tilde{y}}{(z+2) \tilde{\gamma} \tilde{y}+2}-\right. \\
		&\left.-\frac{\frac{1}{2}(z+2)(z+3)(z+4)\left(\tilde{\gamma} \tilde{y}^2+3(z+2)(z+3) \tilde{y}\right.}{(z+2)(z+3) \tilde{\gamma} \tilde{y}+3(z+2)}\right].
	\end{aligned}
\end{equation}
Производные в левой части~\ref{eq:scary_system} берутся по переменной $\tau$, а сама система решается численно.
\section{Диффузия мономера в слое полимера}
Ключевая величина, описывающая в процесс диффузии произвольной примеси в слое вещества -- коэффициент диффузии. В настоящее время существуют два основных подхода к определению коэффициента диффузии мономера ММА в слое ПММА. Первый из них основан на использовании теории свободного объема, что позволяет непосредственно определить коэффициент диффузии на основе различных параметров вещества. Второй подход основан на косвенном определении коэффициента диффузии за счет моделирования выхода мономера из слоя ПММА и сравнения результатов моделирования с экспериментальными данными.

\subsection{Теория свободного объема}
Точный расчет коэффициента диффузии возможен на основе теории свободного объема~\cite{Vrentas_free_volume, Zielinski_free_volume}, что требует задания большого числа параметров:
\begin{equation}
	\ln D=\ln \bar{D}_0-\frac{E^*}{\mathrm{R} T}-\left\{\frac{\left(1-\omega_2\right) \hat{V}_1^*+\xi \omega_2 \hat{V}_2^*}{\hat{V}_{\mathrm{FH}} / \gamma}\right\}.
\end{equation}
Здесь $\bar{D}_0$ -- константа, $E^*$ -- энергия на моль частиц примеси, необходимая для преодоления сил притяжения, $R$ -- универсальная газовая постоянная, $T$ -- температура, $\xi$ и $\gamma$ -- параметры, $\hat{V}_1^*$ и $\hat{V}_2^*$ -- удельные объемы вещества и примеси, соответственно, $\hat{V}_{\mathrm{FH}}$ -- средний свободный объем полостей в смеси вещества и примеси, $\omega_2$ -- массовая доля полимера в смеси ($w_p$). Величина $\hat{V}_{\mathrm{FH}} / \gamma$ определяется выражением
\begin{equation}
	\hat{V}_{\mathrm{FH}} / \gamma=\left(1-\omega_2\right)\left(\frac{K_{11}}{\gamma_1}\right)\left(K_{21}+T-T_{\mathrm{g} 1}\right)+\omega_2 \hat{V}_{\mathrm{FH} 2} / \gamma_2,
\end{equation}
где $\left(K_{11} / \gamma_1\right)$ и $\left(K_{21}-T_{\mathrm{g} 1}\right)$ -- параметры примеси, величина $\hat{V}_{\mathrm{FH} 2} / \gamma_2$ описывает вклад полимерной матрицы в средний свободный объем полостей. Эта величина зависит от того, находится система выше или ниже температуры стеклования полимера ($T_{g2}$):
\begin{equation}
	\begin{aligned}
		&\hat{V}_{FH2} =
		\hat{V}_2^0 (T_{g2}) \left[ f_{H2}^{G}+\alpha_2 (T-T_{g2}) \right], & T \geq T_{g2} \\
		&\hat{V}_{FH2} =
		\hat{V}_2^0 (T_{g2})\left[f_{H2}^{G}+(\alpha_2-\alpha_{c2})(T-T_{g2})\right], \hspace{1em} & T<T_{g2}
	\end{aligned}
\end{equation}
В этом выражении $\hat{V}_2^0\left(T_{\mathrm{g} 2}\right)$ -- удельный объем полимера при температуре $T_{g2}$, \linebreak $\alpha_2$ -- коэффициент температурного расширения полимера в состоянии равновесия, $f_{\mathrm{H} 2}^{\mathrm{G}}$ -- доля объема пустот в полимере при температуре $T_{g2}$:
\begin{equation}
	f_{\mathrm{H} 2}^{\mathrm{G}}=\alpha_2 K_{22},
\end{equation}
\begin{equation}
	\alpha_{\mathrm{c} 2}=\frac{1}{T_{\mathrm{g} 2}} \ln \left(\frac{\hat{V}_2^0\left(T_{\mathrm{g} 2}\right)\left(1-f_{\mathrm{H} 2}^{\mathrm{G}}\right)}{\hat{V}_2^0(0)}\right),
\end{equation}
\begin{equation}
	\gamma_2=\frac{\hat{V}_2^0\left(T_{\mathrm{g} 2}\right) \alpha_2}{\left(K_{12} / \gamma_2\right)},
\end{equation}
\begin{equation}
	\hat{V}_1^*=\hat{V}_1^0(0); \hspace{1em} \hat{V}_2^*=\hat{V}_2^0(0),
\end{equation}
где $K_{22}$ и $\left(K_{12} / \gamma_2\right)$ -- параметры модели свободного объема, $\hat{V}_1^0(0)$ и $\hat{V}_2^0(0)$ -- удельные объемы примеси и полимера в состоянии равновесия при $T=0$K. Параметры модели свободного объема для диффузии ММА в слое ПММА приведены в таблице~\ref{table:D_free_volume}~\cite{Tonge_free_volume_parameters}.

\begin{table}[h]
	\centering
	\caption{Константы процессов инициирования активного центра и деполимеризации.}
	\begin{tabular}{l c l}
		\hline \hline
		Параметр & \hspace{4em} & Значение \\ \hline
		$\hat{V}_1^0(0)$, см$^{-1}$ & \hspace{1em} & 0.871 \\
		$\tilde{V}_1^0(0)$, см$^3$ моль$^{-1}$ & \hspace{1em} & 86.9 \\
		$\hat{V}_2^0(0)$, см$^3$ г$^{-1}$ & \hspace{1em} & 0.762 \\
		$\tilde{V}_2^*$, см$^3$ моль$^{-1}$ & \hspace{1em} & 135 \\
		$f_{H2}^{G}$ & \hspace{1em} & 0.00456 \\
		$K_{22}$, К & \hspace{1em} & 80 \\
		$(K_{12} / \gamma_2)$, см$^3$ г$^{-1}$ К$^{-1}$ & \hspace{1em} & 1.28$\times$10$^{-4}$ \\
		$\gamma_2$ & \hspace{1em} & 3.88 \\
		$\alpha_{c2}$, K$^{-1}$ & \hspace{1em} & 2.37$\times$10$^{-4}$ \\
		$\xi_L$ & \hspace{1em} & 0.64 \\
		$\xi$ & \hspace{1em} & 0.58 \\
		$E^*$, Дж моль$^{-1}$ & \hspace{1em} & 0.58 \\
		$\bar{D}_0$, см$^2$ с$^{-1}$ & \hspace{1em} & 1.27$\times$10$^{-3}$ \\
		$(K_{11} / \gamma_1)$, см$^3$ г$^{-1}$ К$^{-1}$ & \hspace{1em} & 6.91$\times$10$^{-4}$ \\
		$(K_{21}-T_{g1})$, К & \hspace{1em} & 72.26 \\
		$\hat{V}_2^0(T_{g2})$, см$^3$ г$^{-1}$ & \hspace{1em} & 0.8754 \\
		$\tilde{V}_c$, см$^3$ моль$^{-1}$ & \hspace{1em} & 311 \\
		\hline \hline
	\end{tabular}
	\label{table:D_free_volume}
\end{table}

Результаты применения модели свободного объема для вычисления коэффициента диффузии ММА в ПММА приведены на рис.~\ref{fig:free_volume_diffusion}

\begin{figure}
	\includegraphics[width=0.9\linewidth]{free_volume_diffusion}
	\caption{Результаты применения модели свободного объема для вычисления коэффициента диффузии ММА в ПММА при 296 К (а) и 313 К (б): пунктирная линия -- рассчитанные значения, точки -- экспериментальные значения~\cite{Griffiths_MMA_PMMA_diffusion}.}
	\label{fig:free_volume_diffusion}
\end{figure}


\subsection{Вычисление коэффициента диффузии на основе модели выхода мономера из слоя полимера}
В работе~\cite{Fragala_3_diffusion} проводилось исследование выхода мономера из слоя ПММА при его экспонировании ионным лучом. В модели процесса деполимеризации ПММА рассматривались процессы инициирования активного центра деполимеризации и распространения активного центра деполимеризации вдоль молекулы:

\begin{equation}
	\begin{aligned}
		&{[\text { Полимер }]_N \stackrel{\text { Экспонирование }}{\longrightarrow}[\text { Полимер }]_{N-m}+[\text { Радикал }]_m,} \\
		&{[\text { Радикал }]_m \stackrel{\text { Деполимеризация }}{\longrightarrow} \text { Мономер }+[\text { Радикал }]_{m-1} .}
	\end{aligned}
\end{equation}
$K_i$ и $K_p$.
Концентрация центров инициирования деполимеризации ($c_I$) описывалась уравнением
\begin{equation}
	\frac{\partial c_I}{\partial t}=K_i f(t)-K_p c_I,
\end{equation}
где $K_i$ и $K_p$ -- константы процессов инициирования и деполимеризации, а функция $f(t)$ описывает режим работы ионного луча:
\begin{equation}
	f(t) = 1 - H(t - t_0),
\end{equation}
где $H(t)$ -- функция Хевисайда,  $t_0$ -- время работы ионного луча.

Образование мономера предполагалось однородным по объему полимера, что позволило описать процесс выхода мономера одномерным уравнением диффузии:
\begin{equation} \label{eq:raduino_diff_eq}
	\frac{\partial c_M}{\partial t}=D\left(\frac{\partial^2 c_M}{\partial z^2}\right)+\beta K_p c_I,
\end{equation}
где $c_M$ -- концентрация мономера, $D$ -- коэффициент диффузии мономера в слое ПММА, считающийся постоянным по всему объему слоя, $\beta$ -- количество мономеров, образующихся при инициирования активного центра деполимеризации.

Уравнение~\ref{eq:raduino_diff_eq} дополнялось начальными и граничными условиями, описывающими беспрепятственный переход мономера через границу ПММА/вакуум, отражение мономера от подложки и его отсутствие до и по истечении большого времени после работы ионного луча, имели вид:
\begin{equation} \label{eq:diff_eq_conditions}
	\begin{aligned}
		&\left.c_M\right|_{z=z_0}=0, \\
		&\left.\frac{\partial c_M}{\partial z}\right|_{z=0}=0, \\
		&\left.c_M\right|_{t=0}=0, \\
		&\lim _{t \rightarrow \infty} c_M=0, \\
		&\lim _{t \rightarrow \infty} c_I=0,
	\end{aligned}
\end{equation}
где $z = 0$ и $z = z_0$ -- границы слоя ПММА.

Решение уравнения~\ref{eq:raduino_diff_eq} с условиями~\ref{eq:diff_eq_conditions} позволило рассчитать поток мономера через границу полимер/вакуум во время работы ионного луча ($t < t_0$):
\begin{equation}
	\begin{aligned}
		J_{+}(t)= & A \beta K_i z_0
		\left[
		1-\frac{8}{\pi^2} \sum_{n=1}^{\infty}
		\frac{1}{(2n-1)^2} \frac{1}{(1-\alpha_n / K_p)} \times \right. \\ & \left.
		\times
		\left(
			\exp (-\alpha_n t)-\frac{\alpha_n}{K_p} \exp (-K_p t)
		\right)
		\right]
	\end{aligned}
\end{equation}
и после выключения ионного луча ($t > t_0$):
\begin{equation}
	\begin{aligned}
		J_{-}(t)= A \beta K_i z_0 \left[
			\frac{8}{\pi^2} \sum_{n=1}^{\infty} \frac{1}{(2 n-1)^2} \frac{1}{\left(1-\alpha_n / K_p\right)}\right. \times \quad \quad \quad \quad \\
	\times \left.
	\left(
	\exp (-\alpha_n t) (\exp (\alpha_n t_0)-1)- \frac{\alpha_n}{K_p}
	\exp (-K_p t) (\exp (K_p t_0)-1)
	\right)
	\right].
	\end{aligned}
\end{equation}

Параметры модели $K_i$, $K_p$ и $\beta$ были подобраны за счет сравнения промоделированной зависимости потока мономера от времени с экспериментальной, их значения приведены в таблице~\ref{table:Ki_Kp_D}.

\begin{figure}
	\begin{minipage}{0.48\textwidth}
		\includegraphics[width=\linewidth]{135_20.png}
	\end{minipage}
	\begin{minipage}{0.48\textwidth}
		\includegraphics[width=\linewidth]{150_15.png}
	\end{minipage}
	\caption{Экспериментальная зависимость скорости выхода мономера из слоя ПММА при экспонировании ионным лучом и результаты моделирования, полученные в работе для температур 135$^\circ$C и 150$^\circ$C~\cite{Fragala_3_diffusion}}
\end{figure}

\begin{table}[h]
	\begin{center}
	\caption{Константы процессов инициирования активного центра и деполимеризации и коэффициенты диффузии, полученные в работе~\cite{Fragala_3_diffusion} для различных температур.}
	\begin{tabular}{lc rc rc r}
		\hline \hline
		Температура, $^\circ$C & \hspace{4em} & $K_i \beta$, с$^{-1}$ & \hspace{1em} & $K_p$, с$^{-1}$ & \hspace{1em} & $D$, см$^2$с$^{-1}$ \\ \hline
		135 & \hspace{4em} & $7 \times 10^{-4}$ & \hspace{1em} & 90 & \hspace{1em} & $2 \times 10^{-10}$ \\  
		150 & \hspace{4em} & $1.85 \times 10^{-3}$ & \hspace{1em} & 100 & \hspace{1em} & $3.5 \times 10^{-10}$ \\
		160 & \hspace{4em} & $1.6 \times 10^{-3}$ & \hspace{1em} & 100 & \hspace{1em} & $1.1 \times 10^{-9}$ \\
		170 & \hspace{4em} & $3.2 \times 10^{-3}$ & \hspace{1em} & 120 & \hspace{1em} & $1.2\times10^{-9}$ \\
		185 & \hspace{4em} & $3.1 \times 10^{-3}$ & \hspace{1em} & 300 & \hspace{1em} & $2.1\times10^{-9}$ \\ \hline \hline
	\end{tabular}
	\label{table:Ki_Kp_D}
	\end{center}
\end{table}

\section{Моделирование термического растекания резиста}


\subsection{Аналитический подход}
Моделирование оплавления резиста может быть проведено аналитически на основе подхода, предложенного для моделирования оплавления периодических структур, полученных методом НИЛ~\cite{Leveder_2008, Leveder_2011}. В его основе лежит Фурье-преобразование профиля резиста $h(t)$:
\begin{equation}
	\begin{aligned}
		& h(x, t) = h_0 + \tilde{h}(x, t) \\
		& \tilde{h}(x, t) = \sum_{-\infty}^{+\infty} a_n(t) \exp \left(i n \frac{2 \pi}{\lambda} x\right),
	\end{aligned}
\end{equation}
где $h_0$ -- средняя высота профиля, $\lambda$ -- пространственный период профиля.

Уравнения Навье-Стокса при условии отсутствия проскальзывания и с учетом расклинивающего и Лапласова давления может быть выражено в виде:
\begin{equation}
	\partial_t \tilde{h}-\frac{A}{6 \pi \eta h_0} \partial_x^2 \tilde{h}+\frac{\gamma h_0^3}{3 \eta} \partial_x^4 \tilde{h} = 0,
\end{equation}
где $A$ -- постоянная Гамакера, $\gamma$ -- коэффициент поверхностного натяжения резиста.

Его решение приводит к выражению для времени затухания $n$-й гармоники профиля $\tau_n$:
\begin{equation}
	\frac{1}{\tau_n}=\left(n \frac{2 \pi}{\lambda}\right)^2 \frac{A}{6 \pi h_0 \eta}+\left(n \frac{2 \pi}{\lambda}\right)^4 \frac{\gamma h_0^3}{3 \eta}.
\end{equation}

При выполнении условия $\left(\frac{\displaystyle h_0^2}{\displaystyle \lambda}\right)^2 \ll \frac{\displaystyle A}{\displaystyle \gamma}$ выражение для $\tau_n$ принимает более простой вид:
\begin{equation}
	\tau_n=\frac{3 \eta}{\gamma h_0^3} \times\left(\frac{\lambda}{2 \pi n}\right)^4.
\end{equation}

Результирующий профиль в момент времени $t$ определяется суммой гармоник:
\begin{equation}
	\tilde{h}(x, t)=\sum_{-\infty}^{+\infty} a_n(0) \exp \left(-\frac{t}{\tau_n}+i n \frac{2 \pi}{\lambda} x\right).
\end{equation}

\begin{figure}
	\begin{center}
		\includegraphics[width=0.6\linewidth]{reflow_analytical}
		\caption{Растекание решетки с периодом 2 мкм, полученной в ПММА методом наноимпринтной литографии~\cite{Leveder_2011}. Нагрев производился при температуре 145 $^\circ$С, время нагрева составляло от 50 до 1200 с: а) 2D профили, полученные методом атомно-силовой микроскопии, б) зависимость от времени высоты профиля решетки (точки) и подгонка линейной функцией (сплошная линия).}
		\label{fig:ferlow_analytical}
	\end{center}
\end{figure}


\subsection{Численный подход}
Второй подход к моделированию растекания профиля резиста основан на использовании численного метода конечных элементов, реализованном в программе \textquotedbl Surface Evolver\textquotedbl{}~\cite{Brakke_SE}. В этом подходе структурные свойства трехмерного объекта задаются свойствами его поверхности, и эволюция формы объекта в различных процессах описываются изменением формы его поверхности. При этом объем внутри поверхности на протяжении процесса эволюции формы поддерживается постоянным. В процессе моделирования поверхность объекта разбивается на треугольные площадки -- грани, задаваемыми тремя узлами -- вершинами, которые, в свою очередь, соединяются ориентированными ребрами~\ref{fig:SE_12}.

В процессе моделирования эволюции формы объекта производится на основе минимизации полной поверхностной энергии. Энергия отдельной грани вычисляется по формуле
\begin{equation}
	E_i=\frac{\gamma_i}{2}\left\|\vec{e}_0 \times \vec{e}_1\right\|,
\end{equation}
где $\gamma_i$ -- коэффициент поверхностного натяжения грани с номером $i$. Сила, действующая на вершину $V_0$ (рис.~\ref{fig:SE_12}), определяется выражением
\begin{equation}
	\vec{F}_{V_0}=\frac{T}{2} \cdot \frac{\vec{e}_1 \times\left(\vec{e}_0 \times \vec{e}_1\right)}{\left\|\vec{e}_0 \times \vec{e}_1\right\|}.
\end{equation}

При моделировании растекания двумерных структур в резисте последние представляются в виде фигуры бесконечной протяженность. При этом моделирование проводится для участка фигуры конечной длины с использованием зеркальных граничных условия на краях участка~\ref{fig:SE_3}. Таким образом, возникают три возможных типа граней -- грани на границе полимер/вакуум ($p$), грани на границе полимер/подложка ($ps$) и боковые (зеркальные) грани ($m$). Полная энергия поверхности $E_{tot}$ вычисляется по формуле
\begin{equation}
	E_{tot}=E_p-(E_{p s}+E_m),
\end{equation}
где 
\begin{equation}
	E_x = \sum_{i} E_{x,i}, \hspace{0.5em} x = p, ps, m
\end{equation}

\begin{figure}
	\begin{minipage}{0.55\textwidth}
		\includegraphics[width=\linewidth]{SE_1}
	\end{minipage}
	\begin{minipage}{0.4\textwidth}
		\includegraphics[width=\linewidth]{SE_2}
	\end{minipage}
	\caption{Схематическое изображение поверхности в программе \textquotedbl Surface Evolver\textquotedbl{}}.
	\label{fig:SE_12}
\end{figure}

В работах по использованию программы \textquotedbl Surface Evolver\textquotedbl{} для моделирования растекания слоя ПММА моделирование проводилось в режиме нормализации площади, использующимся для реалистичного описания движения под действием сил поверхностного натяжения. В этом режиме при вычислении силы, действующей на вершину, учитывается площадь всех граней, окружающих вершину. Поскольку каждая из граней задается тремя вершинами,
сила, действующая на каждую из вершин грани, будет пропорциональна отношению площади грани к 1/3 площади всех граней, окружающих данную вершину ($A$):
\begin{equation}
	\vec{F}_{norm} = \frac{\vec{F}}{A/3} = \frac{3\vec{F}}{A}.
\end{equation}

Связь силы, действующей на вершину, со скоростью движения вершины в процессе эволюции поверхности описывается коэффициентом подвижности вершины $m$:
\begin{equation}
	\vec{v} = \vec{F}_{norm} \cdot m = \frac{\vec{F}}{A/3} \cdot m.
\end{equation}

Наконец, смещение вершины определяется формулой
\begin{equation} \label{eq:SE_delta}
	\boldsymbol{\delta} = \vec{v} \cdot scale,
\end{equation}
где $scale$ -- величина, аналогичная времени.

Следует отметить, что в исходном виде данный метод является чисто математической моделью, на что указывает характер зависимости~\ref{eq:SE_delta}, а также использование величины $scale$ вместо привычного \textquotedbl физического\textquotedbl{} времени.
\begin{figure}[t!]
	\begin{center}
		\includegraphics[width=0.9\linewidth]{SE_3}
		\caption{Модель поверхности образца, полученного методом полутоновой литографии, заданная в программе \textquotedbl Surface Evolver\textquotedbl{}~\cite{Kirchner_reflow}.}
		\label{fig:SE_3}
	\end{center}
\end{figure}






\input{text/sec_heating}

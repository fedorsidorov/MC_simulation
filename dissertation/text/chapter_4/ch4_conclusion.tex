\section{Выводы по главе}

В данной главе описывается верификация разработанной модели процесса СЭЛТР и приводятся результаты моделирования конечного профиля линий, получаемых методом СЭЛТР при различных параметрах процессов экспонирования и охлаждения образца. Резюмирую вышеизложенное, можно сделать следующие выводы:

\begin{itemize}
	\item Разработанная модель процесса СЭЛТР, учитывающая рассеяние электронного пучка, электронно-стимулированные разрывы молекул резиста, термическую деполимеризацию резиста и процессы растекания, позволяет с высокой точностью воспроизвести профиль линий, полученных в эксперименте.
	\item Микрополости, образующиеся в слое резиста за счет диффузии мономера, оказывают существенное влияние на профиль линии, получаемой методом СЭЛТР. Наличие микрополостей в центре линии на момент остывания обеспечивает меньший радиус кривизны профиля линии по отношению к тому случаю, когда микрополости на момент остывания отсутствуют.
	\item Максимально допустимые значения флуктуаций параметров экспонирования в процессе СЭЛТР, при которых возможно получение необходимого профиля, составляют около 0.5 кэВ, 0.1 нА и 1~$^\circ$C для энергии пучка, тока экспонирования и температуры образца соответственно.
	\item Формирование профиля линии в процессе СЭЛТР происходит не только на стадии экспонирования -- при охлаждении образца профиль линии продолжает формироваться за счет процессов растекания, что приводит к уменьшению глубины профиля и увеличению его ширины. Таким образом, для обеспечения высокого контраста получаемого изображения следует обеспечить как можно более высокую скорость охлаждения образца. При этом максимально допустимое значение флуктуации скорости охлаждения образца, при котором возможно получение необходимого профиля, составляет около 0.1~$^\circ$C/с.
	\item Для получения максимального латерального разрешения в методе СЭЛТР следует использовать узкий электронный пучок (шириной до 10~нм) с энергией от 25 кэВ. При этом также следует обеспечить высокую скорость охлаждения образца после экспонирования для предотвращения заполнения микрополостей на краях линии. В этом случае разрешение метода СЭЛТР и максимальный угол наклона стенок линии составят около 300 нм и 70$^\circ$ соответственно.
	\item Метод СЭЛТР может быть использован для получения синусоидальных голографических решеток с плотностью штрихов до 2000 1/мм.
	\item Разработанный алгоритм может быть использован для моделирования рельефа, получаемого методом СЭЛТР при экспонировании по произвольной области.
\end{itemize}

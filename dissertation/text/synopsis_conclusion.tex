\chapter*{Заключение}
\addcontentsline{toc}{chapter}{Заключение}

В данной работе проводится изучение относительно нового метода формирования трехмерных микро- и наноструктур -- сухого электронно-лучевого травления резиста (СЭЛТР). В его основе лежит цепная реакция деполимеризации, протекающая при экспонировании позитивного резиста электронным лучом в условиях повышенной температуры. Свободный мономер, образующийся в слое резиста при экспонировании, в дальнейшем покидает объем травления, что приводит к образованию микрополостей. В условиях метода СЭЛТР вязкость резиста снижается до значений, при которых становится возможным растекание резиста, что приводит к заполнению микрополостей и формированию профиля линии непосредственно на стадии экспонирования.% Таким образом, метод СЭЛТР является одностадийным, при этом он может быть реализован в большинстве электронно-лучевых систем с минимальными модификациями -- необходимо обеспечить возможность нагрева образца и эффективное выведение или захват образующегося мономера.

Ранние экспериментальные исследования метода СЭЛТР выявили его характерные особенности --  простоту и высокую производительность, сглаженный профиль всех получаемых структур, а также ограниченное латеральное разрешение и контраст получаемого в резисте изображения. Поскольку в этом методе профиль линии формируется за счет нескольких одновременно протекающих процессов, определение влияния каждого из них на результирующий профиль, разработка методов оптимизации метода СЭЛТР и оценка его возможностей на основе лишь экспериментальных исследований было затруднительным.

В данной работе разработана физическая модель метода СЭЛТР, учитывающая основные процессы, обеспечивающие формирование линии -- рассеяние электронного пучка в резисте и подложке, электронно-стимулированные разрывы молекул резиста, цепную деполимеризацию резиста, диффузию мономера и растекание резиста. В основе этой модели лежит алгоритм моделирования рассеяния электронного пучка в резисте и подложке методом Монте-Карло. Он был использован для моделирования электронно-стимулированных разрывов молекул резиста в процессе СЭЛТР, что позволило определить константу процесса инициирования деполимеризации в различных областях резиста. Ее значения в различные моменты времени в дальнейшем использовалось моделирования локального распределения молекулярной массы резиста в ходе экспонирования. Учитывая, что среднечисловая молекулярная масса резиста определяет как коэффициент диффузии мономера в слое резиста, так и его вязкость, это позволило определить локальный коэффициент диффузии мономера в слое резиста и локальную вязкость резиста для каждого момента времени при экспонировании резиста. На основе рассчитанных значений коэффициента диффузии мономера в резисте установлено, что временем диффузии мономера из слоя резиста можно пренебречь по сравнению со временем экспонирования. Неоднородность профиля экспонирования обеспечивает неоднородный профиль вязкости резиста в методе СЭЛТР, и для моделирования растекания резиста был разработан подход, позволяющий рассчитать значения подвижности вершин поверхности резиста на основе распределения вязкости резиста. Полученные значения подвижности были в дальнейшем использованы для моделирования растекания численным методом. Модель метода СЭЛТР, основанная на моделях вышеописанных процессов продемонстрировала высокую точность моделирования результирующего профиля линии.

Разработанная модель позволяет промоделировать профиль линии, получаемой методом СЭЛТР, при различных параметрах экспонирования и последующего охлаждения образца, что было использовано для детального изучения метода СЭЛТР.
Это позволило оценить влияние флуктуаций параметров экспонирования и охлаждения на результирующий профиль линии, а также определить предельные разрешение метода и угол наклона профиля. Было продемонстрировано, что при выборе оптимальных параметров экспонирования метод СЭЛТР может быть использован для получения отражательных дифракционных решеток, обеспечивающих высокую эффективность. Также в качестве демонстрации возможностей разработанного алгоритма моделирования были приведены профили линий, полученных в резисте при экспонировании с различными распределениями дозы экспонирования по области линии, что указывает на возможность использования алгоритма для определения параметров экспонирования, необходимых для получения структур с необходимым профилем.

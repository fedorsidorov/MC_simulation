\chapter*{Заключение}
\addcontentsline{toc}{chapter}{Заключение}
В данной работе была разработана физическая модель нового высокопроизводительного метода формирования трехмерных микро- и наноструктур -- сухого электронно-лучевого травления резиста (СЭЛТР).
В разработанной модели учитываются основные процессы, влияющие на профиль линии в методе СЭЛТР -- рассеяние электронного пучка, электронно-стимулированные разрывы молекул резиста, электронно-стимулированная термическая деполимеризация резиста, диффузия мономера и растекание резиста.
На основе данной модели был разработан алгоритм моделирования профиля линии, получаемой методом СЭЛТР при произвольных параметрах экспонирования и последующего охлаждения образца, что было использовано для детального изучения этого метода.
Было установлено, что минимальная ширина и максимальный угол наклона стенок канавки, получаемой методом СЭЛТР при экспонировании в линию, составляют около 300 нм и 70$^\circ$ соответственно.
Было также исследовано влияние флуктуаций энергии пучка, тока экспонирования, температуры и скорости охлаждения образца на конечный профиль линии.
За счет этого были определены максимально допустимые флуктуации параметров процесса СЭЛТР, при которых возможно получение необходимого профиля.
Помимо этого, была продемонстрирована возможность формирования методом СЭЛТР синусоидальных дифракционных и голографических элементов и был предложен метод определения параметров СЭЛТР для формирования необходимого профиля.

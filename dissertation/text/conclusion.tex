\chapter*{Заключение}
\addcontentsline{toc}{chapter}{Заключение}

В данной работе проводится исследование относительно нового метода формирования трехмерных микро- и наноструктур -- сухого электронно-лучевого травления резиста (СЭЛТР).
В основе данного метода лежит реакция цепной термической деполимеризации, которая протекает при экспонировании позитивного полимерного резиста электронным лучом при температурах, превышающих температуру стеклования резиста.
Свободный мономер, образующийся в слое резиста в ходе деполимеризации, в дальнейшем покидает объем травления за счет процессов диффузии, что приводит к образованию микрополостей в слое резиста.
В условиях метода СЭЛТР вязкость резиста снижается до значений, при которых становятся явно выраженными процессы растекания, что приводит к заполнению микрополостей и формированию профиля линии непосредственно на стадии экспонирования.
Таким образом, метод СЭЛТР является одностадийным, при этом он может быть реализован в большинстве электронно-лучевых систем с минимальными модификациями -- необходимо обеспечить возможность нагрева образца и эффективное выведение или захват образующегося мономера.
По сравнению с существующими методами микро- и наноструктурирования преимущества метода СЭЛТР заключаются в относительной простоте и высокой производительности, обеспечиваемой реакцией цепной термической деполимеризации.

До настоящего времени проводились лишь экспериментальные исследования данного метода, которые позволили выявить его характерные особенности -- высокую производительность, сглаженный профиль получаемого рельефа, а также ограниченное латеральное разрешение метода и невысокий угол наклона профиля.
В большинстве экспериментов производилось экспонирование резиста ``в кадр'' с промежутком между линиями в несколько микрон, что продемонстрировало возможность получения в резисте волнообразного рельефа глубиной в несколько сотен нанометров.
Было продемонстрировано, что в текущем виде метод СЭЛТР может быть использован для формирования дифракционных и голографических оптических элементов, однако, низкие латеральное разрешение метода и аспектное отношение получаемых структур ограничивали область применения метода.
Поскольку в методе СЭЛТР рельеф формируется за счет одновременного протекания нескольких различных процессов, определение влияния каждого из них на конечный профиль рельефа, выявление путей оптимизации метода и оценка его возможностей на основе лишь экспериментальных исследований представлялись затруднительными.
Таким образом, целесообразным являлось создание физической модели метода СЭЛТР, которая позволила бы определить возможности метода и оптимизировать его для применения в различных областях.

Основными процессами, протекающими при СЭЛТР, являются рассеяние электронного пучка в резисте и подложке, электронно-стимулированные разрывы молекул резиста, электронно-стимулированная термическая деполимеризация резиста, диффузия мономера и растекание резиста.
Некоторые из этих процессов достаточно хорошо изучены -- так, в настоящее время существуют высокоточные модели упругого и неупругого рассеяния электронного пучка в веществе.
В то же время, некоторые процессы являются изученными относительно слабо -- например, для описания электронно-стимулированных разрывов молекул резиста существует только макроскопический подход, основанный на анализе распределения энергии, выделившейся в слое резисте.
Существующие кинетические модели термической деполимеризации полимеров, в свою очередь, требуют задания различных констант, значения которых приведены в литературе лишь для некоторых частных случаев.
В отдельную группу можно выделить процессы диффузии мономера и растекания резиста -- для этих процессов существуют простые и в то же время достаточно точные модели, однако, они не могут быть использованы в исходном виде в силу неоднородности резиста в методе СЭЛТР.
Таким образом, для описания процессов, протекающих при сухом электронно-лучевом травлении резиста, требовалась существенная доработка существующих моделей либо разработка на их основе новых моделей.

В основе разработанной в данной работе модели СЭЛТР лежит алгоритм моделирования рассеяния электронного пучка в резисте и подложке методом Монте-Карло.
Используемая в нем модель неупругого рассеяния электронного пучка в резисте позволяет промоделировать акты электрон-электронного, электрон-фононного и электрон-поляронного рассеяния.
В качестве приводящих к разрыву молекул резиста рассматривались акты электрон-электронного рассеяния, и для моделирования электронно-стимулированных разрывов молекул была введена вероятность разрыва при электрон-электронном рассеянии.
Ее значения для различных температур были найдены за счет моделирования эксперимента по определению радиационно-химического выхода разрывов путем анализа молекулярно-массового распределения проэкспонированного резиста.
Моделирование слоя резиста и актов электрон-электронного рассеяния в нем позволило промоделировать молекулярно-массовое распределение проэкспонированного резиста при различных значениях вероятности разрыва и для каждой температуры подобрать значение, обеспечивающее соответствие между промоделированным и экспериментальным значениями радиационно-химического выхода разрывов.

Разработанный алгоритм моделирования электронно-стимулированных разрывов молекул резиста позволил промоделировать константу скорости инициирования кинетической цепи при деполимеризации в различных областях резиста.
Ее значения в дальнейшем использовалось для численного решения системы кинетических уравнений, описывающих деполимеризацию резиста, что позволило промоделировать изменение локальной среднечисловой молекулярной массы резиста в ходе экспонирования.
Учитывая, что среднечисловая молекулярная масса и температура резиста определяют как значение коэффициента диффузии мономера в слое резиста, так и вязкость резиста, это позволило промоделировать локальные значения коэффициента диффузии и вязкости в различные моменты процесса СЭЛТР.
Решение уравнения диффузии с промоделированными значениями коэффициента диффузии показало, что временем диффузии мономера из слоя резиста можно пренебречь по сравнению с характерным временем протекания других процессов.

Неравномерное экспонирование резиста в процессе СЭЛТР приводит к неоднородному профилю вязкости резиста, и для моделирования процессов растекания был разработан подход на основе метода конечных элементов.
Для различных значений вязкости резиста были рассчитаны соответствующие значения подвижности вершин его поверхности, что позволило в дальнейшем задать необходимое распределение подвижности вершин и численно промоделировать эволюцию поверхности резиста.
При этом для упрощения задачи растекания слой резиста со внутренними микрополостями представлялся в виде пилообразной структуры, объем внутренних зубьев которой равнялся суммарному объему микрополостей.

Модели отдельных процессов, протекающих при СЭЛТР, были объединены в модель процесса СЭЛТР -- все время экспонирования разделялось на промежутки времени величиной 1 с, и в течении каждого промежутка последовательно моделировались вышеописанные процессы.
После моделирования экспонирования резиста также моделировалось его растекание при остывании.
Для верификации разработанной модели использовались образцы, полученные методом СЭЛТР при экспонировании резиста ``в кадр'' в растровом электронном микроскопе при различных значениях температуры и времени экспонирования.
Сравнение экспериментальных и промоделированных профилей продемонстрировало достоверность разработанной модели.

Алгоритм моделирования, созданный на основе разработанной модели сухого электронно-лучевого травления резиста, позволяет промоделировать профиль линии, получаемой методом СЭЛТР при произвольных параметрах экспонирования и охлаждения образца, что в дальнейшем было использовано для детального изучения данного метода.
Было установлено, что минимальная ширина и максимальный угол наклона стенок канавки, получаемой методом СЭЛТР при экспонировании в линию, составляют около 300 нм и 70$^\circ$ соответственно.
Было исследовано влияние флуктуаций параметров экспонирования и последующего охлаждения образца на результирующий профиль линии, что позволило сформулировать требования к стабильности параметров в методе СЭЛТР. 
Помимо этого, было продемонстрировано, что метод СЭЛТР может быть использован для формирования синусоидальных голографических решеток с плотностью штрихов до 2000 1/мм.
Было также показано, что разработанный алгоритм позволяет промоделировать рельеф, получаемый методом СЭЛТР при экспонировании с произвольным распределением плотности тока по области.
За счет этого алгоритм может быть использован в целях определения параметров СЭЛТР для формирования необходимого профиля.

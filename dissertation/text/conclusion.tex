\chapter*{Заключение}
\addcontentsline{toc}{chapter}{Заключение}

В данной работе описывается исследование относительно нового метода формирования трехмерных микро- и наноструктур -- сухого электронно-лучевого травления резиста (СЭЛТР). В основе данного метода лежит цепная реакция термической деполимеризации, которая протекает при экспонировании позитивного полимерного резиста электронным лучом при температурах, превышающих температуру стеклования резиста. Свободный мономер, образующийся в слое резиста в ходе деполимеризации, в дальнейшем покидает объем травления за счет процессов диффузии, что приводит к образованию микрополостей в слое резиста. В условиях метода СЭЛТР вязкость резиста снижается до значений, при которых становится возможным его интенсивное растекание, что приводит к заполнению микрополостей и формированию профиля линии непосредственно на стадии экспонирования. Таким образом, метод СЭЛТР является одностадийным, при этом он может быть реализован в большинстве электронно-лучевых систем с минимальными модификациями -- необходимо обеспечить возможность нагрева образца и эффективное выведение или захват образующегося мономера. По сравнению с существующими методами микро- и наноструктурирования преимущества метода СЭЛТР заключаются в относительной простоте и высокой производительности, обеспечиваемой цепной реакцией термической деполимеризации.

До настоящего времени проводились лишь экспериментальные исследования данного метода, которые позволили выявить его характерные особенности -- высокую производительность, сглаженный профиль получаемых структур, а также ограниченные латеральное разрешение и аспектное отношение получаемых структур. В большинстве экспериментов производилось экспонирование резиста ``в кадр'' с промежутком между линиями в несколько микрон, что продемонстрировало возможность получения в резисте волнообразного профиля глубиной в несколько сотен нанометров. Было продемонстрировано, что в текущем виде метод СЭЛТР может быть использован для формирования дифракционных и голографических оптических элементов, однако, низкие латеральное разрешение и аспектное отношение получаемых структур ограничивали область применимости метода. Поскольку в методе СЭЛТР рельеф формируется за счет одновременного протекания нескольких различных процессов, определение влияния каждого из них на результирующий профиль рельефа, разработка методов оптимизации метода СЭЛТР и оценка его возможностей на основе лишь экспериментальных исследований представлялось затруднительным. Таким образом, целесообразным являлось создание физической модели метода СЭЛТР, которая позволила бы определить возможности метода и оптимизировать его для применения в различных областях.

Основными процессами, протекающими при СЭЛТР, являются рассеяние электронного пучка в резисте и подложке, электронно-стимулированные разрывы молекул резиста, цепная деполимеризация резиста, диффузия мономера и растекание резиста. Несмотря на то, что их совместное протекание в процессе микроструктурирования до настоящего момента не исследовалось, большинство из них являются относительно хорошо изученными. Так, существует множество моделей упругого и неупругого рассеяния электронного пучка в веществе, кинетическая модель термической деполимеризации полимеров, несколько способов расчета значения коэффициента диффузии мономеров в слое полимера и два основных подхода к моделированию растекания полимеров. Однако, до настоящего времени отсутствовала микроскопическая модель электронно-стимулированных разрывов полимерных молекул при различных температурах, модель электронно-стимулированной термической деполимеризации полимеров, а также подход к моделированию растекания полимера с неоднородным профилем вязкости.

В основе разработанной в данной работе модели процесса СЭЛТР лежит алгоритм моделирования рассеяния электронного пучка в резисте и подложке методом Монте-Карло. Используемая в нем модель неупругого рассеяния позволяет промоделировать акты электрон-электронного, электрон-фононного и электрон-поляронного рассеяния. В качестве приводящих к разрыву полимерных молекул рассматривались акты электрон-электронного рассеяния, и для моделирования электронно-стимулированных разрывов молекул была введена вероятность разрыва при электрон-электронном рассеянии. Ее значения для различных температур были найдены за счет моделирования эксперимента по определению радиационно-химического выхода разрывов путем анализа распределения молекулярной массы проэкспонированного резиста. Моделирование слоя резиста и актов электрон-электронного рассеяния в нем позволило промоделировать распределение молекулярной массы проэкспонированного резиста при различных значениях вероятности разрыва и для каждой температуры подобрать значение, обеспечивающее соответствие между промоделированным и экспериментальным значениями радиационно-химического выхода разрывов.

Полученный алгоритм моделирования электронно-стимулированных разрывов молекул резиста позволил промоделировать константу скорости инициирования кинетической цепи при термической деполимеризации в различных областях резиста. Ее значения в дальнейшем использовалось для численного решения системы дифференциальных уравнений, описывающих распределение молекулярной массы резиста, что позволило промоделировать изменение локальной среднечисловой молекулярной массы резиста в ходе экспонирования. Учитывая, что среднечисловая молекулярная масса и температура резиста определяют как значение коэффициента диффузии мономера в слое резиста, так и вязкость резиста, это позволило промоделировать локальный коэффициент диффузии и локальную вязкость для каждого момента времени в процессе СЭЛТР. На основе промоделированных значений коэффициента диффузии мономера в резисте установлено, что временем диффузии мономера из слоя резиста можно пренебречь по сравнению со временем экспонирования.

Неравномерное экспонирование резиста в процессе СЭЛТР обеспечивает неоднородный профиль вязкости резиста, и для моделирования растекания резиста в процессе СЭЛТР был разработан подход на основе метода конечных элементов. Для различных значений вязкости резиста были рассчитаны соответствующие значения подвижности вершин его поверхности, что позволило в дальнейшем задать необходимое распределение подвижности вершин при моделировании эволюции поверхности резиста методом конечных элементов. При этом для упрощения задачи растекания резиста слой резиста со внутренними микрополостями представлялся в виде пилообразной структуры. Объем зубьев данной структуры под поверхностью резиста равнялся суммарному объему микрополостей.

Модели отдельных процессов, протекающих при СЭЛТР, были объединены в модель процесса СЭЛТР -- все время экспонирования разделялось на промежутки времени величиной 1 с, и в течении каждого промежутка последовательно моделировались вышеописанные процессы. После моделирования экспонирования резиста также моделировалось его растекание при остывании. Для верификации разработанной модели использовались образцы, полученные методом СЭЛТР при экспонировании резиста ``в кадр'' в растровом электронном микроскопе при различных значениях температуры и времени экспонирования. Сравнение экспериментальных и промоделированных профилей продемонстрировало высокую точность разработанной модели.

Алгоритм моделирования, созданный на основе разработанной модели сухого электронно-лучевого травления резиста, позволяет промоделировать профиль линии, получаемой методом СЭЛТР при произвольных параметрах экспонирования и охлаждения образца, что в дальнейшем было использовано для детального изучения данного метода. Было установлено, что предельные латеральное разрешение метода СЭЛТР и угол наклона стенок канавки, получаемой этим методом, составляют около 300 нм и 70$^\circ$, соответственно. Было исследовано влияние флуктуаций параметров экспонирования и охлаждения на результирующий профиль линии, что позволило сформулировать требования к стабильности параметров в методе СЭЛТР.  Помимо этого, было продемонстрировано, что при правильном подборе параметров метод СЭЛТР может быть использован для формирования синусоидальных голографических решеток с плотностью штрихов до 2000 1/мм. Также в качестве демонстрации возможностей разработанного алгоритма были промоделированы профили линий, полученных методом СЭЛТР при экспонировании с различными распределениями плотности тока экспонирования по области. Это указывает на возможность использования алгоритма в целях определения параметров СЭЛТР для формирования необходимого профиля.

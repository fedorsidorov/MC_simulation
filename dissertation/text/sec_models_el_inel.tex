\section{Моделирование рассеяния электронного пучка в веществе}

Поскольку упругие и неупругие процессы, протекающие при рассеянии заряженных частиц в веществе изучаются практически с начала прошлого века, в настоящее время существует множество подходов к их описанию. При выборе конкретных моделей для процессов упругого и неупругого рассеяния на их основе можно реализовать алгоритм моделирования рассеяния электронного пучка в веществе.

\subsection{Модели упругого рассеяния электронов в веществе}
Упругое рассеяние происходит в основном в результате столкновения высокоэнергетических электронов с ядрами атомов, частично экранированными связанными электронами. При этом изменяется направление движения электрона, а его энергия остается практически неизменной. Азимутальный угол рассеяния $\phi$ распределен равномерно в промежутке (0$^\circ$, 360$^\circ$), полярный угол рассеяния $\theta$ распределен в промежутке (0$^\circ$ до 180$^\circ$) со средним значением 5$^\circ$-10$^\circ$.

Основной характеристикой упругого рассеяния электрона на атомах вещества является дифференциальное сечение рассеяния $\frac{d \sigma}{d \Omega}$, определяемое как отношение числа частиц, рассеянных мишенью в элемент телесного угла $d \Omega = d \varphi \sin \theta d \theta$ за единицу времени к плотности потока налетающих частиц. Интеграл от дифференциального сечения по полному телесному углу определяется как полное сечение упругого рассеяния:
\begin{equation} \label{eq:models_1}
	\sigma_{el}=2 \pi \int_0^\pi \frac{d \sigma}{d \Omega} \sin \theta d \theta.
\end{equation}

\subsubsection{Формула Резерфорда}
Для определения дифференциального сечения упругого рассеяния электронов на атомах вещества можно воспользоваться формулой Резерфорда~\cite{Dapor_large_book}:
\begin{equation} \label{eq:models_3}
	\frac{d \sigma_R}{d \Omega}=\frac{Z^2 e^4}{4 E^2(1-\cos \theta+2 \beta)^2},
\end{equation}
где $Z$ -- зарядовое число атомов вещества, $e$ -- заряд электрона, $E$ -- энергия налетающего электрона, $\beta$ -- параметр экранирования поля ядра атома-мишени атомными электронами. Формула Резерфорда хорошо описывает сечения упругого рассеяния электронов на легких атомах, однако, ее точность снижается с ростом зарядового числа атомов, особенно, в области низких энергий электронов (<1 кэВ) (рис. 1) [7–10].

\subsubsection{Моттовские сечения}
Более точные значения сечений упругого рассеяния (моттовские сечения) могут быть получены за счет решения уравнения Дирака для рассеяния релятивистского электрона в центральном статическом поле атома-мишени~\cite{Czyzewski_mott_cs}. В этом подходе дифференциальное сечение упругого рассеяния задается формулой:
\begin{equation}
	\frac{d \sigma_{e l}}{d \Omega}=|f(\theta)|^2+|g(\theta)|^2,
\end{equation}
где $f(\theta)$ и $g(\theta)$ -- амплитуды рассеяния, соответствующими параллельному и антипараллельному направлению спина электрона относительно его направления движения, соответственно, и определяемые выражениями:
\begin{equation}
	\begin{aligned}
		&f(\theta)=\frac{1}{2 i K} \sum_{l=0}^{\infty}\left\{(l+1)\left[\exp \left(2 i \delta_l^{-1}\right)-1\right]+l\left[\exp \left(2 i \delta_l^{+}\right)-1\right]\right\} P_l(\cos \theta) \\
		&g(\theta)=\frac{1}{2 i K} \sum_{l=0}^{\infty}\left[-\exp \left(2 i \delta_l^{-}\right)+\exp \left(2 i \delta_l^{+}\right)\right] P_l^1(\cos \theta).
	\end{aligned}
\end{equation}
Здесь $k$ -- волновое число налетающего релятивистского электрона, $P_l(\cos \theta)$ и $P_l^1 (\cos \theta)$ -- полиномы Лежандра и присоединенные полиномы Лежандра, соответственно, фазовые сдвиги сферических волн, рассчитываемые по формуле:
\begin{equation}
	\tan \left(\eta_l\right)=\frac{K j_{l+1}(K r)-j_l(K r)\left[(W+1) \tan \phi_l^{\pm}+\left(1+l+k^{\pm}\right) / r\right]}{K n_{l+1}(K r)-n_l(K r)\left[(W+1) \tan \phi_l^{\pm}+\left(1+l+k^{\pm}\right) / r\right]},
\end{equation}
где $K^2 = W^2 - 1$ , $W$ -- полная энергия электрона в единицах $mc^2$, $r$ -- расстояние до рассеивающего центра в единицах $h/2 \pi mc$. 
Индексы <<+>> и <<0>> обозначают параллельное и антипараллельное направление спина, соответственно:
\begin{equation}
	\begin{aligned}
		&+: k^{+}=-l-1, \quad & j=l+1 / 2, \\
		&-: k^{-}=l, \quad & j=l-1 / 2 .
	\end{aligned}
\end{equation}

При этом $\phi_l^\pm$ -- предел функции $\phi_l^\pm (r)$ (при $r \rightarrow \infty$), которая находится путем численного интегрирования уравнения Дирака:
\begin{equation}
	\frac{d \phi_l^{\pm}(r)}{d r}=\frac{k^{\pm}}{r} \sin \left[2 \phi_l^{\pm}(r)\right]-\cos \left[2 \phi_l^{\pm}(r)\right]+W-V(r)
\end{equation}
где $V(r)$ -- рассеивающий потенциал.


\subsection{Модели неупругого рассеяния электронов в веществе}
Квазиупругие и неупругие процессы включают в себя все процессы взаимодействия между налетающим электроном и веществом мишени, в которых электрон теряет свою энергию. При этом также происходит изменение направления движения электрона, и полярный угол рассеяния $\theta$ задается выражением~\cite{Ciappa_2010}:
\begin{equation}
	\sin ^2 \theta=\frac{\hbar \omega}{E},
\end{equation}
где $E$ -- энергия электрона до акта рассеяния, $\hbar \omega$ -- потери энергии. В моделях неупругого рассеяния часто рассматривается взаимодействие налетающего электрона с веществом мишени в целом, и для описания такого взаимодействия используется обратная длина свободного пробега $\lambda_{inel}^{-1}(E)$, связанная с сечением неупругого рассеяния формулой:
\begin{equation}
	\lambda_{\text {inel }}^{-1}(E)=n \sigma(E),
\end{equation}
где $n$ -- концентрация рассеивающих центров в веществе.


\subsubsection{Модель непрерывных потерь энергии}
Исторически первые подходы к описанию потерь энергии электрона в веществе основывались на формуле Бете~\cite{Bethe}:
\begin{equation}
	-\left(\frac{d E}{d s}\right)_{\text {Bethe }}=2 \pi e^4 N_{\mathrm{A}} \frac{\rho}{Z} \frac{1}{E} \ln \left(\frac{1.66 E}{J}\right),
\end{equation}
где $N_A$ -- число Авогадро, $\rho$ -- плотность вещества, $Z$ -- его порядковый номер, соответственно, $e$ и $E$ -- заряд и энергия движущегося в веществе электрона, соответственно. Средний потенциал ионизации $J$ определяется экспериментально или вычисляется на основе порядкового номером атомов вещества~\cite{Dapor_large_book}:
\begin{equation}
	\frac{J}{Z}=9.76+58.8 Z^{-1.19}
\end{equation}
Формула Бете с высокой точностью описывает потери энергии в области высоких энергий налетающего электрона ($E \gg J$). Однако, при приближении энергии налетающего электрона к среднему потенциалу ионизации точность формулы снижается, а в области потери энергии, рассчитываемые по ней, становятся отрицательными. Существуют модификации формулы Бете, позволяющие использовать ее в области низких энергий~\cite{Bethe_corrected}, в которых потери энергии описываются степенной функцией при $E \rightarrow 0$:
\begin{equation}
	-\frac{dE}{ds} \propto \frac{1}{\sqrt{E}}
\end{equation}
В таком виде формула Бете может быть использована, например, для оценки количества обратно отраженных и вторичных электронов, что дает правдоподобные результаты~\cite{Bethe_corr_2ndary_e}. Однако, неограниченный рост потерь энергии при $E \rightarrow 0$ противоречит эмпирическим данным, согласно которым при уменьшении энергии электрона, его потери энергии достигают максимума при энергии в несколько сотен электрон вольт, затем стремятся к нулю~\cite{Shimizu_Review}.


\subsubsection{Модель дискретных потерь энергии}
В современных моделях неупругого рассеяния потери энергии электрона в веществе сводятся к дискретным процессам. В них, аналогично случаю с упругим рассеянием, вводится дифференциальная обратная длина свободного пробега $\frac{d \lambda_{inel}^{-1}}{d \hbar \omega}(E, \hbar \omega)$, позволяющая определить обратную длину свободного пробега по формуле~\cite{Dapor_large_book}:
\begin{equation}
	\lambda_{inel}^{-1}(E)=\int_0^{E / 2} \frac{d \lambda_{\text {inel }}^{-1}(E, \hbar \omega)}{d \hbar \omega} d \hbar \omega,
\end{equation}
а также потери энергии электрона на единицу длины пути $\frac{dE}{ds}$ по формуле:
\begin{equation}
	\frac{dE}{ds}(E) = \int_0^{E / 2} \frac{d \lambda_{\text {inel }}^{-1}(E, \hbar \omega)}{d \hbar \omega} \hbar \omega d \hbar \omega.
\end{equation}

Потери энергии $\hbar \omega$ при неупругом рассеянии также определяются на основе функции $\frac{d \lambda_{inel}^{-1}}{d \hbar \omega}$  методом Монте-Карло~\cite{Ciappa_2010}.

Наиболее распространенный подход к определению дифференциальной обратной длины свободного пробега основана на использовании функции потерь энергии (Energy Loss Function, ELF)~\cite{Dapor_large_book}:
\begin{equation}
	\operatorname{ELF}(q, \omega) \equiv \operatorname{Im}\left[\frac{-1}{\varepsilon(q, \omega)}\right]
\end{equation}
где $\varepsilon(q, \omega)$ -- комплексная диэлектрическая функция, $\vec{q}$ и $\hbar \omega$ -- передаваемые среде импульс и энергия, соответственно. При известной функции потерь энергии дифференциальная обратная длина свободного пробега может быть найдена по формуле:
\begin{equation}
	\frac{d \lambda_{\text {inel }}^{-1}}{d \hbar \omega}=\frac{1}{\pi E a_0} \int_{k_{-}}^{k_{+}} \operatorname{Im}\left[\frac{-1}{\varepsilon(q, \omega)}\right] \frac{d q}{q},
\end{equation}
где
\begin{equation}
	q_{\pm}=\frac{\sqrt{2 m}}{\hbar}(\sqrt{E} \pm \sqrt{E-\hbar \omega}),
\end{equation}
$E$ -- энергия налетающего электрона, m -- масса электрона и $a_0$ -- боровский радиус.

Поскольку функция $\varepsilon(q, \omega)$ может быть найдена из первых принципов только в нескольких идеализированных случаях~\cite{Ritchie_ELF}, часто используется подход на основе оптической функции потерь энергии (Optical Energy Loss Function, OELF), получаемой в пределе $q \rightarrow 0$:
\begin{equation}
	\operatorname{OELF}(\omega) \equiv E L F(0, \omega)=\operatorname{Im}\left[\frac{-1}{\varepsilon(0, \omega)}\right]
\end{equation}

Оптическая функция потерь энергии может быть рассчитана на основе значений коэффициентов преломления ($n$) и поглощения ($k$)~\cite{Dapor_2015_oscillators}:
\begin{equation}
	\operatorname{Im}\left[\frac{-1}{\varepsilon(0, \omega)}\right]=\frac{2 n k}{\left(n^2+k^2\right)^2}
\end{equation}
Коэффициенты n и k табулированы для низких энергий (примерно до 2 кэВ)~\cite{Palik}, для более же высоких энергий они могут быть найдены из компонент атомных факторов рассеяния $f = f_1 + i f_2$ (для молекулярных веществ)~\cite{Henke_photoabs}:
\begin{equation}
	\begin{aligned}
		&n=1-\frac{e^2}{2 \pi m c^2} \lambda^2 N \sum_p x_p f_{1 p}, \\
		&k=\frac{e}{2 \pi m c^2} \lambda^2 N \sum_p x_p f_{2 p}
	\end{aligned}
\end{equation}
где $N$ -- концентрация молекул, содержащих $x_p$ атомов каждого вида, $\lambda$ -- длина волны фотона. Для атомарных веществ оптическая функция потерь энергии может быть найдена непосредственно по формуле:
\begin{equation}
	\operatorname{Im}\left[\frac{-1}{\varepsilon(0, \omega)}\right]=\frac{n_m c \sigma_{p h o t}}{\omega}
\end{equation}
где $n_m$ -- концентрация остовных электронов, $\sigma_phot$ – сечение фотоионизации~\cite{Biggs_cs}. При известной оптической функции потерь энергии поведение функция потерь энергии в области $q > 0$ учитывается с помощью одного из подходов, описанных ниже.


\paragraph{Аппроксимация функции потерь энергии эмпирической функцией} \mbox{} \\
\indent Наиболее простым является подход, в котором поведение функции потерь энергии в области $q > 0$ описывается подгоночными функциями $L(x)$ и $S(x)$, что позволяет непосредственно рассчитать обратную длину свободного пробега~\cite{Ashley_LxSx}:
\begin{equation}
	\begin{aligned}
		&\lambda^{-1}(E)=\frac{m e^2}{2 \pi \hbar^2 E} \int_0^{W_{\max }} \operatorname{Im}\left[\frac{-1}{\varepsilon(0, \omega)}\right] L\left(\frac{\hbar \omega}{E}\right) d \hbar \omega, \\
		&L(x)=(1-x) \ln \frac{4}{x}-\frac{7}{4} x+x^{3 / 2}-\frac{33}{32} x^2,
	\end{aligned}
\end{equation}
а также потери энергии на единицу длины пути:
\begin{equation}
	\begin{aligned}
		&\lambda^{-1}(E)=\frac{m e^2}{2 \pi \hbar^2 E} \int_0^{W_{\max }} \operatorname{Im}\left[\frac{-1}{\varepsilon(0, \omega)}\right] L\left(\frac{\hbar \omega}{E}\right) d \hbar \omega, \\
		&L(x)=(1-x) \ln \frac{4}{x}-\frac{7}{4} x+x^{3 / 2}-\frac{33}{32} x^2
	\end{aligned}
\end{equation}


\newpage
\paragraph{Аппроксимация функции потерь энергии суммой осцилляторов Друде} \mbox{} \\
\indent В данном подходе оптическая функция потерь энергии приближается суммой осцилляторов Друде~\cite{Ritchie_Drude}:
\begin{equation}
	\operatorname{Im}\left[\frac{-1}{\varepsilon(0, \omega)}\right]=\sum_i \frac{A_i \Gamma_i \hbar \omega}{\left[E_i^2-(\hbar \omega)^2\right]^2+\left(\Gamma_i \hbar \omega\right)^2}
\end{equation}
параметры $E_i$, $\Gamma_i$ и $A_i$ которых определяются путем подгонки~\cite{Dapor_2015_oscillators}. Продолжение оптической функции потерь энергии в область осуществляется за счет использования квадратичного закон дисперсии:
\begin{equation}
	E_i(q)=E_i+\frac{\hbar^2 q^2}{2 m}
\end{equation}
что в дальнейшем позволяет построить функцию потерь энергии:
\begin{equation}
	\operatorname{Im}\left[\frac{-1}{\varepsilon(q, \omega)}\right]=\sum_i \frac{A_i \Gamma_i \hbar \omega}{\left[\left(E_i+\frac{\hbar^2 q^2}{2 m}\right)^2-(\hbar \omega)^2\right]^2+\left(\Gamma_i \hbar \omega\right)^2},
\end{equation}

\begin{fig}{OLF_Drude}{OLF_Drude}
	Рисунок 5. а) Оптические функции потерь энергии ПММА и Si, б) оптическая функция потерь энергии ПММА, приближенная суммой осцилляторов Друде.
\end{fig}


\newpage
\paragraph{Диэлектрическая функция Мермина} \mbox{} \\
\indent Наиболее точным подходом к построению функции потерь энергии в органических полимерах является подход на основе модели Мермина~\cite{Mermin}. В его основе лежит диэлектрическая функция Мермина для столкновительной плазмы:
\begin{equation}
	\varepsilon_M(q, \omega)=1+\frac{(1+i \gamma / \omega)\left[\varepsilon_L(q, \omega+i \gamma)-1\right]}{1+(i \gamma / \omega)\left[\varepsilon_L(q, \omega+i \gamma)-1\right] /\left[\varepsilon_L(q, 0)-1\right]},
\end{equation}
где $\gamma$ -- постоянная затухания, $\varepsilon_L(q, \omega)$ -- диэлектрическая функция Линдхарда~\cite{Lindhard}:
\begin{equation}
	\varepsilon_L(q, \omega)=1+\frac{\chi^2}{z^2}\left[f_1(u, z)+i f_2(u, z)\right].
\end{equation}
Здесь $u=\omega /\left(q v_F\right)$, $z=q /\left(2 q_F\right)$ и $\chi^2=e^2 /\left(\pi \hbar v_F\right)$, где $v_F$
-- скорость Ферми валентных электронов вещества, $q_F=m v_F / \hbar$. При этом функции $f_1(u, z)$ и $f_2(u, z)$ определяются формулами:
\begin{equation}
	\begin{aligned}
		f_1(u, z) &=\frac{1}{2}+\frac{1}{8 z}[g(z-u)+g(z+u)] \\
		f_2(u, z) &= \begin{cases}\frac{\pi}{2} u, & z+u<1 \\
			\frac{\pi}{8 z}\left[1-(z-u)^2\right], & |z-u|<1<z+u \\
			0, & |z-u|>1\end{cases},
	\end{aligned}
\end{equation}
где
\begin{equation}
	g(x)=\left(1-x^2\right) \ln \left|\frac{1+x}{1-x}\right|.
\end{equation}

Как и в предыдущем подходе, функция потерь энергии вещества суммой функций потерь отдельных осцилляторов, и ее построение проводится в два этапа. Сначала оптическая функция потерь энергии вещества подгоняется суммой функций потерь энергии Мермина (осцилляторов Мермина) для $q=0$:
\begin{equation}
	\operatorname{Im}\left[\frac{-1}{\varepsilon(0, \omega)}\right]=\sum_i A_i \operatorname{Im}\left[\frac{-1}{\varepsilon_M\left(\omega_i, \gamma_i, q=0, \omega\right)}\right],
\end{equation}
что позволяет получить параметры $A_i$, $\omega_i$ и $\gamma_i$ отдельных осцилляторов~\cite{DeVera_MELF_params}. Параметр $\omega_i$ определяет частоту каждого из осцилляторов, что позволяет найти параметр $v_F$, входящий в величины $u$, $z$ и $\chi$, используемые в диэлектрической функции Линдхарда:
\begin{equation}
	\begin{aligned}
		&\omega_i=\sqrt{\frac{4 \pi n_i e^2}{m}} \Rightarrow n_i=\frac{\omega_i^2 m}{4 \pi e^2} \\
		&v_{F_i}=\frac{\hbar}{m}\left(3 \pi^2 n_i\right)^{1/3},
	\end{aligned}
\end{equation}
где $n_i$ -- концентрация электронов, соответствующая осциллятору с индексом $i$, $m$ -- масса электрона. Далее на основе параметров $A_i$, $\omega_i$ и $\gamma_i$ составляется функция потерь энергии:
\begin{equation}
	\operatorname{Im}\left[\frac{-1}{\varepsilon(q, \omega)}\right]=\sum_i A_i \operatorname{Im}\left[\frac{-1}{\varepsilon_M\left(\omega_i, \gamma_i, q, \omega\right)}\right]
\end{equation}
и определяется дифференциальная обратная длина свободного пробега.


\subsection{Алгоритм моделирования на основе кинетической теории транспорта}
Моделирование на основе кинетической теории транспорта заключается в решении кинетического уравнения Больцмана, описывающего распространение электронов в структуре. Этот метод успешно применяется для моделирования электронного пучка в планарных структурах, состоящих из небольшого количества слоев~\cite{Stepanova_2006, Stepanova_2010}. Для задач с более сложной геометрией необходимо введение дополнительных граничных условий, что значительно усложняет расчет.


\paragraph{Определение распределения электронов по глубине} \mbox{} \\
\indent В большинстве случаев для упрощения расчетов, рассматривается точечный пучок электронов, направленный под прямым углом к поверхности (вдоль оси $z$). Распространение электронов в веществе по глубине может быть описано функцией распределения $f(z, E, \cos \theta_v)$, где z , E – глубина проникновения электрона в образец и его энергия, z – угол между скоростью электрона и осью z . В этом случае уравнение Больцмана принимает вид~\cite{ME_rev_60}:
\begin{equation} \label{eq:Boltzman_1}
	\frac{d E}{d s} \frac{\partial f}{\partial E}+\cos \theta_v \frac{\partial f}{\partial z}=\frac{1}{\Lambda} \int w(\cos \gamma)\left[f\left(\cos \theta_{v^{\prime}}\right)-f\left(\cos \theta_v\right)\right] d \Omega_\gamma,
\end{equation}
где $\frac{dE}{ds}$ -- потери энергии на единицу длины пути, $\Lambda$ –- длина свободного пробега при упругом рассеянии, $v$ и $v^{\prime}$ -- скорости до и после рассеяния, соответственно, $w(\cos \gamma)$ -- нормированное дифференциальное сечение упругого рассеяния на угол $\gamma$:
\begin{equation} \label{eq:Boltzman_2}
	w(\cos \gamma)=\frac{1}{\sigma_{\mathrm{el}}} \frac{d \sigma}{d \Omega_\gamma}
\end{equation}

Уравнение \ref{eq:Boltzman_1} может быть решено в диффузионном приближении~\cite{ME_rev_61}, применимом в диапазоне энергий, характерных для электронно-лучевой литографии. Для этого функция распределения электронов раскладывается по полиномам Лежандра $P(\cos \theta)$:
\begin{equation} \label{eq:Boltzman_3}
	f\left(z, E, \cos \theta_v\right)=\sum_0^{\infty} C_n(z, E) P_n\left(\cos \theta_n\right)
\end{equation}

Подстановка \ref{eq:Boltzman_3} в \ref{eq:Boltzman_1} приводит к разностной дифференциальной схеме для коэффициентов $C_n$:
\begin{equation} \label{eq:Boltzman_4}
	\left(\frac{d E}{d s}\right) \frac{\partial C_n}{\partial E}+\frac{n}{2 n-1} \frac{\partial C_{n-1}}{\partial z}+\frac{n+1}{2 n+3} \frac{\partial C_{n+1}}{\partial z}=-\frac{1}{\lambda_n} C_n,
\end{equation}
где
\begin{equation} \label{eq:Boltzman_5}
	\frac{1}{\lambda_n}=\frac{1}{\Lambda} \int\left[1-P_n(\cos \gamma)\right] W(\cos \gamma) d \Omega_\gamma
\end{equation}

Коэффициенты $C_0$ и $C_1$ пропорциональны плотности вероятности и проекции плотности потока вероятности на ось $z$, соответственно:
\begin{equation} \label{eq:Boltzman_6}
	\begin{gathered}
		\rho(z, E)=\int f\left(z, E, \cos \theta_v\right) d \Omega_v=4 \pi C_0(z, E), \\
		J_z(z, E)=\int v \cos \theta_v f\left(z, E, \cos \theta_v\right) d \Omega_v=\frac{4 \pi}{3} v C_1(z, E).
	\end{gathered}
\end{equation}

Точное решение \ref{eq:Boltzman_1} возможно при отбрасывании в  коэффициентов $C_n$ с \break $n>1$, соответствующими турбулентному движению. При этом \ref{eq:Boltzman_4} приводит к уравнению диффузии:
\begin{equation} \label{eq:Boltzman_7}
	\frac{\partial}{\partial E} \rho(z, E)=a(E) \frac{\partial^2}{\partial z^2} \rho(z, E), a(E)=\left(\frac{d E}{d s}\right)^{-1} \frac{\lambda_1}{3} .
\end{equation}
Его решение:
\begin{equation} \label{eq:Boltzman_8}
	\rho(z, E)=\frac{1}{\sqrt{\pi} \sigma(E)} \exp \left(-\frac{z^2}{4 \sigma^2(E)}\right)\left\{1-\frac{\sqrt{\pi} \sigma(E)}{\Delta \lambda_1\left(E_0\right)} \exp \left(\varsigma^2\right) \operatorname{erfc}\left(\varsigma^2\right)\right\},
\end{equation}
где
\begin{equation} \label{eq:Boltzman_9}
	\zeta=\frac{z}{2 \sigma(E)}+\frac{\sigma(E)}{\Delta \lambda_1\left(E_0\right)}, \quad \sigma^2=\int a(E) d E, \quad \Delta=0.71
\end{equation}

Для описания функции распределения электронов в системе, состоящей из нескольких слоев, решение для предыдущего слоя используется как граничное условие для уравнения диффузии в новом слое.


\paragraph{Определение латерального распределения электронов} \mbox{} \\
\indent Латеральное распределение электронов описывается функцией плотности вероятности $\rho(r, z, E)$, определяющей вероятность нахождения электрона с энергией E в кольце радиуса $r$, имеющем объем $2 \pi r dr dz$ и расположенном параллельно поверхности резиста на глубине $z$. Для определения продольного распределения электронов, необходимо решение уравнения Больцмана в более общем виде, чем \ref{eq:Boltzman_1}~\cite{ME_rev_63}, и при этом отдельно учитывается вклад от электронов, рассеянных на малые углы (индекс $f$), обратного рассеянных электронов (индексы $bd$ и $bs$) и вторичных электронов (индекс $s$)~\cite{ME_rev_64}:
\begin{equation} \label{eq:Boltzman_10}
	\rho(r, z, E)=\rho(z, E)\left[\rho_f(r \mid z, E)+\rho_{b d}(r \mid z, E)\right]+\rho_{b s}(r, z, E)+\rho_s(r, z, E).
\end{equation}
Здесь выражения вида $\rho(r|z,E)$ означают плотность вероятности при известных значениях z и E. Слагаемое $\rho_f(r|z,E)$ описывает вклад в продольное уширение пучка за счет малого количества актов рассеяния первичных электронов на малые углы в слое резиста:
\begin{equation} \label{eq:Boltzman_11}
	\rho_f(r \mid z, E)=\frac{3 \lambda_1}{2 \pi z^3} \exp \left(-\frac{3 \lambda_1 r^2}{2 z^3}\right)
\end{equation}
где $\lambda_1$ определяется из соотношения \ref{eq:Boltzman_5}.

Обратное рассеяние электронов происходит за счет малого количества актов рассеяния на большие углы вблизи границы резиста с подложкой ($\rho_{bs}$), либо за счет диффузии электронов в структуре ($\rho_{bd}$):
\begin{equation} \label{eq:Boltzman_12}
	\begin{gathered}
		\rho_{b s}(r, z, E)=\frac{1}{\pi} \int_z^{z_d} \beta(1+\beta) \rho\left(z^{\prime}, E\right) \frac{z^{\prime}-z}{R} \frac{d z^{\prime} / \Lambda}{\left[(1+\beta) R+z^{\prime}-z\right]^2}, \\
		\rho_{b d}(r \mid z, E)=\frac{A^2}{3} \int_{z_d}^{z_{\max }}\left(\frac{1}{4 \pi \sigma_b^2}\right)^{3 / 2} \exp \left(-\frac{R^2}{4 \sigma_b^2}\right) \frac{z^{\prime}-z}{z_{\max }-z_d} d z^{\prime},\\
		R=\sqrt{r^2+\left(z-z^{\prime}\right)^2}, \\ \sigma_b^2=\int_{E\left(z^{\prime}\right)}^{E(z)} a\left(E^{\prime}\right) d E^{\prime},
	\end{gathered}
\end{equation}
где $\beta$ -- параметр экранирования в формуле Резерфорда для дифференциального сечения упругого рассеяния, $\Lambda$ -- длина свободного пробега для упругого рассеяния, a $a(E)$ – коэффициент диффузии \ref{eq:Boltzman_7}, $E(z)$ , $E(z^\prime)$ – средние энергии электрона, получаемые за счет интегрировании функции $Ef(z, E, \cos \theta_v )$. Параметр z d выражает максимальную глубину проникновения электронов, рассеивающихся на большие углы вблизи границы резиста с подложкой, и его значение выбирается исходя из моделирования методом Монте-Карло. Значения параметров $z_max$ и $A$, определяющих максимальную глубину, на которой могут находиться обратно отраженные электроны и коэффициент обратного отражения электронов соответственно, также выбираются исходя из соответствия результатам Монте-Карло моделирования. Например, для слоя полиметилметакрилата (ПММА) толщиной 0.5 мкм на кремниевой подложке выбираются следующие значения этих параметров: $z_d$ = 0.83 мкм, $z_{max}$ = 8.5 мкм, $A$ = 0.19.

Вклад вторичных электронов в уширение пучка описывается плотностью вероятности вторичных электронов:
\begin{equation} \label{eq:Boltzman_13_0}
	\rho_s(r, z, E)=\int_{2 E}^{E_0} d E^{\prime} P_{inel}\left(E^{\prime}\right) \Phi\left(E, E^{\prime}\right) \int d^3 \vec{r}^{\prime} \frac{1}{8 \pi S^3(E)} \exp \left(\frac{-\left|\vec{r}-\vec{r}^{\prime}\right|}{S(E)}\right) \rho\left(r^{\prime}, z, E^{\prime}\right).
\end{equation}
Здесь $E_0$ -- начальная энергия электронов, $P_{inel}$ вероятность неупругого рассеяния, в котором возникает вторичный электрон, определяемая из выражений для сечения упругого и неупругого рассеяния ($\sigma_{el}$ и $\sigma_{inel}$, соответственно):
\begin{equation} \label{eq:Boltzman_13}
	P_{inel}=\frac{\sigma_{inel}}{\sigma_{el}+\sigma_{inel}},
\end{equation}
Функция $\Phi\left(E, E^{\prime}\right)$ представляет дифференциальное сечение неупругого рассеяния, нормированное на полное сечение неупругого рассеяния, $S(E)$ -- максимальная глубина проникновения электронов, выражающаяся через потери энергии на единицу пути:
\begin{equation} \label{eq:Boltzman_14}
	S(E)=\int_{E_0}^{E_{\min }} d E\left(\frac{d E}{d s}\right)^{-1}
\end{equation}

Одним из наиболее важных результатов моделирования является распределение энергии, выделенной в резисте. Плотность выделенной энергии в расчете на один электрон $I(r,z)$ может быть получена за счет интегрирования произведения плотности вероятности и функции потерь энергии~\cite{ME_rev_64}.


\subsection{Алгоритм моделирования методом Монте-Карло}
При моделировании методом Монте-Карло для каждого электрона из пучка рассчитывается его траектория в структуре. Параметры траектории и потери энергии электрона определяются из дифференциальных сечений упругих и неупругих процессов с использованием случайных чисел из равномерного распределения на промежутке [0, 1). Данный метод требует больших вычислительных мощностей, но при этом его сложность практически не зависит от формы структуры и количества входящих в нее материалов. Также, в отличие от моделирования на основе кинетической теории транспорта, алгоритм расчета траектории электрона в структуре методом Монте-Карло позволяет воспроизвести стохастичность процессов рассеяния.

\paragraph{Определение длины пробега электрона} \mbox{} \\
\indent Путем интегрирования дифференциальных сечений вычисляются значения полных сечений упругого и неупругого рассеяния электронов в веществе:
\begin{equation} \label{eq:MC_1}
	\sigma_{el/inel}(E) = \int_Q \frac{d \sigma_{el/inel}(E, q)}{dq} dq
\end{equation}
где индекс \textquotedbl el/inel\textquotedbl{} означает тип рассеяния -- упругое рассеяние или неупругое, соответственно. Дифференциальные сечения рассеяния зависят как от энергии налетающего электрона, так и от второй переменной, которая здесь называется $q$. Для упругого рассеяния это полярный угол рассеяния $\theta$ , для неупругого -- энергия $\Delta E$, передаваемая налетающим электроном среде. Интегрирование в формуле \ref{eq:MC_1} производится по области всех возможных значений $q$. Далее определяется полное сечение рассеяния на атомах всех типов:
\begin{equation} \label{eq:MC_3}
	\lambda_{el/inel}(E)=\left(n \sigma_{el/inel}(E)\right)^{-1},
\end{equation}
длина свободного пробега определяется по формуле:
\begin{equation} \label{eq:MC_4}
	\lambda^{-1}(E) = \lambda_{el}^{-1}(E)+\lambda_{inel}^{-1}(E).
\end{equation}

Вероятность того, что на промежутке пути длиной s не произойдет рассеяния, равна~\cite{ME_rev_49}:
\begin{equation} \label{eq:MC_5}
	p(s) = \lambda(E)^{-1} \exp \left(-\frac{s}{\lambda(E)}\right)
\end{equation}

При Монте-Карло моделировании длина пробега электрона может быть определена по формуле:
\begin{equation} \label{eq:MC_6}
	s = -\lambda(E) \ln \left(\xi_1\right)
\end{equation}
где $\xi_1$ случайное число из промежутка [0, 1). Если пробег электрона начинается и заканчивается в слоях, состоящих из разного вещества (например, моделирование проводится для системы из $m$ слоев с толщинами $s_1$, $s_2$, ..., $s_m$ и длинами свободного пробега электронов $\lambda_1$, $\lambda_2$, ..., $\lambda_m$), длина пробега электрона s должна быть пересчитана с условием пересечения границы между слоями. Например, она может быть вычислена как верхний предел интеграла в формуле~\cite{Han_2002}:
\begin{equation} \label{eq:MC_7}
	\ln \left(\xi_1\right)=-\frac{s_1}{\lambda_1}-\frac{s_2}{\lambda_2} \ldots+\int_{s_k}^s-\frac{d u}{\lambda_k}
\end{equation}
где $k \leq m$.


\paragraph{Определение типа взаимодействия} \mbox{} \\
\indent Далее на основе вероятностей упругого и неупругого рассеяния
\begin{equation} \label{eq:MC_8}
	p_{el/inel}=\sigma_{el/inel} /\left(\sigma_{el}+\sigma_{inel} \right)
\end{equation}
определяется тип взаимодействия (упругое или неупругое рассеяние), в котором электрон примет участие после прохождения пути $s$:
\begin{equation} \label{eq:MC_9}
	\begin{aligned}
		\xi_2 < p_{el} & \Rightarrow \text{упругое рассеяние} \\
		\xi_2 > p_{el} & \Rightarrow \text{неупругое рассеяние}
	\end{aligned}
\end{equation}
где $\xi_2$ -- новое случайное число из промежутка [0, 1).


\paragraph{Определение нового направления электрона и потерь энергии} \mbox{} \\
\indent В случае упругого рассеяния определяется новое направление рассеянного электрона, для чего используются случайные числа -- $\xi_3$ и $\xi_4$. Азимутальный угол рассеяния $\varphi$ считается равномерно распределенным на промежутке [0, 2$\pi$) и определяется выражением:
\begin{equation} \label{eq:MC_11}
	\phi = 2 \pi \xi_3.
\end{equation}

Полярный угол рассеяния $\theta$ вычисляется на основе дифференциального сечения упругого рассеяния по формуле:
\begin{equation} \label{eq:MC_12}
	\xi_4 = \frac
	{\displaystyle \int_0^\theta \frac{d \sigma}{d \Omega} \sin \vartheta d \vartheta}
	{\displaystyle \int_0^\pi \frac{d \sigma}{d \Omega} \sin \vartheta d \vartheta}.
\end{equation}

При известных углах $n$-ого акта рассеяния $\phi_n$ и $\theta_n$ новое направление движения электрона $\vec{x}_n$ определяется начальным направлением движения электронов в пучке $x_0$ (часто выражаемом вектором $(0,0,1)$) и комбинацией матриц поворота~\cite{rotation_matrices}:
\begin{equation} \label{eq:MC_13}
	\vec{x}_n=O_n^T \vec{x}_0, \quad O_n=W_n O_{n-1},
\end{equation}
\begin{equation} \label{eq:MC_14}
	W_n=\left(\begin{array}{ccc}
		\cos \varphi_n & \sin \varphi_n & 0 \\
		-\sin \varphi_n \cos \theta_n & \cos \varphi_n \sin \theta_n & \sin \theta_n \\
		\sin \varphi_n \sin \theta_n & -\cos \varphi_n \sin \theta_n & \cos \theta_n
	\end{array}\right).
\end{equation}
При этом используются начальные значения $O_{-1} = E$ (единичная матрица) и $\phi_0 = \theta_0 = 0$.

В случае неупругого рассеяния определяются потери энергии. При использовании модели непрерывных потерь энергии, потери энергии на пути $s$, определяемом выражением \ref{eq:MC_6} , вычисляются по формуле:
\begin{equation} \label{eq:MC_15}
	\Delta E=\int_0^s \frac{d E}{d s} d s \approx \frac{d E}{d s} s
\end{equation}

При использовании модели дискретных потерь энергии, потери энергии $\Delta E$ определяются с помощью случайного числа $\xi_5$:
\begin{equation} \label{eq:MC_16}
	\xi_5 = \frac
	{\displaystyle \int_{E_{\min }}^{\Delta E} \frac{d \sigma}{d\left(\Delta E^{\prime}\right)} d\left(\Delta E^{\prime}\right)}
	{\displaystyle \int_{E_{\min }}^{E_{\max }} \frac{d \sigma}{d\left(\Delta E^{\prime}\right)} d\left(\Delta E^{\prime}\right)},
\end{equation}
где дифференциальные сечения неупругого рассеяния $\frac{d \sigma}{d \Delta E}$ вычисляются по формуле Гризинского или из диэлектрической функции, а в качестве значений $E_{min}$ и $E_{max}$ выбираются $0$ и $E/2$, соответственно~\cite{Dapor_large_book}. Пример траектории электрона, рассчитываемой по методу Монте-Карло, приведен на рис.~\ref{fig:Monte_Carlo_scheme}.

\begin{figure}
	\centering
	\includegraphics[width=0.8\linewidth]{Monte_Carlo_scheme}
	\caption{Схематическое изображение траектории электрона в веществе, получаемой при Монте-Карло моделировании при использовании модели непрерывных потерь энергии.}
	\label{fig:Monte_Carlo_scheme}
\end{figure}

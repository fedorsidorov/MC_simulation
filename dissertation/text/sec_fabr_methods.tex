\section{Существующие методы микро- и наноструктурирования}

\subsection{Наноимпринтная литография}
Наноимпринтная литография (НИЛ) -- технология, предназначенная для переноса изображения наноструктуры или электронной схемы на полимерный материал путем прямого воздействия на него специальным штампом~\cite{NIL_1, NIL_2}. Существуют два основных метода НИЛ -- термическая и ультрафиолетовая (УФ). В термической НИЛ штамп вдавливается в полимер, нагретый до температур выше температуры стеклования, затем происходит его охлаждение и извлечение штампа (рисунок~\ref{fig:NIL}a)). В ультрафиолетовой НИЛ штамп из материала, прозрачного в УФ области спектра, погружается в жидкий полимер, который отверждается под действием УФ излучения, после чего происходит извлечение штампа (рисунок~\ref{fig:NIL}б)). Штамп обычно изготавливается из металла или кремния (для термической НИЛ) и полимеров или кварца (для УФ НИЛ) с помощью электронно-лучевой литографии. Учитывая прямой контакт штампа с основным материалом, а также масштаб печати 1:1, к штампу предъявляются повышенные требования по плоскопараллельности и бездефектности.  Перед проведением процесса НИЛ штамп покрывается специальным антиадгезионным покрытием, что позволяет избежать прилипания полимера к штампу при его отделении. Также после печати неизбежно остаётся тонкий остаточный слой полимера, который удаляют с помощью плазменного травления. Преимуществами НИЛ являются простота процесса (при наличии штампа), высокая производительность и возможность достижения высокого разрешения (менее 100 нм). К недостаткам этого метода относятся трудоемкость и дороговизна процесса изготовления штампа надлежащего качества, необходимость частого его обслуживания (удаления остатков основного материала), а также сложность совмещения штампа с низлежащим слоем. Несмотря на то, что технология НИЛ изначально создавалась как альтернатива фото- и электронно-лучевой литографии, она может применяется для получения трехмерных микро- и наноструктур, таких как фотонные кристаллы~\cite{NIL_nanophotonics}, микроканалы~\cite{NIL_microfluidics} и др.~\cite{NIL_3D_1, NIL_3D_2}

\begin{figure}
	\centering
	\includegraphics{jpg/NIL_14pt}
	\caption{Схематическое изображение метода термической и УФ НИЛ.}
	\label{fig:NIL}
\end{figure}


\subsection{Двухфотонная лазерная литография}

Двухфотонная лазерная литография (ДЛЛ) — технология создания микро- и наноструктур, основанная на двухфотонном поглощении внутри фокального объёма лазерного излучения~\cite{Hohmann2015, Kawata2001}. Фотовозбуждение компонент литографической смол приводящее к ее отверждению, происходит лишь в окрестности перетяжки сфокусированного лазерного излучения благодаря нелинейному характеру поглощения (рисунок~\ref{fig:TPL}а)). Процесс отверждения имеет пороговый характер, что позволяет регулировать размер отверждаемого объёма, изменяя дозу или плотность энергии поглощённого лазерного излучения. Последующее погружение смолы в растворитель приводит к удалению тех участков, которые не были подвергнуты воздействию излучения. В качестве источника излучения в ДЛЛ обычно используется фемптосекундный лазер, работающий в инфракрасном диапазоне, в качестве литографической смолы -- вещество, содержащее реакционно-способные олигомеры и фотоинициатор. При точной фокусировке ДЛЛ способна обеспечить разрешение менее 1 мкм (рисунок~\ref{fig:TPL}б)). Поскольку в ДЛЛ положение центров отвреждения может задаваться произвольно, эта технология нашла применение для формирования трехмерных во многих областях -- микрофлюидике~\cite{TPL_microfluidics_1, TPL_microfluidics_2}, биологии и \break медицины~\cite{TPL_biology_1, TPL_biology_2}, оптике и нанофотонике \cite{TPL_optics, TPL_nanophotonics}, и др. При этом, силу своей природы, данная технология обладает низкой производительностью, что является ее главным недостатком.

\begin{figure}
	\centering
	\includegraphics{jpg/TPL_14pt}
	\caption{Схематическое изображение метода двухфотонной литографии и пример структуры, полученной этим методом~\cite{TPL_castle}.}
	\label{fig:TPL}
\end{figure}


\subsection{Интерференционная литография}

Интерференционная литография (ИЛ) -- метод формирования периодической структуры в резисте, основанный на экспонировании резиста пространственно упорядоченным стоячим электромагнитным полем, возникающим при интерференции двух и более когерентных монохроматических или квазимонохроматических пучков излучения~\cite{IL_general} (рисунок~\ref{fig:IL}). Когерентность интерферирующих пучков обычно обеспечивается путем разделения исходного когерентного пучка на соответствующее число пучков с помощью различных интерференционных схем. При наноструктурировании ИЛ применяется для получения  метаматериалов~\cite{IL_metamaterials}, нанофотонных и наноплазмонных устройств~\cite{IL_nanophotonics}, биомедицинских объектов~\cite{IL_biomedical}, изделий на основе выращиваемых наноэлементов и самоорганизующихся структур~\cite{IL_self-assembly} и др.
В оптическом и УФ-диапазонах используются зеркальные схемы (Френеля, Ллойда и др.), схемы на преломляющей оптике (бипризма Френеля, билинза Бийе) или комбинированные зеркально-линзовые схемы. В этих диапазонах в качестве источника исходного пучка с высокой степенью монохроматичности и когерентности используются мощные лазеры, позволяющие получить разрешение до 100 нм. Вопрос обеспечения высокого разрешения ИЛ решается путем перехода в область рентгеновского излучения~\cite{IL_X-ray}. Преимущества метода заключаются в относительной легкости формирования дву- и трехмерных структур, к недостаткам можно отнести не самую высокую производительность и возможность получения исключительно периодических структур.

\begin{figure}
	\centering
	\includegraphics{jpg/IL_14pt}
	\caption{Схематическое изображение процесса получения двумерных (a)) и трехмерных (б)) структур методом интерференционной литографии.}
	\label{fig:IL}
\end{figure}


\subsection{Полутоновая литография}

Полутоновая литография (ПЛ) -- общее название для методов, позволяющих получить сложный трехмерный рельеф в резисте в литографическом процессе с одной одной стадией экспонирования~\cite{GL_general}. В их основе лежит пространственная модуляция дозы при экспонировании, приводящая к локальному увеличению или уменьшению скорости растворения резиста при проявлении. Таким образом, конечный рельеф имеет ступенчатую форму и состоит из участков резиста, растворенных в различной степени. Сглаживание границы между участками, проэкспонированных с различными дозами, может быть в дальнейшем достигнуто за счет оплавления образца при температурах вблизи его температуры стеклования (рисунок~\ref{fig:GL}a)). При этом существующие методы моделирования эволюции поверхности полимеров при их оплавлении позволяют использовать этот процесс как дополнительный этап структурирования~\cite{Kirchner_reflow} (рисунок~\ref{fig:GL}: б)-г)). Таким образом, полутоновая литография с последующим оплавлением образца является гибкой технологией микро- и наноструктурирования, использующейся в оптике и нанофотонике~\cite{GL_optics}, микрофлюидике~\cite{GL_microfluidics}, формировании микроэлектромеханических систем~\cite{GL_MEMS} и других областях. Существует как электронно-лучевая, так и фото-ПЛ, однако, фото-ПЛ имеет некоторые ограничения, связанные с оплавлением резиста. Так, например, вязкость широко распространенного негативного фоторезиста SU-8 при экспонировании увеличивается, что усложняет процесс контролируемого оплавления~\cite{Kirchner_GL_review}. Преимущества метода заключаются в его универсальности -- путем вариации дозы экспонирования и последующего нагрева образца можно добиться получения практически произвольного рельефа. Недостатками метода являются его сложность и производительность, еще более низкая, чем при электронно-лучевой литографии (за счет дополнительной стадии нагрева образца).

\begin{figure}
	\centering
	\includegraphics{jpg/GL_14pt}
	\caption{Схематическое изображение процесса ПЛ с последующим оплавлением (а)) и примеры структуры в ПММА, полученной непосредственно при проявлении (б)) и при последующем нагреве (в) и г))~\cite{Kirchner_reflow}.}
	\label{fig:GL}
\end{figure}


\subsection{Сканирующая зондовая литография}

Сканирующая зондовая литография (СЗЛ) включает в себя семейство технологий формирования структур с наноразмерным разрешением. Каждая из технологий основана на применении специального сканирующего зонда для воздействия на поверхность образца, приводящего к локальным изменениям поверхности. В зависимости от природы воздействия зонда на поверхность можно выделить следующие основные виды СЗЛ:
\begin{itemize}
	\item механическая, в которой изменение поверхности образца происходит в результате механического воздействия зонда ~\cite{SPL_mechanical};
	\item  термохимическая, в которой воздействие нагретого зонда на образец приводит к термической активации различных химических реакций в нем~\cite{SPL_termochemical};
	\item СЗЛ с приложением напряжения, при которой высокая напряженность электростатического поля в области зонда приводит к разложению молекул жидкости~\cite{SPL_bias_liquid} или газа~\cite{SPL_bias_gas}, окружающего образец, и локальному отложению материала на образце;
	\item окислительная СЗЛ, основанная на модификации поверхности путем ее локального окисления~\cite{SPL_oxidation};
	\item перьевая СЗЛ, в которой сканирующий зонд используется для нанесения на поверхность образца органических, полимерных или коллоидных наночернил~\cite{SPL_dip_pen_1, SPL_dip_pen_2}
\end{itemize}

Поскольку сканирующий зонд воздействует только на поверхность образца, этот метод может быть использован только для послойного формирования рельефа (в отличие от, например, ДЛЛ). Однако, высокое разрешение этой технологии и возможность ее реализации с использованием различных материалов обеспечили ей широкое применение. При этом, как и ДЛЛ, производительность сканирующей зондовой литографии крайне низка.


\subsection{Методы на основе цепной деполимеризации}

Процесс цепной деполимеризации полимерных молекул~\cite{depol_general_1}, обратный процессу полимеризации, может быть использован для формирования рельефа в полимерном резисте. Цепная реакция деполимеризации резиста становится возможна при повышенных температуре (выше температуры стеклования резиста), и для инициирования этого процесса требуется нарушение целостности главной цепи полимерной молекулы, приводящее к радикализации концов молекулы в месте разрыва~\cite{depol_general_2}. В процессе цепной деполимеризации резиста от полимерной молекулы последовательно отделяется большое число мономеров (по разным данным от нескольких сотен до нескольких тысяч~\cite{Cowley_1952_1, Mita_PMMA_zip_lengths_T, Inaba_zip_len}), который вследствие диффузии покидает область, в которой находилась молекула. Это приводит к образованию свободного пространства в резисте, что и позволяет использовать этот процесс для микро- и наноструктурирования.

Существуют два устоявшихся подхода в области формирования трехмерного рельефа в резисте на основе процесса цепной деполимеризации. В каждом из них нагревание резиста происходит локально, что ограничивает область деполимеризации резиста. Первый подход по своей сути является термической сканирующей зондовой литографией, в который для разрушения целостности полимерных молекул используется нагретый зонд~\cite{depol_fabrication_probe}. Разрывы молекул в этом случае происходят случайно за счет повышенной температуры резиста. Во втором подходе используется сфокусированный лазерный луч, который вызывает локальный нагрев резиста и разрушение в главной цепи его молекул~\cite{depol_fabrication_laser}.

Однако, существует еще один подход, основанный на нагреве всего образца, что позволяет реакции цепной деполимеризации протекать в любой его области, при условии возникновения активного центра деполимеризации. На нем основан метод термостимулированной электронно-лучевой литографии, в котором резист при температуре выше его температуры стеклования экспонируется электронным лучом~\cite{Bruk_2016_mee}. Отличительными особенностями этого метода является высокая производительность и возможность формирования в резисте дву- и трехмерных структур со сглаженным профилем в простом одностадийном процессе. Описанию этого метода будет посвящена вторая часть этой главы.

%%% Макет страницы %%%
% Выставляем значения полей (ГОСТ 7.0.11-2011, 5.3.7)
\geometry{a4paper,top=2cm,bottom=2cm,left=2.5cm,right=1cm}

%%% Кодировки и шрифты %%%
\ifxetexorluatex
    \setmainlanguage[babelshorthands=true]{russian}  % Язык по-умолчанию русский с поддержкой приятных команд пакета babel
    \setotherlanguage{english}                       % Дополнительный язык = английский (в американской вариации по-умолчанию)
    \ifXeTeX
        \defaultfontfeatures{Ligatures=TeX,Mapping=tex-text}
    \else
        \defaultfontfeatures{Ligatures=TeX}
    \fi
    \setmainfont{Times New Roman}
    \newfontfamily\cyrillicfont{Times New Roman}
    \setsansfont{Arial}
    \newfontfamily\cyrillicfontsf{Arial}
    \setmonofont{Courier New}
    \newfontfamily\cyrillicfonttt{Courier New}
\else
	\IfFileExists{pscyr.sty}{\renewcommand{\rmdefault}{ftm}}{}
%    \IfFileExists{pscyr.sty}{\renewcommand{\rmdefault}{faq}}{}
\fi

%%% Интервалы %%%
%linespread-реализация ближе к реализации полуторного интервала в ворде.
%setspace реализация заточена под шрифты 10, 11, 12pt, под остальные кегли хуже, но всё же ближе к типографской классике. 
%\linespread{1.3}                    % Полуторный интервал (ГОСТ Р 7.0.11-2011, 5.3.6)

%%% Выравнивание и переносы %%%
\sloppy                             % Избавляемся от переполнений
\clubpenalty=10000                  % Запрещаем разрыв страницы после первой строки абзаца
\widowpenalty=10000                 % Запрещаем разрыв страницы после последней строки абзаца

%%% Подписи %%%
\captionsetup{%
singlelinecheck=off,                % Многострочные подписи, например у таблиц
skip=2pt,                           % Вертикальная отбивка между подписью и содержимым рисунка или таблицы определяется ключом
justification=centering,            % Центрирование подписей, заданных командой \caption
}

%%% Рисунки %%%
\DeclareCaptionLabelSeparator*{emdash}{~--- }             % (ГОСТ 2.105, 4.3.1)
\captionsetup[figure]{labelsep=emdash,font=onehalfspacing,position=bottom}

%%% Таблицы %%%
\ifthenelse{\equal{\thetabcap}{0}}{%
    \newcommand{\tabcapalign}{\raggedright}  % по левому краю страницы или аналога parbox
}

\ifthenelse{\equal{\thetablaba}{0} \AND \equal{\thetabcap}{1}}{%
    \newcommand{\tabcapalign}{\raggedright}  % по левому краю страницы или аналога parbox
}

\ifthenelse{\equal{\thetablaba}{1} \AND \equal{\thetabcap}{1}}{%
    \newcommand{\tabcapalign}{\centering}    % по центру страницы или аналога parbox
}

\ifthenelse{\equal{\thetablaba}{2} \AND \equal{\thetabcap}{1}}{%
    \newcommand{\tabcapalign}{\raggedleft}   % по правому краю страницы или аналога parbox
}

\ifthenelse{\equal{\thetabtita}{0} \AND \equal{\thetabcap}{1}}{%
    \newcommand{\tabtitalign}{\raggedright}  % по левому краю страницы или аналога parbox
}

\ifthenelse{\equal{\thetabtita}{1} \AND \equal{\thetabcap}{1}}{%
    \newcommand{\tabtitalign}{\centering}    % по центру страницы или аналога parbox
}

\ifthenelse{\equal{\thetabtita}{2} \AND \equal{\thetabcap}{1}}{%
    \newcommand{\tabtitalign}{\raggedleft}   % по правому краю страницы или аналога parbox
}

\DeclareCaptionFormat{tablenocaption}{\tabcapalign #1\strut}        % Наименование таблицы отсутствует
\ifthenelse{\equal{\thetabcap}{0}}{%
    \DeclareCaptionFormat{tablecaption}{\tabcapalign #1#2#3}
    \captionsetup[table]{labelsep=emdash}                       % тире как разделитель идентификатора с номером от наименования
}{%
    \DeclareCaptionFormat{tablecaption}{\tabcapalign #1#2\par%  % Идентификатор таблицы на отдельной строке
        \tabtitalign{#3}}                                       % Наименование таблицы строкой ниже
    \captionsetup[table]{labelsep=space}                        % пробельный разделитель идентификатора с номером от наименования
}
\captionsetup[table]{format=tablecaption,singlelinecheck=off,font=onehalfspacing,position=top,skip=0pt}  % многострочные наименования и прочее
\DeclareCaptionLabelFormat{continued}{Продолжение таблицы~#2}

%%% Подписи подрисунков %%%
\renewcommand{\thesubfigure}{\asbuk{subfigure}}           % Буквенные номера подрисунков
\captionsetup[subfigure]{font={normalsize},               % Шрифт подписи названий подрисунков (не отличается от основного)
    labelformat=brace,                                    % Формат обозначения подрисунка
    justification=centering,                              % Выключка подписей (форматирование), один из вариантов            
}
%\DeclareCaptionFont{font12pt}{\fontsize{12pt}{13pt}\selectfont} % объявляем шрифт 12pt для использования в подписях, тут же надо интерлиньяж объявлять, если не наследуется
%\captionsetup[subfigure]{font={font12pt}}                 % Шрифт подписи названий подрисунков (всегда 12pt)

%%% Настройки гиперссылок %%%
\ifLuaTeX
    \hypersetup{
        unicode,                % Unicode encoded PDF strings
    }
\fi

\hypersetup{
    linktocpage=true,           % ссылки с номера страницы в оглавлении, списке таблиц и списке рисунков
%    linktoc=all,                % both the section and page part are links
%    pdfpagelabels=false,        % set PDF page labels (true|false)
    plainpages=false,           % Forces page anchors to be named by the Arabic form  of the page number, rather than the formatted form
    colorlinks,                 % ссылки отображаются раскрашенным текстом, а не раскрашенным прямоугольником, вокруг текста
    linkcolor={linkcolor},      % цвет ссылок типа ref, eqref и подобных
    citecolor={citecolor},      % цвет ссылок-цитат
    urlcolor={urlcolor},        % цвет гиперссылок
%    hidelinks,                  % Hide links (removing color and border)
    pdftitle={\thesisTitle},    % Заголовок
    pdfauthor={\thesisAuthor},  % Автор
    pdfsubject={\thesisSpecialtyNumber\ \thesisSpecialtyTitle},      % Тема
%    pdfcreator={Создатель},     % Создатель, Приложение
%    pdfproducer={Производитель},% Производитель, Производитель PDF
    pdfkeywords={\keywords},    % Ключевые слова
    pdflang={ru},
}

%%% Шаблон %%%
\DeclareRobustCommand{\todo}{\textcolor{red}}       % решаем проблему превращения названия цвета в результате \MakeUppercase, http://tex.stackexchange.com/a/187930/79756 , \DeclareRobustCommand protects \todo from expanding inside \MakeUppercase
\setlength{\parindent}{2.5em}                       % Абзацный отступ. Должен быть одинаковым по всему тексту и равен пяти знакам (ГОСТ Р 7.0.11-2011, 5.3.7).

%%% Списки %%%
% Используем дефис для ненумерованных списков (ГОСТ 2.105-95, 4.1.7)
\renewcommand{\labelitemi}{\normalfont\bfseries{--}} 
\setlist{nosep,%                                    % Единый стиль для всех списков (пакет enumitem), без дополнительных интервалов.
    labelindent=\parindent,leftmargin=*%            % Каждый пункт, подпункт и перечисление записывают с абзацного отступа (ГОСТ 2.105-95, 4.1.8)
}

%%% Изображения %%%
\graphicspath{{images/}}         % Пути к изображениям

\LoadInterface {titlesec}                   % Подгружаем интерфейсы для дополнительных опций управления некоторыми пакетами

%%% Блок управления параметрами для выравнивания заголовков в тексте %%%
\newlength{\otstuplen}
\setlength{\otstuplen}{\theotstup\parindent}
\ifthenelse{\equal{\theheadingalign}{0}}{% выравнивание заголовков в тексте
    \newcommand{\hdngalign}{\filcenter}                % по центру
    \newcommand{\hdngaligni}{\hfill\hspace{\otstuplen}}% по центру
}{%
    \newcommand{\hdngalign}{\filright}                 % по левому краю
    \newcommand{\hdngaligni}{\hspace{\otstuplen}}      % по левому краю
} % В обоих случаях вроде бы без переноса, как и надо (ГОСТ Р 7.0.11-2011, 5.3.5)

%%% Оглавление %%%
\renewcommand{\cftchapdotsep}{\cftdotsep}                % отбивка точками до номера страницы начала главы/раздела
\renewcommand{\cfttoctitlefont}{\hdngaligni\fontsize{14pt}{16pt}\selectfont\bfseries}% вместе со следующей строкой
\renewcommand{\cftaftertoctitle}{\hfill}                 % устанавливает заголовок по центру
\setlength{\cftbeforetoctitleskip}{-1.4\curtextsize}     % Поскольку этот заголовок всегда является первым на странице, то перед ним отделять пустым тройным интервалом не следует. Независимо от основного шрифта, в этом случае зануление (почти) происходит при -1.4\curtextsize.
\setlength{\cftaftertoctitleskip}{\theintvl\curtextsize} % Если считаем Оглавление заголовком, то выставляем после него тройной интервал через наше определённое значение

%% Переносить слова в заголовке не допускается (ГОСТ Р 7.0.11-2011, 5.3.5). Заголовки в оглавлении должны точно повторять заголовки в тексте (ГОСТ Р 7.0.11-2011, 5.2.3). Прямого указания на запрет переносов в оглавлении нет, но по той же логике невнесения искажений в смысл, лучше в оглавлении не переносить:
\cftsetrmarg{2.55em plus1fil}                       %To have the (sectional) titles in the ToC, etc., typeset ragged right with no hyphenation
\renewcommand{\cftchappagefont}{\normalfont}        % нежирные номера страниц у глав в оглавлении
\renewcommand{\cftchapleader}{\cftdotfill{\cftchapdotsep}}% нежирные точки до номеров страниц у глав в оглавлении
%\renewcommand{\cftchapfont}{}                       % нежирные названия глав в оглавлении

\ifthenelse{\theheadingdelim > 0}{%
    \renewcommand\cftchapaftersnum{.\ }   % добавляет точку с пробелом после номера раздела в оглавлении
}{%
\renewcommand\cftchapaftersnum{\quad}     % добавляет \quad после номера раздела в оглавлении
}
\ifthenelse{\theheadingdelim > 1}{%
    \renewcommand\cftsecaftersnum{.\ }    % добавляет точку с пробелом после номера подраздела в оглавлении
    \renewcommand\cftsubsecaftersnum{.\ } % добавляет точку с пробелом после номера подподраздела в оглавлении
}{%
\renewcommand\cftsecaftersnum{\quad}      % добавляет \quad после номера подраздела в оглавлении
\renewcommand\cftsubsecaftersnum{\quad}   % добавляет \quad после номера подподраздела в оглавлении
}

\ifthenelse{\equal{\thepgnum}{1}}{%
    \addtocontents{toc}{~\hfill{Стр.}\par}% добавить Стр. над номерами страниц
}

%%% Оформление названий глав %%%
%% настройки заголовка списка рисунков
\renewcommand{\cftloftitlefont}{\hdngaligni\fontsize{14pt}{16pt}\selectfont\bfseries}% вместе со следующей строкой
\renewcommand{\cftafterloftitle}{\hfill}                                             % устанавливает заголовок по центру
\setlength{\cftbeforeloftitleskip}{-1.5\curtextsize}     % Поскольку этот заголовок всегда является первым на странице, то перед ним отделять пустым тройным интервалом не следует. Независимо от основного шрифта, в этом случае зануление (почти) происходит при -1.5\curtextsize.
\setlength{\cftafterloftitleskip}{\theintvl\curtextsize} % выставляем после него тройной интервал через наше определённое значение

%% настройки заголовка списка таблиц
\renewcommand{\cftlottitlefont}{\hdngaligni\fontsize{14pt}{16pt}\selectfont\bfseries}% вместе со следующей строкой
\renewcommand{\cftafterlottitle}{\hfill}                                             % устанавливает заголовок по центру
\setlength{\cftbeforelottitleskip}{-1.5\curtextsize}     % Поскольку этот заголовок всегда является первым на странице, то перед ним отделять пустым тройным интервалом не следует. Независимо от основного шрифта, в этом случае зануление (почти) происходит при -1.5\curtextsize.
\setlength{\cftafterlottitleskip}{\theintvl\curtextsize} % выставляем после него тройной интервал через наше определённое значение

\ifnum\curtextsize>\bigtextsize     % Проверяем условие использования базового шрифта 14 pt
\setlength{\headheight}{17pt}       % Исправляем высоту заголовка
\else
\setlength{\headheight}{15pt}       % Исправляем высоту заголовка
\fi

%%% Колонтитулы %%%
% Порядковый номер страницы печатают на середине верхнего поля страницы (ГОСТ Р 7.0.11-2011, 5.3.8)
\makeatletter
\let\ps@plain\ps@fancy              % Подчиняем первые страницы каждой главы общим правилам
\makeatother
\pagestyle{fancy}                   % Меняем стиль оформления страниц
\fancyhf{}                          % Очищаем текущие значения
\fancyhead[C]{\thepage}             % Печатаем номер страницы на середине верхнего поля
\renewcommand{\headrulewidth}{0pt}  % Убираем разделительную линию

%%% Оформление заголовков глав, разделов, подразделов %%%
%% Работа должна быть выполнена ... размером шрифта 12-14 пунктов (ГОСТ Р 7.0.11-2011, 5.3.8). То есть не должно быть надписей шрифтом более 14. Так и поставим.
%% Эти установки будут давать одинаковый результат независимо от выбора базовым шрифтом 12 пт или 14 пт
\titleformat{\chapter}[block]                                % default display;  hang = with a hanging label. (Like the standard \section.); block = typesets the whole title in a block (a paragraph) without additional formatting. Useful in centered titles
        {\hdngalign\fontsize{14pt}{16pt}\selectfont\bfseries}% 
        %\fontsize{<size>}{<skip>} % второе число ставим 1.2*первое, чтобы адекватно отрабатывали команды по расчету полуторного интервала (домножая разные комбинации коэффициентов на этот)
        {\thechapter\cftchapaftersnum}                       % Заголовки в оглавлении должны точно повторять заголовки в тексте (ГОСТ Р 7.0.11-2011, 5.2.3).
        {0em}% отступ от номера до текста
        {}%

\titleformat{\section}[block]                                % default hang;  hang = with a hanging label. (Like the standard \section.); block = typesets the whole title in a block (a paragraph) without additional formatting. Useful in centered titles
        {\hdngalign\fontsize{14pt}{16pt}\selectfont\bfseries}% 
        %\fontsize{<size>}{<skip>} % второе число ставим 1.2*первое, чтобы адекватно отрабатывали команды по расчету полуторного интервала (домножая разные комбинации коэффициентов на этот)
        {\thesection\cftsecaftersnum}                        % Заголовки в оглавлении должны точно повторять заголовки в тексте (ГОСТ Р 7.0.11-2011, 5.2.3).
        {0em}% отступ от номера до текста
        {}%

\titleformat{\subsection}[block]                             % default hang;  hang = with a hanging label. (Like the standard \section.); block = typesets the whole title in a block (a paragraph) without additional formatting. Useful in centered titles
        {\hdngalign\fontsize{14pt}{16pt}\selectfont\bfseries}% 
        %\fontsize{<size>}{<skip>} % второе число ставим 1.2*первое, чтобы адекватно отрабатывали команды по расчету полуторного интервала (домножая разные комбинации коэффициентов на этот)
        {\thesubsection\cftsubsecaftersnum}                  % Заголовки в оглавлении должны точно повторять заголовки в тексте (ГОСТ Р 7.0.11-2011, 5.2.3).
        {0em}% отступ от номера до текста
        {}%

\ifthenelse{\equal{\thechapstyle}{1}}{%
    \makeatletter\sectionformat{\chapter}{% Параметры заголовков разделов в тексте
        label=\chaptername\ \thechapter\cftchapaftersnum,
        labelsep=0em,
    }\makeatother
    %% Следующие две строки: будет вписано слово Глава перед каждым номером раздела в оглавлении   
    \renewcommand{\cftchappresnum}{\chaptername\ }
    \setlength{\cftchapnumwidth}{\widthof{\cftchapfont\cftchappresnum\thechapter\cftchapaftersnum}}
}%

%% Интервалы между заголовками
% На эти величины titlespacing множит через *
\beforetitleunit=\curtextsize% привязались к нашему размеру шрифта
\aftertitleunit=\curtextsize% привязались к нашему размеру шрифта

% Счётчик intvl и длина \otstup определены в файле setup
\titlespacing{\chapter}{\theotstup\parindent}{-1.7em}{*\theintvl}       % Заголовки отделяют от текста сверху и снизу тремя интервалами (ГОСТ Р 7.0.11-2011, 5.3.5). Поскольку название главы всегда является первым на странице, то перед ним отделять пустым тройным интервалом не следует. Независимо от основного шрифта, в этом случае зануление происходит при -1.7em.
\titlespacing{\section}{\theotstup\parindent}{*\theintvl}{*\theintvl}
\titlespacing{\subsection}{\theotstup\parindent}{*\theintvl}{*\theintvl}
\titlespacing{\subsubsection}{\theotstup\parindent}{*\theintvl}{*\theintvl}

%%% Блок дополнительного управления размерами заголовков
\ifthenelse{\equal{\theheadingsize}{1}}{% Пропорциональные заголовки и базовый шрифт 14 пт
    \renewcommand{\cfttoctitlefont}{\hdngaligni\Large\bfseries} % Исправляем размер заголовка оглавления
    \setlength{\cftbeforetoctitleskip}{-1.2\curtextsize}        % Исправляем вертикальный отступ перед заголовком оглавления
    \renewcommand{\cftloftitlefont}{\hdngaligni\Large\bfseries} % Исправляем размер заголовка списка рисунков
    \setlength{\cftbeforeloftitleskip}{-1.4\curtextsize}        % Исправляем вертикальный отступ перед заголовком списка рисунков
    \renewcommand{\cftlottitlefont}{\hdngaligni\Large\bfseries} % Исправляем размер заголовка списка таблиц 
    \setlength{\cftbeforelottitleskip}{-1.4\curtextsize}        % Исправляем вертикальный отступ перед заголовком списка таблиц
    \sectionformat{\chapter}{% Параметры заголовков разделов в тексте
        format=\hdngalign\Large\bfseries, % Исправляем размер заголовка
        top-=0.4em,                       % Исправляем вертикальный отступ перед заголовком
    }
    \sectionformat{\section}{% Параметры заголовков подразделов в тексте
        format=\hdngalign\large\bfseries, % Исправляем размер заголовка
    }
}

\ifthenelse{\equal{\theheadingsize}{1}\AND \curtextsize < \bigtextsize}{% Пропорциональные заголовки и базовый шрифт 14 пт
    \sectionformat{\chapter}{% Параметры заголовков разделов в тексте
        top-=0.2em, % Исправляем вертикальный отступ перед заголовком
    }
}

%%% Счётчики %%%

%% Упрощённые настройки шаблона диссертации: нумерация формул, таблиц, рисунков
\ifthenelse{\equal{\thecontnumeq}{1}}{%
    \counterwithout{equation}{chapter} % Убираем связанность номера формулы с номером главы/раздела
}
\ifthenelse{\equal{\thecontnumfig}{1}}{%
    \counterwithout{figure}{chapter}   % Убираем связанность номера рисунка с номером главы/раздела
}
\ifthenelse{\equal{\thecontnumtab}{1}}{%
    \counterwithout{table}{chapter}    % Убираем связанность номера таблицы с номером главы/раздела
}


%%http://www.linux.org.ru/forum/general/6993203#comment-6994589 (используется totcount)
\makeatletter
\def\formbytotal#1#2#3#4#5{%
    \newcount\@c
    \@c\totvalue{#1}\relax
    \newcount\@last
    \newcount\@pnul
    \@last\@c\relax
    \divide\@last 10
    \@pnul\@last\relax
    \divide\@pnul 10
    \multiply\@pnul-10
    \advance\@pnul\@last
    \multiply\@last-10
    \advance\@last\@c
    \total{#1}~#2%
    \ifnum\@pnul=1#5\else%
    \ifcase\@last#5\or#3\or#4\or#4\or#4\else#5\fi
    \fi
}
\makeatother

\AtBeginDocument{
%% регистрируем счётчики в системе totcounter
    \regtotcounter{totalcount@figure}
    \regtotcounter{totalcount@table}       % Если иным способом поставить в преамбуле то ошибка в числе таблиц
    \regtotcounter{TotPages}               % Если иным способом поставить в преамбуле то ошибка в числе страниц
}

% для вертикального центрирования ячеек в tabulary
\def\zz{\ifx\[$\else\aftergroup\zzz\fi}
\def\zzz{\setbox0\lastbox
\dimen0\dimexpr\extrarowheight + \ht0-\dp0\relax
\setbox0\hbox{\raise-.5\dimen0\box0}%
\ht0=\dimexpr\ht0+\extrarowheight\relax
\dp0=\dimexpr\dp0+\extrarowheight\relax 
\box0
}



\lstdefinelanguage{Renhanced}%
{keywords={abbreviate,abline,abs,acos,acosh,action,add1,add,%
        aggregate,alias,Alias,alist,all,anova,any,aov,aperm,append,apply,%
        approx,approxfun,apropos,Arg,args,array,arrows,as,asin,asinh,%
        atan,atan2,atanh,attach,attr,attributes,autoload,autoloader,ave,%
        axis,backsolve,barplot,basename,besselI,besselJ,besselK,besselY,%
        beta,binomial,body,box,boxplot,break,browser,bug,builtins,bxp,by,%
        c,C,call,Call,case,cat,category,cbind,ceiling,character,char,%
        charmatch,check,chol,chol2inv,choose,chull,class,close,cm,codes,%
        coef,coefficients,co,col,colnames,colors,colours,commandArgs,%
        comment,complete,complex,conflicts,Conj,contents,contour,%
        contrasts,contr,control,helmert,contrib,convolve,cooks,coords,%
        distance,coplot,cor,cos,cosh,count,fields,cov,covratio,wt,CRAN,%
        create,crossprod,cummax,cummin,cumprod,cumsum,curve,cut,cycle,D,%
        data,dataentry,date,dbeta,dbinom,dcauchy,dchisq,de,debug,%
        debugger,Defunct,default,delay,delete,deltat,demo,de,density,%
        deparse,dependencies,Deprecated,deriv,description,detach,%
        dev2bitmap,dev,cur,deviance,off,prev,,dexp,df,dfbetas,dffits,%
        dgamma,dgeom,dget,dhyper,diag,diff,digamma,dim,dimnames,dir,%
        dirname,dlnorm,dlogis,dnbinom,dnchisq,dnorm,do,dotplot,double,%
        download,dpois,dput,drop,drop1,dsignrank,dt,dummy,dump,dunif,%
        duplicated,dweibull,dwilcox,dyn,edit,eff,effects,eigen,else,%
        emacs,end,environment,env,erase,eval,equal,evalq,example,exists,%
        exit,exp,expand,expression,External,extract,extractAIC,factor,%
        fail,family,fft,file,filled,find,fitted,fivenum,fix,floor,for,%
        For,formals,format,formatC,formula,Fortran,forwardsolve,frame,%
        frequency,ftable,ftable2table,function,gamma,Gamma,gammaCody,%
        gaussian,gc,gcinfo,gctorture,get,getenv,geterrmessage,getOption,%
        getwd,gl,glm,globalenv,gnome,GNOME,graphics,gray,grep,grey,grid,%
        gsub,hasTsp,hat,heat,help,hist,home,hsv,httpclient,I,identify,if,%
        ifelse,Im,image,\%in\%,index,influence,measures,inherits,install,%
        installed,integer,interaction,interactive,Internal,intersect,%
        inverse,invisible,IQR,is,jitter,kappa,kronecker,labels,lapply,%
        layout,lbeta,lchoose,lcm,legend,length,levels,lgamma,library,%
        licence,license,lines,list,lm,load,local,locator,log,log10,log1p,%
        log2,logical,loglin,lower,lowess,ls,lsfit,lsf,ls,machine,Machine,%
        mad,mahalanobis,make,link,margin,match,Math,matlines,mat,matplot,%
        matpoints,matrix,max,mean,median,memory,menu,merge,methods,min,%
        missing,Mod,mode,model,response,mosaicplot,mtext,mvfft,na,nan,%
        names,omit,nargs,nchar,ncol,NCOL,new,next,NextMethod,nextn,%
        nlevels,nlm,noquote,NotYetImplemented,NotYetUsed,nrow,NROW,null,%
        numeric,\%o\%,objects,offset,old,on,Ops,optim,optimise,optimize,%
        options,or,order,ordered,outer,package,packages,page,pairlist,%
        pairs,palette,panel,par,parent,parse,paste,path,pbeta,pbinom,%
        pcauchy,pchisq,pentagamma,persp,pexp,pf,pgamma,pgeom,phyper,pico,%
        pictex,piechart,Platform,plnorm,plogis,plot,pmatch,pmax,pmin,%
        pnbinom,pnchisq,pnorm,points,poisson,poly,polygon,polyroot,pos,%
        postscript,power,ppoints,ppois,predict,preplot,pretty,Primitive,%
        print,prmatrix,proc,prod,profile,proj,prompt,prop,provide,%
        psignrank,ps,pt,ptukey,punif,pweibull,pwilcox,q,qbeta,qbinom,%
        qcauchy,qchisq,qexp,qf,qgamma,qgeom,qhyper,qlnorm,qlogis,qnbinom,%
        qnchisq,qnorm,qpois,qqline,qqnorm,qqplot,qr,Q,qty,qy,qsignrank,%
        qt,qtukey,quantile,quasi,quit,qunif,quote,qweibull,qwilcox,%
        rainbow,range,rank,rbeta,rbind,rbinom,rcauchy,rchisq,Re,read,csv,%
        csv2,fwf,readline,socket,real,Recall,rect,reformulate,regexpr,%
        relevel,remove,rep,repeat,replace,replications,report,require,%
        resid,residuals,restart,return,rev,rexp,rf,rgamma,rgb,rgeom,R,%
        rhyper,rle,rlnorm,rlogis,rm,rnbinom,RNGkind,rnorm,round,row,%
        rownames,rowsum,rpois,rsignrank,rstandard,rstudent,rt,rug,runif,%
        rweibull,rwilcox,sample,sapply,save,scale,scan,scan,screen,sd,se,%
        search,searchpaths,segments,seq,sequence,setdiff,setequal,set,%
        setwd,show,sign,signif,sin,single,sinh,sink,solve,sort,source,%
        spline,splinefun,split,sqrt,stars,start,stat,stem,step,stop,%
        storage,strstrheight,stripplot,strsplit,structure,strwidth,sub,%
        subset,substitute,substr,substring,sum,summary,sunflowerplot,svd,%
        sweep,switch,symbol,symbols,symnum,sys,status,system,t,table,%
        tabulate,tan,tanh,tapply,tempfile,terms,terrain,tetragamma,text,%
        time,title,topo,trace,traceback,transform,tri,trigamma,trunc,try,%
        ts,tsp,typeof,unclass,undebug,undoc,union,unique,uniroot,unix,%
        unlink,unlist,unname,untrace,update,upper,url,UseMethod,var,%
        variable,vector,Version,vi,warning,warnings,weighted,weights,%
        which,while,window,write,\%x\%,x11,X11,xedit,xemacs,xinch,xor,%
        xpdrows,xy,xyinch,yinch,zapsmall,zip},%
    otherkeywords={!,!=,~,$,*,\%,\&,\%/\%,\%*\%,\%\%,<-,<<-},%
    alsoother={._$},%
    sensitive,%
    morecomment=[l]\#,%
    morestring=[d]",%
    morestring=[d]'% 2001 Robert Denham
}%

%решаем проблему с кириллицей в комментариях (в pdflatex) https://tex.stackexchange.com/a/103712/79756
\lstset{extendedchars=true,literate={Ö}{{\"O}}1
    {Ä}{{\"A}}1
    {Ü}{{\"U}}1
    {ß}{{\ss}}1
    {ü}{{\"u}}1
    {ä}{{\"a}}1
    {ö}{{\"o}}1
    {~}{{\textasciitilde}}1
    {а}{{\selectfont\char224}}1
    {б}{{\selectfont\char225}}1
    {в}{{\selectfont\char226}}1
    {г}{{\selectfont\char227}}1
    {д}{{\selectfont\char228}}1
    {е}{{\selectfont\char229}}1
    {ё}{{\"e}}1
    {ж}{{\selectfont\char230}}1
    {з}{{\selectfont\char231}}1
    {и}{{\selectfont\char232}}1
    {й}{{\selectfont\char233}}1
    {к}{{\selectfont\char234}}1
    {л}{{\selectfont\char235}}1
    {м}{{\selectfont\char236}}1
    {н}{{\selectfont\char237}}1
    {о}{{\selectfont\char238}}1
    {п}{{\selectfont\char239}}1
    {р}{{\selectfont\char240}}1
    {с}{{\selectfont\char241}}1
    {т}{{\selectfont\char242}}1
    {у}{{\selectfont\char243}}1
    {ф}{{\selectfont\char244}}1
    {х}{{\selectfont\char245}}1
    {ц}{{\selectfont\char246}}1
    {ч}{{\selectfont\char247}}1
    {ш}{{\selectfont\char248}}1
    {щ}{{\selectfont\char249}}1
    {ъ}{{\selectfont\char250}}1
    {ы}{{\selectfont\char251}}1
    {ь}{{\selectfont\char252}}1
    {э}{{\selectfont\char253}}1
    {ю}{{\selectfont\char254}}1
    {я}{{\selectfont\char255}}1
    {А}{{\selectfont\char192}}1
    {Б}{{\selectfont\char193}}1
    {В}{{\selectfont\char194}}1
    {Г}{{\selectfont\char195}}1
    {Д}{{\selectfont\char196}}1
    {Е}{{\selectfont\char197}}1
    {Ё}{{\"E}}1
    {Ж}{{\selectfont\char198}}1
    {З}{{\selectfont\char199}}1
    {И}{{\selectfont\char200}}1
    {Й}{{\selectfont\char201}}1
    {К}{{\selectfont\char202}}1
    {Л}{{\selectfont\char203}}1
    {М}{{\selectfont\char204}}1
    {Н}{{\selectfont\char205}}1
    {О}{{\selectfont\char206}}1
    {П}{{\selectfont\char207}}1
    {Р}{{\selectfont\char208}}1
    {С}{{\selectfont\char209}}1
    {Т}{{\selectfont\char210}}1
    {У}{{\selectfont\char211}}1
    {Ф}{{\selectfont\char212}}1
    {Х}{{\selectfont\char213}}1
    {Ц}{{\selectfont\char214}}1
    {Ч}{{\selectfont\char215}}1
    {Ш}{{\selectfont\char216}}1
    {Щ}{{\selectfont\char217}}1
    {Ъ}{{\selectfont\char218}}1
    {Ы}{{\selectfont\char219}}1
    {Ь}{{\selectfont\char220}}1
    {Э}{{\selectfont\char221}}1
    {Ю}{{\selectfont\char222}}1
    {Я}{{\selectfont\char223}}1
    {і}{{\selectfont\char105}}1
    {ї}{{\selectfont\char168}}1
    {є}{{\selectfont\char185}}1
    {ґ}{{\selectfont\char160}}1
    {І}{{\selectfont\char73}}1
    {Ї}{{\selectfont\char136}}1
    {Є}{{\selectfont\char153}}1
    {Ґ}{{\selectfont\char128}}1
}

% Ширина текста минус ширина надписи 999
\newlength{\twless}
\newlength{\lmarg}
\setlength{\lmarg}{\widthof{999}}   % ширина надписи 999
\setlength{\twless}{\textwidth-\lmarg}


\lstset{ %
%    language=R,                     %  Язык указать здесь, если во всех листингах преимущественно один язык, в результате часть настроек может пойти только для этого языка
    numbers=left,                   % where to put the line-numbers
    numberstyle=\fontsize{12pt}{14pt}\selectfont\color{Gray},  % the style that is used for the line-numbers
    firstnumber=2,                  % в этой и следующей строках задаётся поведение нумерации 5, 10, 15...
    stepnumber=5,                   % the step between two line-numbers. If it's 1, each line will be numbered
    numbersep=5pt,                  % how far the line-numbers are from the code
    backgroundcolor=\color{white},  % choose the background color. You must add \usepackage{color}
    showspaces=false,               % show spaces adding particular underscores
    showstringspaces=false,         % underline spaces within strings
    showtabs=false,                 % show tabs within strings adding particular underscores
    frame=leftline,                 % adds a frame of different types around the code
    rulecolor=\color{black},        % if not set, the frame-color may be changed on line-breaks within not-black text (e.g. commens (green here))
    tabsize=2,                      % sets default tabsize to 2 spaces
    captionpos=t,                   % sets the caption-position to top
    breaklines=true,                % sets automatic line breaking
    breakatwhitespace=false,        % sets if automatic breaks should only happen at whitespace
%    title=\lstname,                 % show the filename of files included with \lstinputlisting;
    % also try caption instead of title
    basicstyle=\fontsize{12pt}{14pt}\selectfont\ttfamily,% the size of the fonts that are used for the code
%    keywordstyle=\color{blue},      % keyword style
    commentstyle=\color{ForestGreen}\emph,% comment style
    stringstyle=\color{Mahogany},   % string literal style
    escapeinside={\%*}{*)},         % if you want to add a comment within your code
    morekeywords={*,...},           % if you want to add more keywords to the set
    inputencoding=utf8,             % кодировка кода
    xleftmargin={\lmarg},           % Чтобы весь код и полоска с номерами строк была смещена влево, так чтобы цифры не вылезали за пределы текста слева
} 

%http://tex.stackexchange.com/questions/26872/smaller-frame-with-listings
% Окружение, чтобы листинг был компактнее обведен рамкой, если она задается, а не на всю ширину текста
\makeatletter
\newenvironment{SmallListing}[1][]
{\lstset{#1}\VerbatimEnvironment\begin{VerbatimOut}{VerbEnv.tmp}}
{\end{VerbatimOut}\settowidth\@tempdima{%
        \lstinputlisting{VerbEnv.tmp}}
    \minipage{\@tempdima}\lstinputlisting{VerbEnv.tmp}\endminipage}    
\makeatother


\DefineVerbatimEnvironment% с шрифтом 12 пт
{Verb}{Verbatim}
{fontsize=\fontsize{12pt}{14pt}\selectfont}

\RawFloats[figure,table]            % Отмена установок пакета floatrow для всех флотов (плавающих окружений) выбранных типов или подтипов. А то будто мы зря задавали настройки подписей рисунков и таблиц. 

\DeclareNewFloatType{ListingEnv}{
    placement=htb,
    within=chapter,
    fileext=lol,
    name=Листинг,
}

\captionsetup[ListingEnv]{
    format=tablecaption,
    labelsep=space,                 % Точка после номера листинга задается значением period
    singlelinecheck=off,
    font=onehalfspacing,
    position=top,
}


\floatsetup[ListingEnv]{
    style=plaintop,
    captionskip=4pt,
}

\captionsetup[lstlisting]{
    format=tablecaption,
    labelsep=space,                 % Точка после номера листинга задается значением period
    singlelinecheck=off,
    font=onehalfspacing,
    position=top,
}

\renewcommand{\lstlistingname}{Листинг}

%Общие счётчики окружений листингов
%http://tex.stackexchange.com/questions/145546/how-to-make-figure-and-listing-share-their-counter
% Если смешивать плавающие и не плавающие окружения, то могут быть проблемы с нумерацией
\makeatletter
\AtBeginDocument{%
    \let\c@ListingEnv\c@lstlisting
    \let\theListingEnv\thelstlisting
    \let\ftype@lstlisting\ftype@ListingEnv % give the floats the same precedence
}
\makeatother

% значок С++ — используйте команду \cpp
\newcommand{\cpp}{%
    C\nolinebreak\hspace{-.05em}%
    \raisebox{.2ex}{+}\nolinebreak\hspace{-.10em}%
    \raisebox{.2ex}{+}%
}


%%% Русская традиция начертания математических знаков
%\renewcommand{\le}{\ensuremath{\leqslant}}
%\renewcommand{\leq}{\ensuremath{\leqslant}}
%\renewcommand{\ge}{\ensuremath{\geqslant}}
%\renewcommand{\geq}{\ensuremath{\geqslant}}
%\renewcommand{\emptyset}{\varnothing}

%%% Русская традиция начертания греческих букв (греческие буквы вертикальные, через пакет upgreek)
%\renewcommand{\epsilon}{\ensuremath{\upvarepsilon}}   %  русская традиция записи
%\renewcommand{\phi}{\ensuremath{\upvarphi}}
%%\renewcommand{\kappa}{\ensuremath{\varkappa}}
%\renewcommand{\alpha}{\upalpha}
%\renewcommand{\beta}{\upbeta}
%\renewcommand{\gamma}{\upgamma}
%\renewcommand{\delta}{\updelta}
%\renewcommand{\varepsilon}{\upvarepsilon}
%\renewcommand{\zeta}{\upzeta}
%\renewcommand{\eta}{\upeta}
%\renewcommand{\theta}{\uptheta}
%\renewcommand{\vartheta}{\upvartheta}
%\renewcommand{\iota}{\upiota}
%\renewcommand{\kappa}{\upkappa}
%\renewcommand{\lambda}{\uplambda}
%\renewcommand{\mu}{\upmu}
%\renewcommand{\nu}{\upnu}
%\renewcommand{\xi}{\upxi}
%\renewcommand{\pi}{\uppi}
%\renewcommand{\varpi}{\upvarpi}
%\renewcommand{\rho}{\uprho}
%%\renewcommand{\varrho}{\upvarrho}
%\renewcommand{\sigma}{\upsigma}
%%\renewcommand{\varsigma}{\upvarsigma}
%\renewcommand{\tau}{\uptau}
%\renewcommand{\upsilon}{\upupsilon}
%\renewcommand{\varphi}{\upvarphi}
%\renewcommand{\chi}{\upchi}
%\renewcommand{\psi}{\uppsi}
%\renewcommand{\omega}{\upomega}

\renewcommand{\footnotesize}{\fontsize{14pt}{16pt}\selectfont}
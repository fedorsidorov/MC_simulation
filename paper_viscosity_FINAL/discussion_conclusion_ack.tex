\section{Discussion}

The developed method of PMMA vertex mobility determination has several remarkable features.
First, it demonstrates the agreement between reflowed profiles, simulated by analytical spectral method and ones obtained by numerical SE simulation in area normalization mode.
This emphasizes the applicability of soapfilm modeling for the simulation of polymer reflow and confirms the linear relation between $scale$ and time, proposed by Kirchner~\cite{Kirchner_SE_1}.
Moreover, the relation of inverse mobility of PMMA surface vertices and PMMA viscosity turns out to be linear too.
It is also noteworthy that developed method brings clarity into SE reflow simulation process -- in case of setting the mobilities obtained by equation~\ref{eq:mu_eta} the $scale$ factor denotes exactly the actual reflow time.

Second, this method could be applied for reflow simulation of any structure obtained in PMMA by lithographic methods.
In case profile, obtained in PMMA by grayscale e-beam lithography, simulation of PMMA main-chain scission distribution could be directly converted to PMMA viscosity profile and, finally, to PMMA vertex mobility profile.
In case of structures obtained by nanoimprint lithography the PMMA viscosity dependence on temperature and molecular weight only should be taken into account.
If sample size is much greater than structure period one can neglect the edge effects (contact angle etc.) for the reflow simulation far from sample edges.
Then structure geometry and vertex mobility distribution become the only parameters required for simulation.

Third, the described algorithm allows to expand the boundaries of thermal reflow application. At present, thermal reflow is used predominantly for profile smoothing, which not causes dramatic profile transformation~\cite{Kirchner_GL_review,Kirchner_2017}.
This results in a need for sophisticated grayscale e-beam lithography process for achievement of staircase profile close enough to required one.
The complex but predictable thermal reflow process in turn could simplify the whole fabrication method provided that greater part of profile is being formed by reflow.

\section{Conclusion} \label{conclusion}
This paper presents simulation technique for viscoelastic thermal reflow of PMMA with non-uniform viscosity profile caused by e-beam exposure with dose modulation.
Thermal reflow simulation bases on numerical search of minimal surface by finite elements method, processed by free software ``Surface Evolver'' (SE).
PMMA non-uniform viscosity profile is described by specific mobilities of PMMA surface vertices, which are embedded in surface model in SE simulation processed in area normalization mode.
PMMA surface vertex mobilities determination bases on PMMA viscosity profile which is calculated in three steps.
First, Monte-Carlo algorithm is used for the simulation e-beam induced PMMA main-chain scissions.
Then, statistical model is used to convert PMMA main-chain scission distribution to PMMA number-average molecular weight distribution.
Finally, empirical equations are used for determination of PMMA viscosity profile relying in simulated PMMA number-average molecular weight distribution
The relation between PMMA viscosity and vertex mobility of its surface was determined by simulation of rectangular grating reflow by two approaches -- analytical, based on decay simulation of profile spatial harmonics, and numerical one, processed by SE.
The obtained agreement between reflowed profiles reveal direct proportionality between SE scale factor and reflow time.
Reflow simulation in wide viscosity range resulted in turn in almost inverse proportionality between PMMA surface vertex mobility and PMMA viscosity.
Using this relation, one can calculate proper vertex mobility of PMMA structure surface vertices and process clear SE reflow simulation with scale factor denoting exactly the actual reflow time.
The developed approach could simplify complex 3D structure fabrication by deeper insight into thermal reflow processes.

\ack
The investigation was supported by the Program no. FFNN-2022-0021 of the Ministry of Science and Higher Education of Russia for Valiev Institute of Physics and Technology of RAS.

\chapter{Методы формирования структур и моделирования} \label{chapter:methods}

В настоящее время существует ряд областей, требующих формирования трехмерных микро- и наноструктур (нанофотоника, микро- и нанофлюидика, МЭМС и др.). Существующие методы формирования имеют как свои характерные достоинства, так и недостатки.

\section{Методы формирования трехмерных микро- и наноструктур}

\subsection{Наноимпринтная литография}
Наноимпринтная литография (НИЛ) -- технология, предназначенная для переноса изображения наноструктуры или электронной схемы на полимерный материал путем прямого воздействия на него специальным штампом. Существуют два основных метода НИЛ -- термическая и ультрафиолетовая (УФ). В термической НИЛ штамп вдавливается в полимер, нагретый до температур выше температуры стеклования, затем происходит его охлаждение и извлечение штампа. В ультрафиолетовой НИЛ штамп из материала, прозрачного в УФ области спектра, погружается в жидкий полимер, который отверждается под действием УФ излучения, после чего происходит извлечение штампа. Штамп обычно изготавливается из металла или кремния (для термической НИЛ) и полимеров или кварца (для УФ НИЛ) с помощью электронно-лучевой литографии. Учитывая прямой контакт штампа с основным материалом, а также масштаб печати 1:1, к штампу предъявляются повышенные требования по плоскопараллельности и бездефектности.  Перед проведением процесса НИЛ штамп покрывается специальным антиадгезионным покрытием, что позволяет избежать прилипания полимера к штампу при его отделении. Также после печати неизбежно остаётся тонкий остаточный слой полимера, который удаляют с помощью плазменного травления.

Преимуществами НИЛ являются простота процесса (при наличии штампа), высокая производительность и возможность достижения высокого разрешения (менее 30 нм). К недостаткам этого метода относятся трудоемкость и дороговизна процесса изготовления штампа надлежащего качества, необходимость частого его обслуживания (удаления остатков основного материала), а также сложность совмещения штампа с низлежащим слоем.


\subsection{Двухфотонная лазерная литография}

Двухфотонная лазерная литография (ДЛЛ) — технология создания микро- и наноструктур, основанная на двухфотонном поглощении внутри фокального объёма лазерного луча~\cite{Hohmann2015, Kawata2001}. Фотовозбуждение компонент литографической смол приводящее к ее отверждению, происходит лишь в окрестности перетяжки сфокусированного лазерного излучения благодаря нелинейному характеру поглощения. Процесс отверждения имеет пороговый характер, что позволяет регулировать размер отверждаемого объёма, изменяя дозу или плотность энергии поглощённого лазерного излучения. Последующее погружение смолы в растворитель приводит к удалению тех участков, которые не были подвергнуты воздействию излучения. В качестве источника излучения в ДЛЛ обычно используется фемптосекундный лазер, работающий в инфракрасном диапазоне, в качестве литографической смолы -- вещество, содержащее реакционно-способные олигомеры и фотоинициатор. При точной фокусировке лазерного луча ДЛЛ способна обеспечить разрешение менее 1 мкм. Поскольку в ДЛЛ положение центров отвреждения может задаваться произвольно, эта технология нашла применение для формирования трехмерных во многих областях -- микрофлюидике~\cite{TPL_microfluidics_1, TPL_microfluidics_2}, биологии и медицины~\cite{TPL_biology_1, TPL_biology_2}, оптике и нанофотонике \cite{TPL_optics, TPL_nanophotonics}, и др. При этом, силу своей природы, данная технология обладает низкой производительностью, что является ее главным недостатком.


\subsection{Интерференционная литография}

Интерференционная литография (ИЛ) -– это метод формирования периодической топологической структуры резистной маски, основанный на экспонировании резиста пространственно упорядоченным стоячим электромагнитным полем, возникающим при интерференции двух и более когерентных монохроматических или квазимонохроматических пучков излучения. Когерентность интерферирующих пучков (точное поддержание разности фаз между ними) обычно обеспечивается путем разделе ния исходного когерентного пучка (в идеале пло ской волны или сферической волны от точечного источника) на соответствующее число пучков с помощью различных интерференционных схем. В оптическом и УФ-диапазонах это, например, хорошо известные зеркальные схемы (Френеля, Ллойда и др.), схемы на преломляющей оптике (бипризма Френеля, билинза Бийе) или комбинированные зеркально-линзовые схемы. В этих диапазонах в качестве источника исходного пучка с высокой степенью монохроматичности и когерентности естественно использовать мощные лазеры.


\subsection{Полутоновая литография}

Здесь будет ссылка \cite{TPL_1}







%{\let\clearpage\relax \chapter*{Содержание работы}}
\section*{Структура диссертации}
%\chapter*{Содержание работы}
В диссертации рассматривается возможность использования узкозонных полупроводников для межзонной лазерной генерации в ТГц диапазоне. Особое внимание уделяется процессам оже-рекомбинации и рекомбинации с испусканием плазмонов, поскольку именно они будут являться основными механизмами рекомбинации на пороге лазерной генерации и определять пороговые токи ТГц лазерных диодов. Подавление оже-рекомбинации ожидается в материалах с дираковским законом дисперсии $E = \pm\sqrt{v_0^2 p^2 + E_g^2/4}$, поэтому в главах 1 и 2 рассмотрены случаи дираковского и приближённо дираковского закона дисперсии на примере графена и квантовых ям из теллурида кадмия-ртути. Глава 3 посвящена рекомбинации с испусканием плазмонов в квантовых ямах из теллурида кадмия-ртути.

\chapter{Терагерцовая генерация в материалах с дираковским законом дисперсии} \label{chapter:graphene}
\section{Расчёт темпа рекомбинации методом неравновесных функций Грина} \label{sec:NEGF}
%\subsection{Краткое введение в метод неравновесных функций Грина} \label{sec:NEGF-basics}
%\subsection{Кинетическое уравнение и темп рекомбинации} \label{sec:Kadanoff-Baym}
%\subsection{Самосогласованное $GW$-приближение} \label{sec:GW}
%\subsection{Темп рекомбинации в $GW$-приближении} \label{sec:GW-recombination}
%\subsection{Численное решение уравнений самосогласованного $GW$-приближения} \label{sec:GW-Fourier}
%\subsection{Основные результаты раздела} \label{sec:GW-summary}
%
%
Большая часть первой главы посвящена проблеме расчёта темпа оже-рекомбинации в материалах с дираковским законом дисперсии. В бесщелевом случае среди всех трёхчастичных оже-процессов разрешены законами сохранения энергии и импульса лишь те, в которых импульсы всех носителей сонаправлены~\cite{Rana-Auger}, причём и эти процессы оказываются подавлены при учёте динамического экранирования кулоновского взаимодействия~\cite{Tomadin-theory}. При наличии щели трёхчастичная оже-рекомбинация оказывается вовсе запрещена. Поэтому для корректного расчёта темпа оже-рекомбинации в дираковских материалах требуется учёт многочастичных эффектов, таких как уширение спектра носителей из-за их рассеяния друг на друге, искривление дираковского конуса под действием межэлектронного взаимодействия, а также динамическое экранирование кулоновского взаимодействия.

Для решения этой задачи нами предложено использовать метод неравновесных функций Грина~\cite{NEGFhandbook}, который используется для моделирования кинетики носителей с учётом многочастичных эффектов~\cite{Haug}. Оказывается, что в рамках $GW$-приближения~\cite{NEGF-GW} темп оже-рекомбинации можно записать в следующем виде:
\begin{eq}{ImPccImPcv-Auger}
    R_{\rm Auger} &= 4 \int_{-\infty}^{+\infty} \frac{d\omega}{2\pi} \int_{\mathbb{R}^D} \frac{d^D \vec{q}}{\left( 2\pi \right)^D} &&\left[ n_B(\hbar\omega - \Delta\mu_{cv}) - n_B(\hbar\omega)\right]\\
    &&&\times \Im\Pi^{R}_{cc}(\vec{q},\omega) \Im\Pi^{R}_{vc}(\vec{q},\omega) \abs{W^{R}(\vec{q}, \omega)}^2\\
    &+ 4 \int_{-\infty}^{+\infty} \frac{d\omega}{2\pi} \int_{\mathbb{R}^D} \frac{d^D \vec{q}}{\left( 2\pi \right)^D} &&\left[ n_B(\hbar\omega) - n_B(\hbar\omega + \Delta\mu_{cv}) \right]\\
    &&&\times \Im\Pi^{R}_{cv}(\vec{q},\omega) \Im\Pi^{R}_{vv}(\vec{q},\omega) \abs{W^{R}(\vec{q}, \omega)}^2\\
    &+4 \int_{-\infty}^{+\infty} \frac{d\omega}{2\pi} \int_{\mathbb{R}^D} \frac{d^D \vec{q}}{\left( 2\pi \right)^D} &&\left[ n_B(\hbar\omega - \Delta\mu_{cv}) - n_B(\hbar\omega + \Delta\mu_{cv}) \right]\\
    &&&\times \Im\Pi^{R}_{cv}(\vec{q},\omega) \Im\Pi^{R}_{vc}(\vec{q},\omega) \abs{W^{R}(\vec{q}, \omega)}^2,\\
\end{eq}
где $\Im\Pi^{R}_{cc}$, $\Im\Pi^{R}_{vv}$, $\Im\Pi^{R}_{cv}$, $\Im\Pi^{R}_{vc}$ --- внутризонные и межзонные поляризуемости электрон-дырочной системы, $W^{R}$ --- динамически экранированное кулоновское взаимодействие, $n_B$  --- распределение Бозе-Эйнштейна, $\Delta\mu_{cv}$ --- разность квазиуровней Ферми в зоне проводимости и валентной зоне, $\hbar\vec{q}$ и $\hbar\omega$ --- передача импульса и энергии в оже-процессе. Три вклада в темп оже-рекомбинации \eqref{ImPccImPcv-Auger} соответствуют CHCC процессу (излишек энергии, выделяющийся при рекомбинации, передаётся второму электрону), CHHH процессу (излишек энергии передаётся второй дырке) и CHCH процессу (<<двойная рекомбинация>>, два электрона рекомбинируют с двумя дырками, разрешён только при учёте уширения спектра). Также наличие динамически экранированного кулоновского взаимодействия в формуле \eqref{ImPccImPcv-Auger} позволяет учесть оже-рекомбинацию через промежуточное плазмонное состояние, которую можно рассматривать как рекомбинацию с испусканием плазмона и последующим его внутризонным поглощением.

Формула \eqref{ImPccImPcv-Auger} соответствует золотому правилу Ферми без обменных слагаемых, если поляризуемости вычислять для невзаимодействующих частиц. Если вычислять поляризуемости и экранированное кулоновское взаимодействие с учётом эффектов межэлектронного взаимодействия, формула \eqref{ImPccImPcv-Auger} даст темп оже-рекомбинации с учётом многочастичных эффектов. Для расчёта поляризуемостей нами предложено использовать самосогласованное $GW$-приближение~\cite{NEGF-GW}, так как оно позволяет учесть все интересующие нас многочастичные эффекты. В этом состоит отличие нашего подхода от более ранних работ~\cite{Ziep-Mocker,Yevick-GW_Auger}, в которых также выводилась формула \eqref{ImPccImPcv-Auger}, но учёт межэлектронного взаимодействия не производился, либо производился в более грубом приближении~\cite{Auger_scattering} (пренебрежение вкладом носителей в экранирование, пренебрежение рассеянием электронов, пренебрежением искривлением зон из-за межэлектронного взаимодействия).
%
%\begin{figure}[!t]
%    \centering
%    \includegraphics[width=0.4\linewidth]{synopsis-graphene-dielectric}
%    \caption{\label{fig:graphene-dielectric}Время оже-рекомбинации в слабонеравновесном нелегированном графене. Кружки соответствуют расчётам с учётом частотной зависимости диэлектрических проницаемостей гексагонального нитрида бора и диоксида гафния, при этом учтена и рекомбинация с испусканием оптических фононов диэлектрика. Нефизичное поведение графиков в области малых диэлектрических проницаемостей связано с тем, что при сильном межэлектронном взаимодействии $GW$-приближение работает плохо.}
%\end{figure}
\section{Применение метода неравновесных функций Грина для расчёта темпа рекомбинации в графене}   
\begin{narrowfig}{graphene-dielectric}{synopsis-graphene-dielectric}
Время оже-рекомбинации в слабонеравновесном нелегированном графене. Кружки соответствуют расчётам с учётом частотной зависимости диэлектрических проницаемостей гексагонального нитрида бора и диоксида гафния, при этом учтена и рекомбинация с испусканием оптических фононов диэлектрика. Нефизичное поведение графиков в области малых диэлектрических проницаемостей связано с тем, что при сильном межэлектронном взаимодействии $GW$-приближение работает плохо.
\end{narrowfig}

Разработанный метод был применён для расчёта темпа оже-рекомбинации в нелегированном слабонеравновесном графене. Соответствующие времена жизни неравновесных носителей $\tau_{\rm Auger}$ составляют 1--3 пс при 300 К (рис.~\ref{fig:graphene-dielectric}), а их температурная зависимость имеет приближённый вид $\tau_{\rm Auger} \propto T^{-1}$, который можно получить из анализа размерностей. При температуре носителей 1000--3000 К времена рекомбинации оказываются в диапазоне десятков-сотен фемтосекунд, что согласуется с экспериментами по наблюдению кинетики носителей в фотовозбуждённом графене~\cite{Gierz2013}%~\cite{Gierz2013,Gierz2014,Gierz2015,Gierz2016}
. Такие времена оже-рекомбинации свидетельствуют о том, что она является доминирующим механизмом рекомбинации в графене в широком диапазоне параметров.

Наличие низкоэнергетических оптических фононов в high-$\kappa$ диэлектриках не позволяет использовать их для подавления рекомбинации в графене (чёрный круг на рис.~\ref{fig:graphene-dielectric}).

Также были рассчитаны времена оже-рекомбинации в графене в более простых приближениях на основе золотого правила Ферми, использовавшихся в предыдущих работах~\cite{Rana-Auger, Tomadin-theory, Malic-dynamic}. Показано, что эти приближения могут давать ошибку в несколько раз из-за неточного учёта многочастичных эффектов.

\section{Оценка пороговых токов лазерных диодов на основе графена} \label{sec:graphene-threshold_currents}
В конце главы с использованием рассчитанных времён рекомбинации были оценены пороговые токи лазерных диодов на основе графена:
\begin{eq}{graphene-threshold_current}
J_{\rm th} \approx J_0 \left( \frac{T}{T_0} \right)^3,
\end{eq}
где $J_0 \approx$~50 кА/см$^2$ на один графеновый слой при $T_0 = 300$~К для частот генерации $kT < \hbar\omega < 3 kT$ (для больших частот пороговые токи увеличиваются из-за необходимости обеспечить $\Delta\mu_{cv} > \hbar\omega$). Это свидетельствует в пользу возможности межзонной ТГц генерации в графене при температуре жидкого азота.

\chapter{Терагерцовая генерация в материалах с приближённо дираковским законом дисперсии} \label{chapter:HgCdTe}
Во второй главе рассматривается ТГц усиление и рекомбинационные процессы в квантовых ямах \HgCdTe{}, выращенных вдоль кристаллографического направления [013]. Для таких ям имеется отлаженная технология производства~\cite{HgCdTe-technology}, и в них уже наблюдалось вынужденное излучение на 15 ТГц~\cite{HgCdTe-stimulated_emission}.

\section{Электронный и фононный спектр квантовых ям из теллурида кадмия-ртути} \label{sec:HgCdTe-basics}
Для расчёта зонной структуры и волновых функций электронов в квантовых ямах из теллурида кадмия-ртути мы использовали четырёхзонную модель Кейна и приближение огибающих функций, показавшие хорошее согласие с магнетоспектроскопическими экспериментами~\cite{Kane_model}. Примеры рассчитанных зонных структур приведены на рис.~\ref{fig:HgCdTe-all-in-one}. 
%
%\begin{fig}{HgCdTe-permittivity}{HgCdTe-permittivity} (а) --- действительная и мнимая части диэлектрической проницаемости теллурида кадмия-ртути $\kappa(\omega)$ в зависимости от частоты, рассчитанные в многоосцилляторной модели Лоренца с параметрами из работы~\cite{HgCdTe-phonon_params}. (б) --- аналогичные графики для $\abs{\kappa(\omega)^{-1}}$ и $(- \Im \kappa(\omega)^{-1})$.
%\end{fig}

Для расчёта темпа рекомбинации в этих ямах нам также требуется знать их диэлектрическую проницаемость и спектр оптических фононов. Мы использовали высокочастотную диэлектрическую проницаемость $\kappa_{\infty} = 12$~\cite{HgCdTe-high-frequency_kappa} и параметры фононов из работы~\cite{HgCdTe-phonon_params}; частотная дисперсия диэлектрической проницаемости вычислялась в многоосцилляторной модели Лоренца. Частоты фононов в теллуриде кадмия-ртути лежат в диапазоне 3--5 ТГц, из-за чего область сильного решёточного поглощения оказывается в более длинноволновом диапазоне по сравнению с AlGaAs, поэтому диапазон 6--10 ТГц, недоступный квантово-каскадным лазерам на GaAs/AlGaAs, теоретически доступен для лазеров на квантовых ямах из теллурида кадмия-ртути.
\begin{fig}{HgCdTe-all-in-one}{synopsis-HgCdTe-all-in-one} Зонная структура, заселённости электронных (красный цвет) и дырочных (голубой цвет) состояний на пороге генерации, длина волны генерации и соответствующий вертикальный переход (оранжевые стрелки), пороговые токи генерации и различные пороговые оже-процессы в квантовых ямах \HgCdTe{} различной толщины при 77 К. Толщина синих линий показывает величину анизотропии зонной структуры. Красная и голубая полоса показывают квазиуровни Ферми для электронов и дырок, их ширина равна тепловой энергии $k T$.
\end{fig}
\section{Расчёт темпа рекомбинации и пороговых уровней накачки в квантовых ямах из теллурида кадмия-ртути} \label{sec:HgCdTe-results}
\begin{fig}{HgCdTe-recombination}{synopsis-HgCdTe-recombination} (a) --- пороговые энергии оже-рекомбинации в единицах запрещённой зоны в квантовых ямах \HgCdTe{} различной толщины при 4.2 К. Сплошные линии --- расчёты с использованием модели Кейна и приближения огибающих функций, пунктирные линии --- по формуле для параболического спектра с использованием эффективных масс в экстремумах зон. На вставках показаны пороговые оже-процессы для разных ям и пороговые энергии в абсолютных единицах. (б) --- времена жизни неравновесных носителей в квантовых ямах \HgCdTe{}, связанные с различными процессами рекомбинации. На графиках также показаны стандартные отклонения результатов, полученных интегрированием методом Монте-Карло. 
\end{fig}

В отличие от графена, в квантовых ямах из теллурида кадмия-ртути область дираковского спектра ограничена десятками мэВ (рис.~\ref{fig:HgCdTe-all-in-one}), поэтому оже-рекомбинация оказывается разрешена даже без учёта многочастичных эффектов. Однако рассчитанные пороговые энергии оже-рекомбинации (минимально возможные суммарные кинетические энергии участвующих носителей) достигают половины запрещённой зоны и более для ям с запрещённой зоной в ТГц области (рис.~\ref{fig:HgCdTe-recombination}а), что соответствует подавлению оже-рекомбинации на один-два порядка при 77 К по сравнению с беспороговым случаем. В материалах с параболическими зонами пороговые энергии равны $E_g \times \min\{m_e, m_h\}/(m_e + m_h)$ и обычно гораздо меньше $0.5 E_g$ из-за наличия тяжёлых дырок с $m_h \gg m_e$. В рассматриваемых квантовых ямах взаимодействие краевых состояний на границах между материалом барьеров с нормальной зонной структурой и материалом ямы с инвертированной зонной структурой приводит к появлению приближённо дираковского спектра носителей, и большой электрон-дырочной асимметрии удаётся избежать.

\begin{fig}{HgCdTe-gaps_and_concentrations}{HgCdTe-gaps_and_concentrations} (а) --- ширина запрещённой зоны, оптическая запрещённая зона (длинноволновая граница вертикальных переходов) и пороговая частота генерации для ям \HgCdTe{} (с учётом уширения спектра $\gamma = 1$~мэВ). (б) --- пороговые концентрации носителей для достижения оптического усиления в нелегированных ямах без учёта уширения спектра и с $\gamma = 1$~мэВ.
\end{fig}

Для определения пороговых концентраций носителей, необходимых для достижения лазерной генерации, была рассчитана оптическая проводимость рассматриваемых ям с учётом всех межзонных и межподзонных переходов, а также друдевского поглощения на свободных носителях в яме, и определены концентрации электронов и дырок, при которых она становится отрицательной на некоторой частоте (рис.~\ref{fig:HgCdTe-gaps_and_concentrations}б). Затем для этих концентраций носителей были рассчитаны времена различных процессов рекомбинации: оже-рекомбинации, рекомбинации с испусканием оптических фононов (возможной в достаточно узкозонных ямах) и излучательной рекомбинации в подпороговом режиме (т. е. без учёта вынужденных переходов под действием генерируемого излучения). Учёт возможности небольшого несохранения энергии в оже-процессах из-за конечного времени жизни квазичастиц, связанного с рассеянием носителей, не приводит к существенному снижению пороговых энергий, поэтому для расчёта темпа оже-рекомбинации мы воспользовались золотым правилом Ферми.

Расчёты показали, что в ямах ТГц диапазона основным механизмом рекомбинации на пороге генерации является оже-рекомбинация (наряду с плазмонной рекомбинацией, рассмотренной в главе 3). Полученные времена рекомбинации достигают сотен пс при 77 К в ямах нормальной зонной структуры (рис.~\ref{fig:HgCdTe-recombination}б), при том что в широких ямах, имеющих инвертированную зонную структуру, эти времена составляют около 1 пс, поскольку оже-рекомбинация оказывается беспороговым процессом. Это подтверждает выводы о подавлении оже-рекомбинации на один-два порядка, сделанные на основе анализа пороговых энергий.

\begin{fig}{HgCdTe-current-frequency}{HgCdTe-current-frequency} Зависимость пороговой плотности тока электрической накачки и пороговой интенсивности оптической накачки, необходимых для достижения оптического усиления в квантовых ямах \HgCdTe{} различной толщины, от пороговых частот и длин волн генерации в этих ямах. (а) --- температура решётки 4.2 К. (б) --- температура решётки равна температуре носителей.  Зелёные точки --- экспериментальные данные из работы~\cite{HgCdTe-stimulated_emission}; треугольники, квадраты и окружности --- экспериментальные данные по квантово-каскадным~\cite{QCL_5.1um_300K,QCL_7.66um_300K,QCL_15.1um_300K,QCL_17.8um_77K,QCL_24.5um_77K}, межзонным каскадным лазерам~\cite{ICL_3.67um_300K,ICL_6um_300K,ICL_7um_300K,ICL_10.4um_77K} и лазерным диодам на солях свинца~\cite{LeadSalt_DoubleHeterostructure,Lead_Salt_LaserCharacteristics}. Красным и зелёным закрашены области сильного решёточного поглощения (Reststrahlen band) в Hg$_{1-x}$Cd$_{x}$Te и GaAs соответственно. На вставках изображены пороговые оже-процессы в различных ямах.
\end{fig}

Рассчитанные частоты лазерной генерации (рис.~\ref{fig:HgCdTe-gaps_and_concentrations}а), пороговые концентрации носителей (рис.~\ref{fig:HgCdTe-gaps_and_concentrations}б) и времена рекомбинации (рис.~\ref{fig:HgCdTe-recombination}б) были использованы для оценки пороговых токов электрической накачки и пороговых интенсивностей оптической накачки, необходимых для достижения лазерной генерации на той или иной частоте. Полученные значения пороговых токов лежат в диапазоне 50--400 А/см$^2$ на одну яму для генерации в области 6--10 ТГц при 77 К (рис.~\ref{fig:HgCdTe-current-frequency}б). При учёте необходимости использовать более одной ямы для преодоления оптических потерь в резонаторе пороговые токи оказываются в диапазоне единиц кА/см$^2$, характерном для квантово-каскадных лазеров дальнего ИК-диапазона. 

Минимальная частота генерации теоретически могла бы составить 4.7 ТГц, если рассматривать только оптические потери в самой яме, однако учёт сильного решёточного поглощения вблизи частот оптических фононов теллурида кадмия-ртути увеличивает минимальную частоту генерации примерно до 6 ТГц. Стоит отметить, что частоты 6--10 ТГц попадают в область сильного решёточного поглощения в AlGaAs и недоступны существующим квантово-каскадным лазерам.

Рассчитанные пороговые интенсивности оптической накачки находятся в качественном согласии с экспериментальными данными по получению вынужденного излучения в квантовых ямах \HgCdTe{}~\cite{HgCdTe-stimulated_emission} (зелёные точки на рис.~\ref{fig:HgCdTe-current-frequency}а), хотя точное сравнение затрудняется тем, что температура носителей в этих экспериментах неизвестна.

\chapter{Роль плазмонов в терагерцовой генерации в дираковских материалах на примере квантовых ям HgCdTe} \label{chapter:plasmon}
В третьей главе изучается рекомбинация с испусканием двумерных плазмонов в  квантовых ямах \HgCdTe{} и её влияние на пороговые токи лазерных диодов.
\section{Расчёт закона дисперсии плазмонов и границы области межзонных переходов} \label{sec:plasmon-dispersion}
Вначале был рассчитан закон дисперсии плазмонов и нижняя граница области межзонных переходов в этих ямах и определены концентрации носителей, при которых закон дисперсии плазмонов пересекает область межзонных переходов и рекомбинация с испусканием плазмонов становится возможна. Показано, что рекомбинационный процесс, который становится разрешён при наименьшей концентрации носителей, представляет собой переход электрона из минимума зоны проводимости в побочный локальный максимум валентной зоны с испусканием плазмона (рис.~\ref{fig:plasmons-intersection}а).
\begin{fig}{plasmons-intersection}{plasmons-intersection}(а) --- закон дисперсии плазмонов и область межзонных переходов в квантовой яме \HgCdTe{} толщиной 6.5 нм при 77 К и концентрации каждого типа носителей $5\times10^{10}$ см$^{-2}$. На вставке изображён единственный возможный при этой концентрации процесс рекомбинации с испусканием плазмона. (б) --- то же самое, но для концентрации $5\times10^{11}$ см$^{-2}$. На вставке изображён возможный процесс плазмонной рекомбинации с участием носителей из дираковской части закона дисперсии.
\end{fig}
\section{Пороговые энергии и пороговые концентрации носителей для плазмонной рекомбинации} \label{sec:plasmon-thresholds}
Процесс плазмонной рекомбинации с участием дырки из побочного максимума валентной зоны имеет энергетический порог, равный разности энергий между центральным и боковым максимумом валентной зоны, и может быть подавлен при достаточно низких температурах. Процессы с участием только носителей из дираковской части зонной структуры требуют на порядок больших концентраций носителей (рис.~\ref{fig:plasmons-intersection}б), которые обычно превышают пороговые энергии, необходимые для лазерной генерации в ямах нормальной зонной структуры, за исключением наиболее узкозонных ям (рис.~\ref{fig:plasmons-concentrations_and_times}а).
\section{Времена плазмонной рекомбинации} \label{sec:plasmon-recombination_times}
\begin{fig}{plasmons-concentrations_and_times}{synopsis-plasmons-concentrations_and_times}(а) --- пороговые концентрации для плазмонной рекомбинации в квантовых ямах \HgCdTe{} (сплошные линии), для плазмонной рекомбинации в пределах дираковской части закона дисперсии (длинный пунктир) и для достижения оптического усиления (короткий пунктир). (б) --- времена жизни неравновесных носителей на пороге лазерной генерации при 77 К, связанные с плазмонной рекомбинацией (пунктирные линии) и оже-рекомбинацией (сплошная линия). Длинный пунктир --- результаты для пороговых концентраций, полученных в наших расчётах, короткий пунктир --- для вдвое больших концентраций носителей.
\end{fig}

В конце главы были рассчитаны времена плазмонной рекомбинации в ямах различной толщины на пороге лазерной генерации. Учитывая сильную чувствительность темпа плазмонной рекомбинации к концентрации носителей, мы также рассмотрели ситуацию, когда реальные пороговые концентрации носителей оказываются вдвое больше рассчитанных нами из-за значительных оптических потерь в резонаторе. Расчёты показали, что в широкозонных ямах, где плазмонная рекомбинация разрешена только из-за уширения спектра плазмонов, связанного с их конечным временем жизни, плазмонная рекомбинация оказывается медленнее оже-рекомбинации, а в наиболее узкозонных --- быстрее, с характерными временами около 0.3 пс (рис.~\ref{fig:plasmons-concentrations_and_times}б). Частота генерации, для которой темпы плазмонной и оже-рекомбинации на пороге генерации одинаковы, лежит в ТГц области и чувствительна к оптическим потерям в резонаторе, так как они влияют на пороговую концентрацию носителей. Для пороговых концентраций, рассчитанных только с учётом потерь в самой яме, учёт плазмонной рекомбинации увеличивает пороговые токи ТГц генерации не более чем в 2 раза, а для вдвое больших концентраций --- не более чем в 4 раза. Такие пороговые токи свидетельствуют о возможности межзонной ТГц лазерной генерации в квантовых ямах из теллурида кадмия-ртути при температуре жидкого азота или более низкой.
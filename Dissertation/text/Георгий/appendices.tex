\chapter{Частотные зависимости диэлектрических проницаемостей различных материалов} \label{appendix:dielectric_functions}
В основном тексте диссертации рассчитаны темпы рекомбинации в графене с различным диэлектрическим окружением, а также в квантовых ямах из теллурида кадмия-ртути. В этих расчётах использованы диэлектрические проницаемости в многоосцилляторной модели Лоренца 
\begin{eq}{Lorentz_kappas}
    \kappa(\omega) = \kappa_{\infty} + \sum_{i=1}^{N} \frac{(\kappa_{i-1}-\kappa_i)\omega^2_i}{\omega^2_i-\omega(\omega + i\gamma_i)},
\end{eq}
где $\omega_i$ и $\gamma_i$ --- частоты и постоянные затухания $N$ поперечных оптических фононных мод, $\kappa_i$ --- параметры, связанные с силами осцилляторов. $\kappa_0$ равна статической диэлектрической проницаемости, $\kappa_N \equiv \kappa_{\infty}$ --- высокочастотная диэлектрическая проницаемость; остальные <<промежуточные>> проницаемости $\kappa_i$ не соответствуют реальной диэлектрической проницаемости на какой-либо частоте, но наглядно иллюстрируют относительную <<силу>> различных фононных мод. Используемые нами значения этих параметров для различных материалов приводятся в таблице \ref{tab:phonons}. Данные для HfO$_2$ взяты из работы~\cite{HfO2kappa}, для hBN --- из работы~\cite{hBNkappa}, для 6H-SiC --- из работы~\cite{SiCkappa}, для HgCdTe --- из работы~\cite{HgCdTe-phonon_params}. Высокочастотная диэлектрическая проницаемость HgCdTe взята из работы~\cite{HgCdTe-high-frequency_kappa}. Для расчётов темпа рекомбинации в графене \emph{на} SiC мы использовали $\kappa(\omega)=(\kappa_{\text{SiC}}(\omega)+\kappa_{\text{air}}(\omega))/2=(\kappa_{\text{SiC}}(\omega)+1)/2$. Большое число значащих цифр для некоторых $\kappa_i$ в HgCdTe приведено в тех случаях, когда это необходимо для того, чтобы они отличались от $\kappa_{i\pm1}$ (некоторые фононные моды в работе~\cite{HgCdTe-phonon_params} имеют малые силы осцилляторов и слабо влияют на диэлектрическую проницаемость). Реальная точность значений диэлектрических проницаемостей ограничена двумя-тремя значащими цифрами.

\begin{table}
\centering
\caption{\label{tab:phonons}Параметры модели Лоренца для различных диэлектриков.}
\begin{tabular}{@{\extracolsep{70pt}}cddd}
\toprule
$i$ & \multicolumn{1}{c}{$\omega_i$ (мэВ)} & \multicolumn{1}{c}{$\gamma_i$ (мэВ)} & \multicolumn{1}{c}{$\kappa_i$}\\
\midrule
\multicolumn{4}{c}{HfO$_2$}\\
0 & & & 14.2\\
1 & 23.2 & 26.8 & 12.4\\
2 & 31.6 & 5.6 & 10.3\\
3 & 41.8 & 7.7 & 6.2\\
4 & 50.0 & 7.0 & 4.4\\
5 & 62.7 & 6.7 & 3.9\\
6 & 73.8 & 3.2 & 3.8\\
\midrule
\multicolumn{4}{c}{hBN}\\
0 & & & 7.0\\
1 & 95.1 & 4.3 & 6.8\\
2 & 169.5 & 3.6 & 5.0\\
\midrule
\multicolumn{4}{c}{6H-SiC}\\
0 & & & 10.0\\
1 & 98.4 & 0.6 & 6.7\\
\midrule
\multicolumn{4}{c}{HgCdTe}\\
0 & & & 15.4\\
1 & 12.9 & 1.2 & 15.0\\
2 & 13.6 & 1.1 & 14.7\\
3 & 14.3 & 0.7 & 14.4\\
4 & 15.0 & 1.1 & 12.8\\
5 & 15.5 & 1.2 & 12.455\\
6 & 15.9 & 1.0 & 12.437\\
7 & 16.4 & 0.7 & 12.425\\
8 & 16.7 & 1.0 & 12.419\\
9 & 18.1 & 0.6 & 12.358\\
10 & 18.4 & 0.9 & 12.2\\
11 & 18.9 & 0.9 & 12.0\\
\bottomrule
\end{tabular}
\end{table}

\chapter{Алгоритм Монте-Карло для расчёта темпа оже-рекомбинации} \label{appendix:Auger}
Численный расчёт темпа оже-рекомбинации требует вычисления многомерных интегралов, поэтому для этой задачи используют интегрирование методом Монте-Карло (например, \cite{Auger_Monte_Carlo}). Из-за наличия функций распределения носителей в подынтегральном выражении и ограничений со стороны законов сохранения лишь небольшая часть фазового пространства вносит вклад в интеграл, что учитывают ограничением области интегрирования~\cite{Auger_Monte_Carlo_phase_space_restriction}. В данном приложении мы приводим несколько усовершенствованную версию этого метода, в которой наиболее <<острые>> части подынтегрального выражения (дельта-функция от энергий и распределения Ферми-Дирака) целиком перемещены в плотность вероятности распределения, в соответствии с которым выбираются точки в области интегрирования. Это позволяет вычислять темп оже-рекомбинации в широком диапазоне температур и концентраций носителей с разумной точностью и вычислительными затратами.

Мы рассчитываем темп рекомбинации в квантовых ямах из теллурида кадмия-ртути по золотому правилу Ферми:
\begin{eq}{appendix-Auger_rate}
       R_{\rm CHCC} = \frac{2 \pi}{\hbar} \int \frac{d^2 \vec{k}_{1,2,3}}{(2\pi)^6} |M_{fi}|^2 f_1 f_2 \bar{f}_3 (1-f_4) \delta(E_1+E_2+\bar{E}_3-E_4)
\end{eq}
(для CHCC процесса, аналогично для CHHH, CHHH2 и CHHH3 процессов). Здесь $M_{fi}$ --- матричный элемент кулоновского взаимодействия, просуммированный по проекциям спина участвующих носителей; $k_i$, $E_i$, $f_i$ --- квазиволновые векторы, энергии и числа заполнения начальных (1,2,3) и конечных (4) состояний; черты над символами обозначают дырочные энергии и числа заполнения. В матричном элементе учтены как прямые, так и обменные слагаемые; кулоновское взаимодействие экранировано решёточной диэлектрической проницаемостью \eqref{Lorentz}.

Выражение \eqref{appendix-Auger_rate} описывает только рекомбинацию, без обратного процесса ударной ионизации. В основном тексте диссертации под темпом рекомбинации подразумевается разность темпа собственно рекомбинации и темпа ударной ионизации. Это учитывается домножением \eqref{appendix-Auger_rate} на $1-\exp\left[-(\mu_c-\mu_v)/kT\right]$.

Чтобы эффективно вычислить многомерный интеграл в \eqref{appendix-Auger_rate}, мы переписываем его в переменных $E_{1,2,4}$ и $\varphi_{1,2,4}$ (углы, характеризующие направления квазиимпульсов $\vec{k}_{1,2,4}$):
\begin{eq}{to_Ephi}
       \int d^2 \vec{k}_{1,2,3} \rightarrow \frac{1}{\hbar^3}\int\displaylimits_{\substack{
       E_4 - E_c \geq E_{\rm th}^{\rm CHCC}+E_g\\
       0 \leq E_2 - E_c \leq E_4 - E_c - E_g\\
       0 \leq E_1 - E_c \leq E_4 - E_2 - E_g
       }} \frac{k_{1,2,4}^2}{|\vec{v}\cdot\vec{k}|_{1,2,4}}dE_{1,2,4}d\varphi_{1,2,4},
\end{eq}
поскольку наиболее <<острые>> части подынтегрального выражения (дельта-функция и распределения Ферми-Дирака) зависят только от энергий. Область интегрирования ограничена краями зон $E_{c/v}$ и пороговой энергией CHCC процесса $E_{\rm th}^{\rm CHCC}$.

Затем мы устраняем дельта-функцию интегрированием по $\varphi_2$:
\begin{eq}{removing_delta}
       \int \frac{k_2^2}{|\vec{v}_2\cdot\vec{k}_2|}dE_{1,2,4}d\varphi_{1,2,4}\delta(E_1+E_2+\bar{E}_3-E_4) \rightarrow \frac{1}{\hbar}\int \frac{dE_{1,2,4}d\varphi_{1,4}}{\lVert\vec{v}_2 \times \vec{v}_3\rVert}.
\end{eq}

Интеграл по-прежнему зависит от угла $\varphi_2$, который мы не можем выразить через остальные переменные аналитически, поэтому в каждой выбранной точке в области интегрирования мы находим $\varphi_2$ численно из законов сохранения энергии и импульса.

Оставшийся пятимерный интеграл всё ещё плохо подходит для непосредственного применения метода Монте-Карло, так как распределения Ферми-Дирака не являются плавными функциями, особенно в больцмановском пределе. Чтобы перейти к интегрированию плавной функции, мы используем выборку по значимости (importance sampling), выбирая точки в соответствии с плотностью вероятности $(1/Z) f_1 f_2 \bar{f}_3 (1-f_4)$. При этом $\varphi_{1,4}$ выбираются из равномерного распределения на отрезке $[0, 2\pi]$.

Такая выборка достигается в две стадии:
\begin{enumerate}
\item $E_4$ выбирается в соответствии с плотностью вероятности $(1/Z')\{\exp[(E_4-\mu_1-\mu_2-\mu_3)/kT]+1\}^{-1}\{\exp[(\mu_4-E_4)/kT]+1\}^{-1}$. Такая плотность вероятности выбрана из соображений наилучшей аппроксимации для $(1/Z) f_1 f_2 \bar{f}_3 (1-f_4)$ функцией одной переменной, чтобы для генерации случайных точек можно было использовать метод обратного преобразования (inverse transform sampling).\\
\item Для получения набора точек, распределённых в соответствии с желаемой плотностью вероятности $(1/Z) f_1 f_2 \bar{f}_3 (1-f_4)$, используется выборка с отклонением. Нормировочный множитель $Z = (2\pi)^2\int f_1 f_2 \bar{f}_3 (1-f_4) dE_{1,2,4}$ вычисляется непосредственным численным интегрированием (пределы интегрирования такие же, как в уравнении \eqref{to_Ephi}).\\
\end{enumerate}

Наконец, остаётся проблема бесконечной дисперсии: после замены переменных и устранения дельта-функции в подынтегральном выражении появляется знаменатель, который может обращаться в ноль в некоторых точках, из-за чего среднее значение подынтегрального выражения в области интегрирования конечно, но дисперсия бесконечна. По этой причине известная теорема о корневой сходимости метода Монте-Карло здесь не работает. На самом деле, алгоритм Монте-Карло по-прежнему сходится~\cite{Monte_Carlo_infinite_variance}, но медленнее, чем для случая конечной дисперсии. Чтобы улучшить сходимость и уменьшить численный шум на графиках, мы использовали для оценки значения интеграла усечённое среднее (truncated mean) вместо среднего арифметического. Этот приём часто используется в статистике как раз для оценки математического ожидания распределений с плотностью вероятности, медленно спадающей на бесконечности. Суть техники усечённого среднего заключается в том, что мы отбрасываем одну или две точки, в которых подынтегральное выражение принимает наибольшие значения. Этого достаточно, чтобы распределение значений подынтегрального выражения в оставшихся точках имело конечную дисперсию, при этом возникающее систематическое занижение значения интеграла оказывается существенно меньше случайных погрешностей метода Монте-Карло.

Ниже приводится краткое изложение использованного нами алгоритма:
\begin{enumerate}
  \item Генерируем $N_{\rm start}$ чисел $\xi_1,...,\xi_{N_{\rm start}}$,  равномерно распределённых на отрезке $[0, 1]$. Мы использовали $N_{\rm start}=10^3$.
    \item Используем метод обратного преобразования (inverse transform sampling), чтобы получить значения энергии конечного носителя $E_4$, распределённые между $E_{4{\rm min}}$ (определяется пороговой энергией рассматриваемого процесса, см. уравнение \eqref{to_Ephi}) и $E_{4{\rm max}}$ (определяется максимальной энергией, до которой просчитана соответствующая зона) с плотностью вероятности $(1/Z')\{\exp[(E_4-\mu_1-\mu_2-\mu_3)/kT]+1\}^{-1}\{\exp[(\mu_4-E_4)/kT]+1\}^{-1}$:
    \begin{eq}{transform}
    E_{4,i} &= \mu_4 + kT \ln \frac{F_1(\xi_i)-1}{\exp\left(\frac{\mu_4-\mu_1-\mu_2-\mu_3}{kT}\right)-F_1(\xi_i)},\\
    F_1(\xi_i) &\equiv F_2(E_{4{\rm max}})^{\xi_i}F_2(E_{4{\rm min}})^{1-\xi_i},\\
    F_2(E) &\equiv \frac{F_{\rm Fermi}(E-\mu_4)}{F_{\rm Fermi}(E-\mu_1-\mu_2-\mu_3)}, \\
    F_{\rm Fermi}(E) &\equiv \left[\exp\left(E/kT\right)+1\right]^{-1}.
    \end{eq}
    \item Для каждого значения $E_{4,i}$ выбираем энергии $E_{1,i}$, $E_{2,i}$ двух начальных носителей из равномерного распределения на треугольнике $E_{1,i}>E_{1{\rm min}}$, $E_{2,i}>E_{2{\rm min}}$, $E_{1,i}+E_{2,i}<E_{4,i}-E_{3{\rm min}}$ ($E_{{1,2,3}{\rm min}}$ --- края зон), а также углы $\varphi_{1,i}$, $\varphi_{4,i}$ из равномерного распределения на отрезке $[0, 2\pi]$. Энергия третьего носителя $E_{3,i}$ определяется из закона сохранения энергии: $E_{3,i}=E_{4,i}-E_{1,i}-E_{2,i}$.
    \item Используем выборку с отклонением (rejection sampling), чтобы сгенерировать сэмплы в соответствии с плотностью вероятности $(1/Z) f_1 f_2 \bar{f}_3 (1-f_4)$. А именно, каждая сгенерированная ранее пятёрка чисел $\{E_{1,i},E_{2,i},E_{4,i},\varphi_{1,i},\varphi_{4,i}\}$ должна быть отброшена с вероятностью $1-p$, $p = (1/Z'')f_{1,i} f_{2,i} \bar{f}_{3,i} (1-f_{4,i}) \{\exp[(E_{4,i}-\mu_1-\mu_2-\mu_3)/kT]+1\}\{\exp[(\mu_4-E_{4,i})/kT]+1\}$. Нормировочный множитель $Z''$ может быть любым положительным числом, обеспечивающим $p<1$ во всей области интегрирования, однако наиболее эффективно использовать $Z'' = \max_{1 \leq i \leq N_{\rm start}}\{f_{1,i} f_{2,i} \bar{f}_{3,i} (1-f_{4,i}) \{\exp[(E_{4,i}-\mu_1-\mu_2-\mu_3)/kT]+1\}\{\exp[(\mu_4-E_{4,i})/kT]+1\}\}$, чтобы максимальное значение $p$ равнялось 1 и количество отброшенных сэмплов было минимально.
    \item Так как на шаге 4 некоторые сэмплы отбрасываются, повторяем шаги 1--4, пока число сгенерированных сэмплов не достигнет $N_{\rm samples}$. Мы использовали $N_{\rm samples}=10^3$.
    \item Для каждой сгенерированной пятёрки чисел $\{E_{1,i},E_{2,i},E_{4,i},\varphi_{1,i},\varphi_{4,i}\}$ находим соответствующие квазиволновые векторы носителей $\{{\vec k}_{1,ij},{\vec k}_{2,ij},{\vec k}_{3,ij},{\vec k}_{4,ij}\}$ путём численного решения законов сохранения. Если одной пятёрке $\{E_{1,i},E_{2,i},E_{4,i},\varphi_{1,i},\varphi_{4,i}\}$ могут соответствовать несколько разных наборов квазиволновых векторов, оставляем все и будем их нумеровать индексом $j$.
    \item На шаге 6 число сэмплов может существенно уменьшиться, поскольку не все наборы чисел $\{E_{1,i},E_{2,i},E_{4,i},\varphi_{1,i},\varphi_{4,i}\}$ способны удовлетворить законам сохранения, особенно вблизи пороговой энергии. Поэтому мы вычисляем $N_{\rm consistent}$, количество сгенерированных сэмплов $\{E_{1,i},E_{2,i},E_{4,i},\varphi_{1,i},\varphi_{4,i}\}$, согласующихся с законами сохранения, и возвращаемся к шагу 1, меняя $N_{\rm start}$ и $N_{\rm samples}$ таким образом, чтобы получить примерно $N_{\rm target}$ сэмплов после шагов 1--6: $N_{\rm start, new} = N_{\rm samples, new} = N_{\rm start, old}N_{\rm target}/\max\{N_{\rm consistent},10\}$. Мы использовали $N_{\rm target} = 10^3$; $\max\{N_{\rm consistent},10\}$ в знаменателе применяется для того, чтобы избежать больших вычислительных затрат в случаях, когда $N_{\rm consistent}$ оказывается слишком малым.
    \item Для каждого сгенерированного набора квазиволновых векторов $\{{\vec k}_{1,i},{\vec k}_{2,i},{\vec k}_{3,i},{\vec k}_{4,i}\}$ вычисляем подынтегральное выражение
    \begin{eq}{integrand}
    J_{ij} = \frac{1}{(2 \pi\hbar)^5}\frac{k_1^2 k_4^2}{|\vec{k}_1\cdot\vec{v}_1|\lVert\vec{v}_2\times\vec{v}_3\rVert|\vec{v}_4\cdot\vec{k}_4|}|M_{fi}|^2
    \end{eq}
(без распределений Ферми-Дирака, так как они были перемещены в плотность вероятности) и суммируем $J_{ij}$, соответствующие одному и тому же набору энергий и углов $\{E_{1,i},E_{2,i},E_{4,i},\varphi_{1,i},\varphi_{4,i}\}$, но разным квазиволновым векторам: $J_i = \sum_j J_{ij}$.
\item Отбрасываем два максимальных $J_i$ для улучшения сходимости (см. выше).
\item Вычисляем нормировочный множитель
\begin{eq}{normalization}
Z = (2\pi)^2\int\displaylimits_{\substack{
       E_{1,2,4}>E_{1,2,4{\rm min}}\\
       E_4<E_{4{\rm max}}\\
       E_4-E_1-E_2>E_{3{\rm min}}
       }} f_1 f_2 \bar{f}_3 (1-f_4) dE_{1,2,4}
\end{eq}
($(2\pi)^2$ берётся из интегрирований по углам). Это всего лишь трёхмерный интеграл от произведения простых аналитически заданных функций, поэтому для его вычисления можно использовать готовые библиотеки численного интегрирования.
\item Получаем оценки для искомого интеграла \eqref{appendix-Auger_rate} и погрешности по формулам $Z\left<J_i\right>$ и $Z\sqrt{\left(\left<J_i^2\right>-\left<J_i\right>^2\right)/{N_{\rm consistent}}}$ соответственно.
\item Повторяем алгоритм для всех типов оже-рекомбинации, рассматриваемых в настоящей диссертации: CHCC, CHHH, CHHH2 и CHHH3.
\end{enumerate}

\chapter{Связь темпа излучательной рекомбинации в двумерном материале с межзонной оптической проводимостью} \label{appendix:radiative}
Рассмотрим двумерную электронную систему, расположенную в плоскости $z=0$ внутри бездисперсионной среды с диэлектрической проницаемостью $\kappa$ и взаимодействующую с поперечным электромагнитным полем. Электронные состояния будем нумеровать индексом (под)зоны $s$ и квазиволновым вектором $\vec{k}$, фотонные --- волновым вектором $\vec{q}$ (с компонентами $\vec{q}_{\parallel}$ в плоскости двумерной системы и $q_z$ в перпендикулярном направлении) и индексом поляризации $\mu$ (вектор поляризации $\vec{e}_{\mu \vec{q}}$). Энергии этих состояний равны $\epsilon_{s \vec{k}}$ и $\hbar \omega_{q} = \hbar c q/\sqrt{\kappa}$, заселённости --- $f_{s\vec{k}}$ и $n_{\vec{q}}$, соответствующие операторы уничтожения --- ${\hat a}_{s \vec{k}}$ и ${\hat b}_{\mu\vec{q}}$. Также будем использовать обозначения ${\bar f}_{s\vec{k}} \equiv 1 - f_{s\vec{k}}$ и ${\bar n}_{\vec{q}} \equiv n_{\vec{q}} + 1$.

Гамильтониан такой системы состоит из электронного гамильтониана, фотонного гамильтониана и гамильтониана электрон-фотонного взаимодействия:
\begin{eq}{radiative_Hamiltonian}
{\hat H} &= {\hat H}_e + {\hat H}_{ph} + {\hat H}_{e-ph},\\
{\hat H}_e &= \sum_{s \vec{k}} \epsilon_{s \vec{k}} {\hat a}_{s \vec{k}}^{\dagger} {\hat a}_{s \vec{k}},\\
{\hat H}_{ph} &= \sum_{\mu\vec{q}} \hbar \omega_{q} {\hat b}_{\mu\vec{q}}^{\dagger} {\hat b}_{\mu\vec{q}},\\
{\hat H}_{e-ph} &= - \frac{1}{c} \sum_{\vec{q}_{\parallel}} {\hat{\vec{A}}}_{\vec{q}_{\parallel}}(z=0) \cdot {\hat{\vec{j}}}_{-\vec{q}_{\parallel}}.
\end{eq}
Здесь ${\hat{\vec{A}}}_{\vec{q}_{\parallel}}(z=0)$ --- фурье-компонента оператора векторного потенциала в плоскости $z=0$, соответствующая волновому вектору $\vec{q}_{\parallel}$ в плоскости $z=0$; ${\hat{\vec{j}}}_{-\vec{q}_{\parallel}} = (-e/2)\times(e^{i \vec{q}_{\parallel} {\hat{\vec{r}}}} {\hat{\vec{v}}} + {\hat{\vec{v}}} e^{i \vec{q}_{\parallel} {\hat{\vec{r}}}})$ --- фурье-компонента оператора плотности тока. В гамильтониане электрон-фотонного взаимодействия оставлены только линейные по векторному потенциалу слагаемые.

Стандартная процедура квантования электромагнитного поля даёт следующее выражение для оператора векторного потенциала~\cite{electromagnetic_field_quantization}: 
\begin{eq}{electromagnetic_quantizarion}
{\hat{\vec{A}}}_{\vec{q}_{\parallel}}(z=0) = \sum_{\mu q_z} \sqrt{ \frac{2\pi \hbar c^2}{\kappa \omega_q} } \vec{e}_{\mu \vec{q}} \left( {\hat b}_{\mu\vec{q}} + {\hat b}^{\dagger}_{\mu,-\vec{q}} \right).
\end{eq}

С учётом малости волнового вектора фотона, матричный элемент электрон-фотонного взаимодействия равен
\begin{eq}{electron-photon_matrix_element}
M_{ss'\mu\vec{k}\vec{q}} \approx -\sqrt{ \frac{2\pi\hbar}{\kappa \omega_q} } \vec{e}_{\mu \vec{q}} \cdot \left\langle s \vec{k} \right\rvert {\hat{\vec{j}}}_0 \left\lvert s' \vec{k} \right\rangle = e \sqrt{ \frac{2\pi\hbar}{\kappa \omega_q} } \vec{e}_{\mu \vec{q}} \cdot \vec{v}_{s s' \vec{k}}.
\end{eq}

Согласно золотому правилу Ферми, темп излучательной рекомбинации равен
\begin{eq}{electron-phonon_golden_rule}
R_{\rm rad} &= \frac{2\pi}{\hbar} \sum_{s \in c, s' \in v,\mu,\vec{k},\vec{q}} \abs{M_{ss'\mu\vec{k}\vec{q}}}^2\\
&\times \left[f_{s \vec{k}} {\bar f}_{s' \vec{k}} {\bar n}_{\vec{q}} -  {\bar f}_{s \vec{k}} f_{s' \vec{k}} n_{\vec{q}}\right] \delta(\epsilon_{s\vec{k}} - \epsilon_{s'\vec{k}} - \hbar\omega_q)\\
&= \frac{2\pi}{\hbar} \sum_{s \in c, s' \in v,\vec{k},\vec{q}} \frac{2\pi \hbar e^2}{\kappa \omega_q} \left(\delta_{\alpha\beta} - \frac{q_{\alpha} q_{\beta}}{q^2} \right) v_{\alpha,s s' \vec{k}}v_{\beta,s' s \vec{k}}\\
&\times \left[f_{s \vec{k}} {\bar f}_{s' \vec{k}} {\bar n}_{\vec{q}} -  {\bar f}_{s \vec{k}} f_{s' \vec{k}} n_{\vec{q}}\right] \delta(\epsilon_{s\vec{k}} - \epsilon_{s'\vec{k}} - \hbar\omega_q)\\
&= \frac{2\pi}{\hbar} \sum_{s \in c, s' \in v,\vec{k}} \int_0^{\infty} \frac{4\pi \omega^2 d\omega}{(2 \pi c/\sqrt{\kappa})^3} \frac{2\pi \hbar e^2}{\kappa \omega} \frac{2}{3}\delta_{\alpha\beta} v_{\alpha,s s' \vec{k}}v_{\beta,s' s \vec{k}}\\
&\times \left\{f_{s \vec{k}} {\bar f}_{s' \vec{k}} \left[n_B(\hbar\omega) + 1\right] -  {\bar f}_{s \vec{k}} f_{s' \vec{k}} n_B(\hbar\omega) \right\} \delta(\epsilon_{s\vec{k}} - \epsilon_{s'\vec{k}} - \hbar\omega).
\end{eq}

Используя формулу \eqref{bare_interband_conductivity} для межзонной оптической проводимости и тождество \eqref{Fermi-Bose_trick} для распределений Ферми-Дирака, получаем
\begin{eq}{appendix-radiative_recombination_rate}
       R_{\rm rad} &= \frac{8}{3\pi}\frac{\sqrt{\kappa_\infty}}{c^3}\int_0^{+\infty}\omega^2 d\omega\left[n_B(\omega-\Delta\mu_{cv})-n_B(\omega)\right]\\
      &\times \Re\frac{\sigma_{xx}^{(0),cv}(\omega)+\sigma_{yy}^{(0),cv}(\omega)+\sigma_{zz}^{(0),cv}(\omega)}{2},
\end{eq}
где $\sigma_{\alpha\beta}^{(0),cv}(\omega)$ --- межзонная часть оптической проводимости без учёта эффектов уширения спектра.

Как и при расчёте пороговых концентраций носителей, мы заменяем $\sigma_{\alpha\beta}^{(0),cv}(\omega)$ на $\sigma_{\alpha\beta}^{cv}(\omega)$, полученную из $\sigma_{\alpha\beta}^{(0),cv}(\omega)$ свёрткой с лоренцианом шириной $\gamma$ \eqref{broadened_interband_conductivity}, что позволяет учесть <<размытие>> спектра спонтанного излучения из-за эффектов рассеяния носителей.

Формула \eqref{appendix-radiative_recombination_rate} является двумерным аналогом известной формулы, связывающей темп излучательной рекомбинации в \emph{трёхмерном} материале с коэффициентом поглощения~\cite{radiative_rate_from_absorption}.


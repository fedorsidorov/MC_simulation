\chapter{Роль плазмонов в терагерцовой генерации в дираковских материалах на примере квантовых ям HgCdTe} \label{chapter:plasmon}
В предыдущих главах мы показали, что одним из основных процессов безызлучательной рекомбинации в узкозонных дираковских материалах является оже-рекомбинация, даже несмотря на ограничения со стороны законов сохранения. Однако есть ещё один механизм безызлучательной рекомбинации, который невозможен в широкозонных полупроводниках, но может играть существенную роль в узкозонных материалах.

Таким механизмом является рекомбинация с испусканием плазмонов~\cite{recombination_in_narrow_gap}. В лазерных диодах на халькогенидах свинца-олова этот процесс приводит к резкому росту пороговых токов при приближении запрещённой зоны к частоте гибридной плазмон-фононной моды~\cite{lead_salt-plasmon}, а в объёмном теллуриде кадмия-ртути в присутствии магнитного поля наблюдался провал в фотопроводимости, когда расстояние между нижними уровнями Ландау для электронов и дырок совпадало с частотой плазмона~\cite{HgCdTe-plasmon_photoconductivity}.

В трёхмерных полупроводниках этот механизм рекомбинации включается, когда ширина запрещённой зоны оказывается меньше плазменной частоты. В материалах, поддерживающих \emph{двумерные} плазмоны, закон дисперсии плазмонов имеет корневой вид, $\omega_{\rm pl}(q) \propto \sqrt{q}$, из-за чего всегда найдётся плазмон с энергией больше ширины запрещённой зоны. С другой стороны, этот плазмон будет иметь ненулевой волновой вектор и может лежать за пределами области межзонных переходов даже при $\omega_{\rm pl} > E_g$.

В данной главе исследуется возможность рекомбинации с испусканием плазмонов в двумерных дираковских материалах на примере квантовых ям \HgCdTe{} и обсуждается её влияние на пороговые характеристики лазерных диодов на основе этого материала.

\section{Расчёт закона дисперсии плазмонов и границы области межзонных переходов} \label{sec:plasmon-dispersion}
Рекомбинация с испусканием плазмонов возможна, если закон дисперсии плазмонов пересекается с областью межзонных переходов. Под областью межзонных переходов мы понимаем область в пространстве $(\vec{q}, \omega)$, содержащую все возможные межзонные переходы в данном материале ($\hbar\vec{q}$ --- изменение квазиимпульса, $\hbar\omega$ --- изменение энергии при межзонном переходе).

Для расчёта закона дисперсии двумерных плазмонов в квантовых ямах из теллурида кадмия-ртути мы использовали формулу~\cite{plasmon_dispersion}
\begin{eq}{plasmon_dispersion}
\omega_{\rm pl}(q) = \sqrt{\frac{2 \pi e^2 q}{\kappa_{\infty}} \left(\frac{n_e}{\left\langle m_e \right\rangle} + \frac{n_h}{\left\langle m_h \right\rangle} \right) },
\end{eq}
где эффективные массы электронов и дырок $m_e, m_h$, как и при вычислении друдевской проводимости \eqref{Drude_conductivity}, усреднены по распределению носителей:
\begin{eq}{carrier-averaged_masses}
\frac{n_e}{\left\langle m_e \right\rangle} &\equiv \sum_{s \in c,\vec{k}}\frac{f_{s \vec{k}}}{2}  \left[ \frac{\partial^2 \epsilon_{s \vec{k}}}{\partial (\hbar k_x)^2} + \frac{\partial^2 \epsilon_{s \vec{k}}}{\partial (\hbar k_y)^2} \right],\\
\frac{n_h}{\left\langle m_h \right\rangle} &\equiv - \sum_{s \in v,\vec{k}} \frac{{\bar f}_{s \vec{k}}}{2}\left[ \frac{\partial^2 \epsilon_{s \vec{k}}}{\partial (\hbar k_x)^2} + \frac{\partial^2 \epsilon_{s \vec{k}}}{\partial (\hbar k_y)^2} \right].\\
\end{eq}
Так как для ТГц генерации интересны ямы с шириной запрещённой зоны больше энергий оптических фононов, в формуле \eqref{plasmon_dispersion} мы используем высокочастотную диэлектрическую проницаемость теллурида кадмия-ртути $\kappa_{\infty} = 12$.

Расчёт длинноволновой границы области межзонных переходов производился следующим образом:
\begin{enumerate}
\item Законы дисперсии электронов $\epsilon_{s \vec{k}}$, рассчитанные методом, описанным в разделе \ref{sec:HgCdTe-Kane}, усреднялись по углу.
\item Для каждой пары модулей квазиволновых векторов $k_c, k_v$, принадлежащих сетке, на которой рассчитывался закон дисперсии, вычислялось минимально и максимально возможное изменение квазиволнового вектора при межзонном переходе, $q_{\rm min}(k_c,k_v) = \abs{k_c - k_v}$ и $q_{\rm max}(k_c,k_v) = k_c + k_v$, а также изменение энергии $\hbar\omega(k_c,k_v) = \epsilon_{c1}(k_c) - \epsilon_{v1}(k_v)$.
\item Для каждого квазиволнового вектора $q$, принадлежащего сетке, находились все такие пары $\{k_c, k_v\}$, что $q_{\rm min}(k_c,k_v) \leq q \leq q_{\rm max}(k_c,k_v)$.
\item Среди найденных пар $\{k_c, k_v\}$ выбиралась пара, дающая минимальное $\omega(k_c,k_v)$.
\end{enumerate}

\begin{fig}{plasmons-intersection}{plasmons-intersection}(а) --- закон дисперсии плазмонов и область межзонных переходов в квантовой яме \HgCdTe{} толщиной 6.5 нм при 77 К и концентрации каждого типа носителей $5\times10^{10}$ см$^{-2}$. На вставке изображён единственный возможный при этой концентрации процесс рекомбинации с испусканием плазмона. (б) --- то же самое, но для концентрации $5\times10^{11}$ см$^{-2}$. На вставке изображён возможный процесс плазмонной рекомбинации с участием носителей из дираковской части закона дисперсии.
\end{fig}

Пример рассчитанного закона дисперсии плазмонов и области межзонных переходов показан на рис.~\ref{fig:plasmons-intersection}а. Расчёты показали, что при увеличении концентрации носителей пересечение закона дисперсии плазмонов с областью межзонных переходов впервые происходит в точке, соответствующей межзонному переходу между дном зоны проводимости и боковым максимумом валентной зоны (вставка на рис.~\ref{fig:plasmons-intersection}а), а переходы с испусканием плазмонов, в которых участвуют только носители из (приближённо) дираковской части спектра (т. е. до того квазиимпульса, при котором зона v1 начинает загибаться вверх) становятся возможными лишь при гораздо больших концентрациях носителей (рис.~\ref{fig:plasmons-intersection}б).

Рекомбинация с испусканием плазмонов, в которой участвуют дырки из бокового максимума валентной зоны, имеет энергетический порог, равный разности энергий между центральным и боковым максимумом валентной зоны. Этот порог снижается с увеличением толщины ямы и обращается в ноль для непрямозонных ям, однако он остаётся не ниже порога для оже-рекомбинации (рис.~\ref{fig:plasmons-thresholds}а). Наличие порога частично компенсируется большой плотностью состояний в боковом максимуме, которая может на два порядка превосходить плотность состояний в центральном максимуме (рис.~\ref{fig:plasmons-DOS}), из-за чего количество дырок в боковом максимуме может быть сравнимо или даже превосходить их количество в центральном максимуме. Тем не менее, при достаточно низких температурах рекомбинация с испусканием плазмонов может быть подавлена.

\section{Пороговые энергии и пороговые концентрации носителей для плазмонной рекомбинации} \label{sec:plasmon-thresholds}
\begin{fig}{plasmons-thresholds}{plasmons-thresholds}(а) --- пороговые энергии для плазмонной и оже-рекомбинации в квантовых ямах \HgCdTe{} различной толщины. Пороговые энергии для плазмонной рекомбинации рассчитаны при минимальной концентрации носителей, при которой она возможна. (б) --- пороговые концентрации для плазмонной рекомбинации (сплошные линии), для плазмонной рекомбинации в пределах дираковской части закона дисперсии (длинный пунктир) и для достижения оптического усиления (короткий пунктир).
\end{fig}

\begin{narrowfig}{plasmons-DOS}{plasmons-DOS}Плотность состояний электронов в квантовых ямах \HgCdTe{} различной толщины при 4.2 К.
\end{narrowfig}

Также этот вид рекомбинации подавлен при достаточно низких концентрациях носителей, когда закон дисперсии плазмонов лежит вне области мезонных переходов. Рассчитанные пороговые концентрации носителей, ниже которых плазмонная рекомбинация невозможна (рассматриваются нелегированные ямы, поэтому концентрации электронов и дырок одинаковы), приведены на рис.~\ref{fig:plasmons-thresholds}б (короткий пунктир). Для относительно широкозонных ям эта концентрация существенно превосходит пороговую концентрацию, необходимую для достижения оптического усиления (сплошные линии на рис.~\ref{fig:plasmons-thresholds}б), поэтому плазмонная рекомбинация не вносит вклад в пороговые токи лазерной генерации. Однако для ям толщиной более 6.2 нм при 77 К (частота генерации 6.3 ТГц) пороговая концентрация для оптического усиления становится больше, чем пороговая концентрация для плазмонной рекомбинации, поэтому для этих ям можно ожидать увеличения пороговых токов по сравнению с значениями, рассчитанными в разделе~\ref{sec:HgCdTe-currents}.

В ситуации, когда количество дырок в боковом максимуме валентной зоны пренебрежимо мало, пороговые концентрации плазмонной рекомбинации оказываются гораздо выше (длинный пунктир на рис.~\ref{fig:plasmons-thresholds}б) и плазмонная рекомбинация влияет на пороговые токи только в наиболее узкозонных ямах, в которых частота генерации выходит за пределы ТГц области из-за возросшего друдевского и межподзонного поглощения (см. раздел~\ref{sec:HgCdTe-concentrations}). Таким образом, при достаточно низких температурах (когда в боковом максимуме валентной зоны почти нет дырок) пороговые токи ТГц лазерных диодов на основе квантовых ям теллурида кадмия-ртути будут определяться прежде всего оже-рекомбинацией, а не рекомбинацией с испусканием плазмонов.

Плазмоны могут играть не только отрицательную роль для лазерной генерации, увеличивая темп безызлучательной рекомбинации, но также могут сами усиливаться, вызывая вынужденные межзонные переходы. Такое явление усиления плазмонов за счёт вынужденных переходов называется спазерной генерацией~\cite{spaser} и может использоваться как для получения интенсивных когерентных плазменных волн, так и для конвертации генерируемых плазмонов в свободное электромагнитное излучение, позволяя создать сверхкомпактные лазеры~\cite{subwavelength_spaser} размерами меньше дифракционного предела.

Для спазерной генерации в квантовых ямах из теллурида кадмия-ртути, вероятно, необходимо участие \emph{центрального} максимума валентной зоны, так как межзонное усиление на переходах с участием бокового максимума валентной зоны будет конкурировать с внутризонным и межподзонным поглощением (затуханием Ландау) в пределах валентной зоны. Большие пороговые концентрации, требуемые для возможности испускания плазмонов при рекомбинации в пределах дираковской части спектра, могут свидетельствовать о том, что спазерная генерация достижима лишь в наиболее узкозонных ямах и в импульсном режиме (так как большие концентрации носителей приведут к большому темпу безызлучательной рекомбинации, высоким пороговым токам и интенсивному тепловыделению). Однако окончательный ответ на вопрос о возможности спазерной генерации в квантовых ямах из теллурида кадмия-ртути требует более точного расчёта закона дисперсии плазмонов с использованием условия зануления диэлектрической проницаемости вместо корневой формулы \eqref{plasmon_dispersion} и оставляется нами для дальнейших исследований.

\section{Времена плазмонной рекомбинации} \label{sec:plasmon-recombination_times}
В предыдущем разделе мы показали, что плазмонная рекомбинация, скорее всего, не будет влиять на пороговые токи лазерной генерации при достаточно низких температурах. Однако для практических применений наиболее интересны лазеры, работающие хотя бы при температуре жидкого азота: при гелиевых температурах уже работают лазеры на халькогенидах свинца-олова~\cite{lead_salt_record_wavelength} вплоть до 6.5 ТГц.

При 77 К пороговые энергии плазмонной рекомбинации для ям ТГц диапазона составляют всего 3--5 $k_B T_e$, что недостаточно для того, чтобы скомпенсировать эффект огромной плотности состояний в боковом максимуме валентной зоны. Кроме того, мы используем в расчётах классическую корневую формулу для закона дисперсии плазмонов \eqref{plasmon_dispersion}. При использовании точного закона дисперсии плазмонов, полученного из условия зануления диэлектрической проницаемости, а также учёте дополнительных механизмов потерь, не учтённых при расчёте пороговых концентраций для оптического усиления (см. раздел~\ref{sec:HgCdTe-discussion}), диапазон толщин ям, для которых плазмонная рекомбинация важна, может измениться. Наконец, конечные времена жизни плазмонов приводят к тому, что плазмонная рекомбинация возможна даже когда закон дисперсии не пересекает область межзонных переходов.

Поэтому в данном разделе мы оценим характерные времена плазмонной рекомбинации на пороге лазерной генерации для ям различной толщины при 77 К. Так как плазмоны, как и продольные оптические фононы, соответствуют полюсам в обратной диэлектрической проницаемости, расчёт темпа плазмонной рекомбинации производился по той же формуле~\eqref{phonon_recombination}, что и для фононной рекомбинации. Отличие состоит лишь в том, что в качестве $\kappa(\omega)$ вместо решёточной диэлектрической проницаемости мы берём диэлектрическую проницаемость электрон-дырочной системы в приближении плазмонного полюса:
\begin{eq}{plasmon_pole_kappa}
\kappa_{pl}(q, \omega) = \kappa_{\infty} \left[ 1 - \frac{\omega^2_{\rm pl}(q)}{\omega(\omega + i \gamma)} \right],
\end{eq}
где $\gamma = 1$~мэВ, как и в разделе~\ref{sec:HgCdTe-optical_conductivity}. Затуханием Ландау мы пренебрегаем, хотя для более точного расчёта темпа плазмонной рекомбинации его учёт необходим. Хотя формула \eqref{plasmon_pole_kappa} даёт неправильную диэлектрическую проницаемость в статическом пределе, для расчёта темпа плазмонной рекомбинации нам нужна лишь $\Im \kappa_{pl}(q, \omega)^{-1}$ вблизи закона дисперсии плазмонов. Если мы учитываем только вклад плазмонного полюса в $\Im \kappa_{pl}(q, \omega)^{-1}$, то вычет в полюсе однозначно связан с частотой плазмона правилом сумм $\int_0^{\infty} \omega \kappa_{\infty} \Im \kappa_{pl}(q, \omega)^{-1} d \omega = -\pi \omega^2_{\rm pl}(q)/2$, и формула \eqref{plasmon_pole_kappa} удовлетворяет этому соотношению.

\begin{narrowfig}{plasmons-times}{plasmons-times} Времена жизни неравновесных носителей на пороге лазерной генерации в квантовых ямах \HgCdTe{} при 77 К, связанные с плазмонной рекомбинацией (пунктирные линии) и оже-рекомбинацией (сплошная линия). Длинный пунктир --- результаты для пороговых концентраций, рассчитанных в разделе~\ref{sec:HgCdTe-concentrations}, короткий пунктир --- результаты для вдвое больших концентраций носителей. На графиках также показаны стандартные отклонения результатов, полученных интегрированием методом Монте-Карло. 
\end{narrowfig}

Рассчитанные времена рекомбинации на пороге лазерной генерации при 77 К приведены на рис.~\ref{fig:plasmons-times} (длинный пунктир). Как и для оже-рекомбинации, времена плазмонной рекомбинации снижаются с уменьшением запрещённой зоны и для ям толщиной более 6.5 нм плазмонная рекомбинация оказывается быстрее оже-рекомбинации; для непрямозонных ям характерные времена составляют около 0.3 пс. Однако для ям, для которых рассчитанные частоты генерации при 77 К лежат в ТГц области (т. е. ям толщиной 5.7 --- 6.5 нм), плазмонная рекомбинация увеличивает суммарный темп рекомбинации не более, чем в два раза.

Как видно из рис.~\ref{fig:plasmons-times}, плазмонная рекомбинация резко ускоряется вблизи критической толщины $d_c = 7.4$~нм, при которой ширина запрещённой зоны обращается в ноль. Это связано как с уменьшением пороговых концентраций, начиная с которых закон дисперсии плазмонов пересекает область межзонных переходов, так и с увеличением концентрации носителей, необходимой для достижения лазерной генерации. Так как различные оптические потери, описанные в разделе \ref{sec:HgCdTe-discussion}, могут приводить к тому, что пороговые концентрации для лазерной генерации будут больше своего фундаментального нижнего предела, мы решили исследовать влияние концентрации носителей на темп плазмонной рекомбинации. Коротким пунктиром на рис.~\ref{fig:plasmons-times} показаны времена плазмонной рекомбинации, рассчитанные для концентраций носителей вдвое выше тех пороговых значений, которые получены в разделе~\ref{sec:HgCdTe-concentrations}. Снижение времён рекомбинации по сравнению с исходными значениями пороговых концентраций $n_{\rm th}$ наиболее заметно для ям, в которых закон дисперсии плазмонов не пересекает область межзонных переходов при $n_{\rm th}$, но пересекает при $2 n_{\rm th}$. Это наблюдается как раз для ям ТГц диапазона, поэтому при концентрациях $2 n_{\rm th}$ плазмонная рекомбинация играет существенную роль во всей ТГц области, увеличивая пороговые токи в 2--4 раза по сравнению с рассчитанными в разделе \ref{sec:HgCdTe-currents}.

Таким образом, плазмонная рекомбинация, наряду с оже-рекомбинацией, вносит основной вклад в темп безызлучательной рекомбинации в квантовых ямах из теллурида кадмия-ртути с запрещённой зоной в ТГц диапазоне. Темп плазмонной рекомбинации чувствителен к температуре и концентрации носителей, поэтому избежать возрастания пороговых токов, связанных с этим процессом, можно либо минимизацией оптических потерь (и, соответственно, пороговых концентраций носителей), либо снижением температуры.
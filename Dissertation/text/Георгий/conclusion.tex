%% Согласно ГОСТ Р 7.0.11-2011:
%% 5.3.3 В заключении диссертации излагают итоги выполненного исследования, рекомендации, перспективы дальнейшей разработки темы.
%% 9.2.3 В заключении автореферата диссертации излагают итоги данного исследования, рекомендации и перспективы дальнейшей разработки темы.
\chapter*{Заключение}						% Заголовок
\addcontentsline{toc}{chapter}{Заключение}	% Добавляем его в оглавление
В настоящей диссертации рассмотрена возможность использования узкозонных полупроводников для межзонной лазерной генерации в ТГц диапазоне. Особое внимание уделено процессам оже-рекомбинации и рекомбинации с испусканием плазмонов, поскольку именно они будут определять пороговые токи ТГц лазерных диодов. Так как подавление оже-рекомбинации ожидается в материалах с дираковским законом дисперсии $E = \pm\sqrt{v_0^2 p^2 + E_g^2/4}$, были рассмотрены случаи дираковского и приближённо дираковского закона дисперсии на примере графена и квантовых ям \HgCdTe{}.

В случае дираковского закона дисперсии трёхчастичная оже-рекомбинация запрещена законами сохранения (кроме коллинеарных процессов в бесщелевом случае), и темп оже-рекомбинации определяется многочастичными процессами. Для учёта этих процессов нами разработан подход, основанный на методе неравновесных функций Грина и самосогласованном $GW$-приближении. С использованием этого подхода рассчитаны времена оже-рекомбинации в нелегированном графене со слабой инверсией населённостей и продемонстрировано их согласие с экспериментальными данными. Полученные времена составляют около 1--3 пс при комнатной температуре в зависимости от диэлектрической проницаемости окружающего материала. Это свидетельствует о том, что для бесщелевого дираковского спектра запрет на оже-рекомбинацию со стороны законов сохранения фактически исчезает из-за эффектов уширения спектра, связанных с рассеянием носителей друг на друге. Также мы сравнили разработанный нами метод расчёта темпа оже-рекомбинации с более простыми методами, использовавшимися в литературе, и показали, что более простые методы могут давать ошибку в несколько раз из-за неучтённых или некорректно учтённых многочастичных эффектов, таких как уширение и искривление спектра носителей, а также динамическое экранирование кулоновского взаимодействия. 

Наличие запрещённой зоны в дираковском спектре приводит к подавлению оже-рекомбинации даже в случае, когда трёхчастичные оже-процессы разрешены из-за отклонений реального закона дисперсии от дираковского, как показано нами на примере квантовых ям \HgCdTe{}. Рассчитанные пороговые энергии оже-рекомбинации в ямах нормальной зонной структуры достигают половины запрещённой зоны и более, что обусловлено близкими эффективными массами электронов и дырок (в отличие от трёхмерных полупроводников \AIIIBV{}) и непараболичностью закона дисперсии. Для ям с шириной запрещённой зоны в диапазоне 6--10 ТГц такие пороговые энергии соответствуют подавлению оже-рекомбинации на полтора-два порядка при 77 К по сравнению со случаем большой электрон-дырочной асимметрии, реализующимся в ямах инвертированной зонной структуры. Это подтверждается рассчитанными временами оже-рекомбинации, достигающими сотен пс при 77 К в ямах ТГц диапазона с нормальной зонной структурой, при том что в ямах с инвертированной зонной структурой эти времена составляют около 1 пс.

Помимо оже-рекомбинации, также была исследована рекомбинация с испусканием двумерных плазмонов. На примере квантовых ям \HgCdTe{} продемонстрировано, что существует некоторая пороговая концентрация носителей, ниже которой межзонные переходы с испусканием плазмонов в первом приближении невозможны (т. е. если не учитывать неопределённость энергии плазмонов из-за конечного времени жизни). Эта пороговая концентрация определяется условием пересечения закона дисперсии плазмонов с областью межзонных переходов в пространстве $(\vec{q}, \omega)$. При концентрации носителей выше пороговой плазмонная рекомбинация в квантовых ямах \HgCdTe{} оказывается быстрее оже-рекомбинации и имеет характерные времена порядка сотен фемтосекунд при 77 К; при концентрациях ниже пороговой плазмонная рекомбинация замедляется до 1 нс и более при 77 К и оказывается медленнее оже-рекомбинации. По нашим расчётам, при обеспечении достаточно низких оптических потерь за пределами активной среды пороговые концентрации для достижения лазерной генерации при 77 К в ямах \HgCdTe{} нормальной зонной структуры оказываются ниже пороговых концентраций для плазмонной рекомбинации, за исключением наиболее узкозонных ям.

Пороговая концентрация для плазмонной рекомбинации сильно чувствительна к поведению зонной структуры в области больших квазиимпульсов. Так, наличие побочного локального максимума в валентной зоне квантовых ям \HgCdTe{} снижает пороговые концентрации для плазмонной рекомбинации примерно на порядок по сравнению с дираковским законом дисперсии. Однако такая рекомбинация с участием дырок из побочного максимума валентной зоны имеет энергетический порог, равный разности энергий между основным и побочным максимумами, поэтому плазмонная рекомбинация может быть подавлена при низких температурах даже при превышении пороговой концентрации.

Рассчитанные времена рекомбинации в графене и квантовых ямах \HgCdTe{} были использованы для оценки пороговых токов ТГц лазерных диодов на основе этих материалов. Полученные значения при 77 К составляют сотни А/см$^2$ на одну квантовую яму и единицы кА/см$^2$ на один графеновый слой. Это свидетельствует в пользу достижимости межзонной ТГц генерации при азотной температуре, хотя окончательный ответ на этот вопрос требует аккуратного моделирования конкретной конструкции лазера. При 300 К достижению ТГц усиления в квантовых ямах \HgCdTe{} препятствует межподзонное и друдевское поглощение, а в графене пороговые токи составляют около 50 кА/см$^2$ на один графеновый слой, поэтому межзонная ТГц генерация в непрерывном режиме, скорее всего, возможна лишь при криогенных температурах. Достижимые частоты генерации в квантовых ямах \HgCdTe{} ограничены снизу областью сильного решёточного поглощения и составляют около 6 ТГц. В гексагональном нитриде бора, использующемся в качестве диэлектрика в высококачественных графеновых гетероструктурах, область сильного решёточного поглощения лежит выше ТГц диапазона, поэтому для графена минимальная частота генерации будет ограничена друдевским поглощением в волноводных слоях и самом графене и может быть меньше 6 ТГц при достаточно низких температурах.

На основании полученных результатов можно сформулировать общие рекомендации по проектированию ТГц лазерных диодов. Для достижения ТГц лазерной генерации на межзонных переходах в узкозонных материалах и минимизации пороговых токов требуется обеспечить подавление основных механизмов безызлучательной рекомбинации: оже-рекомбинации и рекомбинации с испусканием плазмонов. Этому благоприятствуют следующие факторы:
\begin{itemize}
\item наличие ненулевой запрещённой зоны;
\item дираковский закон дисперсии в как можно более широком диапазоне энергий;
\item отсутствие в этом диапазоне энергий каких-либо других зон/подзон, кроме нижней подзоны зоны проводимости и верхней подзоны валентной зоны;
\item низкие оптические потери в резонаторе и, соответственно, низкие пороговые концентрации носителей для достижения лазерной генерации;
\item низкий уровень остаточного легирования активной среды;
\item криогенные температуры.
 \end{itemize}

В заключение обсудим возможности дальнейшего развития темы исследования. С теоретической точки зрения представляет интерес обобщение разработанного метода расчёта темпа оже-рекомбинации с учётом многочастичных эффектов на случай сильного межэлектронного взаимодействия. Для этого требуется использование ещё более продвинутых приближений, нежели самосогласованное $GW$-приближение, и возникает проблема сохранения разумной вычислительной сложности. Также разработанный метод можно применить для определения точной границы между режимом, когда основной вклад в темп оже-рекомбинации связан с многочастичными эффектами, и режимом, когда основной вклад в темп оже-рекомбинации определяется отклонением закона дисперсии от дираковского. Другой важной теоретической проблемой для узкозонных материалов является корректный учёт динамического экранирования в условиях сильной инверсии населённостей, при которой возможно появление незатухающих плазмонов и зануление диэлектрической проницаемости на действительных частотах.

С практической точки зрения интересно применение результатов настоящей диссертации для моделирования конкретных конструкций ТГц лазеров на основе узкозонных материалов, поиск оптимального состава активной среды, а также исследование возможности ТГц генерации на плазмонных модах в двумерных материалах.
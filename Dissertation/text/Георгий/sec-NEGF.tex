\section{Расчёт темпа рекомбинации методом неравновесных функций Грина} \label{sec:NEGF}
Метод неравновесных функций Грина широко используется для описания кинетики носителей в полупроводниках в случаях, когда важны эффекты взаимодействия между квазичастицами~\cite{Haug}. Суть метода заключается в том, что любую наблюдаемую величину можно записать во вторичном квантовании и свести к средним значениям произведений операторов рождения и уничтожения (функциям Грина). Из уравнения Шрёдингера можно получить самосогласованную систему уравнений для функций Грина, наглядно записывающуюся в виде диаграмм Фейнмана.

Достоинство метода неравновесных функций Грина заключается в том, что, хотя точное решение многоэлектронной задачи, как правило, невозможно на практике, на основании физических соображений можно выбрать некоторое подмножество диаграмм Фейнмана, охватывающее интересующие нас эффекты, и получить хорошее приближение к точному решению.

Метод неравновесных функций Грина применялся ранее для вывода формул для темпа оже-рекомбинации и других видов рекомбинации~\cite{Ziep-Mocker,Yevick-GW_Auger}, однако они, как правило, использовались для учёта влияния рассеяния на фононах, а не межэлектронного взаимодействия. Влияние межэлектронного взаимодействия на темп оже-рекомбинации на примере полупроводников A$^{\rm III}$B$^{\rm V}$ рассматривалось в работе~\cite{Auger_scattering}, но в ней использовались приближения, непригодные для дираковских материалов (пренебрежение вкладом электронов и дырок в экранирование, учёт эффектов рассеяния только для дырок).

В данном разделе мы предложим метод, основанный на так называемом самосогласованном $GW$-приближении, который позволяет учесть и уширение спектра носителей из-за межэлектронного рассеяния, и искривление дираковского конуса (также вызванное межэлектронным взаимодействием), и динамическое экранирование кулоновского взаимодействия.

\subsection{Краткое введение в метод неравновесных функций Грина} \label{sec:NEGF-basics}
В методе неравновесных функций Грина~\cite{NEGFhandbook, Haug} основными математическими объектами являются одночастичные функции Грина, которые содержат всю информацию о свойствах квазичастиц. Существует четыре основных вида одночастичных функций Грина: запаздывающая ($G^{R}$), опережающая ($G^{A}$), <<меньшая>> ($G^{<}$) и <<большая>> ($G^{>}$). Для электронов в периодическом потенциале они определяются следующим образом:
\begin{eq}{Green's_functions_definitions}
     G^{R}_{s_1 s_2}(\vec{k},t_1,t_2) &= - \frac{i}{\hbar} \left\langle \hat{\psi}_{s_1 \vec{k}}(t_1)  \hat{\psi}^{\dagger}_{s_2 \vec{k}}(t_2) +  \hat{\psi}^{\dagger}_{s_2 \vec{k}}(t_2) \hat{\psi}_{s_1 \vec{k}}(t_1) \right\rangle \theta(t_1 - t_2),\\
    G^{A}_{s_1 s_2}(\vec{k},t_1,t_2) &= + \frac{i}{\hbar} \left\langle \hat{\psi}_{s_1 \vec{k}}(t_1)  \hat{\psi}^{\dagger}_{s_2 \vec{k}}(t_2) +  \hat{\psi}^{\dagger}_{s_2 \vec{k}}(t_2) \hat{\psi}_{s_1 \vec{k}}(t_1) \right\rangle \theta(t_2 - t_1),\\
     G^{<}_{s_1 s_2}(\vec{k},t_1,t_2) &= + \frac{i}{\hbar} \left\langle \hat{\psi}^{\dagger}_{s_2 \vec{k}}(t_2) \hat{\psi}_{s_1 \vec{k}}(t_1) \right\rangle,\\
     G^{>}_{s_1 s_2}(\vec{k},t_1,t_2) &= - \frac{i}{\hbar} \left\langle \hat{\psi}_{s_1 \vec{k}}(t_1)  \hat{\psi}^{\dagger}_{s_2 \vec{k}}(t_2) \right\rangle,\\
\end{eq}
где $\hat{\psi}_{s \vec{k}}(t)$ --- оператор уничтожения электрона с квазиволновым вектором $\vec{k}$ в зоне $s$ в момент времени $t$ (оператор берётся в представлении взаимодействия), $\theta(t)$ --- функция Хевисайда, а угловые скобки обозначают усреднение по заданному начальному состоянию системы.

В большинстве практически значимых ситуаций состояние системы слабо меняется за время порядка $2\pi\hbar/E$ ($E$ --- характерная энергия носителей), поэтому на таких временных масштабах систему можно рассматривать как однородную во времени и перейти к энергетическому представлению:
\begin{eq}{Green's_functions_energy_representation}
     G^{R/A/</>}_{s_1 s_2}(\vec{k}, E) &= \int_{-\infty}^{+\infty} d \Delta t \ G^{R/A/</>}_{s_1 s_2}(\vec{k}, t_0+\Delta t, t_0) e^{\frac{i E \Delta t}{\hbar}},\\
\end{eq}
где $t_0$ может быть выбрано произвольным в силу временной однородности.

Недиагональные по зонным индексам компоненты описывают эффекты <<смешивания зон>>. Такие эффекты могут быть важны, например, в сильных лазерных полях, создающих заметную межзонную поляризацию в полупроводниках~\cite{interband_coherence}, однако в данной диссертации мы интересуемся прежде всего пороговыми характеристиками лазеров, поэтому такие компоненты функций Грина будут полагаться нулевыми, хотя их учёт не представляет принципиальных затруднений.

Диагональные компоненты запаздывающей и опережающей функций Грина комплексно сопряжены друг другу и взаимно-однозначно связаны со спектральной функцией ${\cal A}_{s}(\vec{k}, E) = -\frac{1}{\pi} \Im G^{R}_{s}(\vec{k}, E) = \frac{1}{\pi} \Im G^{A}_{s}(\vec{k}, E)$ соотношениями Крамерса-Кронига, а диагональные компоненты <<меньшей>> и <<большей>> функций Грина можно записать в физически понятном виде, используя обобщённую функцию распределения ${\cal F}_s(\vec{k}, E)$:
\begin{eq}{Green's_functions_through_spectral_and_distribution}
     G^{<}_s(\vec{k}, E) &= 2\pi i {\cal F}_s(\vec{k}, E) {\cal A}_s(\vec{k}, E), \\
    G^{>}_s(\vec{k}, E) &= 2\pi i ({\cal F}_s(\vec{k}, E) - 1) {\cal A}_s(\vec{k}, E).
\end{eq}

Спектральная функция всегда неотрицательна и нормирована на единицу:
\begin{eq}{spectral_function_normalization}
\int_{-\infty}^{+\infty} {\cal A}_{s}(\vec{k}, E) dE = \left\langle \hat{\psi}_{s \vec{k}}(t_0)  \hat{\psi}^{\dagger}_{s \vec{k}}(t_0) +  \hat{\psi}^{\dagger}_{s \vec{k}}(t_0) \hat{\psi}_{s \vec{k}}(t_0) \right\rangle = 1.
\end{eq}

Для невзаимодействующих электронов спектральная функция имеет вид ${\cal A}_{s}(\vec{k}, E) = \delta(E - \epsilon_{s \vec{k}})$ ($\epsilon_{s \vec{k}}$ --- энергия состояния с квазиволновым вектором $\vec{k}$ в зоне $s$), а обобщённая функция распределения равна числам заполнения ${\cal F}_s(\vec{k}, E) = f_{s \vec{k}}$. Соответствующие выражения для функций Грина невзаимодействующих электронов выглядят следующим образом:
\begin{eq}{noninteracting_Green's_functions}
     G^{0\, R}_s(\vec{k}, E) &= \frac{1}{E - \epsilon_{s\vec{k}} + i0}, \\
     G^{0\, A}_s(\vec{k}, E) &= \frac{1}{E - \epsilon_{s\vec{k}} - i0}, \\
    G^{0\, <}_s(\vec{k}, E) &= 2\pi i f_{s \vec{k}} \delta\left(E - \epsilon_{s\vec{k}} \right), \\
    G^{0\, >}_s(\vec{k}, E) &= 2\pi i (f_{s \vec{k}} - 1) \delta\left(E - \epsilon_{s\vec{k}} \right).
\end{eq}

Таким образом, запаздывающая и опережающая функции Грина содержат информацию о спектре квазичастиц, а <<меньшая>> и <<большая>> --- о распределении носителей по этому спектру. Влияние межэлектронного взаимодействия (а также любых других отклонений точного гамильтониана от $\sum_{s,\vec{k}} \epsilon_{s\vec{k}} \hat{\psi}^{\dagger}_{s \vec{k}} \hat{\psi}_{s \vec{k}}$) на функции Грина описывается так называемыми собственными энергиями $\Sigma^{R/A/</>}_s(\vec{k}, E)$, отвечающими за уширение спектральной функции, сдвиг энергий квазичастиц, усложнение вида спектральной функции (например, появление плазмонных сателлитов~\cite{plasmon_satellites}) и модификацию функции распределения. Связь между точными функциями Грина и собственными энергиями даётся уравнениями Дайсона:
\begin{eq}{interacting_Green's_functions}
     G^{R}_s(\vec{k}, E)^{-1} &= G^{0\, R}_s(\vec{k}, E)^{-1} - \Sigma^{R}_s(\vec{k}, E) = E - \epsilon_{s\vec{k}} - \Sigma^R_s(\vec{k},E) + i0, \\
     G^{A}_s(\vec{k}, E) &= G^{R}_s(\vec{k}, E)^{*}, \\
     G^{<}_s(\vec{k}, E) &= \Sigma^{<}_s(\vec{k}, E)\abs{G^{R}_s(\vec{k}, E)}^2, \\
     G^{>}_s(\vec{k}, E) &= \Sigma^{>}_s(\vec{k}, E)\abs{G^{R}_s(\vec{k}, E)}^2. \\
\end{eq}

В состоянии равновесия обобщённая функция распределения принимает ферми-дираковский вид ${\cal F}_s(\vec{k}, E) = f(E) \equiv \left\{ \exp\left[ (E - \mu)/k_B T_e \right] + 1 \right\}^{-1}$ независимо от конкретного вида межэлектронного взаимодействия, и уравнения Дайсона можно упростить:
\begin{eq}{equilibrium_interacting_Green's_functions}
     G^{R}_s(\vec{k}, E) &= \frac{1}{E - \epsilon_{s\vec{k}} - \Sigma^R_s(\vec{k},E) + i0}, \\
     G^{A}_s(\vec{k}, E) &= G^{R}_s(\vec{k}, E)^{*}, \\
     G^{<}_s(\vec{k}, E) &= - 2 i f(E) \Im G^{R}_s(\vec{k}, E), \\
     G^{>}_s(\vec{k}, E) &= 2 i \left[1 - f(E)\right] \Im G^{R}_s(\vec{k}, E). \\
\end{eq}

Межэлектронное взаимодействие через скалярный потенциал (то есть экранированное кулоновское взаимодействие) описывается следующими функциями:
\begin{eq}{screened_Coulomb_definitions}
     W^{R}(\vec{r}_1,\vec{r}_2,t_1,t_2) &= - \frac{i e^2}{\hbar} \left\langle \hat{\varphi}(\vec{r}_1,t_1)  \hat{\varphi}(\vec{r}_2,t_2) - \hat{\varphi}(\vec{r}_2,t_2) \hat{\varphi}(\vec{r}_1,t_1) \right\rangle \theta(t_1 - t_2),\\
    W^{A}(\vec{r}_1,\vec{r}_2,t_1,t_2) &= + \frac{i e^2}{\hbar} \left\langle \hat{\varphi}(\vec{r}_1,t_1)  \hat{\varphi}(\vec{r}_2,t_2) - \hat{\varphi}(\vec{r}_2,t_2) \hat{\varphi}(\vec{r}_1,t_1) \right\rangle \theta(t_2 - t_1),\\
     W^{<}(\vec{r}_1,\vec{r}_2,t_1,t_2) &= - \frac{i e^2}{\hbar} \left\langle \hat{\varphi}(\vec{r}_2,t_2) \hat{\varphi}(\vec{r}_1,t_1) \right\rangle,\\
     W^{>}(\vec{r}_1,\vec{r}_2,t_1,t_2) &= -\frac{i e^2}{\hbar} \left\langle \hat{\varphi}(\vec{r}_1,t_1)  \hat{\varphi}(\vec{r}_2,t_2) \right\rangle,\\
\end{eq}
где $\hat{\varphi}(\vec{r},t)$ --- оператор электрического потенциала в точке $\vec{r}$ в момент времени $t$ (в представлении взаимодействия), $e$ --- элементарный заряд.

Воспользовавшись временной однородностью и частичной трансляционной инвариантностью, можно перейти к фурье-представлению:
\begin{eq}{screened_Coulomb_Fourier_representation}
     &W^{R/A/</>}(\vec{q}+\vec{G}_1, \vec{q}+\vec{G}_2, \omega)\\
&= \frac{1}{\cal V} \int_{-\infty}^{+\infty} d \Delta t e^{i \omega \Delta t} \int_{\cal V} d^D \vec{r}_1 e^{-i (\vec{q}+\vec{G}_1) \vec{r}_1} \int_{\cal V} d^D \vec{r}_2 e^{+i (\vec{q}+\vec{G}_2) \vec{r}_2}\\
&\times W^{R/A/</>}(\vec{r}_1,\vec{r}_2, t_0+\Delta t, t_0),\\
\end{eq}
где $\vec{G}_1$, $\vec{G}_2$ --- векторы обратной решётки, вектор $\vec{q}$ принадлежит первой зоне Бриллюэна, $D$ --- размерность системы\footnote{Вообще говоря, даже для низкоразмерных систем нужно делать трёхмерное преобразование Фурье, так как орбитали имеют ненулевую протяжённость и $\vec{r},\vec{r'}$ не привязаны \emph{строго} к одной плоскости или прямой. Однако пренебрегая ненулевыми $\vec{G}_1$, $\vec{G}_2$, мы оставим только длинноволновую часть межэлектронного взаимодействия, для которой систему можно считать истинно двумерной/одномерной.}, ${\cal V}$ --- нормировочный объём.

В силу малости кулоновского взаимодействия для больших передач импульса в расчётах темпа рекомбинации часто пренебрегают слагаемыми с ненулевыми $\vec{G}_1$, $\vec{G}_2$. Мы тоже будем использовать это приближение, так как в настоящей работе исследуются прямозонные материалы с узкой запрещённой зоной, для которых оно работает хорошо~\cite{neglecting_short-range_Coulomb}.

В отсутствие экранирования со стороны валентных электронов межэлектронное взаимодействие выглядит следующим образом:
\begin{eq}{lattice_screened_Coulomb}
   V^{R}(\vec{q},\omega) &= \frac{V^{0}(\vec{q})}{\kappa(\omega)},\\
   V^{A}(\vec{q},\omega) &= V^{R}(\vec{q},\omega)^{*},\\
   V^{<}(\vec{q},\omega) &= 2i n_{\rm phon}(\vec{q},\omega) \Im V^{R}(\vec{q},\omega),\\
   V^{>}(\vec{q},\omega) &= 2i \left[ n_{\rm phon}(\vec{q},\omega) + 1 \right] \Im V^{R}(\vec{q},\omega).
\end{eq}
Запаздывающее межэлектронное взаимодействие представляет собой обычный кулоновский потенциал $V^{0}(\vec{q}) = \int_{\mathbb{R}^D} d^D \vec{r} (e^2/r) \exp(-i \vec{q}\vec{r})$ ($= 2\pi e^2/q$ для двумерных систем), делённый на диэлектрическую проницаемость $\kappa(\omega)$, связанную с экранированием ионами решётки и электронами внутренних оболочек. <<Меньшее>> и <<большее>> межэлектронное взаимодействие, подобно <<меньшей>> и <<большей>> электронным функциям Грина, содержат информацию о числах заполнения электромагнитных мод. Так как в $\kappa(\omega)$ учитывается решёточное экранирование, под электромагнитными модами здесь подразумеваются оптические фононы. В случае, когда решётка находится в равновесии при температуре $T_{\rm lat}$, числа заполнения фононов даются распределением Бозе-Эйнштейна $n_{\rm phon}(\vec{q},\omega) \equiv n_{\rm phon}(\hbar \omega) = \left[ \exp\left( \hbar\omega/k_B T_{\rm lat} \right) - 1 \right]^{-1}$.

Подобно тому, как электронные функции Грина изменяются из-за взаимодействия электронов с электромагнитным полем, кулоновский потенциал изменяется из-за взаимодействия электромагнитного поля с электронами. Это дополнительное экранирование свободными носителями заряда описывается поляризационными операторами $\Pi^{R/A/</>}(\vec{q},\omega)$, которые связаны с экранированным межэлектронным взаимодействием уравнениями Дайсона (аналогично уравнениям \eqref{interacting_Green's_functions} для электронных функций Грина):
\begin{eq}{screened_Coulomb}
     W^{R}(\vec{q}, \omega)^{-1} &= V^{R}(\vec{q}, \omega)^{-1} - \Pi^{R}(\vec{q}, \omega) = \left[\frac{V^{R}(\vec{q}, \omega)}{\epsilon^{R}(\vec{q}, \omega)}\right]^{-1},\\
     \epsilon^{R}(\vec{q}, \omega) &= 1 - V^{R}(\vec{q}, \omega)\Pi^{R}(\vec{q}, \omega), \\
     W^{A}(\vec{q}, \omega) &= W^{R}(\vec{q}, \omega)^{*}, \\
     W^{</>}(\vec{q}, \omega) &= \left[\Pi^{</>}(\vec{q}, \omega) + \Pi^{0\, </>}(\vec{q}, \omega)\right]\abs{W^{R}(\vec{q}, \omega)}^2,\\
     \Pi^{0\, </>}(\vec{q}, \omega) &= \frac{V^{</>}(\vec{q}, \omega)}{\abs{V^{R}(\vec{q}, \omega)}^2}.
\end{eq}
В последней строчке определены поляризационные операторы, связанные с \emph{решёточным} экранированием, поэтому в выражениях для $W^{</>}(\vec{q}, \omega)$ два слагаемых, в отличие от аналогичных уравнений для электронных функций Грина \eqref{interacting_Green's_functions}.


\subsection{Кинетическое уравнение и темп рекомбинации} \label{sec:Kadanoff-Baym}
Кинетика носителей в методе неравновесных функций Грина описывается уравнением Каданова-Бейма:
\begin{eq}{Kadanoff-Baym}
     &i \hbar \left( \frac{\partial}{\partial t_1} + \frac{\partial}{\partial t_2} \right) G^{<}_s(\vec{k}, t_1, t_2)\\
&= \frac{1}{2} \int_{-\infty}^{+\infty} dt_3 \left[ G^{<}_s(\vec{k},t_1,t_3) \Sigma^{>}_s(\vec{k},t_3,t_2) + \Sigma^{>}_s(\vec{k},t_1,t_3) G^{<}_s(\vec{k},t_3,t_2) \right.\\
 &- \left. G^{>}_s(\vec{k},t_1,t_3) \Sigma^{<}_s(\vec{k},t_3,t_2) - \Sigma^{<}_s(\vec{k},t_1,t_3) G^{>}_s(\vec{k},t_3,t_2) \right],
\end{eq}
где $\Sigma^{</>}$ --- так называемые <<меньшая>> и <<большая>> собственные энергии, связанные с временем жизни электрона в данном состоянии.

Из уравнения Каданова-Бейма можно найти темп рекомбинации как скорость изменения концентрации электронов в зоне проводимости:
\begin{eq}{Kadanoff-Baym_recombination_rate}
    R &= -\frac{d {\cal N}_c (t)}{dt} =  -g\frac{d}{dt} \sum_{\vec{k}} \left\langle \hat{\psi}^{\dagger}_{c \vec{k}}(t) \hat{\psi}_{c \vec{k}}(t) \right\rangle = i g \hbar \frac{d}{dt} \sum_{\vec{k}} G^{<}_c(\vec{k}, t, t)\\
&= g \sum_{\vec{k}, E} \left[ G^{<}_c(\vec{k}, E) \Sigma^{>}_c(\vec{k}, E) - G^{>}_c(\vec{k}, E) \Sigma^{<}_c(\vec{k}, E) \right],\\
\end{eq}
где $\sum_{\vec{k}, E} \equiv \int_{-\infty}^{+\infty} \frac{dE}{2\pi\hbar} \int_{BZ} \frac{d^D \vec{k}}{\left( 2\pi \right)^D}$, $g$ --- фактор вырождения, то есть число независимых <<разновидностей>> электронов, которые отличаются друг от друга проекцией спина либо индексом долины и не могут превращаться друг в друга в рамках выбранной модели. В стационарном состоянии, когда рекомбинация уравновешена накачкой, в уравнении \eqref{Kadanoff-Baym_recombination_rate} нужно брать собственные энергии без учёта вклада накачки.

Таким образом, вычисление темпа рекомбинации сводится к вычислению функций Грина и собственных энергий в рассматриваемом материале, в которых оказывается скрыта вся сложность многочастичной квантовой системы. Очевидно, найти их точно так же сложно, как и решить многочастичное уравнение Шрёдингера, поэтому приходится использовать приближения. В данной работе используется так называемое самосогласованное $GW$-приближение, которое позволяет учесть основные многочастичные эффекты, которые могут влиять на темп рекомбинации в дираковских материалах: размытие и искривление дираковского конуса, динамическое экранирование кулоновского взаимодействия, рекомбинация с участием плазмонов.

\subsection{Самосогласованное $GW$-приближение} \label{sec:GW}
В самосогласованном $GW$-приближении~\cite{NEGF-GW} собственная энергия раскладывается по степеням функции Грина $G$ и экранированного кулоновского взаимодействия $W$, и оставляется только слагаемое низшего порядка $\Sigma = iGW$. В более подробных обозначениях выражения для каждого вида собственной энергии выглядят так:
\begin{eq}{SigmaGWwithmatrixelements}
    \Sigma^R_s(\vec{k},E) &= i \hbar \sum_{s',\vec{k}',\omega} &&\left[{G^R_{s'}(\vec{k}',E+\hbar\omega) W^{<}_{s\vec{k},s'\vec{k}',s'\vec{k}',s\vec{k}}(\omega)} \right. \\
    & &&+ \left. {G^{<}_{s'}(\vec{k}',E+\hbar\omega) W^{A}_{s\vec{k},s'\vec{k}',s'\vec{k}',s\vec{k}}(\omega)} \right],\\
     \Sigma^A_s(\vec{k},E) &= \mathrlap{\Sigma^R_s(\vec{k},E)^{*},}\\
    \Sigma^{<}_s(\vec{k},E) &= i \hbar \sum_{s',\vec{k}',\omega} &&{G^{<}_{s'}(\vec{k}',E+\hbar\omega) W^{>}_{s\vec{k},s'\vec{k}',s'\vec{k}',s\vec{k}}(\omega)},\\ 
    \Sigma^{>}_s(\vec{k},E) &= i \hbar \sum_{s',\vec{k}',\omega} &&{G^{>}_{s'}(\vec{k}',E+\hbar\omega) W^{<}_{s\vec{k},s'\vec{k}',s'\vec{k}',s\vec{k}}(\omega)},\\       
\end{eq}
где 
\begin{eq}{Wmatrixelement_definition}
&W^{R/A/</>}_{s\vec{k},s'\vec{k}',s'\vec{k}',s\vec{k}}(\omega)\\
 &\equiv \frac{1}{\cal V}\iint d^D\vec{r} d^D \vec{r'} \psi^{\dagger}_{s'\vec{k}'}(\vec{r})\psi_{s\vec{k}}(\vec{r}) W^{R/A/</>}(\vec{r},\vec{r'},\omega) \psi^{\dagger}_{s\vec{k}}(\vec{r'})\psi_{s'\vec{k}'}(\vec{r'})\\
 &\approx \frac{1}{{\cal V}^2} \sum_{\vec{G}_1,\vec{G}_2} W^{R/A/</>}(\vec{k}'-\vec{k}+\vec{G}_1,\vec{k}'-\vec{k}+\vec{G}_2,\omega)\\
&\times \left\langle s'\vec{k}' \right| e^{i(\vec{k}'-\vec{k}+\vec{G}_1)\vec{r}}\left| s\vec{k} \right\rangle \left\langle s\vec{k} \right| e^{-i(\vec{k}'-\vec{k}+\vec{G}_2)\vec{r}}\left| s'\vec{k'} \right\rangle\\
\end{eq}
--- матричный элемент экранированного кулоновского взаимодействия,\footnote{См. сноску к формуле \eqref{screened_Coulomb_Fourier_representation}.} $\sum_{\vec{k}', \omega} \equiv \int_{-\infty}^{+\infty} \frac{d\omega}{2\pi} \int_{BZ} \frac{d^D \vec{k}'}{\left( 2\pi \right)^D}$.

Пренебрегая слагаемыми с ненулевыми $\vec{G}_1,\vec{G}_2$, выражение для матричного элемента можно упростить:
\begin{eq}{Wmatrixelement}
&W^{R/A/</>}_{s\vec{k},s'\vec{k}',s'\vec{k}',s\vec{k}}(\omega) = W^{R/A/</>}(\vec{k}'-\vec{k},\omega) u^{s s'}_{\vec{k},\vec{k}'},
\end{eq}
где $u^{s s'}_{\vec{k},\vec{k}'} \equiv \frac{1}{{\cal V}^2} \abs{ \left\langle s\vec{k} \right\rvert e^{i(\vec{k}-\vec{k}')\vec{r}}\left\lvert s'\vec{k'} \right\rangle }^2$ --- перекрытие периодических частей блоховских функций $\psi_{s \vec{k}}(\vec{r})$, $\psi_{s' \vec{k}'}(\vec{r})$. При этом выражения \eqref{SigmaGWwithmatrixelements} примут вид
\begin{eq}{SigmaGW}
    \Sigma^R_s(\vec{k},E) &= i \hbar \sum_{s',\vec{q},\omega} &&\left( {G^R_{s'}(\vec{k}+\vec{q},E+\hbar\omega) W^{<}(\vec{q},\omega) u^{s s'}_{\vec{k},\vec{k}+\vec{q}}} \right. \\
    & &&+ \left. {G^{<}_{s'}(\vec{k}+\vec{q},E+\hbar\omega) W^{A}(\vec{q},\omega) u^{s s'}_{\vec{k},\vec{k}+\vec{q}}} \right),\\
    \Sigma^A_s(\vec{k},E) &= \mathrlap{\Sigma^R_s(\vec{k},E)^{*},}\\
    \Sigma^{<}_s(\vec{k},E) &= i \hbar \sum_{s',\vec{q},\omega} &&{G^{<}_{s'}(\vec{k}+\vec{q},E+\hbar\omega) W^{>}(\vec{q},\omega) u^{s s'}_{\vec{k},\vec{k}+\vec{q}}}, \\ 
    \Sigma^{>}_s(\vec{k},E) &= i \hbar \sum_{s',\vec{q},\omega} &&{G^{>}_{s'}(\vec{k}+\vec{q},E+\hbar\omega) W^{<}(\vec{q},\omega) u^{s s'}_{\vec{k},\vec{k}+\vec{q}}}. \\       
\end{eq}

Для поляризационных операторов в $GW$-приближении тоже используются выражения, содержащие лишь слагаемые низшего порядка по $G$ и $W$, а именно:
\begin{eq}{PiGW}
     \Pi^R_{ss'}(\vec{q},\omega) &= -ig\hbar \sum_{\vec{k},E} &&\left({G^R_{s'}(\vec{k}+\vec{q},E+\hbar\omega) G^{<}_s(\vec{k},E) u^{s s'}_{\vec{k},\vec{k}+\vec{q}}} \right. \\
    & &&+ \left. {G^{<}_{s'}(\vec{k}+\vec{q},E+\hbar\omega) G^A_s(\vec{k},E) u^{s s'}_{\vec{k},\vec{k}+\vec{q}}} \right),\\
      \Pi^A_{ss'}(\vec{q},\omega) &= \mathrlap{\Pi^R_{ss'}(\vec{q},\omega)^{*},}\\
      \Pi^{<}_{ss'}(\vec{q},\omega) &= -ig\hbar \sum_{\vec{k},E} &&{G^{<}_{s'}(\vec{k}+\vec{q},E+\hbar\omega) G^{>}_s(\vec{k},E) u^{s s'}_{\vec{k},\vec{k}+\vec{q}}},\\
      \Pi^{>}_{ss'}(\vec{q},\omega) &= -ig\hbar \sum_{\vec{k},E} &&{G^{>}_{s'}(\vec{k}+\vec{q},E+\hbar\omega) G^{<}_s(\vec{k},E) u^{s s'}_{\vec{k},\vec{k}+\vec{q}}}\\
       &= \mathrlap{\Pi^{<}_{s's}(-\vec{q},-\omega).}\\
\end{eq}
Полные поляризуемости получаются суммированием по зонным индексам: $\Pi^{R/A/</>}(\vec{q},\omega) = \sum_{s,s'} \Pi^{R/A/</>}_{ss'}(\vec{q},\omega)$.

В свою очередь, функции Грина и экранированное кулоновское взаимодействие связаны с собственными энергиями и поляризационными операторами уравнениями Дайсона \eqref{interacting_Green's_functions}, \eqref{screened_Coulomb}, которые вместе с уравнениями \eqref{SigmaGW}, \eqref{PiGW} $GW$-приближения образуют замкнутую систему уравнений, позволяющих найти функции Грина взаимодействующих электронов $G$ из функций Грина невзаимодействующих электронов $G^0$ и решёточно-экранированного кулоновского потенциала $V$. Функции Грина, полученные самосогласованным решением этой системы, удовлетворяют законам сохранения~\cite{NEGFhandbook}.

Следует сделать несколько оговорок о применении метода неравновесных функций Грина в изложенной формулировке к моделированию активных сред лазеров. Во-первых, мы рассматривали систему, однородную во времени (т. е. находящуюся в стационарном состоянии), а таковой в отсутствие накачки является равновесная система. Для того, чтобы получить решение, соответствующее системе с инверсией населённостей, строго говоря, нужно добавить в собственную энергию слагаемое, связанное с накачкой. Дабы избежать усложнений, связанных с тем, что строгий учёт накачки добавит в гамильтониан слагаемые, неоднородные в пространстве (для электрической накачки) и/или времени (для оптической накачки), мы будем полагать, что влияние накачки на функции Грина сводится к заданию функции распределения в ферми-дираковском виде $f_s(E) = \left\{ \exp\left[ (E - \mu_s)/k_B T_e \right] + 1 \right\}^{-1}$ с разными квазиуровнями Ферми $\mu_c, \mu_v$ в зоне проводимости и валентной зоне, а изменение спектра квазичастиц под влиянием накачки пренебрежимо мало. Математически это означает замену уравнений \eqref{interacting_Green's_functions} на \eqref{equilibrium_interacting_Green's_functions} (с $f_s(E)$ вместо $f(E)$).

Сделанные допущения, как правило, оправданы на пороге генерации, так как темп накачки в стационарном состоянии определяется темпом рекомбинации и обычно мал по сравнению с темпом межэлектронного рассеяния в отсутствие быстрой рекомбинации за счёт вынужденных переходов. В сильно надпороговом режиме работы лазера эти допущения могут нарушаться: например, возможно <<выжигание дырок>> в распределении носителей~\cite{spectral_hole_burning1, spectral_hole_burning2}. В настоящей диссертации мы интересуемся прежде всего пороговыми характеристиками лазеров и подобные эффекты не рассматриваем.

Во-вторых, электроны взаимодействуют не только с продольным электрическим полем (кулоновским потенциалом, создаваемым другими электронами), но и с поперечным (лазерным излучением, тепловым излучением, нулевыми колебаниями электромагнитного поля). В лазере выше порога генерации поперечное электромагнитное поле вызывает рекомбинацию за счёт вынужденных переходов, темп которой может превышать темп остальных механизмов рекомбинации. До порога генерации взаимодействие электронов с поперечным электромагнитным полем вызывает спонтанные излучательные переходы, темп которых в узкозонных материалах обычно мал по сравнению с темпом безызлучательной (прежде всего, оже-) рекомбинации. Учёт поперечного электромагнитного поля в методе неравновесных функций Грина не представляет принципиальных трудностей, однако несколько усложняет изложение метода, так как, например, вместо кулоновское взаимодействия, являющегося скалярной функцией, появляется фотонный пропагатор, имеющий вид тензора $4\times4$. Так как нас будет интересовать прежде всего темп безызлучательной рекомбинации, необходимый для определения пороговых токов накачки, рассматривать взаимодействие электронов с поперечным электромагнитным полем мы не будем.

В-третьих, на каждой итерации самосогласованной схемы края зон могут сдвигаться по энергии, поэтому если накачка моделируется введением квазиуровней Ферми $\mu_c, \mu_v$, нужно следить за тем, от какой точки они отсчитываются. Например, в расчётах для графена мы будем энергию отсчитывать от дираковской точки, что достигается использованием $\Sigma^R_s(\vec{k},E) - [\Sigma^R_c(0,0) + \Sigma^R_v(0,0)]/2$ вместо $\Sigma^R_s(\vec{k},E)$ в уравнениях Дайсона \eqref{equilibrium_interacting_Green's_functions}.

\subsection{Темп рекомбинации в $GW$-приближении} \label{sec:GW-recombination}
Если подставить в выражение \eqref{Kadanoff-Baym_recombination_rate} для темпа рекомбинации собственные энергии, взятые в $GW$-приближении \eqref{SigmaGW}, окажется, что темп рекомбинации можно выразить через поляризационные операторы:
\begin{eq}{ImPccImPcv-lesser-greater}
    R &= i g \hbar \sum_{s',\vec{k}, E,\vec{q},\omega} &&\left[ G^{<}_c(\vec{k}, E) {G^{>}_{s'}(\vec{k}+\vec{q},E+\hbar\omega) W^{<}(\vec{q},\omega) u^{c s'}_{\vec{k},\vec{k}+\vec{q}}} \right.\\
    & &&\left. - G^{>}_c(\vec{k}, E) {G^{<}_{s'}(\vec{k}+\vec{q},E+\hbar\omega) W^{>}(\vec{q},\omega) u^{c s'}_{\vec{k},\vec{k}+\vec{q}}} \right]\\
    &= \mathrlap{\sum_{s',\vec{q},\omega} \left[ \Pi^{<}_{cs'}(\vec{q},\omega) W^{>}(\vec{q},\omega) - \Pi^{>}_{cs'}(\vec{q},\omega) W^{<}(\vec{q},\omega) \right]}\\
        &= \mathrlap{\sum_{s',s'',s''',\vec{q},\omega} \left[ \Pi^{<}_{cs'}(\vec{q},\omega) \Pi^{>}_{s''s'''}(\vec{q},\omega) - \Pi^{>}_{cs'}(\vec{q},\omega) \Pi^{<}_{s''s'''}(\vec{q},\omega) \right]\abs{W^{R}(\vec{q}, \omega)}^2}\\
        &+\mathrlap{\sum_{s',\vec{q},\omega} \left[ \Pi^{<}_{cs'}(\vec{q},\omega) V^{>}(\vec{q},\omega) - \Pi^{>}_{cs'}(\vec{q},\omega) V^{<}(\vec{q},\omega) \right]\abs{\epsilon^{R}(\vec{q}, \omega)}^{-2}.}\\
\end{eq}

С использованием тождества
\begin{eq}{Fermi-Bose_trick}
     f_s(E + \hbar\omega) \left[ 1 - f_{s'}(E) \right] = n_B(\hbar\omega - \Delta\mu_{ss'}) \left[ f_{s'}(E) - f_{s}(E + \hbar\omega) \right],
\end{eq}
выполняющегося для распределений Ферми-Дирака, <<меньшие>> и <<большие>> поляризационные операторы можно выразить через мнимые части запаздывающих:

\begin{eq}{Pi_fluctuation-dissipation}
   \Pi^{<}_{ss'}(\vec{q},\omega) &= 2i n_B(\hbar\omega + \Delta\mu_{ss'})\Im \Pi^{R}_{ss'}(\vec{q},\omega),\\
   \Pi^{>}_{ss'}(\vec{q},\omega) &= 2i \left[ n_B(\hbar\omega + \Delta\mu_{ss'}) + 1 \right] \Im \Pi^{R}_{ss'}(\vec{q},\omega),\\
\end{eq}
где $n_B(\hbar\omega) = \left[ \exp\left( \hbar\omega/k_B T_{\rm e} \right) - 1 \right]^{-1}$ --- распределение Бозе-Эйнштейна, $\Delta\mu_{ss'} = \mu_s - \mu_{s'}$ --- разность квазиуровней Ферми в зонах $s$ и $s'$. Темп рекомбинации также можно выразить через мнимые части запаздывающих поляризационных операторов:
\begin{eq}{ImPccImPcv-all_bands}
    R &= 4 \sum_{s',s'',s''',\vec{q},\omega} &&\left[ n_B(\hbar\omega + \Delta\mu_{s''s'''}) - n_B(\hbar\omega + \Delta\mu_{cs'}) \right]\\
    & &&\times \Im\Pi^{R}_{cs'}(\vec{q},\omega) \Im\Pi^{R}_{s''s'''}(\vec{q},\omega) \abs{W^{R}(\vec{q}, \omega)}^2\\
        &\ \mathrlap{\begin{aligned}
        +\, 4\sum_{s',\vec{q},\omega} &\left[ n_{\rm phon}(\hbar\omega) - n_B(\hbar\omega + \Delta\mu_{cs'})\right]\\
        &\times \Im \Pi^{R}_{cs'}(\vec{q},\omega) \Im V^{R}(\vec{q},\omega) \abs{\epsilon^{R}(\vec{q}, \omega)}^{-2}.
        \end{aligned}}
\end{eq}

Какие виды рекомбинации учтены в выражении \eqref{ImPccImPcv-all_bands} --- определяется выбранным приближением для собственных энергий. Как видно из формул \eqref{SigmaGW}, \eqref{screened_Coulomb}, \eqref{lattice_screened_Coulomb}, в собственных энергиях мы учли межэлектронное взаимодействие через экранированный кулоновский потенциал, а также взаимодействие электронов с оптическими фононами.

 Межэлектронное взаимодействие приводит к появлению первого слагаемого в выражении \eqref{ImPccImPcv-all_bands}, которое описывает процесс передачи излишка энергии, выделяющейся при рекомбинации, другом электрону или дырке (оже-рекомбинацию). Электрон-фононное взаимодействие приводит к появлению второго слагаемого, описывающего рекомбинацию с испусканием оптических фононов (такой процесс возможен в достаточно узкозонных материалах, в которых ширина запрещённой зоны сравнима с энергиями оптических фононов).

Из всевозможных комбинаций зонных индексов в формуле \eqref{ImPccImPcv-all_bands} многие соответствуют процессам внутризонного рассеяния, а не рекомбинации, и дают нулевой вклад в $R$. Остаются лишь три слагаемых для оже-рекомбинации:
\begin{eq}{ImPccImPcv-Auger}
    R_{\rm Auger} &= 4 \sum_{\vec{q},\omega} &&\left[ n_B(\hbar\omega - \Delta\mu_{cv}) - n_B(\hbar\omega)\right]\\
    & &&\times \Im\Pi^{R}_{cc}(\vec{q},\omega) \Im\Pi^{R}_{vc}(\vec{q},\omega) \abs{W^{R}(\vec{q}, \omega)}^2\\
    &+ 4 \sum_{\vec{q},\omega} &&\left[ n_B(\hbar\omega) - n_B(\hbar\omega + \Delta\mu_{cv}) \right]\\
    & &&\times \Im\Pi^{R}_{cv}(\vec{q},\omega) \Im\Pi^{R}_{vv}(\vec{q},\omega) \abs{W^{R}(\vec{q}, \omega)}^2\\
    &+4 \sum_{\vec{q},\omega} &&\left[ n_B(\hbar\omega - \Delta\mu_{cv}) - n_B(\hbar\omega + \Delta\mu_{cv}) \right]\\
    & &&\times \Im\Pi^{R}_{cv}(\vec{q},\omega) \Im\Pi^{R}_{vc}(\vec{q},\omega) \abs{W^{R}(\vec{q}, \omega)}^2,\\
\end{eq}
и одно слагаемое для фононной рекомбинации:
\begin{eq}{ImPccImPcv-phonon}
    R_{\rm phonon} &= 4 \sum_{\vec{q},\omega} &&\left[n_{\rm phon}(\hbar\omega) -  n_B(\hbar\omega + \Delta\mu_{cv})\right]\\
        & &&\times \Im \Pi^{R}_{cv}(\vec{q},\omega) \Im V^{R}(\vec{q},\omega) \abs{\epsilon^{R}(\vec{q}, \omega)}^{-2}.
\end{eq}

Три вклада в темп оже-рекомбинации \eqref{ImPccImPcv-Auger} соответствуют CHCC процессу (излишек энергии, выделяющийся при рекомбинации, передаётся второму электрону), CHHH процессу (излишек энергии передаётся второй дырке) и CHCH процессу (<<двойная рекомбинация>>, два электрона рекомбинируют с двумя дырками). Последний процесс был бы невозможен в отсутствие уширения спектра носителей из-за большого несохранения энергии, но при учёте уширения спектра он становится возможен, хоть и сильно подавлен по сравнению с CHCC/CHHH процессами.

Физический смысл формулы \eqref{ImPccImPcv-Auger} становится понятен, если учесть, что мнимые части внутри- и межзонных поляризационных операторов $\Im\Pi^{R}_{cc/vv}(\vec{q},\omega), \Im\Pi^{R}_{cv/vc}(\vec{q},\omega)$ описывают поглощение электромагнитного поля, связанное с внутри- и межзонными переходами. Например, для CHCC процесса межзонные переходы, вероятность которых пропорциональна $\Im\Pi^{R}_{vc}(\vec{q},\omega)$, генерируют электрическое поле, которое распространяется в пространстве и времени в виде экранированного кулоновского потенциала $W^{R}(\vec{q}, \omega)$ и поглощается на внутризонных переходах с вероятностью, пропорциональной $\Im\Pi^{R}_{cc}(\vec{q},\omega)$.

\begin{fig}{graphene-diagrams}{graphene-diagrams} Иллюстрация вывода формулы \eqref{ImPccImPcv-Auger} из золотого правила Ферми на языке диаграмм Фейнмана на примере CHCC процесса. Взятие квадрата модуля матричного элемента оже-рекомбинации и суммирование по всем начальным и конечным состояниям с учётом их заселённостей приводит к <<склейке>> диаграммы оже-рекомбинации с комплексно сопряжённой (с противоположным направлением линий), в результате чего для темпа оже-рекомбинации получается диаграмма с поляризационными операторами.
\end{fig}

Таким образом, вычисление темпа рекомбинации в $GW$-приближении сводится к вычислению поляризационных операторов с учётом поправок, связанных с рассеянием носителей. Если не учитывать эти поправки, то формула \eqref{ImPccImPcv-Auger} эквивалентна золотому правилу Ферми без учёта обменных слагаемых и её можно наглядно представить как результат <<склеивания>> диаграмм Фейнмана для оже-рекомбинации (рис.~\ref{fig:graphene-diagrams}). Пренебрежение обменными слагаемыми является артефактом $GW$-приближения, однако оно оправдано для материалов с большим числом неэквивалентных <<разновидностей>> электронов $g$ (так, для графена $g = 4$, так как электроны могут отличаться друг от друга долиной и проекцией спина).

Учёт эффектов межэлектронного взаимодействия, таких как уширение и искривление спектра носителей, может быть произведён путём использования в формуле \eqref{ImPccImPcv-Auger} поляризационных операторов, рассчитанных с учётом этих эффектов. Для этого мы будем использовать в формулах \eqref{ImPccImPcv-Auger}, \eqref{ImPccImPcv-phonon} поляризационные операторы, рассчитанные в самосогласованном $GW$-приближении, как описано в предыдущих разделах. Стоит отметить, что более грубые приближения, такие как приближение времени релаксации, могут быть непригодны для применения в формулах \eqref{ImPccImPcv-Auger}, \eqref{ImPccImPcv-phonon}, так как приближённые $\Im\Pi^{R}_{vc/cv}(\vec{q},\omega)$ могут менять знак не при $\omega = \pm \Delta\mu_{cv}$, что приводит к не положительно определённым подынтегральным выражениям и может давать отрицательный темп рекомбинации.

Основной вклад в темп оже-рекомбинации вносят две области $(\vec{q},\omega)$. Первая из них --- область максимального произведения $\Im\Pi^{R}_{cc/vv}(\vec{q},\omega) \times \Im\Pi^{R}_{cv/vc}(\vec{q},\omega)$, то есть область максимальной <<плотности состояний>> оже-процессов. Например, для графена с законом дисперсии $\epsilon_{s \vec{k}} = \pm \hbar v_0 k$ это область вблизи прямой $\omega = v_0 q$, потому что остальные процессы подавлены законами сохранения.

Вторая область расположена вокруг максимумов экранированного кулоновского взаимодействия $W^{R}(\vec{q}, \omega)$ и фактически соответствует оже-рекомбинации через промежуточное плазмонное состояние. Стоит отметить, что в рассматриваемой модели плазмоны всегда затухают, поглощаясь носителями. Иные процессы, такие как конверсия в электромагнитное излучение или утечка в контакты, нами не рассматриваются. Поэтому вся рекомбинация с испусканием плазмонов в конечном итоге сводится к оже-рекомбинации и введение дополнительного слагаемого, описывающего плазмонную рекомбинацию, не требуется.

Формулы, аналогичные \eqref{ImPccImPcv-Auger}, неоднократно выводились в литературе~\cite{Ziep-Mocker, Yevick-GW_Auger}, однако учёт эффектов межэлектронного взаимодействия в рамках $GW$-приближения до настоящего момента не производился. 

Формула \eqref{ImPccImPcv-phonon} для темпа рекомбинации с испускание оптических фононов также сводится к золотому правилу Ферми~\cite{phonon_through_kappa}, если электрон-фононное взаимодействие описывать гамильтонианом Фрёлиха, в выражении для решёточной диэлектрической проницаемости $\kappa(\omega)$ пренебречь конечными временами жизни фононов, а поляризационные операторы вычислять без поправок, связанных с рассеянием носителей. При использовании точной $\kappa(\omega)$ и поляризационных операторов в $GW$-приближении формула \eqref{ImPccImPcv-phonon} обобщает золотое правило Ферми, допуская некоторую неопределённость энергий квазичастиц, участвующих в рекомбинации.

\subsection{Численное решение уравнений самосогласованного $GW$-приближения} \label{sec:GW-Fourier}
Численное решение уравнений $GW$-приближения в том виде, в котором они приведены в предыдущих разделах, является очень затратным по вычислительным ресурсам. Если мы моделируем двумерный материал и используем сетку из $N$ точек по каждой компоненте квазиволнового вектора и по энергии, все фигурирующие в самосогласованной схеме величины (электронные функции Грина $G$, экранированное кулоновское взаимодействие $W$, собственные энергии $\Sigma$, поляризационные операторы $\Pi$) занимают объём памяти $\mathcal{O}(N^3)$, а вычисление собственных энергий и поляризационных операторов по формулам \eqref{SigmaGW}, \eqref{PiGW} требует $\mathcal{O}(N^6)$ операций. Так как для расчёта темпа рекомбинации требуется точность в энергии порядка 1 мэВ ($k_B T_e \approx 26$ мэВ при 300 К, (1 ТГц) $\times 2\pi \hbar \approx$ 4 мэВ), а ширина зоны проводимости/валентной зоны составляет единицы эВ, получаем $N \sim 10^3$ и порядка $10^{18}$ операций, что довольно много даже по меркам суперкомпьютеров.

К счастью, интегралы в формулах \eqref{SigmaGW}, \eqref{PiGW} имеют вид свёртки и их можно вычислять с помощью быстрого преобразования Фурье~\cite{Godby-space-time}, сократив вычислительные затраты до $\mathcal{O}(N^3 \log N)$ операций. Чтобы показать, что это действительно так, введём тензорные функции Грина и собственные энергии:
\begin{eq}{tensor_Green's_functions_definitions}
     G^{R/A/</>}_{ij}(\vec{k},E) &= \sum_{s} G^{R/A/</>}_{s,ij}(\vec{k},E) = \sum_{s} G^{R/A/</>}_{s}(\vec{k},E) u_{s \vec{k},i}u^{\dagger}_{s \vec{k},j},\\
     \Sigma^{R/A/</>}_{ij}(\vec{k},E) &= \sum_{s} \Sigma^{R/A/</>}_{s}(\vec{k},E) u_{s \vec{k},i}u^{\dagger}_{s \vec{k},j},
\end{eq}
где $u_{s \vec{k},i}$ --- периодическая часть блоховской волны $\left\lvert s\vec{k} \right\rangle$ в некотором ортонормированном базисе.

Теперь формулы \eqref{SigmaGW}, \eqref{PiGW} можно переписать без множителей $u^{s s'}_{\vec{k},\vec{k}+\vec{q}}$:
\begin{eq}{tensorSigmaPiGW}
    \Sigma^{<}_{ij}(\vec{k},E) &= i \hbar \sum_{\vec{q},\omega} {G^{<}_{ij}(\vec{k}+\vec{q},E+\hbar\omega) W^{>}(\vec{q},\omega)}, \\ 
    \Pi^{<}_{ss'}(\vec{q},\omega) &= - i g \hbar \sum_{\vec{k},E,i,j} {G^{<}_{s',ij}(\vec{k}+\vec{q},E+\hbar\omega) G^{>}_{s,ji}(\vec{k},E)}
\end{eq}
(и аналогично для остальных видов собственных энергий и поляризационных операторов). В таком представлении явно видно, что выражения для собственных энергий и поляризационных операторов в $GW$-приближении имеют вид свёрток и после преобразования Фурье превращаются в произведения:
\begin{eq}{FourierSigmaPiGW}
    \Sigma^{<}_{ij}(\vec{r},t) &= i \hbar {G^{<}_{ij}(\vec{r},t) W^{>}(-\vec{r},-t)}, \\ 
    \Pi^{<}_{ss'}(\vec{r},t) &= - i g \hbar \sum_{i,j} {G^{<}_{s',ij}(\vec{r},t) G^{>}_{s,ji}(-\vec{r},-t)}.
\end{eq}

Найдя тензорную собственную энергию, из неё можно получить скалярную по формуле
\begin{eq}{tensor_to_scalar_Sigma}
     \Sigma^{R/A/</>}_{s}(\vec{k},E) = \sum_{i,j} u^{\dagger}_{s \vec{k},i} \Sigma^{R/A/</>}_{ij}(\vec{k},E) u_{s \vec{k},j}.
\end{eq}

Ещё большего снижения вычислительной сложности можно добиться, если моделируемая система изотропна, то есть
\begin{itemize}
\item закон дисперсии изотропен в интересующем диапазоне энергий;
\item перекрытие периодических частей блоховских функций $u^{s s'}_{\vec{k},\vec{k}'}$ зависит только от угла между $\vec{k}, \vec{k'}$, но не от направления каждого из них в отдельности;
\item накачка изотропна (например, моделируется введением ферми-дираковской функции распределения с разными квазиуровнями Ферми для электронов и дырок);
\item анизотропия диэлектрической проницаемости окружения пренебрежимо мала для рассматриваемых волновых векторов $\vec{q}$.
\end{itemize}

Изотропность закона дисперсии и перекрытий периодических частей блоховских функций приближённо выполняется в определённом диапазоне квазиимпульсов, если кристаллическая решётка достаточно симметрична и долины находятся в высокосимметричных точках зоны Бриллюэна. К этому случаю относятся многие прямозонные полупроводники с решёткой типа цинковой обманки, а также графен и родственные материалы, в которых долины расположены в углах гексагональной зоны Бриллюэна. В узкощелевых материалах характерное изменение импульса $\hbar\vec{q}$ в процессах рассеяния и рекомбинации, как правило, мало по сравнению с обратной постоянной решётки, поэтому диэлектрическую проницаемость окружения тоже можно считать изотропной. Наконец, если внутризонная термализация существенно быстрее рекомбинации, распределение носителей в каждой зоне будет ферми-дираковским, то есть изотропным при изотропном законе дисперсии.

Для изотропной системы скалярные $G, W, \Sigma, \Pi$ не зависят от направления волновых векторов, так что при реализации $GW$-схемы через преобразования Фурье приходится иметь дело только либо с изотропными функциями, либо с функциями, зависящими от угла так же, как и проекторы на периодические части блоховских функций
\begin{eq}{projector_definition}
P_{s \vec{k},ij} \equiv u_{s \vec{k},i}u^{\dagger}_{s \vec{k},j}.
\end{eq}

Одно из свойств преобразования Фурье --- сохранение угловой зависимости функции, если она имеет вид $D$-мерной сферической гармоники~\cite{hyperspherical_Fourier}, поэтому если проекторы рассматриваемой системы разложить по $D$-мерным сферическим гармоникам, при преобразованиях Фурье потребуется находить лишь радиальную часть фурье-образа. В двумерном случае это делается с помощью преобразования Ханкеля~\cite{from_Fourier_to_Hankel}:
\begin{eq}{Hankel}
F(k) = 2\pi i^{-m} \int_0^{+\infty} f(r)J_m(k r) r dr,
\end{eq}
где $J_m$ --- функция Бесселя первого рода порядка $m$ (для фурье-прообраза вида $f(r) \exp(i m \varphi_{\vec{r}})$). Аналогичные формулы с функциями Бесселя полуцелых порядков существуют и для трёхмерного случая.

Для вычисления таких интегралов тоже существует эффективный алгоритм (быстрое преобразование Ханкеля), основанный на замене $R = - \ln(r/r_0), K = \ln(k/k_0)$, после которой интеграл \eqref{Hankel} принимает вид свёртки и вычисляется с помощью быстрого преобразования Фурье~\cite{Talman-Hankel}. Таким образом, если в разложении проекторов изотропной системы по $D$-мерным сферическим гармоникам оставить лишь несколько низших гармоник, то вычислительная сложность $GW$-приближения с применением быстрого преобразования Ханкеля оказывается $\mathcal{O}(N^2 \log N)$ по числу операций и $\mathcal{O}(N^2)$ по объёму памяти, что позволяет использовать этот метод на обычных персональных компьютерах.

Стоит отметить, что разложение проекторов на несколько низших гармоник является \emph{точным} для многих модельных гамильтонианов: дираковского~\cite{Dirac_materials-review} (гармоники с $m = 0, \pm 1$), гамильтониана двухслойного графена~\cite{BLG_Hamiltonian} ($m = 0, \pm 1, \pm 2$), гамильтониана BHZ, описывающего квантовые ямы из теллурида кадмия-ртути~\cite{BHZ} ($m = 0, \pm 1$).

При численной реализации самосогласованного $GW$-приближения через преобразования Фурье следует обратить особое внимание на расчёт запаздывающих/опережающих собственных энергий и поляризационных операторов. Если их вычислять непосредственно по формулам \eqref{SigmaGW}, \eqref{PiGW}, то из-за численных погрешностей запаздывающие функции могут оказаться <<не совсем запаздывающими>>, то есть их фурье-образы $\Sigma_{s}^{R}(\vec{r},t), \Pi_{ss'}^{R}(\vec{r},t)$ могут не обращаться строго в 0 при $t < 0$. Это, в свою очередь, может приводить к неправильному поведению функций Грина (например, отрицательной спектральной функции), и проблема будет усугубляться с каждой новой итерацией самосогласованной схемы. Чтобы избежать этого, мы вычисляем эти величины из <<меньших>> и <<больших>>:
\begin{eq}{retarded_from_<>}
\Pi_{ss'}^{R}(\vec{r},t) &= \left[ \Pi_{ss'}^{>}(\vec{r},t) - \Pi_{ss'}^{<}(\vec{r},t) \right] \theta(t),\\
\Sigma_{s}^{R}(\vec{r},t) &= \left[ \Sigma_{s}^{>}(\vec{r},t) - \Sigma_{s}^{<}(\vec{r},t) \right] \theta(t) + \Sigma_{s}^{\delta}(\vec{r},t),\\
\end{eq}
где $\Sigma_{s}^{\delta}(\vec{r},t)$ --- сингулярная во времени часть собственной энергии, в импульсно-энергетическом представлении записывающаяся в виде
\begin{eq}{Sigma_singular}
    \Sigma^{\delta}_s(\vec{k},E) &= i \hbar \sum_{s',\vec{q},\omega} {G^{<}_{s'}(\vec{k}+\vec{q},E+\hbar\omega) W^{\delta}(\vec{q}) u^{s s'}_{\vec{k},\vec{k}+\vec{q}}}.
\end{eq}
Здесь $W^{\delta}(\vec{q}) = V^{R}(\vec{q}, \omega \rightarrow \infty)$ --- сингулярная часть межэлектронного взаимодействия, соответствующая мгновенному кулоновскому взаимодействию.

Этот приём позволяет снизить численные погрешности, но не до нуля, поэтому спектральная функция по-прежнему может принимать отрицательные значения. Чтобы избежать накопления ошибок с каждой новой итерацией и обеспечить сходимость, мы на каждой итерации зануляем отрицательные значения спектральной функции и домножаем её на некоторое число, чтобы обеспечить нормировку на единицу \eqref{spectral_function_normalization}, которая должна выполняться для точной спектральной функции.

Наконец, при итерационном решении самосогласованной системы уравнений $GW$-приближения необходимо обеспечить сходимость итераций. Для этого можно на $i$-й итерации использовать некоторую величину не из $(i-1)$-й итерации, а линейную комбинацию её значений на $(i-1)$-й и $(i-2)$-й итерации. Этот метод применим, если выбранная величина является достаточно плавной функцией от $\vec{q}$ и $\omega$, поэтому, например, спектральная функция не подходит. Мы выбрали в качестве такой величины запаздывающую собственную энергию, то есть на $i$-й итерации мы вычисляем функции Грина \eqref{equilibrium_interacting_Green's_functions} с использованием в качестве запаздывающей собственной энергии
\begin{eq}{Sigma_convergence}
\xi_{\rm convergence} \Sigma_s^{R,i-1}(\vec{k}, E) + (1- \xi_{\rm convergence}) \Sigma_s^{R,i-2}(\vec{k}, E),
\end{eq}
где $\Sigma_s^{R,i-1}(\vec{k}, E)$, $\Sigma_s^{R,i-2}(\vec{k}, E)$ --- запаздывающие собственные энергии, рассчитанные на $(i-1)$-й и $(i-2)$-й итерациях. В наших расчётах использовалось значение параметра сходимости $\xi_{\rm convergence} = 0.5$.

\subsection{Основные результаты раздела} \label{sec:GW-summary}
Подводя итог раздела \ref{sec:NEGF}, приведём краткое описание разработанного нами алгоритма расчёта времени рекомбинации, который будет применён к графену в следующем разделе.

Рассматривается $D$-мерный материал с $g$-кратно вырожденными зонами, законом дисперсии $\epsilon_{s\vec{k}}$ и волновыми функциями электронов $\vec{u}_{s \vec{k}} \exp(i \vec{k} \vec{r})$. Температура носителей --- $T_e$, температура решётки --- $T_{\rm lat}$, квазиуровни Ферми в зоне проводимости и валентной зоне --- $\mu_c$, $\mu_v$, их разность --- $\Delta\mu_{cv}$. Заселённости электронных состояний --- $f_s(E) = \left\{ \exp\left[ (E - \mu_s)/k_B T_e \right] + 1 \right\}^{-1}$, фононных --- $n_{\rm phon}(\hbar \omega) = \left[ \exp\left( \hbar\omega/k_B T_{\rm lat} \right) - 1 \right]^{-1}$. $V^{0}(\vec{q})$ --- $D$-мерный кулоновский потенциал ($= 2\pi e^2/q$ для $D = 2$), $\kappa(\omega)$ --- диэлектрическая проницаемость материала без учёта вклада зоны проводимости и валентной зоны (для $D < 3$ --- диэлектрическая проницаемость окружающего диэлектрика), $n_B(\hbar\omega) = \left[ \exp\left( \hbar\omega/k_B T_{\rm e} \right) - 1 \right]^{-1}$ --- распределение Бозе-Эйнштейна с электронной температурой. $\sum_{\vec{q},\omega}$ --- сокращённое обозначение $\int_{\mathbb{R}^D} \frac{d^D \vec{q}}{(2 \pi)^D} \int_{-\infty}^{+\infty} \frac{d\omega}{2 \pi}$.

Алгоритм:
\begin{enumerate}
\item Вычисляем запаздывающее решёточно-экранированное кулоновское взаимодействие:
\begin{eq}{algorithm-VR}
V^{R}(\vec{q},\omega) &= \frac{V^{0}(\vec{q})}{\kappa(\omega)}.
\end{eq}
\item Вычисляем сингулярную часть экранированного кулоновского взаимодействия:
\begin{eq}{algorithm-Wdelta}
W^{\delta}(\vec{q}) = V^{R}(\vec{q}, \omega \rightarrow \infty).
\end{eq}
\item Вычисляем <<меньшую>> решёточную поляризуемость:
\begin{eq}{algorithm-Pi0<}
\Pi^{0\, <}(\vec{q}, \omega) &= - 2i n_{\rm phon}(\hbar\omega) \frac{\Im \kappa(\omega)}{V^{0}(\vec{q})}.
\end{eq}
\item Начинаем самосогласованную схему с рассмотрения невзаимодействующих носителей. Для этого полагаем запаздывающую собственную энергию нулевой: $\Sigma^R_s(\vec{k},E) = 0$.
\item Вычисляем запаздывающую функцию Грина из уравнения Дайсона:
\begin{eq}{algorithm-GR}
     G^{R}_s(\vec{k}, E) &= \frac{1}{E - \epsilon_{s\vec{k}} - \Sigma^R_s(\vec{k},E) + i0}.
\end{eq}
\item Вычисляем спектральную функцию:
\begin{eq}{algorithm-A}
    {\cal A}_{s}(\vec{k}, E) = -\frac{1}{\pi} \Im G^{R}_{s}(\vec{k}, E).
\end{eq}
\item Если спектральная функция из-за численных погрешностей где-то отрицательна, зануляем эти значения. Также обеспечиваем нормировку на единицу делением ${\cal A}_{s}(\vec{k}, E)$ на $\int_{-\infty}^{+\infty} {\cal A}_{s}(\vec{k}, E) dE$.
\item Вычисляем <<меньшую>> и <<большую>> функции Грина:
\begin{eq}{algorithm-G<>}
     G^{<}_s(\vec{k}, E) &= 2 \pi i f_s(E) {\cal A}_s(\vec{k}, E), \\
     G^{>}_s(\vec{k}, E) &= - 2 \pi i \left[1 - f_s(E)\right] {\cal A}_s(\vec{k}, E). \\
\end{eq}
\item Вычисляем <<меньшую>> и <<большую>> тензорные функции Грина:
\begin{eq}{algorithm-Gij<>}
G^{</>}_{s,ij}(\vec{k},E) = G^{</>}_{s}(\vec{k},E) u_{s \vec{k},i}u^{\dagger}_{s \vec{k},j}.
\end{eq}
\item Переходим к пространственно-временному представлению с помощью быстрых преобразований Фурье и Ханкеля (раздел \ref{sec:GW-Fourier}):
\begin{eq}{algorithm-Gij<>x}
G^{</>}_{s,ij}(\vec{r},t) = \int_{\mathbb{R}^D} \frac{d^D \vec{k}}{(2 \pi)^D} \int_{-\infty}^{+\infty} \frac{dE}{2 \pi \hbar} e^{i (\vec{k} \vec{r} - Et/\hbar)} G^{</>}_{s,ij}(\vec{k},E).
\end{eq}
\item Вычисляем <<меньший>> поляризационный оператор:
\begin{eq}{algorithm-Pi<x}
\Pi^{<}_{ss'}(\vec{r},t) &= -i g \hbar \sum_{i,j} {G^{<}_{s',ij}(\vec{r},t) G^{>}_{s,ji}(-\vec{r},-t)}.
\end{eq}
\item Вычисляем <<больший>> поляризационный оператор:
\begin{eq}{algorithm-Pi>x}
\Pi^{>}_{ss'}(\vec{r},t) &= \Pi^{<}_{s's}(-\vec{r},-t).
\end{eq}
\item Вычисляем запаздывающий поляризационный оператор:
\begin{eq}{algorithm-PiRx}
\Pi_{ss'}^{R}(\vec{r},t) &= \left[ \Pi_{ss'}^{>}(\vec{r},t) - \Pi_{ss'}^{<}(\vec{r},t) \right] \theta(t).
\end{eq}
\item Переходим к импульсно-энергетическому представлению с помощью быстрых преобразований Фурье и Ханкеля (раздел \ref{sec:GW-Fourier}):
\begin{eq}{algorithm-PiR}
\Pi_{ss'}^{R}(\vec{q},\omega) = \int_{\mathbb{R}^D} d^D \vec{r} \int_{-\infty}^{+\infty} dt e^{- i (\vec{q} \vec{r} - \omega t)} \Pi_{ss'}^{R}(\vec{r},t).
\end{eq}
\item Вычисляем диэлектрическую проницаемость, связанную с экранированием носителями:
\begin{eq}{algorithm-epsilonR}
\epsilon^{R}(\vec{q}, \omega) = 1 - V^{R}(\vec{q}, \omega) \sum_{s,s'}\Pi_{ss'}^{R}(\vec{q}, \omega).
\end{eq}
\item Вычисляем запаздывающее экранированное кулоновское взаимодействие:
\begin{eq}{algorithm-WR}
W^{R}(\vec{q}, \omega) = \frac{V^{R}(\vec{q}, \omega)}{\epsilon^{R}(\vec{q}, \omega)}.
\end{eq}
\item Вычисляем <<меньшее>> экранированное кулоновское взаимодействие:
\begin{eq}{algorithm-W<}
W^{<}(\vec{q}, \omega) &= \left[\sum_{s,s'}\Pi_{ss'}^{<}(\vec{q}, \omega) + \Pi^{0\, <}(\vec{q}, \omega)\right]\abs{W^{R}(\vec{q}, \omega)}^2.
\end{eq}
\item Переходим к пространственно-временному представлению с помощью быстрых преобразований Фурье и Ханкеля (раздел \ref{sec:GW-Fourier}):
\begin{eq}{algorithm-W<x}
W^{<}(\vec{r},t) = \int_{\mathbb{R}^D} \frac{d^D \vec{q}}{(2 \pi)^D} \int_{-\infty}^{+\infty} \frac{d\omega}{2 \pi} e^{i (\vec{q} \vec{r} - \omega t)} W^{<}(\vec{q},\omega).
\end{eq}
\item Вычисляем <<большее>> экранированное кулоновское взаимодействие:
\begin{eq}{algorithm-W>x}
W^{>}(\vec{r}, t) &= W^{<}(-\vec{r}, -t).
\end{eq}
\item Вычисляем <<меньшую>> и <<большую>> тензорные собственные энергии:
\begin{eq}{algorithm-Sigma<>x}
\Sigma^{</>}_{ij}(\vec{r},t) &= i \hbar {G^{</>}_{ij}(\vec{r},t) W^{>/<}(-\vec{r},-t)}.
\end{eq}
\item Вычисляем сингулярную часть тензорной собственной энергии:
\begin{eq}{algorithm-Sigmadeltax}
\Sigma^{\delta}_{ij}(\vec{r},t) &= i \hbar {G^{<}_{ij}(\vec{r},t) W^{\delta}(-\vec{r},-t)}.
\end{eq}
\item Вычисляем запаздывающую тензорную собственную энергию:
\begin{eq}{algorithm-SigmaRx}
\Sigma_{s}^{R}(\vec{r},t) &= \left[ \Sigma_{s}^{>}(\vec{r},t) - \Sigma_{s}^{<}(\vec{r},t) \right] \theta(t) + \Sigma_{s}^{\delta}(\vec{r},t).
\end{eq}
\item Вычисляем запаздывающую скалярную собственную энергию:
\begin{eq}{algorithm-Sigma<>x}
\Sigma^{R}_{s}(\vec{k},E) = \sum_{i,j} u^{\dagger}_{s \vec{k},i} \Sigma^{R}_{ij}(\vec{k},E) u_{s \vec{k},j}.
\end{eq}
\item Компенсируем сдвиг начала отсчёта энергии (для графена --- дираковской точки), вычитая $[\Sigma^R_c(0,0) + \Sigma^R_v(0,0)]/2$ из $\Sigma^{R}_{s}(\vec{k},E)$.
\item Подмешиваем к запаздывающей собственной энергии её величину на предыдущей итерации для сходимости итераций:
\begin{eq}{algorithm-Sigma_convergence}
\Sigma_s^{R,i}(\vec{k}, E) \rightarrow \xi_{\rm convergence} \Sigma_s^{R,i}(\vec{k}, E) + (1- \xi_{\rm convergence}) \Sigma_s^{R,i-1}(\vec{k}, E)
\end{eq}
($i$ --- номер текущей итерации, $\xi_{\rm convergence} = 0.5$ --- параметр сходимости).
\item Возвращаемся к шагу 5 с новой собственной энергией, повторяем шаги 5--25 до сходимости.
\item Вычисляем темп оже-рекомбинации:
\begin{eq}{algorithm-RAuger}
    R_{\rm Auger} &= 4 \sum_{\vec{q},\omega} &&\left[ n_B(\hbar\omega - \Delta\mu_{cv}) - n_B(\hbar\omega)\right]\\
    & &&\times \Im\Pi^{R}_{cc}(\vec{q},\omega) \Im\Pi^{R}_{vc}(\vec{q},\omega) \abs{W^{R}(\vec{q}, \omega)}^2\\
    &+ 4 \sum_{\vec{q},\omega} &&\left[ n_B(\hbar\omega) - n_B(\hbar\omega + \Delta\mu_{cv}) \right]\\
    & &&\times \Im\Pi^{R}_{cv}(\vec{q},\omega) \Im\Pi^{R}_{vv}(\vec{q},\omega) \abs{W^{R}(\vec{q}, \omega)}^2\\
    &+4 \sum_{\vec{q},\omega} &&\left[ n_B(\hbar\omega - \Delta\mu_{cv}) - n_B(\hbar\omega + \Delta\mu_{cv}) \right]\\
    & &&\times \Im\Pi^{R}_{cv}(\vec{q},\omega) \Im\Pi^{R}_{vc}(\vec{q},\omega) \abs{W^{R}(\vec{q}, \omega)}^2,\\
\end{eq}
\item Вычисляем темп рекомбинации с испусканием фононов:
\begin{eq}{algorithm-Rphonon}
        R_{\rm phonon} &= 4 \sum_{\vec{q},\omega} &&\left[n_{\rm phon}(\hbar\omega) -  n_B(\hbar\omega + \Delta\mu_{cv})\right]\\
        & &&\times \Im \Pi^{R}_{cv}(\vec{q},\omega) \Im V^{R}(\vec{q},\omega) \abs{\epsilon^{R}(\vec{q}, \omega)}^{-2}.
\end{eq}
\item Вычисляем концентрацию неравновесных носителей:
\begin{eq}{algorithm-nnoneq}
n_{\rm noneq} = -i g \hbar \int_{\mathbb{R}^D} \frac{d^D \vec{k}}{(2 \pi)^D} \int_{-\infty}^{+\infty} \frac{dE}{2 \pi \hbar} [G^{<}_c(\vec{k}, E) - G^{<}_{c, \mathrm{eq}}(\vec{k}, E)].
\end{eq}
Здесь $G^{<}_{c, \mathrm{eq}}(\vec{k}, E)$ --- равновесная функция Грина, рассчитанная по этому же алгоритму, но с одинаковыми химпотенциалами $\mu_c = \mu_v$, подобранными таким образом, чтобы разность концентраций электронов и дырок была такой же, как в моделируемом неравновесном состоянии. В слабонеравновесной ситуации можно не прогонять алгоритм два раза, а вычислить $G^{<}_{c, \mathrm{eq}}(\vec{k}, E)$ из той же спектральной функции, что и $G^{<}_c(\vec{k}, E)$, но с одинаковыми химпотенциалами в функциях распределения $f_s(E)$ (см. раздел \ref{sec:graphene-Hamiltonian}).
\item Вычисляем времена рекомбинации как времена жизни неравновесных носителей:
\begin{eq}{algorithm-tau}
\tau_{\rm Auger} = \frac{n_{\rm noneq}}{R_{\rm Auger}},\quad \tau_{\rm  phonon} = \frac{n_{\rm noneq}}{R_{\rm  phonon}},\quad \tau_{r} = \frac{n_{\rm noneq}}{R_{\rm Auger} + R_{\rm  phonon}}.
\end{eq}
\end{enumerate}
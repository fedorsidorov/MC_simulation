\section{Электронный и фононный спектр квантовых ям из теллурида кадмия-ртути} \label{sec:HgCdTe-basics}
\subsection{Расчёт зонной структуры} \label{sec:HgCdTe-Kane}
Для расчёта зонной структуры и волновых функций электронов в квантовых ямах из теллурида кадмия-ртути мы используем метод, развитый в работе~\cite{Kane_model}. Этот метод ранее использовался для расчёта уровней Ландау в рассмотренных ямах и дал хорошее согласие с магнетоспектроскопическими данными~\cite{Kane_model}.  Ниже мы изложим основные положения данного метода.

Основой рассматриваемого метода является четырёхзонная модель Кейна для \emph{объёмного} теллурида кадмия-ртути с гамильтонианом $8 \times 8$, описывающим зону проводимости и зоны тяжёлых, лёгких и спин-отщеплённых дырок. В гамильтониане учитываются также слагаемые, описывающие изменение зонной структуры в напряжённых структурах (мы полагаем, что ямы выращены на буферном слое теллурида кадмия и постоянные решётки в ямах и барьерах оказываются равны постоянной решётки недеформированного CdTe).

Затем производится поворот базиса, так чтобы ось $z$ была направлена вдоль направления роста гетероструктуры, т. е. [013]. Записанный в новом базисе гамильтониан имеет вид ${\hat H}_0(k_x,k_y) + {\hat H}_1(k_x,k_y) {\hat k}_z + {\hat H}_2(k_x,k_y) {\hat k}_z^2$, где ${\hat H}_0, {\hat H}_1, {\hat H}_2$ --- некоторые матрицы $8 \times 8$ в кейновском базисе, содержащие параметры материала и зависящие от квазиволнового вектора электрона в плоскости ямы.

Переход от объёмного материала к квантовой яме производится с использованием приближения огибающих функций: волновая функция электрона по-прежнему записывается в кейновском базисе, но коэффициенты разложения зависят от координаты $z$ и имеют вид не плоских волн, а некоторых огибающих функций, локализованных в яме и прилегающих областях барьеров. При этом полагается, что гамильтониан внутри ямы либо барьера совпадает с кейновским гамильтонианом соответствующего объёмного материала, из-за чего матрицы ${\hat H}_0, {\hat H}_1, {\hat H}_2$ приобретают зависимость от $z$ и уже не коммутируют с оператором ${\hat k}_z$. В соответствии с выбором порядка операторов, предложенным Бёртом и Фореманом~\cite{k-p_handbook}, гамильтониан гетероструктуры принимает вид 
\begin{eq}{heterostructure_Hamiltonian}
{\hat H}_0(z,k_x,k_y) + {\hat k}_z {\hat H}_L(z,k_x,k_y) + {\hat H}_R(z,k_x,k_y) {\hat k}_z + {\hat k}_z {\hat H}_2(z,k_x,k_y) {\hat k}_z
\end{eq}
(явное выражение для гамильтониана приведено в работе~\cite{Kane_model}).\footnote{В работе~\cite{Kane_model} содержатся некоторые опечатки: знак перед $\frac{18}{25}k_y\{\gamma_2-\gamma_3,k_z\}$ в определении $V$ должен быть положительным, в определении $R$ имеется лишнее слагаемое $+\gamma_2 k_x^2$, а $\{\kappa,k_z\}$ следует заменить на $[\kappa,k_z]$ везде в определениях $\bar{S}_{\pm}$ и $\tilde{S}_{\pm}$.}

Численное решение уравнения Шрёдингера с гамильтонианом \eqref{heterostructure_Hamiltonian} для нахождения огибающих функций, как и в работе~\cite{Zholudev_dissertation}, проводилось с помощью разложения огибающих по плоским волнам, что позволяет легко найти дискретный спектр ямы и избежать проблем с экспоненциально растущими и быстро осциллирующими решениями, которые могут возникать при применении приближений $\vec{k}\cdot\vec{p}$-типа к гетероструктурам~\cite{k-p_spurious_solutions}.

Так как разложение огибающих по дискретному набору плоских волн приводит к периодичности решения, фактически моделировалась сверхрешётка с толщиной барьеров 20 нм, что существенно превышает характерную глубину проникновения волновой функции в барьеры и позволяет считать ямы независимыми. Опытным путём установлено, что для получения достаточно точной зонной структуры хватает базиса из 41 плоской волны.

\subsection{Основные особенности зонной структуры} \label{sec:HgCdTe-bandstructure}
Пример рассчитанной зонной структуры приведён на рис. \ref{fig:HgCdTe-bands}а. Она состоит из ряда подзон, которые мы будем обозначать с1, c2 и т. д. (для подзон зоны проводимости, начиная с самой нижней) и v1, v2 и т. д. (для подзон валентной зоны, начиная с самой верхней). Расстояние между подзонами c$n$ составляет сотни мэВ, поэтому для наших задач достаточно рассматривать только нижнюю подзону зоны проводимости (c1), которая практически изотропна и имеет вид гиперболы (как для дираковского спектра). Подзоны валентной зоны расположены гораздо плотнее на шкале энергий и имеют существенную анизотропию в области больших квазиволновых векторов (величина анизотропии обозначена толщиной линий на рис. \ref{fig:HgCdTe-bands}а). Верхняя из этих подзон, v1, содержит центральный максимум, расположенный в центре зоны Бриллюэна, а также может содержать боковой максимум, имеющий форму окружности (приблизительно, учитывая анизотропию).

\begin{fig}{HgCdTe-bands}{HgCdTe-bands} (а) --- рассчитанная для температуры 77 К зонная структура квантовой ямы \HgCdTe{} толщиной 6 нм, выращенной вдоль направления [013]. Толщина линий показывает величину анизотропии. (б) --- зависимость положений краёв зон от толщины ямы при 4.2 К. Для непрямозонных ям пунктирной линией показано положение центрального максимума валентной зоны. Вставки изображают вид подзон c1, v1 для ям разной толщины.
\end{fig}

В наиболее узких ямах запрещённая зона составляет сотни мэВ (рис. \ref{fig:HgCdTe-bands}б), а подзона v1 имеет только центральный максимум. С увеличением толщины ямы её запрещённая зона сужается, а у подзоны v1 появляется боковой максимум. При некоторой критической толщине $d_c$ яма становится бесщелевым полупроводником. В ещё более широких ямах опять открывается небольшая (порядка 10 мэВ) запрещённая зона, а боковой максимум валентной зоны поднимается и при толщине $d_{\rm in}$ чуть выше критической яма становится непрямозонной. Такая зависимость запрещённой зоны от толщины ямы связана с тем, что при толщине $d_c$ происходит переход от нормальной зонной структуры (CdTe-типа, как в материале барьеров) к инвертированной (HgTe-типа, как в материале ямы), а в наиболее широких ямах зонная структура стремится к объёмному HgTe, который является бесщелевым полупроводником с инвертированной зонной структурой.

В окрестности центра зоны Бриллюэна подзоны c1 и v1 в ямах докритической толщины могут быть приближённо описаны гамильтонианом BHZ (Bernevig--Hughes--Zhang)~\cite{BHZ}:
\begin{eq}{BHZ-Hamiltonian}
{\hat H}_{\rm BHZ} =
\begin{pmatrix}
C - D k^2 + M - B k^2 & A(k_x - i k_y)\\
A(k_x + i k_y) & C - D k^2 - M + B k^2
\end{pmatrix}.
\end{eq}

Если параметры этого гамильтониана найти из сравнения законов дисперсии с рассчитанными в кейновской модели, то $B = 0$, а безразмерная величина $D M/A^2$ обычно не превышает 0.1, поэтому гамильтониан оказывается близок к дираковскому
\begin{eq}{Dirac-Hamiltonian}
{\hat H}_{\rm Dirac} =
\begin{pmatrix}
{E_g}/{2} & \hbar v_0(k_x - i k_y)\\
\hbar v_0(k_x + i k_y) & - {E_g}/{2}
\end{pmatrix}
\end{eq}
 и ямы имеют дираковский спектр носителей. Однако, в отличие от графена, в ямах из теллурида кадмия-ртути <<дираковость>> валентной зоны сохраняется только в диапазоне энергий порядка 20 мэВ, поэтому оже-рекомбинация оказывается разрешена даже без учёта многочастичных эффектов, как будет показано в последующих разделах.

Также стоит отметить температурную зависимость зонной структуры. С увеличением температуры критическая толщина увеличивается и составляет 6.3 нм при 0 К, 7.4 нм при 77 К и 14 нм при 300 К. Поэтому, если рассматривать ямы докритической толщины, у ямы заданной толщины запрещённая зона растёт с температурой.

\subsection{Энергии оптических фононов и диэлектрическая проницаемость} \label{sec:HgCdTe-dielectric}
Диэлектрическая проницаемость теллурида кадмия-ртути зависит от состава, увеличиваясь примерно в 2 раза при переходе от чистого CdTe к чистому HgTe~\cite{HgCdTe-high-frequency_kappa}. Так как электрическое поле распространяется как внутри ямы, так и в барьерах, в расчётах темпа оже-рекомбинации требуется использование некоторой усреднённой диэлектрической проницаемости, которая может зависеть от толщины ямы и волновых функций носителей, участвующих в оже-рекомбинации. Чтобы избежать излишнего усложнения модели, мы будем использовать диэлектрическую проницаемость, найденную из измерений коэффициента отражения объёмного Cd$_{0.2}$Hg$_{0.8}$Te при 300 К~\cite{HgCdTe-phonon_params}. Зависимость диэлектрической проницаемости от температуры мала~\cite{HgCdTe-high-frequency_kappa}, и мы ею пренебрегаем.

В работе~\cite{HgCdTe-phonon_params} диэлектрическая проницаемость представлена в виде многоосцилляторной модели Лоренца:
\begin{eq}{Lorentz}
    \kappa(\omega) = \kappa_{\infty} + \sum_{i=1}^{N} \frac{S_i}{\omega^2_i-\omega(\omega + i\gamma_i)},
\end{eq}
где $\omega_i$, $\gamma_i$ и $S_i$ --- частоты, постоянные затухания и силы осцилляторов $N$ поперечных оптических фононных мод (численные значения приведены в работе~\cite{HgCdTe-phonon_params} и приложении~\ref{appendix:dielectric_functions}), $\kappa_{\infty}$ --- высокочастотная диэлектрическая проницаемость (равная 12 для  Cd$_{0.2}$Hg$_{0.8}$Te~\cite{HgCdTe-high-frequency_kappa}).

\begin{fig}{HgCdTe-permittivity}{HgCdTe-permittivity} (а) --- действительная и мнимая части диэлектрической проницаемости теллурида кадмия-ртути $\kappa(\omega)$ в зависимости от частоты, рассчитанные по формуле \eqref{Lorentz} с параметрами из работы~\cite{HgCdTe-phonon_params}. (б) --- аналогичные графики для $\abs{\kappa(\omega)^{-1}}$ и $(- \Im \kappa(\omega)^{-1})$.
\end{fig}

Диэлектрическая проницаемость теллурида кадмия-ртути, рассчитанная по формуле \eqref{Lorentz} с параметрами из работы~\cite{HgCdTe-phonon_params}, показана на рис.~\ref{fig:HgCdTe-permittivity}. В диапазоне 3--5 ТГц имеется ряд фононных мод, приводящих к сильному оптическому поглощению (большой $\Im \kappa(\omega)$) в этом диапазоне. Постоянные затухания фононных мод, согласно работе~\cite{HgCdTe-phonon_params}, составляют 0.15--0.28 ТГц, поэтому 11 фононных мод, найденных в работе~\cite{HgCdTe-phonon_params}, сливаются всего в два пика в $\Im \kappa(\omega)$.
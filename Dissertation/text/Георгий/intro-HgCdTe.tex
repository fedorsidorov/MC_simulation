В предыдущей главе мы рассмотрели рекомбинационные процессы в материалах с дираковским законом дисперсии и показали, что запрет на оже-рекомбинацию со стороны законов сохранения снимается при учёте многочастичных оже-процессов (с участием более трёх носителей). В этой главе мы рассмотрим случай \emph{приближённо} дираковского закона дисперсии, при котором запрет на оже-рекомбинацию может сниматься либо за счёт отклонения формы зон от дираковской, либо за счёт участия других подзон в оже-рекомбинации. Рассмотрение будет проведено на примере квантовых ям из теллурида кадмия-ртути, для которых имеется отлаженная технология производства~\cite{HgCdTe-technology} и в которых уже наблюдалось вынужденное излучение на частоте 15 ТГц~\cite{HgCdTe-stimulated_emission}. Будут рассматриваться ямы из чистого HgTe с барьерами состава Cd$_{0.7}$Hg$_{0.3}$Te, выращенные вдоль кристаллографического направления [013], так как именно такие ямы используются в работе~\cite{HgCdTe-stimulated_emission} и других работах той же группы экспериментаторов.
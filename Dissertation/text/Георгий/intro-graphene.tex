Как было упомянуто во введении, одной из основных проблем, препятствующих ТГц генерации на межзонных переходах, является оже-рекомбинация, темп которой резко возрастает с уменьшением ширины запрещённой зоны. Однако известно, что в материалах с дираковским законом дисперсии $E = \pm\sqrt{v_0^2 p^2 + E_g^2/4}$ оже-рекомбинация запрещена законами сохранения, если учитывать только процессы низшего порядка (т. е. с участием трёх частиц), поэтому такие материалы могут оказаться перспективными для создания лазерных диодов ТГц диапазона.

Невозможность трёхчастичной оже-рекомбинации затрудняет моделирование лазеров на основе дираковских материалов, так как для расчёта темпа рекомбинации требуется выход за рамки золотого правила Ферми и использование более сложных приближений, учитывающих многочастичные эффекты. В данной главе будет предложено решение этой проблемы, основанное на методе неравновесных функций Грина и $GW$-приближении. Разработанный подход будет применён для расчёта темпа рекомбинации в графене, наиболее известном дираковском материале. Результаты будут сравнены с экспериментальными данными по кинетике носителей в фотовозбуждённом графене, а также будет обсуждаться влияние различных многочастичных эффектов на темп оже-рекомбинации и недостатки использовавшихся ранее подходов, основанных на золотом правиле Ферми.

В конце главы рассчитанные времена рекомбинации в графене будут использованы для оценки пороговых токов и рабочих температур лазерных диодов на основе графена.

Рассмотрение материалов, в которых зоны существенно отклоняются от дираковской формы, а также многозонного случая, будет произведено в главе~\ref{chapter:HgCdTe}. 
\section{Применение метода неравновесных функций Грина для расчёта темпа рекомбинации в графене} \label{sec:graphene}
\subsection{Гамильтониан и функции Грина электронов в графене} \label{sec:graphene-Hamiltonian}
Графен --- это двумерный материал с кристаллической решёткой типа <<пчелиных сот>>, состоящей из двух неэквивалентных треугольных подрешёток. Первая зона Бриллюэна имеет вид шестиугольника, в углах которого расположены так называемые дираковские точки --- точки соприкосновения зоны проводимости и валентной зоны.

Зона проводимости и валентная зона в графене сформированы из $p_z$-орбиталей атомов углерода и хорошо описываются безмассовым дираковским гамильтонианом вплоть до энергий порядка 1 эВ~\cite{graphene_Hamiltonian}:
\begin{eq}{graphene-Hamiltonian}
\hat{h} = \hbar v_0 (\sigma_x \hat{k}_x + \sigma_y \hat{k}_y),
\end{eq}
где $\sigma_x, \sigma_y$ --- матрицы Паули, $\hat{k}_x, \hat{k}_y$ --- компоненты квазиволнового вектора электрона (отсчитанного от угла зоны Бриллюэна), $v_0 = 0.85 \times 10^6$~м/c~\cite{velocity_renormalization} --- дираковская скорость.\footnote{После учёта межэлектронного взаимодействия она перенормируется и принимает значения около $10^6$~м/c или более, в зависимости от степени легирования и температуры.} Базисными функциями, по которым раскладываются периодические части блоховских функций, являются орбитали, локализованные на двух неэквивалентных подрешётках графена.

Из гамильтониана \eqref{graphene-Hamiltonian} можно найти закон дисперсии:
\begin{eq}{dispersion}
    \epsilon_{s\vec{k}} = s \hbar v_0 k,
\end{eq}
и волновые функции:
\begin{eq}{graphene-spinors}
     \psi_{s\vec{k}}(\vec{r}) = \frac{1}{\sqrt{2}}
    \begin{pmatrix}
    e^{-i\varphi_{\vec{k}}/2} \\
    s e^{+i\varphi_{\vec{k}}/2}
    \end{pmatrix}
    e^{i\vec{k}\vec{r}},
\end{eq}
где $s = +1(-1)$ для зоны проводимости (валентной зоны), $\varphi_{\vec{k}}$ --- полярный угол, характеризующий направление квазиволнового вектора $\vec{k}$.

Перекрытие периодических частей блоховских функций равно
\begin{eq}{graphene-overlap}
    u^{s s'}_{\vec{k},\vec{k}'} = \abs{\frac{1}{\sqrt{2}}
    \begin{pmatrix}
    e^{-i\varphi_{\vec{k}}/2} \\
    s e^{+i\varphi_{\vec{k}}/2}
    \end{pmatrix}^{\dagger}
    \cdot
    \frac{1}{\sqrt{2}}
    \begin{pmatrix}
    e^{-i\varphi_{\vec{k}'}/2} \\
    s' e^{+i\varphi_{\vec{k}}'/2}
    \end{pmatrix}}^2
    = \frac{ 1 + ss'\cos\varphi_{\vec{k}\vec{k}'} }{2},
\end{eq}
где $\varphi_{\vec{k}\vec{k}'}$ --- угол между квазиволновыми векторами $\vec{k}$ и $\vec{k}'$.

Гамильтониан \eqref{graphene-Hamiltonian} описывает электроны вблизи одной из дираковских точек. В графене есть две дираковские точки (два неэквивалентных угла зоны Бриллюэна), поэтому электроны в графене имеют двукратное долинное вырождение вдобавок к двукратному вырождению по проекции спина, то есть фактор вырождения $g = 4$. Междолинные переходы мы не рассматриваем ввиду малости кулоновского потенциала для больших передач импульса.

Как видно, графен является изотропным материалом (закон дисперсии $\epsilon_{s\vec{k}}$ и факторы перекрытия $u^{s s'}_{\vec{k},\vec{k}'}$ изотропны), причём проектор на состояние $\left\lvert s\vec{k} \right\rangle$  содержит только круговые гармоники $\exp(im\varphi_{\vec{k}})$ с $m = 0, \pm 1$:
\begin{eq}{graphene-projector}
     {\hat P}_{s \vec{k}} = \frac{1}{2} \begin{pmatrix}
    1 & s e^{-i\varphi_{\vec{k}}} \\
    s e^{+i\varphi_{\vec{k}}} & 1\\
    \end{pmatrix}.
\end{eq}
Это означает, что для расчёта темпа рекомбинации в графене в рамках $GW$- приближения хорошо подходит метод, основанный на быстрых преобразованиях Ханкеля (см. раздел \ref{sec:GW-Fourier}).

В настоящей диссертации мы рассчитаем время рекомбинации в нелегированном графене в пределе слабой инверсии населённостей ($\mu_c = -\mu_v \rightarrow 0$). Для этого при решении уравнений самосогласованного $GW$-приближения (раздел \ref{sec:GW-summary}) мы будем использовать использовать равновесные функции распределения $f_s(E)$ (с $\mu_c = \mu_v = 0$).

При учёте межэлектронного взаимодействия свойства двумерной системы дираковских электронов начинают зависеть от протяжённости дираковского конуса $\Lambda$ (максимальной энергии, до которой простирается дираковский спектр). Например, фермиевская скорость $v_F$ отличается от $v_0$ и логарифмически зависит от $\Lambda$~\cite{graphene_e-e_interactions}. По порядку величины $\Lambda$ совпадает с шириной зоны проводимости в графене, а точное значение следует искать из сравнения рассчитанных физических величин, таких как $v_F$, с экспериментальными данными. Мы будем использовать значение 2.5 эВ, полученное в работе~\cite{velocity_renormalization} (если не оговорено иное).

С учётом вышесказанного, выражения для электронных функций Грина \eqref{algorithm-GR}, \eqref{algorithm-A}, \eqref{algorithm-G<>} примут следующий вид:
\begin{eq}{graphene-Green's_functions}
     G^{R}_s(\vec{k}, E) &= \frac{\theta(\Lambda - \hbar v_0 k)}{E - s \hbar v_0 k - \Sigma^R_s(\vec{k},E) + i0}, \\
     {\cal A}_{s}(\vec{k}, E) &= -\frac{1}{\pi} \Im G^{R}_{s}(\vec{k}, E), \\
    G^{<}_s(\vec{k}, E) &= \frac{2 \pi i {\cal A}_s(\vec{k}, E)} {e^{{E}/{k_B T_e}} + 1}, \\
     G^{>}_s(\vec{k}, E) &= -\frac{2 \pi i {\cal A}_s(\vec{k}, E)} {e^{{-E}/{k_B T_e}} + 1}.
\end{eq}

В рамках самосогласованного $GW$-приближения были рассчитаны поляризационные операторы по алгоритму, описанному в разделе \ref{sec:GW-summary}. При расчёте темпа рекомбинации по формулам \eqref{algorithm-RAuger}, \eqref{algorithm-Rphonon} использовалась ненулевая, но малая разность квазиуровней Ферми ($\Delta\mu_{cv} \ll k_B T_e$).

В выражении \eqref{algorithm-nnoneq} для концентрации неравновесных носителей и неравновесная, и равновесная <<меньшие>> функции Грина были найдены из одной и той же равновесной спектральной функции, но с разными функциями распределения. Другими словами, использовалась формула
\begin{eq}{graphene-nnoneq}
n_{\rm noneq} = g \int_{\mathbb{R}^D} \frac{d^D \vec{k}}{(2 \pi)^D} \int_{-\infty}^{+\infty} dE {\cal A}_c(\vec{k}, E) \left(\frac{1} {e^{ \frac{E - \mu_c}{k_B T_e}} + 1} - \frac{1} {e^{\frac{E}{k_B T_e}} + 1}\right).
\end{eq}

При использовании $\mu_c = -\mu_v \ll k_B T_e$ и темп рекомбинации, и концентрация неравновесных носителей оказываются линейны по $\Delta\mu_{cv}$, поэтому время рекомбинации в пределе слабой инверсии населённостей не зависит от конкретного значения $\Delta\mu_{cv}$.

\subsection{Результаты расчётов времени рекомбинации} \label{sec:graphene-recombination-time}
Используя метод, описанный в разделе \ref{sec:GW-summary}, мы рассчитали темп оже-рекомбинации $R_{\rm Auger}$ в нелегированном графене со слабой инверсией населённостей по формуле \eqref{algorithm-RAuger} и пересчитали его в характерное время жизни неравновесных носителей по формуле $\tau_{\rm Auger} = n_{\rm noneq}/R_{\rm Auger}$, где $n_{\rm noneq} = n - n_{\rm eq}$ --- концентрация неравновесных электронов (она же равна концентрации неравновесных дырок).

Обсуждение результатов начнём со случая, когда окружающий диэлектрик имеет постоянную диэлектрическую проницаемость $\kappa$. В рассматриваемой модели нелегированного графена есть только два безразмерных параметра: константа взаимодействия $\alpha_0 = e^2/(\kappa \hbar v_0)$ и протяжённость дираковского конуса в единицах тепловой энергии, $\Lambda/k_B T_e$. Поэтому из соображений размерности ясно, что обратное время оже-рекомбинации в нелегированном графене пропорционально температуре и некой безразмерной функции от этих двух параметров, $\tau^{-1}_{\rm Auger} = (k_B T_e/\hbar)\times F(\alpha_0,\, \Lambda/k_B T_e)$.

Рассчитанная зависимость $F(\alpha_0,\, \Lambda/k_B T_e)$ показана на рис.~\ref{fig:graphene-dielectric}а. Без учёта экранирования кулоновского взаимодействия электронным газом, а также эффектов размытия и искривления дираковского конуса, темп оже-рекомбинации должен быть пропорционален $\alpha_0^2$, так как в выражение для него входит квадрат кулоновского потенциала. Однако при значениях $\alpha_0 \sim 1$, характерных для графена на типичных диэлектрических подложках, эффекты экранирования ослабляют зависимость $\tau^{-1}_{\rm Auger}(\alpha_0)$ до линейной и даже более слабой, как видно на  рис.~\ref{fig:graphene-dielectric}а.

\begin{fig}{graphene-dielectric}{graphene-dielectric}
(а) --- обезразмеренный темп оже-рекомбинации в слабонеравновесном нелегированном графене в зависимости от константы взаимодействия $\alpha_0 = e^2/(\kappa \hbar v_0)$ и полувысоты дираковского конуса $\Lambda$ в единицах тепловой энергии $k_B T_e$. (б) --- время оже-рекомбинации в зависимости от диэлектрической проницаемости окружения и температуры. Кружки показывают результаты расчётов с учётом частотной зависимости диэлектрических проницаемостей гексагонального нитрида бора и диоксида гафния, при этом также учтена рекомбинация с испусканием фононов диэлектрика. Нефизичное поведение графиков в области малых диэлектрических проницаемостей связано с тем, что при сильном межэлектронном взаимодействии ($\alpha_0 \sim 1$) $GW$-приближение перестаёт работать (см. раздел \ref{sec:graphene-GW_justification}).
\end{fig}

В размерных единицах зависимость времени рекомбинации от диэлектрической проницаемости и температуры приведена на рис.~\ref{fig:graphene-dielectric}б. При комнатной температуре характерные времена составляют около 1--2 пс, а при температурах $T_e = 1000$--3000 К, характерных для экспериментов по наблюдению кинетики носителей в фотовозбуждённом графене~\cite{Gierz2013,Gierz2014,Gierz2015,Gierz2016}, времена рекомбинации оказываются в диапазоне десятков-сотен фемтосекунд.

Теперь рассмотрим влияние оптических фононов в окружающем диэлектрике на темп рекомбинации. Это влияние двоякое. С одной стороны, возникает дополнительный механизм рекомбинации: рекомбинация с испусканием фононов. С другой стороны, наличие фононных мод приводит к появлению частотной дисперсии диэлектрической проницаемости, причём вблизи частот оптических фононов диэлектрическая проницаемость может принимать большие значения, что приводит к усилению экранирования и ослаблению оже-рекомбинации.

Эти эффекты мы рассмотрели на примере графена, инкапсулированного в гексагональный нитрид бора (этот диэлектрик имеет кристаллическую решётку, аналогичную графену, и часто используется для создания высококачественных гетероструктур на основе графена~\cite{high-quality_graphene}), а также графена, инкапсулированного в диоксид гафния (типичный пример high-$\kappa$ диэлектрика). Для расчётов мы использовали диэлектрические проницаемости hBN и HfO$_2$ с учётом частотной дисперсии (приведены в приложении~\ref{appendix:dielectric_functions}). Процессы рекомбинации с участием оптических фононов окружающего диэлектрика учитывались по формуле \eqref{algorithm-Rphonon}, при этом температура решётки диэлектрика $T_{\rm lat}$ полагалась равной температуре носителей в графене $T_e$.

Рассчитанные времена рекомбинации при 300 К приведены на рис.~\ref{fig:graphene-dielectric}б (белый круг --- для графена в нитриде бора, чёрный круг --- для графена в диоксиде гафния). Для нитрида бора темп рекомбинации с испусканием фононов оказывается малым из-за больших энергий оптических фононов (95 и 170 мэВ), поэтому эффект снижения темпа оже-рекомбинации за счёт усиления экранирования вблизи частот оптических фононов оказывается сильнее. В результате рекомбинация оказывается медленнее, чем в случае постоянной диэлектрической проницаемости, равной статическому значению ($\kappa(\omega) = \kappa_0$, показана абсциссами кругов на рис.~\ref{fig:graphene-dielectric}б).\footnote{Данные о частотной дисперсии в HfO$_2$ взяты из работы~\cite{HfO2kappa}, в которой статическая проницаемость равна 14.2. Это меньше обычно используемого значения $\kappa_0 = 25$, поэтому чёрный круг на рис.~\ref{fig:graphene-dielectric}б расположен левее, чем можно было бы ожидать.} Для диоксида гафния, наоборот, наличие фононных мод с низкими энергиями (в диапазоне 23--74 мэВ) приводит к существенному ускорению рекомбинации. Так как наличие таких мод характерно для всех диэлектриков с большой диэлектрической проницаемостью, возможности подавления рекомбинации с помощью выбора окружающего диэлектрика оказываются весьма ограничены.

Рассчитанные времена оже-рекомбинации и рекомбинации с испусканием фононов окружающего диэлектрика свидетельствуют о том, что именно эти механизмы рекомбинации будут определять пороговые токи лазерных диодов на основе графена. Так, рекомбинация с испусканием оптических фононов \emph{самого графена} подавлена за счёт высоких энергий фононов в графене ($\sim$ 150---200 мэВ~\cite{graphene-phonon_dispersion}) и неполярности материала, так что соответствующие времена рекомбинации составляют десятки пикосекунд при комнатной температуре~\cite{Rana-phonons}. Время рекомбинации, связанной со спонтанными излучательными переходами, может быть оценено по формуле~\cite{graphene-radiative}
\begin{eq}{radiative}
\tau_{\text{rad}}\sim\left(\frac{e^2 \sqrt{\kappa}}{\hbar c}\right)^{-1}\frac{c^2}{v^2_0} \frac{\hbar}{k_B T_e}
\end{eq}
и составляет сотни наносекунд при комнатной температуре.

\subsection{Сравнение результатов с экспериментальными данными} \label{sec:graphene-experiment}
Для проверки использованного метода расчёта времени рекомбинации в графене мы сравним рассчитанные значения с экспериментальными данными по кинетике фотовозбуждённых носителей в графене. В серии работ~\cite{Gierz2013,Gierz2014,Gierz2015,Gierz2016} методом фотоэмиссионной спектроскопии с развёрткой по углу и времени (time-resolved ARPES) проводились измерения распределения носителей по энергиям в зависимости от времени, прошедшего с момента освещения графена на SiC лазерным импульсом. Картина эволюции распределения носителей, полученная в этих работах, выглядит следующим образом:
\begin{enumerate}
\item После поглощения импульса накачки формируется неравновесное распределение носителей, которое за время порядка десятков фемтосекунд термализуется отдельно в каждой зоне. Температура носителей при этом составляет около 1000--3000 К (одинаковая для электронов и дырок), квазиуровни Ферми для электронов и дырок различаются на 0.5--1 эВ.
\item За время около 130 фс квазиуровни Ферми для электронов и дырок схлопываются в единый уровень Ферми.
\item Далее происходит постепенное охлаждение равновесного распределения по биэкспоненциальному закону с двумя характерными временами: $\sim 100$~фс и $\sim 1$~пс.
\end{enumerate}
Время жизни инверсии населённостей около 130 фс также подтверждается наблюдением отрицательной оптической проводимости в течение примерно такого же времени в pump-probe экспериментах~\cite{Li-pump-probe}.

Для сравнения теории с экспериментом дадим определения времени рекомбинации $\tau_r$, времени жизни инверсии населённостей $\tau_{\Delta \mu}$ и времени охлаждения носителей $\tau_T$:
\begin{eq}{cooling-recombination-times_definitions}
    \frac{n_{\rm noneq}}{\tau_r} &= -\frac{dn_e}{dt} = -\frac{dn_h}{dt},\\
    \frac{\Delta \mu}{\tau_{\Delta\mu}} &= \frac{\mu_e+\mu_h}{\tau_{\Delta\mu}} = -\frac{d(\mu_e + \mu_h)}{dt},\\
    \frac{T_e - T_{\rm lat}}{\tau_T} &= -\frac{d T_e}{dt}.
\end{eq}
Здесь $n_e, n_h$ --- концентрации электронов и дырок в графене, $\mu_e, \mu_h$ --- их химические потенциалы.

Из наблюдаемого времени схлопывания квазиуровней Ферми $\tau_{\Delta \mu}$ нельзя непосредственно найти время рекомбинации $\tau_r$, так как изменение квазиуровней Ферми может быть связано не только с изменением концентрации носителей, но и с изменением температуры носителей:
\begin{eq}{dndt-partial_derivatives}
    \frac{dn_{e/h}}{dt} &= \frac{\partial n_{e/h}}{\partial \mu_{e/h}}\frac{d\mu_{e/h}}{dt}+\frac{\partial n_{e/h}}{\partial T_{e}}\frac{dT_{e}}{dt}.
\end{eq}
Пренебрегая температурой решётки $T_{\rm lat} \sim 300$~К по сравнению с температурой носителей $T_e \sim 1000$--3000~К, получаем связь между временем рекомбинации, временем жизни инверсии населённостей и временем охлаждения:
\begin{eq}{cooling-recombination}
   \frac{n_{\rm neq}}{\tau_r} = \frac{\left( \dfrac{\partial n_e}{\partial \mu_e}\dfrac{\partial n_h}{\partial T_e} + \dfrac{\partial n_h}{\partial \mu_h}\dfrac{\partial n_e}{\partial T_e} \right) \dfrac{T_e}{\tau_T} + \dfrac{\partial n_e}{\partial \mu_e} \dfrac{\partial n_h}{\partial \mu_h} \dfrac{\Delta \mu}{\tau_{\Delta\mu}}}{\dfrac{\partial n_e}{\partial \mu_e} + \dfrac{\partial n_h}{\partial \mu_h}}.
\end{eq}
Из этого выражения видно, что время рекомбинации на самом деле \emph{короче} наблюдаемого времени жизни инверсии населённостей $\sim 130$~фс, так как охлаждение носителей <<расталкивает>> квазиуровни Ферми, замедляя их схлопывание.

\begin{fig}{graphene-mu-T}{graphene-mu-T}
Соотношение между временем рекомбинации $\tau_r$, временем жизни инверсии населённостей $\tau_{\Delta \mu}$ и временем охлаждения носителей $\tau_T$ в графене. Левый столбец: нелегированный графен; правый столбец: избыток электронов или дырок соответствует энергии Ферми $2 k_B T_e$ в равновесии.
\end{fig}

Соотношение \eqref{cooling-recombination} проиллюстрировано на рис.~\ref{fig:graphene-mu-T}. Для концентраций электронов и дырок использовались обычные выражения для дираковского спектра:
\begin{eq}{graphene-e-h_density}
   n_{e/h}(\mu_{e/h},T_e) = \int_0^\infty \frac{gEdE}{2\pi(\hbar v_0)^2} \frac{1}{e^{(E-\mu_{e/h})/{k_B T_e}}+1},
\end{eq}
так как влияние межэлектронного взаимодействия на плотность состояний невелико и сводится в основном к перенормировке дираковской скорости $v_0$, но при выводе соотношения между $\tau_r$ и $\tau_{\Delta \mu}$ она сокращается. В работах~\cite{Gierz2013,Gierz2014,Gierz2015,Gierz2016} графен, выращенный на SiC, получался p-легированным с энергией Ферми 0.2 эВ, что составляет около $2 k_B T_e$ для $T_e$ = 1000 К, поэтому случай $E_F =  2 k_B T_e$ представлен на рис.~\ref{fig:graphene-mu-T} (правый столбец) наряду с нелегированным графеном (левый столбец).

На двух верхних графиках на рис.~\ref{fig:graphene-mu-T} видно, что слабая инверсия населённостей ($\Delta\mu/k_B T_e \lesssim 1$) может существовать значительно дольше времени рекомбинации.

Точное сравнение рассчитанных времён рекомбинации в графене с экспериментальными данными затрудняется тем, что расчёты проводились для слабой инверсии населённостей ($\Delta\mu < k_B T_e$), а такие значения $\Delta\mu$ выходят за пределы точности эксперимента, поэтому ни $\Delta\mu$, ни $\tau_{\Delta \mu}$ точно не известны. Однако известно, что спустя 100--200 фс после освещения графена лазерным импульсом инверсия населённостей становится меньше экспериментального разрешения, то есть, по крайней мере, $\Delta\mu \lesssim k_B T_e$. Также известно время охлаждения $\tau_T \sim 700$ фс~\cite{Gierz2013} и температура носителей $T_e \sim 1500$ К. Воспользовавшись правой нижней картинкой на рис.~\ref{fig:graphene-mu-T} и полагая, что  $\tau_{\Delta \mu} > 0$ (иначе снова возникла бы заметная инверсия населённостей), получаем оценку сверху на время рекомбинации: $\tau_r \lesssim 0.2 \tau_T = 140$~фс, что согласуется с рассчитанным значением $\tau_r = 80$~фс для графена на SiC при $T_e \sim 1500$ К (диэлектрическая проницаемость SiC с учётом частотной дисперсии приведена в приложении~\ref{appendix:dielectric_functions}).

\subsection{Роль многочастичных эффектов в оже-рекомбинации} \label{sec:graphene-many-body-effects}
Итак, мы рассчитали темп оже-рекомбинации в графене в $GW$-приближении и показали, что полученные числа согласуются с экспериментом, но нужно ли вообще было применять этот метод и нельзя ли обойтись, например, золотым правилом Ферми?

В графене применение золотого правила Ферми затруднено тем, что оже-рекомбинация <<почти запрещена>> законами сохранения. Рассмотрим, например, CHCC-процесс, в котором два электрона с квазиимпульсами $\hbar\vec{k}_1, \hbar\vec{k}_2$ и дырка с квазиимпульсом $\hbar\vec{k}_3$ превращаются в электрон с квазиимпульсом $\hbar\vec{k}_4$. Для такого процесса из сохранения импульса $\hbar\vec{k}_1 + \hbar\vec{k}_2 + \hbar\vec{k}_3 = \hbar\vec{k}_4$ и энергии $\hbar v_0 k_1 + \hbar v_0 k_2 + \hbar v_0 k_3 = \hbar v_0 k_4$ следует, что импульсы всех частиц должны быть сонаправлены (мы будем называть такие процессы коллинеарными). Для CHHH-процесса (две дырки $+$ электрон $\rightarrow$ одна дырка) рассуждения аналогичны в силу электрон-дырочной симметрии в графене. Отсюда следует, что в золотом правиле Ферми интегрирование будет производиться по нулевому фазовому объёму, но в подынтегральном выражении будет стоять дельта-функция от нуля, так как для коллинеарных процессов в графене закон сохранения энергии выполняется автоматически при выполнении сохранения импульса.

Впервые регуляризация выражения для темпа оже-рекомбинации в графене была проделана в работе~\cite{Rana-Auger} с использованием соотношения
\begin{eq}{graphene-golden_rule_regularization}
   \int \frac{d^2 \vec{k}_{123}}{(2\pi)^6} 2\pi \delta(k_1+k_2+k_3-k_4) &= \frac{1}{2} \int_0^{+\infty} \frac{dk_{123}}{(2\pi)^3} \sqrt{k_1k_2k_3k_4}
\end{eq}
($k_4 \equiv \abs{\vec{k}_1 + \vec{k}_2 + \vec{k}_3}$), которое можно вывести, допустив бесконечно малое несохранение энергии~\cite{Tomadin-theory} (связанное с возможностью участия дополнительных носителей, забирающих излишек энергии). При этом темп оже-рекомбинации принимает вид
\begin{eq}{Rana-rate}
    R_{\text{Auger}} &= R_{\text{CHCC}} + R_{\text{CHHH}},\\
    R_{\text{CHCC}} &= \frac{g^2}{2v_0} \left(1-e^{-\Delta\mu/k_B T_e}\right) \int_0^{\infty}{\frac{dk_{1234}}{(2\pi)^4}} \sqrt{k_1k_2k_3k_4} \abs{\frac{M_{fi}}{\hbar}}^2\\
 &\times f_1 f_2 {\bar f}_3 (1-f_4) \times 2\pi\delta(k_1+k_2+k_3-k_4),\\
    R_{\text{CHHH}} &= \frac{g^2}{2v_0} \left(1-e^{-\Delta\mu/k_B T_e}\right) \int_0^{\infty}{\frac{dk_{1234}}{(2\pi)^4}} \sqrt{k_1k_2k_3k_4} \abs{\frac{M_{fi}}{\hbar}}^2\\
 &\times {\bar f}_1 {\bar f}_2 f_3 (1-{\bar f}_4) \times 2\pi\delta(k_1+k_2+k_3-k_4),\\
\end{eq}
где $M_{fi}$ --- матричный элемент кулоновского взаимодействия, усреднённый по допустимым комбинациям проекций спина и долин начальных и конечных частиц (т. е. по тем, которые не требуют переворота спина или междолинных переходов), $f_{1,2,3,4}$ --- заселённости начальных (1, 2, 3) и конечных (4) состояний, ${\bar f}_{1,2,3,4} \equiv 1 - f_{1,2,3,4}$ --- заселённость \emph{дырочных} состояний. Выражение \eqref{Rana-rate} впоследствии использовалось в большинстве работ, в которых рассчитывался темп оже-рекомбинации в графене~\cite{Malic-kinetics,Malic-Auger,Malic2017,Malic-dynamic,Tomadin-theory}.

Формула \eqref{Rana-rate} была бы оправдана, если бы матричный элемент $M_{fi}$ являлся плавной функцией квазиимпульсов носителей, а искривление дираковского конуса было пренебрежимо мало. Плавность матричного элемента выполняется при учёте экранирования в статическом приближении, которое использовалось в оригинальной работе~\cite{Rana-Auger} и в ряде последующих работ по теоретическому описанию кинетики фотовозбуждённых носителей в графене~\cite{Malic-kinetics,Malic-Auger,Malic2017}. Однако при учёте динамического экранирования $M_{fi}$ зануляется для коллинеарных процессов из-за корневой расходимости диэлектрической проницаемости $\epsilon(q,\omega) \propto 1/\sqrt{\omega^2 - v_0^2 q^2}$ в бесстолкновительном приближении~\cite{Finite-temperature_polarizability}. Для решения этой проблемы требуется учёт межэлектронных столкновений в том или ином виде, что было приближённо сделано в работе~\cite{Tomadin-theory} с помощью использования диэлектрической проницаемости для слегка неколлинеарных процессов ($\hbar\omega = \hbar v_0 q \pm  \gamma$, $\gamma \neq 0$) и в работе~\cite{Malic-dynamic} в приближении времени релаксации ($\epsilon(q,v_0 q) \rightarrow \epsilon(q,v_0 q + i\gamma/\hbar)$).

Недостаток таких подходов заключается в том, что хотя диэлектрическая проницаемость графена велика для $\omega \approx v_0 q$, она становится малой вблизи закона дисперсии плазмонов $\omega = \omega_{\rm pl}(q)$. Таким образом, динамическое экранирование может не только \scalebox{0.816}[1]{подавлять} коллинеарные процессы, но и усиливать неколлинеарные (разрешённые при учёте столкновительного уширения спектра носителей), если переданные импульс $\hbar q$ и энергия $\hbar \omega$ в них лежат вблизи закона дисперсии плазмонов. Этот эффект не учитывается приближением времени релаксации, если диэлектрическая проницаемость всё равно берётся для коллинеарных процессов. Использование же <<неколлинеарной>> диэлектрической проницаемости даёт результаты, сильно зависящие от того, какие именно точки $(q, \omega)$ используются. Например, линия $\hbar\omega = \hbar v_0 q +  \gamma$ пересекает закон дисперсии плазмонов, а $\hbar\omega = \hbar v_0 q -  \gamma$ --- нет.

Неколлинеарные процессы, для которых $(q, \omega)$ лежат вблизи закона дисперсии плазмонов, можно было бы трактовать как рекомбинацию с испусканием плазмонов~\cite{Rana-plasmons} (которые затем испытывают внутризонное поглощение, и в итоге получается оже-рекомбинация, идущая через промежуточное плазмонное состояние), но разделение полного темпа оже рекомбинации на вклад коллинеарных процессов и процессов с участием плазмонов достаточно произвольно.

\begin{narrowfig}{graphene-different_methods}{graphene-different_methods}Время оже-рекомбинации в слабонеравновесном нелегированном графене при температуре носителей 3000 К, рассчитанное в различных приближениях. Красные кривые соответствуют $GW$-приближению (сплошная красная кривая) и $GW$-приближению без плазмонных вкладов в темп рекомбинации (пунктирная красная кривая). Остальные кривые рассчитаны по формуле \eqref{Rana-rate} с различными приближениями для экранирования: статическим~\cite{Rana-Auger} (зелёная кривая), динамическим в приближении времени релаксации~\cite{Malic-dynamic} (чёрная кривая), динамическим неколлинеарным~\cite{Tomadin-theory} (синяя кривая). Подробное описание этих приближений см. в тексте.
\end{narrowfig}

Для иллюстрации недостатков использовавшихся ранее подходов к расчёту темпа оже-рекомбинации в графене мы рассчитали время оже-рекомбинации в нелегированном графене в различных приближениях (рис.~\ref{fig:graphene-different_methods}). Мы рассмотрели $GW$-приближение, $GW$-приближение без плазмонных вкладов (для этого была сделана замена $\epsilon(q,\omega) \rightarrow \max\{\epsilon(q,\omega),\epsilon(q,0)\}$ на этапе расчёта темпа рекомбинации из уравнения \eqref{algorithm-RAuger}), а также различные приближения, основанные на уравнении \eqref{Rana-rate} с матричным элементом $M_{fi} = 2\pi e^2/(\kappa q \epsilon_{fi}), q = k1+k3$ (без обменных слагаемых, так как они не учитываются $GW$-приближением). В качестве диэлектрической проницаемости графена использовалась либо статическая проницаемость $\epsilon_{fi} = \epsilon(q,0)$~\cite{Rana-Auger}, либо коллинеарная проницаемость в приближении времени релаксации $\epsilon_{fi} = \epsilon(q,v_0 q + i\gamma/\hbar)$~\cite{Malic-dynamic}, либо неколлинеарная проницаемость $\abs{\epsilon_{fi}}^{-2} = [\abs{\epsilon(q,v_0 q + \gamma/\hbar)}^{-2} + \abs{\epsilon(q,v_0 q - \gamma/\hbar)}^{-2}]/2$~\cite{Tomadin-theory}. Уширение спектра $\gamma  = -2\Im \Sigma^R(k,E)$ вычислялось для носителей с энергией $E = k_B T_e$ и квазиволновым вектором $k = k_B T_e/\hbar v_0$ с использованием собственной энергии, рассчитанной в $GW$-приближении.

Как видно из рис.~\ref{fig:graphene-different_methods}, сильное экранирование коллинеарных процессов и <<антиэкранирование>> неколлинеарных приводит к тому, что средняя величина диэлектрической проницаемости близка к статической, и наиболее хорошо с $GW$-приближением согласуется, как ни странно, наиболее простое, статическое экранирование. Тем не менее, этот факт стоит рассматривать как совпадение, а не как фундаментальное обоснование приближения статического экранирования, и в определённых условиях (например, в условиях сильной инверсии населённостей, когда плазмоны могут усиливаться), приближение статического экранирования необоснованно.

\begin{narrowfig}{graphene-renormalization}{graphene-renormalization}Обезразмеренный темп оже-рекомбинации в слабонеравновесном нелегированном графене в зависимости от константы взаимодействия $\alpha_0 = e^2/(\kappa \hbar v_0)$ и полувысоты дираковского конуса $\Lambda$ в единицах тепловой энергии $k_B T_e$ с учётом и без учёта перенормировки дираковской скорости (сплошные/пунктирные линии).
\end{narrowfig}

Кроме уширения спектра и экранирования кулоновского взаимодействия, на темп оже-рекомбинации может влиять и перенормировка дираковской скорости под действием межэлектронного взаимодействия. Эта перенормировка приводит к логарифмическому заострению дираковского конуса по мере приближения к дираковской точке~\cite{velocity_renormalization} и затрудняет выполнение законов сохранения при оже-рекомбинации. Влияние этого эффекта показано на рис.~\ref{fig:graphene-renormalization}, где сравнивается время оже-рекомбинации в $GW$-приближении и $GW$-приближении с <<выключенным>> искривлением дираковского конуса (это достигается заменой $\Sigma_s^R(k,E) \rightarrow \Sigma_s^R(k,E) - \Re \Sigma_s^R(k,\epsilon_{sk})$ на каждой итерации самосогласованного $GW$-приближения, что приводит к занулению действительной части запаздывающей собственной энергии на дираковском конусе). Как видно из рисунка, при низких температурах искривление дираковского конуса может замедлить оже-рекомбинацию в два раза.

Таким образом, мы показали, что темп оже-рекомбинации в графене очень чувствителен к многочастичным эффектам из-за того, что оже-процессы низшего порядка (с участием трёх частиц) оказываются <<почти запрещены>> законами сохранения. Простые приближения, использовавшиеся другими авторами, пренебрегают искривлением дираковского конуса и учитывают его размытие лишь приближённо, либо не учитывают вовсе. Такие приближения могут давать почти правильный темп оже-рекомбинации в некоторых ситуациях за счёт случайной компенсации ошибок, однако для получения надёжных результатов в широкой области параметров требуется более аккуратный учёт многочастичных эффектов, который возможен при использовании метода неравновесных функций Грина в сочетании с разумным приближением для собственной энергии (например, самосогласованным $GW$-приближением).

\subsection{Обоснование применимости самосогласованного $GW$-приближения} \label{sec:graphene-GW_justification}
В свете полученного нефизичного поведения времён оже-рекомбинации для случая сильного межэлектронного взаимодействия ($\alpha_0 \sim 1$), а именно немонотонной зависимости от диэлектрической проницаемости окружения (рис.~\ref{fig:graphene-dielectric}б) и сильной зависимости от протяжённости дираковского конуса (рис.~\ref{fig:graphene-dielectric}а), представляется необходимым обосновать применимость самосогласованного $GW$-приближения для расчёта темпа оже-рекомбинации в графене и оценить достоверность результатов, полученных для меньших значений $\alpha_0$.

Самосогласованное $GW$-приближение учитывает те диаграммы Фейнмана, которые доминируют в пределе большого числа неэквивалентных <<разновидностей>> электронов $g \gg 1$. В графене $g = 4$, что формально обосновывает допустимость применения самосогласованного $GW$-приближения. Тем не менее, известно, что самосогласованное $GW$-приближение может переоценивать роль уширения спектра и приводить к слишком медленно спадающим <<хвостам>> мнимых частей внутризонных поляризуемостей $\Im\Pi^{R}_{cc}(\vec{q},\omega)$, $\Im\Pi^{R}_{vv}(\vec{q},\omega)$ на больших частотах~\cite{GW_sum_rules_violation}. В соответствии с соотношениями Крамерса-Кронига это приводит к переоценке действительных частей поляризуемостей и нефизичному усилению экранирования при малых $\kappa$. При сильном межэлектронном взаимодействии ($\alpha_0 \sim 1$) этот эффект приводит к большим ошибкам в рассчитанных временах оже-рекомбинации, несмотря на достаточно большое число неэквивалентных <<разновидностей>> электронов $g = 4$. Длина <<хвостов>> $\Im\Pi^{R}_{cc}(\vec{q},\omega)$, $\Im\Pi^{R}_{vv}(\vec{q},\omega)$ зависит от протяжённости дираковского конуса, что приводит к сильной зависимости времени рекомбинации от $\Lambda$ при большой константе взаимодействия $\alpha_0$ (рис.~\ref{fig:graphene-dielectric}а).

Устранение этого артефакта требует либо учёта диаграмм Фейнмана за пределами $GW$-приближения, что сильно увеличит вычислительную сложность, так как интегралы больше не будут иметь вид свёрток, либо использования не полностью самосогласованных вариантов $GW$-приближения. Имеются данные, что не полностью самосогласованные варианты $GW$-приближения могут давать более точные спектральные функции, нежели полностью самосогласованный вариант~\cite{GW_sum_rules_violation}, так как некоторые диаграммы, учитываемые в самосогласованном $GW$-приближении, компенсируются неучтёнными. Однако использование несамосогласованного варианта ($G_0 W_0$-приближения) привело бы к формуле \eqref{algorithm-RAuger} с поляризационными операторами для невзаимодействующих электронов, то есть к вычислению темпа оже-рекомбинации в графене по золотому правилу Ферми. Но в этом случае мы бы получили нулевой темп рекомбинации (см. раздел~\ref{sec:graphene-many-body-effects}). Остаётся частично самосогласованное $G W_0$-приближение, в котором поляризационные операторы, вычисленные в первой итерации, используются и в последующих итерациях. Однако проверить, даст ли оно более точный темп оже-рекомбинации, чем самосогласованное $GW$-приближение, можно лишь сравнением с более точными методами, выходящими за рамки $GW$-приближения. Поэтому мы остановились на самосогласованном $GW$-приближении, несмотря на его недостатки при больших константах взаимодействия $\alpha_0$.

При $\alpha_0 \lesssim 0.5$ эффект переоценки экранирования в самосогласованном $GW$-приближении оказывается мал, и нефизичное поведение времён рекомбинации исчезает. В частности, зависимость от $\Lambda$ оказывается слабой (рис.~\ref{fig:graphene-dielectric}а). Такие значения $\alpha_0$ соответствуют диэлектрической проницаемости $\kappa \gtrsim 5$ и реализуются во многих диэлектриках, используемых в графеновых технологиях, таких как гексагональный нитрид бора и карбид кремния.

Мы пробовали полностью устранить артефакты самосогласованного $GW$-приближения, учитывая эффекты уширения спектра только в мнимых частях поляризационных операторов, но не в действительных (нарушение соотношений Крамерса-Кронига не приводит к <<поломке>> самосогласованной схемы, так как запаздывающая функция Грина всегда продолжает им удовлетворять в силу использования \eqref{algorithm-SigmaRx} для вычисления собственной энергии). Полученные результаты близки к результатам, рассчитанным с полным учётом уширения спектра в поляризационных операторах при $\Lambda \sim 5 k_B T_e$. Это свидетельствует в пользу достоверности результатов, рассчитанных для высоких температур, что подтверждается их согласием с экспериментальными данными (см. раздел~\ref{sec:graphene-experiment}). Что касается времён рекомбинации, рассчитанных для $\kappa \sim 5$ при 300 К, они могут оказаться завышенными примерно в два раза, как видно из рис.~\ref{fig:graphene-dielectric}а.

\section{Оценка пороговых токов лазерных диодов на основе графена} \label{sec:graphene-threshold_currents}
Для оценки пороговых токов лазерных диодов на основе графена требуется знать не только времена рекомбинации, но и пороговые концентрации носителей, необходимые для достижения лазерной генерации. Для этого необходимо рассчитать оптическую проводимость графена, определяющую коэффициент усиления электромагнитного излучения в графене, а также величину оптических потерь в резонаторе.

Оптическая проводимость графена в простейшем приближении состоит из межзонной составляющей $\sigma^{\rm inter}(\omega) = (-e^2/4\hbar)\times \left[ f_c(\hbar\omega/2) - f_v(-\hbar\omega/2) \right]$ и друдевской составляющей $\sigma^{\rm Drude}(\omega) = (n_e+n_h)e^2 \tau_{\rm sc}/[m^{*}(1+\omega^2 \tau_{\rm sc}^2)]$, где эффективная масса $m^{*}$ равна $\mu/v_0^2$ в пределе сильной инверсии населённостей $\mu_c = -\mu_v \equiv \mu \gg k_B T_e$. Однако в графене на друдевскую проводимость влияет рассеяние носителей не только на примесях и фононах, но и друг на друге, так как для непараболического закона дисперсии сохранение суммарного импульса при электрон-электронном рассеянии не означает сохранение суммарного тока. Кроме того, в условиях инверсии населённостей важную роль играет электрон-дырочное рассеяние, не сохраняющее суммарный ток даже для параболического закона дисперсии.

Все эти эффекты учитывались в работе~\cite{graphene-Drude_e-e_scattering}. Согласно результатам этой работы, оптическая проводимость в графене, помещённом в среду с диэлектрической проницаемостью $\kappa = 7$ (например, гексагональный нитрид бора), становится отрицательной (что соответствует оптическому усилению) при $\mu \gtrsim k_B T_e$, а минимальная частота, на которой достижимо усиление, составляет около $0.75 k_B T_e$. На самом деле, пороговые значения квазиуровней Ферми и минимальную частоту генерации следует брать с некоторым запасом, так как оптическое усиление в графене должно превзойти оптические потери за пределами активной среды. Мы будем полагать $\mu_{\rm th} = 1.5 k_B T_e$, $\hbar\omega_{\rm th} = k_B T_e$. При 300 К это соответствует пороговой концентрации $n_{\rm th} \approx 2.5 \times 10^{11}$~см$^{-2}$ и минимальной частоте генерации $f_{\rm th} \approx 6$~ТГц.

Стоит напомнить, что для носителей в нелегированном графене единственной характерной энергией является $k_B T_e$ (не считая слабых эффектов перенормировки дираковской скорости, зависящих от протяжённости дираковского конуса), поэтому безразмерные величины $\mu_{\rm th}/k_B T_e$, $\hbar\omega_{\rm th}/k_B T_e$ являются константами во всём диапазоне температур, в котором доминирует рассеяние носителей друг на друге. 

Времена рекомбинации, рассчитанные в разделе~\ref{sec:graphene-recombination-time}, относятся к случаю $\mu \ll k_B T_e$. Для экстраполяции на случай $\mu \gtrsim k_B T_e$ будем полагать, что время рекомбинации зависит только от полной концентрации носителей, но не от конкретного их распределения по энергиям. Вид этой зависимости можно установить для нелегированного графена из соображений размерности: $\tau_r^{-1} \propto k_B T_e \propto \sqrt{n}$ (см. аргументацию в разделе~\ref{sec:graphene-recombination-time}; зависимостью от $\Lambda/k_B T_e$ из-за перенормировки дираковской скорости пренебрегаем). Тогда, используя полученное в разделе~\ref{sec:graphene-recombination-time} значение $\tau_r \approx 1$~пс для графена в нитриде бора при $n = 8 \times 10^{10}$~см$^{-2}$ (равновесная концентрация при 300 К), получаем:
\begin{eq}{graphene-threshold_current}
J_{\rm th} \approx J_0 \left( \frac{T_e}{T_0} \right)^3,
\end{eq}
где $J_0 \approx$~50 кА/см$^2$ при $T_0 = 300$~К.

Формула \eqref{graphene-threshold_current} даёт значение порогового тока для одного графенового слоя. Максимально достижимый коэффициент усиления, который может создать один графеновый слой, составляет $2\pi \alpha_{\rm 0, vac}/ \lambda_{\rm vac}$, где $\alpha_{\rm 0, vac}$ --- постоянная тонкой структуры, $\lambda_{\rm vac}$ --- длина волны генерируемого излучения в вакууме. В ТГц диапазоне соответствующие значения составляют единицы см$^{-1}$, что обычно меньше или сопоставимо с оптическими потерями в резонаторе. Поэтому для достижения лазерной генерации требуется использование нескольких графеновых слоёв, разделённых диэлектриком. При этом пороговые токи при 300 К будут лежать в диапазоне сотен кА/см$^2$, что внушает скептицизм относительно возможности использования графеновых лазерных диодов для ТГц генерации в непрерывном режиме при комнатной температуре.

Однако кубическая зависимость порогового тока от температуры означает сильное снижение пороговых токов при криогенных температурах. Например, температуры жидкого азота вполне достаточно, чтобы снизить пороговый ток для одного графенового слоя до 800 А/см$^2$. При использовании нескольких слоёв пороговые токи возрастут до единиц кА/см$^2$, попадая в типичный диапазон пороговых токов лазеров, работающих в непрерывном режиме.

Эти пороговые токи рассчитаны для вертикальной инжекции носителей (перпендикулярно графеновым слоям). В литературе также рассматривалась возможность изготовления графеновых лазерных диодов с латеральной инжекцией~\cite{graphene-injection_laser_theory}. В этом случае пороговые токи будут измеряться в А/см и могут быть получены из наших оценок для вертикальной инжекции домножением на длину активной области.

В заключение отметим, что формула \eqref{graphene-threshold_current} применима только для частот генерации $\hbar\omega_{\rm th} < \hbar\omega < 2\mu_{\rm th}$. Если желаемая частота генерации превосходит 2$\mu_{\rm th}/\hbar$, необходимо увеличить концентрацию электронов и дырок до достижения $\Delta\mu_{cv} > \hbar\omega$, чтобы обеспечить инверсию населённостей на данной частоте. При этом пороговый ток возрастёт пропорционально полуторной степени концентрации и при дальнейшем снижении температуры будет стремиться к ненулевому пределу, определяемому желаемой частотой генерации. Этот предел пропорционален кубу частоты и составляет около 3 кА/см$^2$ на один графеновый слой для генерации на 10 ТГц.
\section{Расчёт темпа рекомбинации и пороговых уровней накачки в квантовых ямах из теллурида кадмия-ртути} \label{sec:HgCdTe-results}
\subsection{Оптическая проводимость} \label{sec:HgCdTe-optical_conductivity}
В отсутствие рассеяния носителей оптическая проводимость имеет только межзонную и межподзонную составляющую, которые вычисляются через матричные элементы оператора скорости $v_{\alpha, s s' \vec{k}} \equiv (1/{\cal V})\times\left\langle s \vec{k} \right\rvert {\hat v}_{\alpha} \left\lvert s'\vec{k} \right\rangle$ ($\alpha = x, y, z$) по формуле Кубо-Гринвуда~\cite{Kubo}:
\begin{eq}{bare_interband_conductivity}
\frac{\sigma^{(0)}_{\alpha\beta}(\omega) + \sigma^{(0)*}_{\beta\alpha}(\omega)}{2} = \frac{\pi e^2}{\omega} \sum_{s s' \vec{k}} \left(f_{s \vec{k}} - f_{s' \vec{k}} \right) v_{\alpha, s s' \vec{k}} v_{\beta, s' s \vec{k}}\delta(\epsilon_{s \vec{k}} - \epsilon_{s' \vec{k}} + \hbar\omega).
\end{eq}
Здесь не фигурирует фактор вырождения, так как используется соглашение, что каждая из вырожденных зон имеет свой зонный индекс $s$. Все зоны двукратно вырождены, однако из-за наличия спин-орбитального взаимодействия их уже нельзя разбить на две группы, отличающиеся лишь проекцией спина и в остальном идентичные.

С учётом того, что волновые функции мы вычисляем на полярной сетке, формулу \eqref{bare_interband_conductivity} удобно переписать в виде интегрирования по углу и суммирования по модулю импульса:
\begin{eq}{polar_bare_interband_conductivity}
\frac{\sigma^{(0)}_{\alpha\beta}(\omega) + \sigma^{(0)*}_{\beta\alpha}(\omega)}{2} = \frac{e^2}{\hbar} \int_0^{2\pi}\frac{d\varphi_{\vec{k}}}{2\pi}\sum_{\substack{ss' \\ k:\epsilon_{s' \vec{k}}-\epsilon_{s \vec{k}}=\hbar\omega}}
\left( f_{s \vec{k}}-f_{s' \vec{k}} \right) C_{\alpha\beta,ss'\vec{k}},
\end{eq}
где
\begin{eq}{transition_strength}
C_{\alpha\beta,ss'\vec{k}} =  \pi \frac{\hbar k}{2 \pi \abs{v^{(r)}_{s \vec{k}} - v^{(r)}_{s' \vec{k}}}} \frac{v_{\alpha, s s' \vec{k}} v_{\beta, s' s \vec{k}}}{\epsilon_{s' \vec{k}}-\epsilon_{s \vec{k}}}
\end{eq}
--- безразмерный вклад данного меж(под)зонного перехода в оптическую проводимость, пропорциональный силе осциллятора и <<плотности состояний>> меж(под)зонных переходов, $v^{(r)}_{s \vec{k}} \equiv \partial \epsilon_{s \vec{k}}/ \partial(\hbar k)$ --- радиальная составляющая скорости электрона.

Численный расчёт оптической проводимости из уравнения \eqref{polar_bare_interband_conductivity} проводился следующим образом: вначале для каждого значения угла $\varphi_{\vec{k}}$ и для каждой области монотонности $\epsilon_{s \vec{k}}$ как функции модуля квазиволнового вектора рассматривались всевозможные вертикальные переходы и вычислялись их частота и вклад в проводимость. Затем производилась линейная интерполяция, позволяющая получить непрерывную зависимость оптической проводимости от частоты из дискретных точек. Наконец, проводилось суммирование по всем $\varphi_{\vec{k}}$ и областям монотонности радиального закона дисперсии.

\begin{fig}{HgCdTe-conductivity}{HgCdTe-conductivity} Различные оптические переходы в квантовой яме \HgCdTe{} толщиной 6 нм при 77 К и их вклад в оптическую проводимость без учёта заселённости состояний (величина $C_{\alpha\beta,ss'\vec{k}}$, определённая в тексте). Слева показан вклад в проводимость в плоскости ямы, справа --- в перпендикулярном направлении.
\end{fig}

Различные оптические переходы в квантовых ямах из теллурида кадмия-ртути и <<силы>> этих переходов $C_{\alpha\beta,ss'\vec{k}}$ приведены на рис.~\ref{fig:HgCdTe-conductivity}. Видно, что для достижения усиления в ТГц области усиление, связанное с межзонными переходами c1 $\rightarrow$ v1, должно превысить межподзонное поглощение, связанное с переходами между двумя верхними подзонами валентной зоны v2 $\rightarrow$ v1 (переходами между остальными парами соседних подзон валентной зоны можно пренебречь из-за практического отсутствия свободных состояний). Переходы v2 $\rightarrow$ v1 имеют наименьшую частоту вблизи локального кольцеобразного \emph{минимума} подзоны v1, где дырок мало, однако <<плотность состояний>> межподзонных переходов практически сингулярна, так как примерно там же подзона v2 имеет максимум. Поэтому при совпадении запрещённой зоны с минимальной энергией межподзонного перехода v2 $\rightarrow$ v1 возможно существенное увеличение пороговой частоты лазерной генерации и пороговой концентрации носителей, необходимой для достижения усиления.

Как показывают расчёты, усиления проще достичь для излучения, поляризованного в плоскости ямы, так как межзонный матричный элемент скорости $\left\langle {\rm c1}, \vec{k} \right\rvert {\hat v}_{z} \left\lvert {\rm v1}, \vec{k} \right\rangle$ зануляется при нулевом квазиимпульсе (рис.~\ref{fig:HgCdTe-conductivity}). В плоскости яма анизотропия оптической проводимости незначительна, поэтому далее мы будем оперировать усреднённой проводимостью $\sigma^{(0)}(\omega) \equiv \Re \left[ \sigma^{(0)}_{xx}(\omega) + \sigma^{(0)}_{yy}(\omega) \right]/2$.

При учёте неопределённости энергии носителей, связанной с рассеянием, дельта-функцию в выражении \eqref{bare_interband_conductivity} следует заменить на лоренциан шириной $\gamma = \hbar/\tau_{\rm sc}$, где $\tau_{\rm sc}$ --- характерное время релаксации~\cite{Kubo}. Однако удобнее получить тот же результат, вначале рассчитав $\sigma^{(0)}(\omega)$, а затем свернув её с лоренцианом~\cite{Lorentzian_convolution}:
\begin{eq}{broadened_interband_conductivity}
\sigma^{\rm inter}(\omega) = \int_{-\infty}^{+\infty} \frac{1}{\pi} \frac{\gamma}{(\hbar\omega - \hbar\omega')^2 + \gamma^2} \sigma^{(0)}(\omega') d \hbar \omega'.
\end{eq}

Кроме этого, появляется также внутризонный друдевский вклад в оптическую проводимость:
\begin{eq}{Drude_conductivity}
\sigma^{\rm intra}(\omega) = \frac{e^2 \tau_{\rm sc}}{1 + (\omega \tau_{\rm sc})^2} \left[ \sum_{s \in c,\vec{k}} \frac{f_{s \vec{k}}}{m^e_{s \vec{k}}} + \sum_{s \in v, \vec{k}} \frac{{\bar f}_{s \vec{k}}}{m^h_{s \vec{k}}} \right],
\end{eq}
где первая сумма берётся по подзонам зоны проводимости, вторая --- по подзонам валентной зоны; $m^e_{s \vec{k}} \equiv \left\{ (1/2)\times\left[\partial^2 \epsilon_{s \vec{k}}/\partial (\hbar k_x)^2 + \partial^2 \epsilon_{s \vec{k}}/\partial (\hbar k_y)^2\right]\right\}^{-1}$, $m^h_{s \vec{k}} \equiv - \left\{ (1/2)\times\left[\partial^2 \epsilon_{s \vec{k}}/\partial (\hbar k_x)^2 + \partial^2 \epsilon_{s \vec{k}}/\partial (\hbar k_y)^2\right]\right\}^{-1}$ --- эффективные массы электронов и дырок, ${\bar f}_{s \vec{k}} \equiv 1 - f_{s \vec{k}}$ --- заселённость дырочных состояний.

Транспортное время релаксации $\tau_{\rm sc}$, входящее в выражение \eqref{Drude_conductivity} для друдевской проводимости, вообще говоря, зависит от толщины ямы, температуры и концентрации носителей. Также оно может зависеть от частоты и отличаться от времени релаксации меж(под)зонного тока, определяющего величину <<размытия>> меж(под)зонной проводимости. Чтобы избежать чрезмерного усложнения модели, мы будем использовать постоянное значение $\tau_{\rm sc} = 0.66$~пс ($\gamma = 1$~мэВ) и для друдевской, и для меж(под)зонной проводимости. В модели бесщелевого дираковского спектра, оправданной для ям толщиной $d \approx d_c$, такое значение соответствует подвижности
\begin{eq}{Dirac_mobility}
\mu = \frac{e \tau_{\rm sc}}{m^{*}} = \frac{e \tau_{\rm sc} v_0}{\hbar k_F} = \frac{e v_0}{\gamma \sqrt{2 \pi n}} = 6.3 \times 10^4 \text{~см$^2$/(В$\cdot$c)}
\end{eq}
при концентрации электронов $n = 10^{11}$~см$^{-2}$ и дираковской скорости $v_0 = 5 \times 10^7$ см/с. Экспериментально наблюдаемые подвижности при криогенных температурах в 2--2.5 раза больше~\cite{HgCdTe_mobility}.

Межподзонное, межзонное и друдевское поглощение являются основными источниками оптических потерь внутри самой ямы (решёточное поглощение в самой яме пренебрежимо мало за исключением узкой окрестности частот оптических фононов). Поэтому оптическую проводимость $\sigma(\omega) = \sigma^{\rm inter}(\omega) + \sigma^{\rm intra}(\omega)$ можно использовать для определения \emph{фундаментальных} пределов эффективности (минимальных частот генерации, пороговых токов и концентраций носителей) лазеров на межзонных переходах, использующих в качестве активной среды рассматриваемые квантовые ямы. В реальных лазерах существуют другие механизмы потерь --- решёточное и друдевское поглощение в волноводных слоях и потери на зеркалах, --- которые зависят от конкретной конструкции лазера и будут рассмотрены в разделе~\ref{sec:HgCdTe-discussion}.

\subsection{Пороговые концентрации для достижения оптического усиления и частоты генерации} \label{sec:HgCdTe-concentrations}
Для генерации электромагнитного излучения в лазере на основе квантовых ям из теллурида кадмия-ртути необходимо, чтобы усиление превысило потери на частоте генерации. При условии, что основными источниками оптических потерь являются межподзонное и друдевское поглощение в квантовых ямах (см. обсуждение остальных видов потерь в разделе~\ref{sec:HgCdTe-discussion}), условие начала генерации сводится к требованию отрицательности оптической проводимости на данной частоте. Рассчитав оптическую проводимость как функцию частоты и концентрации носителей (которую мы будем полагать одинаковой для электронов и дырок, т. е. ямы нелегированы), мы можем найти пороговую концентрацию как минимальную концентрацию, при которой на какой-то частоте оптическая проводимость становится отрицательной, $\sigma(\omega) = \sigma^{\rm inter}(\omega) + \sigma^{\rm intra}(\omega) < 0$. Частоту, на которой оптическая проводимость впервые становится отрицательной, будем считать частотой генерации.

При расчётах оптической проводимости мы полагаем функции распределения ферми-дираковскими: $f_{s \vec{k}} = \left\{ \exp\left[ (\epsilon_{s \vec{k}} - \mu_s)/k_B T_e \right] + 1 \right\}^{-1}$, где $\mu_s = \mu_c$ для всех подзон зоны проводимости и $\mu_v$ для всех подзон валентной зоны. Это обосновывается тем, что на термализацию в пределах подзон одной зоны нет таких ограничений со стороны законов сохранения, как на межзонную рекомбинацию, поэтому квазиуровни Ферми в пределах одной зоны выравниваются быстрее. Химпотенциалы $\mu_c, \mu_v$ вычисляются по заданной концентрации электронов и дырок.

\begin{fig}{HgCdTe-gaps_and_concentrations}{HgCdTe-gaps_and_concentrations} (а) --- ширина запрещённой зоны, оптическая запрещённая зона (длинноволновая граница вертикальных переходов) и пороговая частота генерации для ям \HgCdTe{} (с учётом уширения спектра $\gamma = 1$~мэВ). (б) --- пороговые концентрации носителей для достижения оптического усиления в нелегированных ямах без учёта уширения спектра и с $\gamma = 1$~мэВ.
\end{fig}

Рассчитанные частоты генерации для ям различной толщины приведены на рис.~\ref{fig:HgCdTe-gaps_and_concentrations}а. В относительно широкозонных ямах генерация начинается на частоте, соответствующей ширине запрещённой зоны, однако когда ширина запрещённой зоны становится достаточно малой, друдевское и межподзонное поглощение препятствуют генерации на $\omega = E_g/\hbar$, и генерация происходит на более высоких частотах. Из-за этого минимальная частота генерации при комнатной температуре составляет 15 ТГц, а генерация в ТГц области (< 10 ТГц) возможна только при азотной температуре для ям толщиной 5.7--6.5 нм. Минимальная частота генерации при 77 К достигается в яме толщиной 6.5 нм и составляла бы 4.7 ТГц в отсутствие решёточного поглощения. Однако эта частота лежит близко к энергиям оптических фононов, поэтому реальная нижняя граница частоты генерации будет, вероятно, около 6 ТГц (см. раздел~\ref{sec:HgCdTe-discussion}).

Усиление поглощения с уменьшением ширины запрещённой зоны также приводит к росту пороговых концентраций (рис.~\ref{fig:HgCdTe-gaps_and_concentrations}б). В отсутствие рассеяния носителей и друдевского поглощения это происходило бы лишь для непрямозонных ям. Однако при учёте друдевского поглощения и <<размытия>> меж(под)зонной проводимости из-за рассеяния носителей рост пороговых концентраций начинается уже для более узких ям. Так, для используемого нами значения уширения спектра $\gamma = 1$~мэВ существенный рост пороговых концентраций при 77 К наблюдается уже начиная с толщины 6 нм, т. е. как раз для ям, способных генерировать в ТГц диапазоне. Величина уширения спектра $\gamma$ связана с подвижностью носителей и зависит от качества материала, поэтому характеристики ТГц лазеров на основе квантовых ям из теллурида кадмия-ртути будут сильно чувствительны к качеству выращиваемых ям.

\subsection{Пороговые энергии оже-рекомбинации} \label{sec:HgCdTe-threshold_energies}
Исследование процессов рекомбинации в квантовых ямах из теллурида кадмия-ртути начнём с оже-рекомбинации, так как при уменьшении ширины запрещённой зоны до десятков мэВ (соответствующих терагерцовому диапазону) она становится доминирующим механизмом рекомбинации, наряду с плазмонной рекомбинацией, которая будет рассмотрена в главе~\ref{chapter:plasmon}.

В рассматриваемых квантовых ямах, в силу наличия множества подзон размерного квантования, возможно множество различных типов оже-рекомбинации. При пороговых концентрациях носителей электронами заселена только нижняя подзона зоны проводимости (c1), а дырками --- в основном верхняя подзона валентной зоны (v1). Существует два типа оже-рекомбинации с участием этих двух подзон: CHCC (электрон из c1 и дырка из v1 рекомбинируют, передавая излишек энергии другому электрону из c1, который остаётся в своей подзоне) и CHHH (излишек энергии передаётся дырке из v1, которая остаётся в своей подзоне). Однако возможны также процессы, в которых дырка, поглотив излишек энергии, переходит в другую подзону v$n$. Такие процессы мы будем обозначать <<CHHH$n$>>. Аналогичные процессы с вылетом электрона в c2 или другие подзоны невозможны в силу большого расстояния между подзонами зоны проводимости (сотни мэВ).

Так как подзоны c1 и v1 имеют дираковский вид лишь в узком диапазоне энергий, CHCC и CHHH процессы разрешены законами сохранения, однако для них требуются носители с ненулевой кинетической энергией, так как оже-рекомбинация с участием только носителей из экстремумов зон запрещена законами сохранения (по крайней мере, для прямозонных ям). Из-за этого темп оже-рекомбинации $R_{\rm CHCC/CHHH}$ для случая больцмановского распределения носителей приобретает экспоненциальный множитель: $R_{\rm CHCC/CHHH} \propto \exp(-E^{\rm CHCC/CHHH}_{\rm th}/k_B T_e)$, где $E^{\rm CHCC/CHHH}_{\rm th}$ --- так называемые пороговые энергии CHCC и CHHH процессов, равные минимально возможной суммарной кинетической энергии (т. е. энергии, отсчитанной от краёв зон) трёх участвующих носителей. 

Напротив, в CHHH$n$ процессах суммарная кинетическая энергия носителей может быть нулевой, если ширина запрещённой зоны совпадает с расстоянием между подзонами. При таких <<резонансах>> в зонной структуре пороговая энергия $E^{{\rm CHHH}n}_{\rm th}$ обращается в ноль, и темп оже-рекомбинации резко возрастает~\cite{resonant_Auger}.

Таким образом, пороговые энергии позволяют оценочно судить о том, насколько оже-рекомбинация подавлена в том или ином материале, а также сравнивать относительную важность различных видов оже-рекомбинации, зная только закон дисперсии.

Для параболического закона дисперсии пороговые энергии CHCC и CHHH процессов, нормированные на ширину запрещённой зоны, равны $m_e/(m_e + m_h)$ и $m_h/(m_e + m_h)$  соответственно, так что общий порог (минимальный среди всех типов процессов) не превосходит половины запрещённой зоны. В узкозонных полупроводниках с параболическими зонами эффективная масса тяжёлых дырок обычно гораздо больше эффективной массы электронов, что сильно снижает пороговую энергию. Так, для InSb при 300 K $m_e = 0.014 m_0$, $m_h = 0.43 m_0$, $E_g = 0.17 эВ$~\cite{semiconductor_handbook}, что даёт $E_{\rm th} = 0.03 E_g = 5$~ мэВ (при учёте непараболичности зон $E_{\rm th} = 11$~мэВ~\cite{InSb_threshold}).

\begin{fig}{HgCdTe-threshold_energies}{HgCdTe-threshold_energies} (a) --- пороговые энергии оже-рекомбинации в единицах запрещённой зоны в квантовых ямах \HgCdTe{} различной толщины при 4.2 К. Сплошные линии --- расчёты с использованием модели Кейна и приближения огибающих функций, пунктирные линии --- по формуле для параболического спектра с использованием эффективных масс в экстремумах зон. На вставках показаны пороговые оже-процессы для разных ям и пороговые энергии в абсолютных единицах. (б) --- пороговые энергии различных оже-процессов, разделённые на кинетическую энергию начального электрона в пороговом процессе (электронный порог) и дырки (дырочный порог). При наличии двух начальных электронов (дырок) они имеют одинаковые энергии в пороговом процессе.
\end{fig}

Однако в квантовых ямах из теллурида кадмия-ртути зоны не параболические, а приближённо дираковские, из-за чего пороговая энергия может в несколько раз превосходить максимально возможное для параболических зон значение $E_g/2$ (рис.~\ref{fig:HgCdTe-threshold_energies}а). Поэтому, например, для ямы толщиной 5 нм пороговая энергия составляет 30 мэВ при 4.2 К, несмотря на то, что запрещённая зона равна 40 мэВ --- в 4 раза меньше, чем в InSb. Такое значение пороговой энергии означает подавление оже-рекомбинации на два порядка при 77 К по сравнению с гипотетическим материалом с той же шириной запрещённой зоны, но параболическим законом дисперсии и большой электрон-дырочной асимметрией (который бы имел практически нулевой порог при таком же $m_h/m_e$, что и в InSb).

Помимо CHCC и CHHH процессов, существенный вклад в темп оже-рекомбинации может вносить и CHHH2 процесс, так как он имеет примерно такую же пороговую энергию для толщин чуть ниже критической (рис.~\ref{fig:HgCdTe-threshold_energies}б). В более узких ямах его пороговая энергия может обращаться в ноль из-за совпадения расстояния между подзонами v1, v2 с шириной запрещённой зоны, однако это происходит за пределами ТГц области. В ещё более узких ямах начинают играть роль CHHH3 и другие CHHH$n$ процессы, а также возможны процессы с вылетом носителей в континуум, однако для ям с запрещённой зоной в ТГц области такие процессы неэффективны.

В широких ямах пороговая энергия обращается в ноль из-за того, что запрещённая зона становится непрямой и появляется беспороговый CHHH-процесс с участием дырок из бокового максимума валентной зоны (изображён на правой нижней вставке на рис.~\ref{fig:HgCdTe-threshold_energies}а). Из-за большого темпа оже-рекомбинации и непрямозонности широких ям в них не наблюдалось вынужденное излучение~\cite{HgCdTe-narrow_vs_wide} и они не представляют интерес для лазерной генерации, несмотря на то, что их запрещённая зона лежит в ТГц диапазоне.

\begin{fig}{HgCdTe-all-in-one}{HgCdTe-all-in-one} Зонная структура, заселённости электронных (красный цвет) и дырочных (голубой цвет) состояний на пороге генерации, длина волны генерации и соответствующий вертикальный переход (оранжевые стрелки) в квантовых ямах \HgCdTe{} различной толщины при 77 К и 300 К, а также пороговые оже-процессы: CHCC (тёмно-красный цвет), CHHH (зелёный цвет) и CHHH2/CHHH3 (тёмно-синий цвет). Толщина синих линий показывает величину анизотропии зонной структуры. Красная и синяя полоса показывают квазиуровни Ферми для электронов и дырок; их ширина равна тепловой энергии $k_B T_e$. Также приведены пороговые токи генерации (см. раздел~\ref{sec:HgCdTe-currents}).
\end{fig}

Пороговые оже-процессы на примере нескольких разных ям показаны на рис.~\ref{fig:HgCdTe-all-in-one}. Для большей информативности на рисунке также изображены распределения носителей на пороге генерации, соответствующие частоты генерации и пороговые токи (см. раздел~\ref{sec:HgCdTe-currents}).

В свете рассмотренного в главе~\ref{chapter:graphene} влияния уширения спектра носителей на темп оже-рекомбинации в графене возникает вопрос, насколько этот эффект важен в квантовых ямах из теллурида кадмия-ртути. Чтобы ответить на этот вопрос, мы рассчитали пороговые энергии, допуская изменение суммарной энергии носителей в процессе оже-рекомбинации на некоторую величину $\delta E$. Так как в главе~\ref{chapter:graphene} показано, что темп оже-рекомбинации без учёта обменных слагаемых пропорционален произведению мнимых частей внутри- и межзонных поляризуемостей (уравнение \eqref{ImPccImPcv-Auger}), каждая из которых <<размыта>> на величину $\gamma$, можно использовать оценку $\delta E \sim 2 \gamma \sim 2$~мэВ. Расчёты показали, что даже использование $\delta E = 5$~мэВ для ямы толщиной 5 нм при 4.2 К приводит к снижению пороговой энергии менее, чем на 1 мэВ. Таким образом, можно сделать вывод, что влияние уширение спектра на темп оже-рекомбинации в рассматриваемых ямах несущественно.

\subsection{Времена рекомбинации на пороге генерации} \label{sec:HgCdTe-times}
Рассмотрение пороговых энергий оже-рекомбинации даёт качественную оценку эффективности различных видов оже-рекомбинации в рассматриваемом материале, однако для понимания того, насколько квантовые ямы из теллурида кадмия-ртути в действительности пригодны для ТГц генерации, полезно рассчитать пороговые токи лазеров на основе этого материала и сравнить с существующими альтернативами. Для расчёта пороговых токов нам понадобится знать времена рекомбинации \emph{на пороге генерации}, т. е. при пороговых концентрациях неравновесных носителей, вычисленных в разделе~\ref{sec:HgCdTe-concentrations}.

В расчётах мы учитываем следующие виды рекомбинации: оже-рекомбинацию (CHCC, CHHH, CHHH2 и CHHH3 процессы), излучательную рекомбинацию и рекомбинацию с испусканием оптических фононов, которая возможна для ям с шириной запрещённой зоны порядка энергии оптических фононов. Темп рекомбинации Шокли-Рида-Холла зависит от качества образца, поэтому мы его рассматривать не будем. Пренебрежение этим процессом оправдано тем, что в узкозонных материалах на пороге генерации основными механизмами рекомбинации являются оже-рекомбинация и рекомбинация с испусканием плазмонов. Рекомбинация с участием двумерных плазмонов, поддерживаемых электрон-дырочной системой в яме, сильно чувствительна к концентрации носителей, так как при малых концентрациях носителей закон дисперсии плазмонов лежит ниже области межзонных переходов. В зависимости от точного значения концентрации носителей доминирующим механизмом рекомбинации может быть как оже-рекомбинация, так и плазмонная рекомбинация. Поэтому мы рассмотрим плазмонную рекомбинацию отдельно в главе~\ref{chapter:plasmon}.

Так как на пороге генерации носители находятся в основном в подзонах c1 и v1, мы рассматриваем только рекомбинацию между этими подзонами (за исключением CHHH2 и CHHH3 оже-процессов, в которых также участвуют подзоны v2 и v3).

Темп оже-рекомбинации вычислялся по золотому правилу Ферми с помощью интегрирования методом Монте-Карло, как описано в приложении \ref{appendix:Auger}. Кулоновское взаимодействие бралось в \emph{трёхмерном} виде $V(q_{\parallel},z-z') = (2\pi e^2 / q_{\parallel})\exp(-q_{\parallel}\abs{z-z'})$ ($\hbar q_{\parallel}$ --- переданный импульс, $z, z'$ --- координаты вдоль направления роста гетероструктуры). В отличие от материалов атомной толщины вроде графена, в квантовых ямах произведение $q_{\parallel}\abs{z-z'}$ может быть порядка единицы и более, поэтому использование двумерного кулоновского потенциала $V_{\rm 2D}(q_{\parallel}) = (2\pi e^2 / q_{\parallel})$ не оправдано. Также учитывалось экранирование кулоновского потенциала диэлектрической проницаемостью $\kappa(\omega)$, изображённой на рис.~\ref{fig:HgCdTe-permittivity}.

Темп излучательной рекомбинации вычислялся интегрированием межзонной части оптической проводимости $\sigma^{cv}(\omega)$ (связанной с переходами c1 $\rightarrow$ v1) с распределениями Бозе-Эйнштейна (вывод формулы приведён в приложении \ref{appendix:radiative}):
 \begin{eq}{radiative_recombination_rate}
       R_{\rm rad} &= \frac{8}{3\pi}\frac{\sqrt{\kappa_\infty}}{c^3}\int_0^{+\infty}\omega^2 d\omega\left[n_B(\omega-\Delta\mu_{cv})-n_B(\omega)\right]\\
      &\times \Re\frac{\sigma_{xx}^{cv}(\omega)+\sigma_{yy}^{cv}(\omega)+\sigma_{zz}^{cv}(\omega)}{2}.
   \end{eq}
В это выражение входит рекомбинация, связанная со спонтанными переходами, а также с вынужденными переходами под действием теплового излучения. Темп рекомбинации, связанной с вынужденными переходами под воздействием излучения, генерируемого в работающем лазере, определяется мощностью накачки и может значительно превышать темп, рассчитанный по формуле \eqref{radiative_recombination_rate}. Однако мы интересуемся порогом генерации, когда заселённость лазерной моды мала и формула \eqref{radiative_recombination_rate} оправдана.

Темп рекомбинации с испусканием фононов, как было показано в разделе \ref{sec:GW-recombination}, можно записать в виде выражения \eqref{ImPccImPcv-phonon}. При пренебрежении эффектами межэлектронного взаимодействия оно принимает вид золотого правила Ферми, обобщённого на случай конечного времени жизни фононов:
\begin{eq}{phonon_recombination}
       R_{\rm phonon} &= \frac{2 \pi}{\hbar} \sum_{i \in c1, f \in v1} \int_{-\infty}^{+\infty} d\hbar\omega (f_i \bar{f}_f \bar{n}_{\rm phon}(\hbar\omega) - \bar{f}_i f_f n_{\rm phon}(\hbar\omega))\\
       &\times \abs{M_{if}(\omega)}^2 \delta(E_i-E_f-\hbar\omega),\\ 
      \abs{M_{if}(\omega)}^2 &= \int_{-\infty}^{+\infty} dz dz' u_{if}(z)u_{fi}(z')
     \times \left[ - \frac{1}{\pi} \operatorname{Im} \frac{V(q_{\parallel},z-z')}{\kappa(\omega)} \right],
     \end{eq}
где $i$ и $f$ --- начальное и конечное электронные состояния, принадлежащие подзонам c1 и v1 соответственно, $E_{i/f}$ --- их энергии, $f_{i/f}$ --- числа заполнения, $ n_{\rm phon}(\hbar \omega) = \left[ \exp\left( \hbar\omega/k_B T_{\rm lat} \right) - 1 \right]^{-1}$ --- числа заполнения фононов, $\bar{f}_{i/f} = 1 - f_{i/f}$, $\bar{n}_{\rm ph} = n_{\rm ph} + 1$. Решёточную температуру $T_{\rm lat}$ мы будем полагать равной электронной, если не оговорено иное.

Из-за учёта конечного времени жизни фононов матричный элемент электрон-фононного взаимодействия $M_{if}(\omega)$ приобретает зависимость от энергии фонона, так как она уже не связана однозначно с энергиями начального и конечного состояний законом сохранения энергии. Сила электрон-фононного взаимодействия оказывается пропорциональна мнимой части решёточно-экранированного кулоновского взаимодействия~\cite{phonon_through_kappa}. Также в $M_{if}(\omega)$ входит перекрытие волновых функций начального и конечного состояния $u_{if}(z) = \int_{\mathbb{R}^2} d\vec{r}_{\parallel} \psi_{f}^{\dagger}(\vec{r}_{\parallel},z)e^{-i\vec{q}_{\parallel}\vec{r}_{\parallel}}\psi_{i}(\vec{r}_{\parallel},z)$. Численный расчёт темпа фононной рекомбинации проводился с помощью интегрирования методом Монте-Карло, аналогично вычислению темпа оже-рекомбинации.

\begin{fig}{HgCdTe-recombination_times}{HgCdTe-recombination_times} Времена жизни неравновесных носителей на пороге лазерной генерации в квантовых ямах \HgCdTe{}, связанные с различными процессами рекомбинации. (а) --- оже-, фононная и излучательная рекомбинация. (б) --- оже-рекомбинация, разбитая на СHCC и CHHH процессы (сплошные линии) и CHHH$n$ процессы (пунктирные линии). На графиках также показаны стандартные отклонения результатов, полученных интегрированием методом Монте-Карло. 
\end{fig}

Рассчитанные времена рекомбинации при пороговых концентрациях носителей приведены на рис.~\ref{fig:HgCdTe-recombination_times}а. Основным механизмом рекомбинации является оже-рекомбинация, как и следовало ожидать для таких величин запрещённой зоны и концентрации носителей. Времена излучательной рекомбинации имеют порядок сотен наносекунд для всех ям с терагерцовой запрещённой зоной. Стоит отметить, что это совсем не означает неэффективность лазеров на этих ямах, так как лазерная генерация происходит за счёт \emph{вынужденных} переходов, темп которых зависит от интенсивности накачки и может значительно превосходить темп излучательной рекомбинации на пороге \eqref{radiative_recombination_rate}, обусловленной в основном спонтанными переходами. Фононная рекомбинация даже для наиболее узкозонных ям как минимум на порядок медленнее, чем оже-рекомбинация. При ширине запрещённой зоны, превосходящей энергии оптических фононов, темп фононной рекомбинации сильно падает, но не до нуля, так как из-за конечного времени жизни фононов допустимо некоторое несохранение энергии при рекомбинации. Неожиданное замедление рекомбинации с температурой для фононной рекомбинации и оже-рекомбинации в широких ямах связано с температурной зависимостью зонной структуры: с ростом температуры для ям с толщиной ниже критической запрещённая зона увеличивается.

В темп оже-рекомбинации вносят вклад как CHCC и CHHH процессы, так и CHHH$n$ процессы с участием различных подзон валентной зоны (рис.~\ref{fig:HgCdTe-recombination_times}б), хотя для широких ям и при высоких температурах роль  CHHH$n$ процессов снижается из-за того, что низкая пороговая энергия уже не может компенсировать меньший объём фазового пространства по сравнению с CHCC/CHHH процессами.

Для широких ям время рекомбинации стремится к постоянному значению в диапазоне десятых долей -- единиц пс, соответствующему времени рекомбинации в объёмном теллуриде ртути, который является бесщелевым полупроводником и поэтому обладает нулевой пороговой энергией оже-рекомбинации. При 77 К для ямы толщиной 5.7 нм (частота генерации 9.4 ТГц) время рекомбинации составляет 140 пс, что соответствует подавлению оже-рекомбинации примерно на два порядка по сравнению с объёмным теллуридом ртути. Это подтверждает наши выводы о возможности подавления оже-рекомбинации на два порядка, сделанные в разделе \ref{sec:HgCdTe-threshold_energies} на основе анализа пороговых энергий.

\subsection{Пороговые уровни накачки и сравнение с экспериментом} \label{sec:HgCdTe-currents}
Зная пороговые концентрации неравновесных носителей $n_{\rm th, noneq}$ и времена рекомбинации $\tau_r$, мы можем найти пороговые плотности тока $J_{\rm th}$, необходимые для достижения генерации в лазерных диодах на основе квантовых ям теллурида кадмия-ртути, по формуле $J_{\rm th} = e \alpha^{-1}_{\rm cap} n_{\rm th, noneq}/\tau_r$, где $\alpha_{\rm cap}$ --- вероятность захвата носителей в квантовую яму. Также можно рассчитать пороговые интенсивности оптической накачки по формуле $I_{\rm th} = \alpha_{\rm abs}^{-1} \hbar \omega_{\rm pump} n_{\rm th, noneq}/\tau_r$, где $\alpha_{\rm abs}$ --- доля энергии излучения накачки, поглощённая на межзонных переходах в яме, $\hbar \omega_{\rm pump}$ --- энергия фотона накачки.

\begin{fig}{HgCdTe-current-frequency}{HgCdTe-current-frequency} Зависимость пороговой плотности тока электрической накачки и пороговой интенсивности оптической накачки, необходимых для достижения оптического усиления в квантовых ямах \HgCdTe{} различной толщины, от пороговых частот и длин волн генерации в этих ямах. (а) --- температура решётки 4.2 К. (б) --- температура решётки равна температуре носителей. Зелёные точки --- экспериментальные данные из работы~\cite{HgCdTe-stimulated_emission}; треугольники, квадраты и окружности --- экспериментальные данные по квантово-каскадным~\cite{QCL_5.1um_300K,QCL_7.66um_300K,QCL_15.1um_300K,QCL_17.8um_77K,QCL_24.5um_77K}, межзонным каскадным лазерам~\cite{ICL_3.67um_300K,ICL_6um_300K,ICL_7um_300K,ICL_10.4um_77K} и лазерным диодам на солях свинца~\cite{LeadSalt_DoubleHeterostructure,Lead_Salt_LaserCharacteristics}\protect\footnotemark (синий цвет обозначает температуру 77 К, красный --- 300 К). Красным и зелёным закрашены области сильного решёточного поглощения (Reststrahlen band) в Hg$_{1-x}$Cd$_{x}$Te и GaAs соответственно. На вставках изображены пороговые оже-процессы в различных ямах.
\end{fig}

Для того, чтобы исследовать фундаментальные ограничения на минимальные пороговые токи лазерных диодов на основе квантовых ям теллурида кадмия-ртути, мы рассчитали пороговые токи в предположении $\alpha_{\rm cap} = 1$ и построили их зависимость от пороговой частоты генерации (рис.~\ref{fig:HgCdTe-current-frequency}). На этих же графиках показаны пороговые интенсивности оптической накачки, рассчитанные в предположении $\alpha_{\rm abs} = 0.4$ \% для накачки на длине волны 2.15 мкм, как в работе~\cite{HgCdTe-stimulated_emission}. Значение $\alpha_{\rm abs}$ рассчитано методом трансфер-матриц для структур, использованных в работе~\cite{HgCdTe-stimulated_emission}. Точное значение осциллирующим образом зависит от толщины подложки, поэтому была выбрана полусумма $\alpha_{\rm abs}$ в максимумах и минимумах осцилляций.

\footnotetext{В работе \cite{LeadSalt_DoubleHeterostructure} длина волны генерации составляет 20 мкм при 20 К. На графике предполагается, что при 77 К она снизится до 15 мкм (на основании температурной зависимости для Pb$_{1-x}$Sn$_x$Se~\cite{Lead_Salt_LaserCharacteristics}).}

В работе~\cite{HgCdTe-stimulated_emission} исследовалось вынужденное излучение из квантовых ям \HgCdTe{} при оптической накачке. Экспериментальные данные о пороговой интенсивности накачки и частотах генерации для различных ям показаны зелёными точками на рис.~\ref{fig:HgCdTe-current-frequency}а. Температура носителей в этих экспериментах неизвестна, однако из рис.~\ref{fig:HgCdTe-current-frequency}а видно, что экспериментальные данные достаточно хорошо согласуются с теоретическими расчётами для $T_e \sim 80$ К, что свидетельствует о том, что развитая в данной работе теория даёт приблизительно правильную скорость роста порогового тока с уменьшением ширины запрещённой зоны и разумные значения самих пороговых токов.

При 77 К для наиболее узких ям пороговые токи близки к соответствующим величинам для межзонных каскадных лазеров (синий квадрат на рис.~\ref{fig:HgCdTe-current-frequency}б) при той же температуре и гораздо ниже пороговых токов квантово-каскадных лазеров (синие треугольники на рис.~\ref{fig:HgCdTe-current-frequency}б) и лазерных диодов на солях свинца (синие окружности на рис.~\ref{fig:HgCdTe-current-frequency}б). Большие пороговые токи в квантово-каскадных лазерах связаны с быстрой фононной рекомбинацией, а в лазерах на солях свинца --- с технологическими проблемами (большим остаточным легированием~\cite{lead_salt_problems}).

С увеличением ширины ямы пороговые токи $J_{\rm th}$ растут в соответствии с ростом пороговых концентраций и уменьшением времён рекомбинации. В области 6--10 ТГц они остаются достаточно низкими и составляют сотни А/см$^2$, однако стоит отметить, что это пороговые токи для \emph{одной} ямы и в реальном лазере, с учётом неучтённых оптических потерь и наличия множества ям, они могут оказаться существенно выше (см. обсуждение в разделе~\ref{sec:HgCdTe-discussion}).

Для ещё более широких ям появляется беспороговый CHHH-процесс (верхняя вставка на рис.~\ref{fig:HgCdTe-current-frequency}а), а возрастающая роль межподзонного и друдевского поглощения приводит к росту пороговых концентраций и частот генерации, в результате чего пороговые токи возрастают до сотен кА/см$^2$, а дальнейшего продвижения в ТГц область не происходит. Это ещё раз подтверждает непригодность широких ям из теллурида кадмия-ртути для лазерной генерации.

При комнатной температуре частоты генерации не достигают ТГц области, а подавление оже-рекомбинации неэффективно из-за того, что протяжённость дираковского участка спектра носителей сравнима с тепловой энергией $k_B T_e$, так что при 300 К лазеры на квантовых ямах из теллурида кадмия-ртути могут быть конкурентоспособны только в средней инфракрасной области, вплоть до $\sim 6$--7~мкм. При дальнейшем продвижении в длинноволновую область квантово-каскадные лазеры имеют существенно более низкие пороговые токи из-за того, что переходы между почти параллельными подзонами позволяют достичь большого усиления при низких концентрациях носителей~\cite{QCL_threshold_concentration}.

\subsection{Обсуждение влияния неучтённых потерь на пороговые характеристики лазерных диодов}
\label{sec:HgCdTe-discussion}
Пороговые токи, представленные в предыдущем разделе, рассчитаны с учётом только \emph{фундаментально неустранимых} оптических потерь, связанных с поглощением в самой яме. В реальном лазере будут присутствовать и другие механизмы потерь, такие как решёточное и друдевское поглощение в волноводных слоях и барьерах, потери на зеркалах, а также токи утечки.

Решёточное поглощение вблизи частот оптических фононов теллурида кадмия-ртути присутствует как в самой яме, так и в барьерах и всех остальных буферных/волноводных слоях, если они состоят из этого материала. Тем не менее, используя выражения для поглощаемой мощности на проводящей плоскости и в трёхмерном материале с комплексной диэлектрической проницаемостью, решёточное поглощение можно описать некоторой эффективной двумерной проводимостью:
\begin{eq}{Reststrahlen_conductivity}
\sigma^{\rm lattice}(\omega) = \frac{\omega l}{4 \pi N_{\rm wells}} \Im \kappa(\omega),
\end{eq}
где $l$ --- эффективный поперечный размер лазерной моды, $N_{\rm wells}$ --- количество ям в активной среде лазера. Если положить $l$ равным $\lambda/2$, половине длины волны излучения в теллуриде кадмия-ртути, то выражение \eqref{Reststrahlen_conductivity} можно упростить до
\begin{eq}{simplified_Reststrahlen_conductivity}
\sigma^{\rm lattice}(\omega) = \frac{c}{2 N_{\rm wells}} n'',
\end{eq}
где $n''$ --- мнимая часть показателя преломления теллурида кадмия-ртути.

Межзонная проводимость в квантовых ямах из теллурида кадмия-ртути может достигать величины $\sigma^{\rm inter}_{\rm max} \approx - e^2/{4 \hbar}$ --- такое значение получается для дираковского спектра при $\hbar\omega = E_g$ и максимальной инверсии населённостей~\cite{Dirac_optical_conductivity}. Отсюда и из формулы \eqref{simplified_Reststrahlen_conductivity} следует, что влияние решёточного поглощения будет несущественно при
\begin{eq}{max_allowed_extinction_coeff}
n'' \ll \frac{N_{\rm wells}}{2} \alpha_0,
\end{eq}
где $\alpha_0$ --- постоянная тонкой структуры. Следовательно, при $n''$ порядка $10^{-2}$ и $N_{\rm wells} \sim 10$ решёточным поглощением можно пренебречь. В чистом теллуриде кадмия такая величина мнимой части показателя преломления достигается при частотах от $\sim 6$ ТГц и выше~\cite{optical_constants_handbook}.

Увеличение количества ям до $N_{\rm wells} \sim 10$ означает увеличение пороговых токов в ТГц области от сотен А/см$^2$ (рис.~\ref{fig:HgCdTe-current-frequency}б) до единиц кА/см$^2$, что сопоставимо с пороговыми токами квантово-каскадных лазеров и лазеров на солях свинца, работающих в диапазоне 10--20 ТГц. Таким образом, решёточное поглощение не препятствует достижению генерации на частотах от 6 ТГц и выше.

Вопрос о длинноволновой границе области сильного решёточного поглощения более сложен, так как точное её положение сильно зависит от температуры и, возможно, других факторов, таких как качество материала и мольная доля ртути. Так, измерения коэффициента пропускания пластины теллурида кадмия толщиной~\cite{CdTe_transmission} показали, что при комнатной температуре отсутствует пропускание в области 1--6 ТГц, а при азотной и более низких температурах область полного непропускания сужается до 4--6 ТГц. Однако генерация ниже $\sim 5$ ТГц, как было показано в предыдущих разделах, сильно осложняется друдевским и межподзонным поглощением, поэтому точное положение длинноволновой границы области сильного решёточного поглощения (Reststrahlen band) несущественно, и мы будем считать, что она охватывает все частоты ниже 6 ТГц.

Серьёзную проблему для ТГц лазеров может представлять друдевское поглощение за пределами ям --- в волноводных слоях и барьерах. Несмотря на то, что уровень легирования в них может быть достаточно низким, за счёт их большой суммарной толщины по сравнению с толщиной ям их вклад в друдевское поглощение может быть сравним со вкладом от носителей в самих ямах. Кроме того, конечное время захвата носителей в ямы может приводить к накоплению носителей в волноводных слоях при больших уровнях инжекции~\cite{carrier_accumulation_in_OCL}.

Минимально допустимая концентрация носителей в волноводных слоях определяется необходимостью обеспечить ток инжекции выше порогового значения. Ограничение на максимально достижимый ток инжекции при заданной концентрации носителей в волноводных слоях может быть связано с конечной дрейфовой скоростью и конечной скоростью захвата носителей в ямы.

Вначале рассмотрим ограничение, связанное с дрейфовой скоростью. При скоростях насыщения $v_{{\rm sat},e/h}$ для электронов и дырок минимально допустимые концентрации электронов и дырок в волноводных слоях составляют
\begin{eq}{min_allowed_doping}
n_{\rm waveguide} \geq \frac{J_{\rm th}}{e v_{{\rm sat},e}},\quad p_{\rm waveguide} \geq \frac{J_{\rm th}}{e v_{{\rm sat},h}}.
\end{eq}
Если использовать значения $v_{{\rm sat},e} \sim 2 \times 10^7$~см/с, $v_{{\rm sat},h} \sim 10^6$~см/c для CdTe~\cite{semiconductor_handbook}, $J_{\rm th} = 400$ А/см$^2$ для генерации на 6 ТГц (рис.~\ref{fig:HgCdTe-current-frequency}б), получаем $n_{\rm waveguide} \gtrsim 1.2 \times 10^{14}$~см$^{-3}$, $p_{\rm waveguide} \gtrsim 2.5 \times 10^{15}$~см$^{-3}$.

При толщинах $n$- и $p$-легированных слоёв $l_n, l_p$ соответствующая двумерная друдевская проводимость равна
\begin{eq}{waveguide_Drude}
\sigma^{\rm Drude, waveguide}(\omega) &= l_n \frac{n_{\rm waveguide} e \mu_e}{1 + \omega^2 \tau_{{\rm sc},e}^2} +  l_p \frac{p_{\rm waveguide} e \mu_h}{1 + \omega^2 \tau_{{\rm sc},h}^2}\\
&\approx \frac{J_{\rm th}}{\omega^2} \left(l_n \frac{\mu_e}{v_{{\rm sat},e} \tau_{{\rm sc},e}^2} +  l_p \frac{\mu_h}{v_{{\rm sat},h} \tau_{{\rm sc},h}^2}\right),
\end{eq}
где $\mu_{e/h}$ и $\tau_{{\rm sc},e/h}$ --- подвижности электронов и дырок и соответствующие транспортные времена релаксации.

Мы будем использовать значения $\mu_e = 2 \times 10^4$~см$^2$/(В$\cdot$с), $\mu_h = 1200$~см$^2$/(В$\cdot$с) для подвижностей и $m_e = 0.1 m_0$, $m_h = 0.8 m_0$ для эффективных масс в CdTe~\cite{semiconductor_handbook}, что даёт транспортные времена релаксации $\tau_{{\rm sc}, e} \approx 1$ пс, $\tau_{{\rm sc}, h} \approx 0.5$ пс.  Полагая $l_n$ и $l_p$ равными четверти длины волны лазерного излучения в теллуриде кадмия-ртути (4 мкм для $\omega/2\pi = 6$~ТГц), получаем $\sigma^{\rm Drude, waveguide}(\omega) \sim 0.003 e^2/\hbar$, что существенно ниже максимально достижимой межзонной проводимости $\sigma^{\rm inter}_{\rm max}(\omega) \sim -0.25 e^2/\hbar$. В случае использования нескольких ям в активной среде пороговый ток, минимальная концентрация носителей в волноводных слоях и сопутствующие друдевские потери возрастут пропорционально количеству ям $N_{\rm wells}$, но и межзонная проводимость также возрастёт в $N_{\rm wells}$ раз.

Роль конечной скорости захвата носителей в ямы оценить сложнее из-за большого разброса данных, имеющихся в литературе. Так, значения времени захвата, полученные в разных работах, охватывают диапазон как минимум от десятых долей пикосекунды~\cite{capture_time_subpicosecond} до единиц наносекунд~\cite{capture_time_oscillations_theory1}. Такой разброс может объясняться, по крайней мере, частично, резонансным характером процесса захвата~\cite{capture_time_oscillations_theory1,capture_time_oscillations_theory2}, из-за чего время захвата чувствительно к толщине ямы~\cite{capture_time_oscillations_exp1, capture_time_oscillations_exp2, capture_time_oscillations_exp3}.

Использованные нами значения скоростей насыщения соответствуют временам пролёта 400 пс для дырок и 20 пс для электронов. Характерные времена рекомбинации на пороге для генерации в ТГц области составляют десятки-сотни пс (рис.~\ref{fig:HgCdTe-recombination_times}а). Следовательно, если время захвата не превышает десятков пс, сопутствующее увеличение общего числа носителей в гетероструктуре будет несущественно.

Существуют также работы, в которых предлагается использовать не время, а \emph{скорость} захвата носителей в ямы, определяемую как отношение плотности тока захватываемых носителей к концентрации носителей в волноводных слоях~\cite{capture_velocity_definition}. Скорости захвата на уровне $2 \times 10^6$ см/c достижимы на практике~\cite{capture_velocity_exp}. Сопоставление этого значения с используемыми нами значениями скоростей насыщения позволяет сделать вывод, что концентрация дырок в волноводных слоях будет определяться прежде всего дрейфовой скоростью, а электронов --- скоростью захвата в яму. Замена $v_{{\rm sat},e}$ на $2 \times 10^6$ см/c в уравнении \eqref{waveguide_Drude} даёт $\sigma^{\rm Drude, waveguide}(\omega) \sim 0.007 e^2/\hbar$, что всё ещё существенно меньше $\abs{\sigma^{\rm inter}_{\rm max}(\omega)}$.  Кроме того, роль конечного времени захвата носителей можно снизить увеличением количества ям, так как суммарная скорость захвата носителей в ямы возрастает примерно пропорционально $N_{\rm wells}$~\cite{capture_time_vs_number_of_wells}, в отличие от дрейфового тока.

Таким образом, друдевское поглощение в волноводных слоях также не препятствует достижению лазерной генерации в ТГц диапазоне, хотя для этого требуется хорошее качество материала волноводных слоёв, чтобы обеспечить достаточную подвижность дырок, а также оптимальный подбор толщины ямы, состава ям/барьеров и профиля легирования, чтобы обеспечить быстрый захват носителей в ямы. Однако эффекты конечной скорости дрейфа и захвата носителей в ямы могут ограничивать максимально достижимый уровень инжекции и, как следствие, выходную мощность лазера.

Паразитные токи, связанные с рекомбинацией в волноводных слоях и надбарьерными утечками, скорее всего, будут пренебрежимы, так как ТГц генерация в рассматриваемых лазерах возможна только при низких концентрациях носителей в волноводных слоях и низких температурах.

Наконец, в реальном лазере имеются потери на зеркалах. Для достаточно большой длины резонатора $L$ они становятся несущественными по сравнению с усилением и потерями в активной среде. Полагая что межзонная оптическая проводимость имеет порядок ($- e^2/{4 \hbar}$), можно показать, что потери на зеркалах пренебрежимо малы при
\begin{eq}{min_allowed_resonator_length}
L \gg \frac{\lambda_{\rm vac}}{2 \pi N_{\rm wells} \alpha_0}
\end{eq}
($\lambda_{\rm vac}$ --- длина волны генерируемого излучения в вакууме). Таким образом, потери на зеркалах не препятствуют ТГц генерации при длине резонатора больше некоторой критической длины, составляющей десятые доли -- единицы мм (в зависимости от количества ям).
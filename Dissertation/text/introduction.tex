{\actuality} 
Электронно-лучевая литография является одним из наиболее распространенных литографических методов формирования субмикронных структур, обеспечивающим высокое разрешение и контраст. Однако, этот метод обладает некоторыми недостатками, главным из которых, пожалуй, является низкая производительность, что вносит некоторые ограничения на применение этого метода в крупномасштабном производстве. Существуют методы решения этой проблемы, связанные с применением химически усиленных резистов, либо использованием многолучевых электронных систем. В настоящее время эти подходы находятся в стадии разработки и также имеют некоторые недостатки. Сухое электронно-лучевое травление резиста (СЭЛТР) – новый одностадийный литографический метод формирования рельефа в слое позитивного резиста, основанный на цепной реакции деполимеризации полимерного резиста и самопроявлении изображения непосредственно в процессе электронно-лучевого экспонирования резиста, проводимого при температурах выше его температуры стеклования. Отличительными особенностями метода являются исключительно высокая чувствительность резиста, высокое разрешение по вертикали и возможность формирование рельефа без этапа проявления, а также скругленные стенки профиля линии. Высокая чувствительность резиста обеспечивает производительность метода в сотни раз превышающую производительность обычной электронно-лучевой литографии. Благодаря этим особенностям метод можно использовать для формирования дифракционных оптических элементов, различных трехмерных микро- и наноструктур или масок ~\cite{Sidorov2018a} Также возможной областью его применения является формирование каналов для использования в микро- и нанофлюидике, поскольку отсутствие острых углов в сечении канала положительно скажется на его гидравлическом диаметре. Однако, латеральное разрешение метода ограничено, и 3 в настоящее время при использовании электронно-лучевых систем с диаметром электронного луча около 10-15 нм удается получать линии шириной около 200 нм. Область применения метода могла бы быть существенно расширена, если бы удалось повысить его латеральное разрешение. Для этого необходимо изучить 

всякое разное

{\previouswork}

\aimsandtasks\ 


Целью данной работы является определение и исследование основных процессов, протекающих при сухом электронно-лучевом травлении резиста, а также создание физической модели метода СЭЛТР, позволяющей определить результирующий профиль линии при различных условиях экспонирования. В большинстве экспериментов, которые были проведены для исследования метода СЭЛТР, в качестве резиста и материала подложки использовались ПММА и Si, соответственно. Учитывая также тот факт, что свойства ПММА достаточно хорошо изучены, при создании модели процесса СЭЛТР в рамках данной работы в качестве резиста рассматривался именно этот материал. Для достижения поставленной цели необходимо было решить следующие задачи:

\begin{enumerate}
  \item На основе существующих моделей взаимодействия электронного излучения с веществом реализовать детальный алгоритм моделирования рассеяния электронного пучка в системе ПММА/Si;
  \item Определить механизмы, приводящие к разрыву молекул ПММА при экспонировании в условиях повышенной температуры;
  \item Разработать алгоритм моделирования электронно-стимулированной деструкции молекул ПММА при температурах метода СЭЛТР;
  \item Разработать модель процесса изменения распределения молекулярной массы ПММА при экспонировании;
  \item Определить температурную зависимость длины кинетической цепи при деполимеризации ПММА в условиях метода СЭЛТР;
  \item Разработать модель диффузии в слое ПММА мономеров, образовавшихся в процессе деполимеризации;
  \item Реализовать алгоритм моделирования растекания линии, вызванного пониженной вязкостью ПММА при температурах процесса СЭЛТР;
  \item Разработать программу моделирования метода СЭЛТР с учетом совместное протекание процессов рассеяния электронного пучка, деполимеризации, диффузии мономеров и растекания профиля линии;
  \item На основе разработанного алгоритма моделирования определить пути оптимизации разрешения метода СЭЛТР.
\end{enumerate}

\defpositions
\begin{enumerate}
	\item При комнатной температуре электронно-стимулированная деструкция ПММА протекает за счет взаимодействия налетающего электрона с валентными электронами ММА, образующими связи между атомами углерода в главной цепи ПММА. Увеличение радиационно-химического выхода разрывов с ростом температуры может быть описано за счет увеличения вероятности разрыва главной цепи ПММА при разрыве связей между атомами водорода и атомами углерода, образующими главную цепи. При температурах в диапазоне от 30 °С до 160 °С данная вероятность увеличивается практически линейно от 0 до 1.
	\item Область оптимальных температур для метода СЭЛТР составляет 120-160 °С. Кинетическая длина цепи при деполимеризации ПММА в этой области изменяется от 500 до 3200 с ростом температуры, при этом имеет место передача активного центра деполимеризации с мономера на полимер;
	\item При использовании в методе СЭЛТР слоев ПММА толщиной до 1 мкм процессы диффузии мономера в слое ПММА не замедляют процесс формирования рельефа;
	\item При экспонировании вдоль серии линий при длительном суммарном времени нагрева форма профиля линии приближается к синусоидальной. Увеличение разрешения метода СЭЛТР может быть достигнуто за счет уменьшения суммарного времени нагрева, до значений, сопоставимых с временем затухания гармоник в фурье-образе профиля линии с высокими частотами (n > 10).
\end{enumerate}

\novelty
\begin{enumerate}
  \item Впервые предложена количественная модель, описывающая электронно-стимулированную деструкция молекул ПММА на молекулярном уровне с учетом температурного эффекта;
  \item Впервые исследовано совместное протекание процессов рассеяния электронного пучка в полимерном резисте, деполимеризация резиста, диффузия продуктов распада молекул резиста и растекание профиля;
  \item Впервые проведено моделирование профиля линии, получаемой методом сухого электронно-лучевого травления резиста.
\end{enumerate}

\influence\
Теоретическая значимость работы состоит в том, что впервые была создана модель формирования рельефа в резисте за счет совместного воздействия основных процессов, характерных для метода СЭЛТР – электронно-стимулированной деструкции резиста при повышенных температурах, термической деполимеризации резиста, диффузии мономеров слое резиста и растекания профиля линии за счет пониженной вязкости. Практическая значимость работы заключается в том, что был разработан алгоритм, позволяющий промоделировать форму профиля линии, получаемой методом СЭЛТР при различных условиях экспонирования и определить оптимальные условия для каждой конкретной задачи.

\methods\
Основным методом исследования основных процессов СЭЛТР являлось математическое моделирование; Для моделирования рассеяния электронного пучка использовался Монте-Карло алгоритм с дискретными потерями энергии. Моделирование слоя ПММА производилось на основе модели идеальной цепи; Моделирование диффузии мономера в слое ПММА проводилось на основе Монте-Карло алгоритма, длины свободного пробега мономеров определялись из функции Грина задачи диффузии частицы в свободном пространстве; Для моделирования растекания профиля линии применялось фурье-преобразование профиля с дальнейшим определением времени затухания различных гармоник из двумерного уравнения Навье-Стокса и уравнения непрерывности в условиях отсутствия скольжения с учетом давления Лапласа и расклинивающего давления.

\probation\
Поскольку на конечный профиль линии, получаемой методом СЭЛТР, влияет сразу несколько процессов, точность их описания проверялась на каждом этапе. Так, при моделировании рассеяния электронного пучка в системе ПММА/Si сечения упругих и неупругих процессов вычислялись с использованием наиболее современных моделей взаимодействия излучения с веществом (моттовские дифференциальные сечения упругого рассеяния и сечения, полученные с использованием диэлектрической функции Мермина и модели обобщенных осцилляторов для неупругого рассеяния). Механизмы разрыва молекул ПММА при комнатной и повышенной температуре определялись на основе моделирования радиационно-химического выхода разрывов, вычисляемого экспериментально из распределения молекулярной массы. Полученные значения для длины кинетической цепи при деполимеризации ПММА при различных температурах согласуются с опубликованными значениями, рассчитанными на основе констант деполимеризации и терминации в кинетических моделях термической деструкции ПММА. Диффузия мономеров в слое ПММА моделировалась с коэффициентами диффузии, соответствующим различным температурам и массовой доле мономера в слое ПММА. Полученная в результате оценка сверху для времени диффузии привела к значению, пренебрежимо малому по сравнению с характерным временем протекания других процессов. Подход, использующийся для моделирования растекания профиля линии в процессе СЭЛТР, эффективно применяется в смежной области – моделировании растекания структур, полученных методом наноимпринтной литографии, и его точность отмечена в ряде работ. Все вышеперечисленное вкупе с соответствием между экспериментальными и промоделированными профилями обеспечивает достоверность полученных результатов.

Основные результаты работы докладывались на следующих конференциях:
\begin{itemize}
	\item 60-я всероссийская научная конференция МФТИ, Долгопрудный (2016);
	\item International conference on information technology and nanotechnology (ITNT), Самара (2017, 2018, 2020, 2022);
	\item III International Conference on modern problems in physics of surfaces and nanostructures (ICMPSN17), Ярославль (2017);
	\item Micro- and Nanoengineering (MNE), Копенгаген (2018), Родос (2019);
	\item International School and Conference "Saint-Petersburg OPEN"on Optoelectronics, Photonics, Engineering and Nanostructures, Санкт-Петербург (2019, 2020).
	
\end{itemize}

Диссертация состоит из трёх глав, основные результаты которых изложены в трёх статьях~\cite{my_graphene,my_HgCdTe,my_plasmon}. Все статьи опубликованы в рецензируемых международных журналах (Physical Review B, ACS Photonics, Journal of Physics: Condensed Matter), включённых в библиографические базы Scopus и Web of Science.
    
\contribution Общая постановка задачи осуществлялась научным руководителем автора Рогожиным А. Е. Для верификации результатов моделирования были использованы структуры, полученные методом СЭЛТР М. А. Бруком, А. Е. Рогожиным и Е. Н. Жихаревым. Все результаты, изложенные настоящей диссертации, получены автором лично.
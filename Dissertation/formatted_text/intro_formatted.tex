\chapter*{\introname}
\addcontentsline{toc}{chapter}{Введение}

\titlespacing{\section}{\theotstup\parindent}{*2}{*1}
\titlespacing{\subsection}{\theotstup\parindent}{*1}{*0.5}
\newcommand{\actuality}{\section*{Актуальность темы исследования}}
\newcommand{\previouswork}{\section*{Степень разработанности темы исследования}}
\newcommand{\aimsandtasks}{\subsection*{Цели и задачи}}
\newcommand{\aim}{\vspace{1em}\textbf{Целью}}
\newcommand{\tasks}{\textbf{задачи}}
\newcommand{\defpositions}{\subsection*{Положения, выносимые на~защиту}}
\newcommand{\novelty}{\subsection*{Научная новизна}}
\newcommand{\influence}{\subsection*{Теоретическая и практическая значимость работы}}
\newcommand{\methods}{\subsection*{Методология и методы исследования}}
\newcommand{\reliability}{\subsection*{Степень достоверности}}
\newcommand{\probation}{\subsection*{Степень достоверности и апробация результатов}}
\newcommand{\contribution}{\subsection*{Личный вклад автора}}
\newcommand{\publications}{\subsection*{Публикации}}

\actuality

Формирование трехмерных микро- и наноструктур является востребованным во множестве областей, таких как микроэлектроника, дифракционная оптика и нанофотоника, микро- и нанофлюидика и др. В настоящее время существует множество подходов к решению этой задачи, однако такие преимущества метода, как универсальность, высокая производительность и доступность зачастую оказываются взаимоисключающими. Универсальные методы с высоким разрешением (например, полутоновая литография~\cite{GL_general}, двухфотонная литография~\cite{TPL_castle} или сканирующая зондовая литография~\cite{SPL_mechanical}) требуют использования сложного высокоточного оборудования и обладают при этом достаточно низкой производительностью. Более производительные и доступные методы позволяют получить только периодические структуры (интерференционная литография~\cite{IL_metamaterials}), либо структуры определенного вида (наноимпринтная литография~\cite{NIL_1}). Таким образом, в настоящее время отсутствует метод получения произвольных микро- и наноструктур, являющийся одновременно высокопроизводительным и простым в реализации.

Ввиду этого внимания заслуживает метод сухого электронно-лучевого травления резиста (СЭЛТР) –- относительно новый одностадийный литографический метод формирования рельефа в слое позитивного резиста, основанный на цепной реакции деполимеризации полимерного резиста и самопроявлении изображения непосредственно в процессе электронно-лучевого экспонирования резиста, проводимого при температурах выше его температуры стеклования~\cite{Bruk_2013, Bruk_2016_mee}. Отличительными особенностями метода являются исключительно высокая чувствительность резиста, высокое разрешение по вертикали и возможность формирование рельефа без этапа проявления, а также скругленные стенки профиля линии. Высокая чувствительность резиста обеспечивает производительность метода в сотни раз превышающую производительность обычной электронно-лучевой литографии. Благодаря этим особенностям метод может быть использован для формирования дифракционных оптических элементов, различных трехмерных микро- и наноструктур или масок. Также возможной областью его применения является формирование каналов для использования в микро- и нанофлюидике, поскольку отсутствие острых углов в сечении канала положительно скажется на его гидравлическом диаметре.

Однако, латеральное разрешение метода ограничено, и в настоящее время при использовании электронно-лучевых систем с диаметром электронного луча около 10-15 нм удается получать линии шириной около 300 нм. Область применения метода могла бы быть существенно расширена, если бы удалось повысить его латеральное разрешение. В силу одновременного протекания в процессе СЭЛТР множества различных процессов точный механизм формирования конечного профиля линии не был понятен, что не позволяло выявить пути оптимизации данного метода. Таким образом, целесообразным является разработка физической модели метода СЭЛТР, что позволит определить возможности метода и оптимизировать его для применения в различных областях.


\previouswork

Первые шаги в изучении метода микролитографии на основе термической деполимеризации резиста описываются в работе~\cite{Bruk_2000}. В ней приводятся результаты инициированной $\gamma$-излучением деполимеризации ПММА в виде нанометрового слоя, адсорбированного на поверхности пор силохрома. Несмотря на то, что в данной работе термическая деполимеризация не использовалась для формирования структуры в резисте, а исследовалась в общем, результаты работы позволили определить особенности потенциально возможного метода микроструктурирования на основе этого явления. Так, например, были получены оценки для времени диффузии мономера в слое ПММА после разрушения молекулы и длины кинетической цепи деполимеризации, сделаны выводы о масштабах протекания процессов передачи активного центра деполимеризации на мономер и полимер. Также было установлено, что при термической деполимеризации ПММА при температурах 120--180 $^\circ$C влияние процессов реполимеризации пренебрежимо мало.

Впоследствии были проведены эксперименты по изучению термической деполимеризации полиметилметакрилата (ПММА), протекающей при экспонировании электронным лучом, а также впервые были продемонстрированы двумерные и трехмерные структуры, полученные в ПММА в этом процессе~\cite{Bruk_2013}.

Наиболее актуальные на сегодняшний день экспериментальные результаты по исследованию метода сухого электронно-лучевого травления резиста приведены в работах~\cite{Bruk_2015_co, Bruk_2016_mee}. Помимо вышеописанных ступенчатых профилей, в этих работах исследовались периодические профили, полученные при экспонировании резиста электронным лучом вдоль серии параллельных линий (рисунок~\ref{fig:DEBER_many_profiles}). Было продемонстрировано, что при таком экспонировании результирующий профиль приобретает практически синусоидальную форму, что является аргументом в пользу применения метода СЭЛТР для формирования некоторых дифракционных оптических элементов~\cite{Mitreska_sin_gratings}. При этом снова была отмечена высокая производительность метода -- при температуре 160 $^\circ$C полное травление в центре линии было достигнуто при дозе экспонирования менее 1 мкКл/см$^2$. Также была продемонстрирована возможность достаточно точного переноса профиля, полученного в ПММА, на поверхность вольфрама и кремния за счет сухого травления в реакторе индуктивно-связанной плазмы (рисунок~\ref{fig:DEBER_Si_W}). Этот факт теоретически позволяет использовать метод СЭЛТР для формирования, например, штампов для термической наноимпринтной литографии.


\aimsandtasks\ 

Целью данной работы является определение и исследование основных процессов, протекающих при сухом электронно-лучевом травлении резиста, а также создание физической модели метода СЭЛТР, позволяющей определить результирующий профиль линии при различных условиях экспонирования. В большинстве экспериментов, которые были проведены для исследования метода СЭЛТР, в качестве резиста и материала подложки использовались ПММА и Si, соответственно. Учитывая также тот факт, что свойства ПММА достаточно хорошо изучены, при создании модели процесса СЭЛТР в рамках данной работы в качестве резиста рассматривался именно этот материал. Для достижения поставленной цели необходимо было решить следующие задачи:

\begin{enumerate}
  \item На основе существующих моделей взаимодействия электронного излучения с веществом реализовать детальный алгоритм моделирования рассеяния электронного пучка в системе ПММА/Si;
  \item Определить механизмы, приводящие к разрыву молекул ПММА при экспонировании в условиях повышенной температуры;
  \item Разработать алгоритм моделирования электронно-стимулированной деструкции молекул ПММА при температурах метода СЭЛТР;
  \item Разработать модель процесса изменения распределения молекулярной массы ПММА при экспонировании;
  \item Определить температурную зависимость длины кинетической цепи при деполимеризации ПММА в условиях метода СЭЛТР;
  \item Разработать модель диффузии в слое ПММА мономеров, образовавшихся в процессе деполимеризации;
  \item Реализовать алгоритм моделирования растекания линии, вызванного пониженной вязкостью ПММА при температурах процесса СЭЛТР;
  \item Разработать программу моделирования метода СЭЛТР с учетом совместное протекание процессов рассеяния электронного пучка, деполимеризации, диффузии мономеров и растекания профиля линии;
  \item На основе разработанного алгоритма моделирования определить пути оптимизации разрешения метода СЭЛТР.
\end{enumerate}


\defpositions

\begin{enumerate}
	\item При комнатной температуре электронно-стимулированная деструкция ПММА протекает за счет взаимодействия налетающего электрона с валентными электронами ММА, образующими связи между атомами углерода в главной цепи ПММА. Увеличение радиационно-химического выхода разрывов с ростом температуры может быть описано за счет увеличения вероятности разрыва главной цепи ПММА при разрыве связей между атомами водорода и атомами углерода, образующими главную цепи. При температурах в диапазоне от 30 °С до 160 °С данная вероятность увеличивается практически линейно от 0 до 1.
	\item Область оптимальных температур для метода СЭЛТР составляет 120-160 °С. Кинетическая длина цепи при деполимеризации ПММА в этой области изменяется от 500 до 3200 с ростом температуры, при этом имеет место передача активного центра деполимеризации с мономера на полимер;
	\item При использовании в методе СЭЛТР слоев ПММА толщиной до 1 мкм процессы диффузии мономера в слое ПММА не замедляют процесс формирования рельефа;
	\item При экспонировании вдоль серии линий при длительном суммарном времени нагрева форма профиля линии приближается к синусоидальной. Увеличение разрешения метода СЭЛТР может быть достигнуто за счет уменьшения суммарного времени нагрева, до значений, сопоставимых с временем затухания гармоник в фурье-образе профиля линии с высокими частотами (n > 10).
\end{enumerate}


\novelty

\begin{enumerate}
  \item Впервые предложена количественная модель, описывающая электронно-стимулированную деструкция молекул ПММА на молекулярном уровне с учетом температурного эффекта;
  \item Впервые исследовано совместное протекание процессов рассеяния электронного пучка в полимерном резисте, деполимеризация резиста, диффузия продуктов распада молекул резиста и растекание профиля;
  \item Впервые проведено моделирование профиля линии, получаемой методом сухого электронно-лучевого травления резиста.
\end{enumerate}


\influence\

Теоретическая значимость работы состоит в том, что впервые была создана модель формирования рельефа в резисте за счет совместного воздействия основных процессов, характерных для метода СЭЛТР – электронно-стимулированной деструкции резиста при повышенных температурах, термической деполимеризации резиста, диффузии мономеров слое резиста и растекания профиля линии за счет пониженной вязкости. Практическая значимость работы заключается в том, что был разработан алгоритм, позволяющий промоделировать форму профиля линии, получаемой методом СЭЛТР при различных условиях экспонирования и определить оптимальные условия для каждой конкретной задачи.


\methods\

Основным методом исследования основных процессов СЭЛТР являлось математическое моделирование; Для моделирования рассеяния электронного пучка использовался Монте-Карло алгоритм с дискретными потерями энергии. Моделирование слоя ПММА производилось на основе модели идеальной цепи; Моделирование диффузии мономера в слое ПММА проводилось на основе Монте-Карло алгоритма, длины свободного пробега мономеров определялись из функции Грина задачи диффузии частицы в свободном пространстве; Для моделирования растекания профиля линии применялось фурье-преобразование профиля с дальнейшим определением времени затухания различных гармоник из двумерного уравнения Навье-Стокса и уравнения непрерывности в условиях отсутствия скольжения с учетом давления Лапласа и расклинивающего давления.


\probation\

Поскольку на конечный профиль линии, получаемой методом СЭЛТР, влияет сразу несколько процессов, точность их описания проверялась на каждом этапе. Так, при моделировании рассеяния электронного пучка в системе ПММА/Si сечения упругих и неупругих процессов вычислялись с использованием наиболее современных моделей взаимодействия излучения с веществом (моттовские дифференциальные сечения упругого рассеяния и сечения, полученные с использованием диэлектрической функции Мермина и модели обобщенных осцилляторов для неупругого рассеяния). Механизмы разрыва молекул ПММА при комнатной и повышенной температуре определялись на основе моделирования радиационно-химического выхода разрывов, вычисляемого экспериментально из распределения молекулярной массы. Полученные значения для длины кинетической цепи при деполимеризации ПММА при различных температурах согласуются с опубликованными значениями, рассчитанными на основе констант деполимеризации и обрыва кинетической цепи деполимеризации в кинетических моделях термической деструкции ПММА. Диффузия мономеров в слое ПММА моделировалась с коэффициентами диффузии, соответствующим различным температурам и массовой доле мономера в слое ПММА. Полученная в результате оценка сверху для времени диффузии привела к значению, пренебрежимо малому по сравнению с характерным временем протекания других процессов. Подход, использующийся для моделирования растекания профиля линии в процессе СЭЛТР, эффективно применяется в смежной области – моделировании растекания структур, полученных методом наноимпринтной литографии, и его точность отмечена в ряде работ. Все вышеперечисленное вкупе с соответствием между экспериментальными и промоделированными профилями обеспечивает достоверность полученных результатов.

Основные результаты работы докладывались на следующих конференциях:
\begin{itemize}
	\item 60-я всероссийская научная конференция МФТИ, Долгопрудный (2016);
	\item International conference on information technology and nanotechnology (ITNT), Самара (2017, 2018, 2020, 2022);
	\item III International Conference on modern problems in physics of surfaces and nanostructures (ICMPSN17), Ярославль (2017);
	\item Micro- and Nanoengineering (MNE), Копенгаген (2018), Родос (2019);
	\item International School and Conference "Saint-Petersburg OPEN"on Optoelectronics, Photonics, Engineering and Nanostructures, Санкт-Петербург (2019, 2020).	
\end{itemize}

Диссертация состоит из трёх глав, основные результаты которых изложены в статьях~\cite{my_CO, my_microlenses, my_evidence, my_detailed, my_review, my_MEE, my_Gvalue, my_microscopic, my_Isaev}. Все статьи опубликованы в рецензируемых международных журналах, включённых в библиографические базы (РИНЦ, Scopus, Web of Science).


\contribution\

Общая постановка задачи осуществлялась научным руководителем автора Рогожиным А. Е. Для верификации результатов моделирования были использованы структуры, полученные методом СЭЛТР М. А. Бруком, А. Е. Рогожиным и Е. Н. Жихаревым. Все результаты, изложенные настоящей диссертации, получены автором лично.
 % Характеристика работы по структуре во введении и в автореферате не отличается (ГОСТ Р 7.0.11, пункты 5.3.1 и 9.2.1), потому её загружаем из одного и того же внешнего файла, предварительно задав форму выделения некоторым параметрам

%Диссертационная работа была выполнена при поддержке грантов ...

%\underline{\textbf{Объем и структура работы.}} Диссертация состоит из~введения, четырех глав, заключения и~приложения. Полный объем диссертации \textbf{ХХХ}~страниц текста с~\textbf{ХХ}~рисунками и~5~таблицами. Список литературы содержит \textbf{ХХX}~наименование.

%\newpage
\titlespacing{\section}{\theotstup\parindent}{*\theintvl}{*\theintvl}
\titlespacing{\subsection}{\theotstup\parindent}{*\theintvl}{*\theintvl}
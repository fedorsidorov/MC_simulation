\begin{abstract}

This paper presents a new approach to simulation of thermal viscoelastic reflow of \\ e-beam exposed PMMA taking into account its non-uniform viscosity profile.
This approach is based on numerical ``soapfilm'' modeling of surface evolution, processed by free software ``Surface Evolver'' in area normalization mode.
PMMA viscosity profile is determined by Monte-Carlo simulation of e-beam induced PMMA main-chain scissions.
The relation between PMMA viscosity and mobility of PMMA surface vertices was determined by thermal reflow simulation of uniform PMMA gratings using both analytical and numerical approaches

 determination of mobilities of PMMA

Vertex mobilities of PMMA surface are determined from PMMA viscosity using empirical formula obtained from


 simulation of thermal reflow of uniform rectangular gratings using both analytical and numerical approaches in wide viscosity range. The agreement between reflowed profiles simulated by these two approaches is obtained, which emphasizes the applicability of soapfilm modeling to simulation of polymer thermal reflow. 
The inverse mobility of PMMA surface vertices appeared to be proportional to PMMA viscosity with high accuracy.
The developed algorithm enables simulation of complex non-uniform structures reflow, which leads to better understanding of polymer thermal reflow and can be applied to usage of predictable reflow as a stage of 3D microfabrication.

\end{abstract}

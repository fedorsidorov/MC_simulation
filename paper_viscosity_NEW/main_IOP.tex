\documentclass[12pt]{iopart}
\newcommand{\gguide}{{\it Preparing graphics for IOP Publishing journals}}
\usepackage{graphicx}
\graphicspath{{./figures/}}
\usepackage{setstack}
\bibliographystyle{iopart-num}
\usepackage{citesort}

\begin{document}

\title{Thermal reflow simulation of PMMA structures with non-uniform viscosity profile}
\author{F A Sidorov, A E Rogozhin}
\address{Valiev Institute of Physics and Technology, Russian Academy of Sciences, Moscow, 117218 Russia}
\ead{sidorov@ftian.ru}

\vspace{10pt}

\begin{abstract}
This paper presents a new approach to simulation of thermal viscoelastic reflow of e-beam exposed polymethyl methacrylate (PMMA) taking into account its non-uniform viscosity profile.
This approach is based on numerical ``soapfilm'' modeling of surface evolution, processed by free software ``Surface Evolver'' in area normalization mode.
PMMA viscosity profile is calculated by simulation of exposed PMMA number average molecular weight distribution using Monte-Carlo method and empirical equations.
The relation between PMMA viscosity and mobility of PMMA surface vertices was determined by thermal reflow simulation of uniform PMMA gratings using both analytical and numerical approaches in wide viscosity range.
The agreement between reflowed profiles simulated by these two approaches emphasizes the applicability of ``soapfilm'' modeling to simulation of polymer thermal reflow. 
The inverse mobility of PMMA surface vertices appeared to be proportional to PMMA viscosity with high accuracy.
The developed algorithm enables thermal reflow simulation of complex non-uniform structures, which allows the usage of predictable reflow as a stage of 3D microfabrication.
\end{abstract}

\vspace{2pc}

\noindent{\it Keywords}: grayscale e-beam lithography, non-uniform viscosity profile, thermal reflow

\section{Introduction}

Thermal reflow can significantly modify the profile of structures obtained in polymer resist and this phenomenon has its advantages and disadvantages.
For example, thermal reflow could be used for smoothing of the relief obtained by grayscale e-beam lithography~\cite{Kirchner_GL_review} which allows to obtain various 3D structures.
On the other hand, thermal reflow leads to relief deformation, which is undesirable in certain cases~\cite{NIL_reflow}.
In the light of the above, the method allowing exact determination of thermal reflow influence on resulting structure profile in any specific processes is highly desirable.

Two common approaches to thermal reflow simulation of polymer structures could be distinguished.
The first include analytical methods based on transfer equations.
For instance, Leveder~\cite{Leveder_2010,Leveder_2011} used an analytical spectral method to simulate the reflow of periodic structures obtained in polystyrene by nanoimprint lithography.
This method is based on two-dimensional Navier-Stokes equation coupled to continuity equation  considering Laplace pressure and Hamaker energy with the assumption of no slip length and no Marangoni effect.
In this method the initial structure profile undergoes Fourier transform and then thermal reflow is being simulated by decay of profile harmonic modes:
\begin{equation} \label{eq:Fourier_1}
	h(x, t) = h_0 + \tilde{h}(x, t),
\end{equation}
\begin{equation} \label{eq:Fourier_2}
	\tilde{h}(x, t) = \sum_{-\infty}^{+\infty} a_n(0) \exp \left(-\frac{t}{\tau_n}+i n \frac{2 \pi}{\lambda} x \right),
\end{equation}
\begin{equation} \label{eq:Fourier_3}
	\tau_n = \frac{3 \eta}{\gamma h_0^3} \times \left( \frac{\lambda}{2 \pi n} \right)^4,
\end{equation}
where $\lambda$ -- profile spatial periodicity, $\eta$, $\gamma$ -- polymer viscosity and surface tension, respectively, $a_n(0)$ -- Fourier coefficients of initial polymer profile, $h_0$ -- polymer layer thickness.
Polymer viscosity depends both on temperature and polymer molecular weight, which should be taken into account.
Temperature dependence of viscosity could be described by Williams–Landel–Ferry (WLF) equation~\cite{Bird_WLF}:
\begin{equation} \label{eq:WLF}
	\log \left( \frac{\eta(T)}{\eta(T_0)} \right) = -\frac{C_1(T-T_0)}{C_2+(T-T_0)},
\end{equation}
which parameters $\eta(T_0)$, $C_1$, $C_2$ and $T_0$ for three different polymers are provided in Table~\ref{table:WLF}~\cite{Aho_WLF}.
The dependence of polymer viscosity on its molecular weight could be described by empirical formula:

\begin{equation} \label{eq:3p4_3p1}
	\eta \propto M_n^\alpha,
\end{equation}
where $M_n$ -- number average polymer molecular weight. For polymethyl methacrylate (PMMA) $\alpha$ comprises 3.4 at $M_n > 48000$ and 1.4 at $M_n < 48000$~\cite{Leveder_2010,Bueche_3p4_1p4}.
Equations~\ref{eq:WLF}, and \ref{eq:3p4_3p1} allow one to calculate polymer viscosity for different temperatures and values of number average molecular weight (Fig.~\ref{fig:eta_vary_T_Mn}).

\begin{figure}
	\begin{center}
		\includegraphics[width=0.6\linewidth]{eta_vary_T_Mn}
	\end{center}
	\vspace{-2em}
	\caption{Temperature viscosity dependencies for PMMA with different number average molecular weights, obtained by equations~(\ref{eq:WLF}, \ref{eq:3p4_3p1}).}
	\label{fig:eta_vary_T_Mn}
\end{figure}

\begin{table}[t]
	\centering
	\caption{Parameters of equation~\ref{eq:WLF}, obtained by Aho et al. for polystyrene 143E by BASF (PS), polymethyl methacrylate Plexiglas 6N by Degussa (PMMA) and polycarbonate Lexan HF1110R by GE Plastics (PC)~\cite{Aho_WLF}.}
	\begin{tabular}{@{}llll}
		\br
		Parameter \hspace{8.9em} & PS \hspace{5em} & PMMA \hspace{5em} & PC \\
		\mr
		$\eta(T_0)$, Pa$\cdot$s \hspace{8.9em} & 7310.4 \hspace{5em} & 13450 \hspace{5em} & 2763 \\
		$C_1$ \hspace{8.9em} & 10.768 \hspace{5em} & 7.6682 \hspace{5em} & 4.7501 \\
		$C_2$, $^\circ$C \hspace{8.9em} & 289.21 \hspace{5em} & 210.76 \hspace{5em} & 110.12 \\
		$T_0$, $^\circ$C \hspace{8.9em} & 190 \hspace{5em} & 200 \hspace{5em} & 200 \\
		\br
	\end{tabular}
	\label{table:WLF}
\end{table}

The second approach, numerical one, is based on search of minimal surface by finite elements method. It can be processed by free software ``Surface Evolver'' (SE) -- the program for the modelling of liquid surfaces shaped by various forces and constraints~\cite{Brakke_SE}. SE allows a wide spectrum of possible energies to be assigned like gravitational energy, surface energy, and further different implementations of mean and Gaussian curvature. For the purpose of polymer reflow simulation only surface energy should be taken into account.

In SE simulation algorithm the structure is only described by its ``outer shell'' (soapfilm modeling)~(Fig.~\ref{fig:SE_basic}a). The resist surface is being divided into triangle facets defined by vertices $v_0$, $v_1$ and $v_2$ and oriented edges $\vec{e_0}$, $\vec{e_1}$ and $\vec{e_2}$, and the polymer reflow is simulated by moving of facet vertices, maintaining the constant volume inside the surface. The force on vertex $v_0$ (the tail of vector $\vec{e_0}$) is

\begin{equation}
	\vec{F}_{v_0}=\frac{\gamma_i}{2} \cdot \frac{\vec{e}_1 \times\left(\vec{e}_0 \times \vec{e}_1\right)}{\left\|\vec{e}_0 \times \vec{e}_1\right\|},
\end{equation}
where $\gamma_i$ -- is surface tension of $i$-th facet~(Fig.~\ref{fig:SE_basic}b). SE could be operated in the area normalization mode to approximate a vertex motion by mean curvature, i.e., a surface tension flow. In this mode, the velocity of a vertex is proportional to force and indirectly proportional to the area of the facets surrounding this vertex. The $i$-th facet has three vertices associated with it, therefore the relative area contribution to the force of one vertex is 1/3 the area of the surrounding facets $A$. The vertex velocity in the area normalization mode is

\begin{equation} \label{eq:SE_v}
	\vec{v} = \frac{\vec{F}}{A/3} \cdot \mu,
\end{equation}
where $\mu$ is so called vertex mobility. The vector of vertex movement $\vec{\delta}$ is then calculated as product of vertex velocity and \textit{scale} factor, the physical representation of simulation step time:

\begin{equation} \label{eq:SE_delta}
	\vec{\delta} = \vec{v} \cdot scale.
\end{equation}
In most cases SE is used for calculation of minimal energy geometries only~\cite{SE_example_1,SE_example_2}, which doesn't imply simulation of liquid or polymer flow dynamics.

\begin{figure}[t]
	\centering
	\includegraphics[width=\linewidth]{Fig_2} \\
	\caption{a) A mound of liquid sitting on a tabletop with gravity acting on it, defined by its surface in SE, b) definition of vertices and oriented edges for $i$-th facet in SE.}
	\label{fig:SE_basic}
\end{figure}

Polymer structures obtained by grayscale e-beam lithography have strongly non-uniform distribution on number average molecular weight, which result in non-uniform viscosity profile.
Analytical spectral approach could not be applicated in this case, but numerical one can still be used.
Kirhner~\cite{Kirchner_SE_1} applied SE for thermal reflow simulation of double-step structures obtained in PMMA by e-beam grayscale patterning and wet development.
The structures consisted of two regions with different values of PMMA number average molecular weight, and the difference in region viscosity was taken into account by setting different vertex mobilities.
Mobility ratios for structure regions were determined empirically by the comparison of simulated profiles to the experimental ones.
In this case, the simulation algorithm is only applicable for the structures obtained with the same exposure doses, and reflow simulation of any other structure requires preliminary measurements.
On the other hand, in case of any structure reflow simulation, carried out using SE, the only question is the distribution of structure vertex mobilities.
Kirchner mentioned that inverse mobility should correlate with PMMA viscosity, but the relation between mobility and viscosity was still unclear.
%Thus, the purpose of this study is to investigate the relation between polymer viscosity and mobility of its surface vertices and to develop the numerical approach for thermal reflow simulation of non-uniform polymer structures obtained by grayscale e-beam lithography using SE as a calculation engine.
Thus, the aim of this study is to develop the numerical approach for thermal reflow simulation of non-uniform PMMA structures obtained by grayscale e-beam lithography using SE as a calculation engine.
For this purpose one should simulate viscosity profile of e-beam exposed PMMA first and then
investigate the relation between PMMA viscosity and mobility of its surface vertices.

\section{Simulation of e-beam exposed PMMA viscosity profile}
In this study, thermal reflow is simulated for PMMA with non-uniform viscosity profile caused by e-beam exposure at room temperature.
According to equations~\ref{eq:WLF} and \ref{eq:3p4_3p1}, one can calculate PMMA viscosity for any specific temperature and number average molecular weight, so PMMA number average molecular weight distribution is of interest.
The distribution of PMMA local $M_n$ could be obtained by simulation of e-beam induced PMMA main-chain scissions.
For the latter, Monte-Carlo simulation of e-beam scattering in PMMA/Si structure was implemented.
Elastic scattering cross-section were obtained by relativistic partial wave expansion method (Mott cross-sections) using free software ``ELSEPA''~\cite{ELSEPA}.
Inelastic scattering was simulated using models based on energy loss functions of PMMA and Si, provided by Dapor~\cite{Dapor} and Valentin~\cite{Valentin}.
Next, according to Aktary~\cite{Aktary}, PMMA main-chain scissions were supposed occur due to inelastic electron-electron scattering.
Electron-electron scattering events, leading to PMMA main-chain scissions were simulated by Monte-Carlo technique with introducing the scission probability ($p_s$):
\begin{equation}
	\mathrm{electron-electron \ inelastic \ scattering: \ } \cases{\xi < p_s: & scission \\
		\xi \geq p_s: & no scission}
\end{equation}
where $\xi$ -- a random variable uniformly distributed on [0, 1).
Value of $p_s$ for room temperature (25 $^\circ$C) was determined by simulation of experimental radiation scission yield using the approach described in our previous paper~\cite{my_MEE} and compised 0.05.

Simulated PMMA main-chain scission events were stored in 3D histograms with 50~nm bin size.
The example of PMMA main-chain scission distribution simulated for line exposure using this approach is shown in Fig.~\ref{fig:sci_hist}.

\begin{figure}[h]
	\centering
	\includegraphics[width=0.8\linewidth]{sci_conc_1uC_cm_LOG}
	\caption{
		Simulation of local PMMA main-chain scission concentration in PMMA layer for line exposure.
		Line dose is 1 nC/cm, line width -- 300~nm, e-beam energy -- 20 keV, PMMA layer thickness -- 500 nm.
		Scission probability in inelastic electron-electron scattering is 0.05, which corresponds to room temperature (25$^\circ$C).
	}
	\label{fig:sci_hist}
\end{figure}

Then, number average molecular weight was calculated for each bin using the model of scissions randomly occurring at the bonds between the monomers~\cite{Ku1969}:
\begin{equation}
	\frac{1}{M_n^\prime} = \frac{w_s}{M_0} + \frac{1}{M_n},
\end{equation}
where $M_n$ and $M_n^\prime$ -- PMMA number average molecular weight before and after exposure, respectively, $M_0$ -- monomer molecular weight (100 for methyl methacrylate, MMA), $w_s$ -- probability of scission at a bond.
$w_s$ values were calculated for each bin using the formula:
\begin{equation}
	w_s = \frac{N_{sci}}{N_{mon}},
\end{equation}
where $N_{sci}$ -- number of scissions and $N_{mon}$ -- number of monomers in the bin, respectively.
Number of monomers in the bin of (50~nm)$^3$ size was calculated from PMMA density (1.19 g/cm$^3$) and MMA molecular weight (100 g/mol) and comprised 894809.
The example of $M_n^\prime$ distribution, simulated for line exposure by this method, is shown in Fig.~\ref{fig:Mn_hist}.

\begin{figure}
	\centering
	\includegraphics[width=0.8\linewidth]{Mn_prime_mat_easy_1_nC_cm_LOG}
	\caption{
		Simulation of local number average PMMA molecular weight in PMMA layer for line exposure at room temperature.
		Line dose is 1 nC/cm, line width -- 300~nm, e-beam energy -- 20 keV, PMMA layer thickness -- 500 nm.
		Initial PMMA number average molecular weight is 271000.
	}
	\label{fig:Mn_hist}
\end{figure}

Finally, local viscosity of exposed PMMA was calculated for required temperature using equation~\ref{eq:WLF} and for the following simulation viscosity distribution is averaged along $z$ axis (Fig.~\ref{fig:eta_arr}).

\begin{figure}
	\centering
	\includegraphics[width=0.6\linewidth]{eta_arr_1_nC_cm_120C_LOG}
	\caption{
			Simulation of averaged (along $z$ axis) viscosity profile in e-beam exposed PMMA layer for 120 $^\circ$C.
			Line dose is 1 nC/cm, e-beam energy -- 20 keV, PMMA layer thickness -- 500 nm.
			Initial PMMA number average molecular weight is 271000, PMMA reflow temperature is 120~$^\circ$C.
		}
	\label{fig:eta_arr}
\end{figure}

\section{Determination of e-beam exposed PMMA vertex mobilities}

Obtained PMMA viscosity profile couldn't be used for thermal reflow simulation so far -- the analytical approach based on profile Fourier transform could be applied in case of uniform viscosity only.
On the other hand, the numerical approach allows reflow simulation of non-uniform structures, but it bases on vertex mobilities, not viscosity.
Therefore, one should investigate the correlation between polymer viscosity and vertex mobilities of its surface.
For this purpose, thermal reflow was simulated by both approaches for the PMMA rectangular gratings, which parameters corresponded to Leveder study~\cite{Leveder_2010} -- 2 $\mu$m pitch and 28 nm depth.
First, for PMMA viscosity values in range 10$^2$--10$^6$ Pa$\cdot$s grating reflow was simulated analytically with constant time steps.
Then the grating surface was reconstructed in SE and surface evolution during grating reflow was simulated numerically with vertex mobility equal to 1.
During the numerical simulation, \textit{scale} values giving the same grating profiles as ones obtained using analytical approach, were determined (Fig.~\ref{fig:gratings_reflow}).
It was found that in the beginning of reflow there is a slight discrepancy between profiles simulated analytically and numerically but then both approaches lead to almost sinusoidal shape as it is predicted by equations~\ref{eq:Fourier_1},~\ref{eq:Fourier_2}~and~\ref{eq:Fourier_3}.
Time-\textit{scale} data obtained for different viscosity values showed almost direct proportionality between \textit{scale} and time ($t$) (Fig.~\ref{fig:alphas}):
\begin{equation} \label{eq:scale_alpha_t}
	scale \approx \alpha \cdot t.
\end{equation}
The values of $\alpha$ obtained by approximation of $t$-\textit{scale} data by a function~\ref{eq:scale_alpha_t} demonstrated quite linear dependence of $\ln(\alpha)$ on $\ln(\eta)$ (Fig.~\ref{fig:final_fit}).
The approximation of $\ln(\eta)$-$\ln(\alpha)$ data by the function
\begin{equation}
	\alpha = \frac{C}{\eta^\beta}
\end{equation}
resulted in $C$ and $\beta$ values equal to 26.142 and 0.989.
Thus, there is almost inverse proportionality between $\alpha$ and polymer viscosity:
\begin{equation}
	\alpha \approx \frac{26.142}{\eta}.
\end{equation}

\begin{figure}[h]
	\centering
	\includegraphics[width=\linewidth]{gratings_reflow}
	\caption{Thermal reflow of PMMA rectangular grating simulated by analytical and numerical approaches for different PMMA viscosity values.}
	\label{fig:gratings_reflow}
\end{figure}

\begin{figure}[h]
	\centering
	\includegraphics[width=\linewidth]{alphas} \\
	\caption{Time-\textit{scale} dependencies for different viscosity values fitted with linear function.}
	\label{fig:alphas}
\end{figure}

\begin{figure}
	\centering
	\includegraphics[width=0.6\linewidth]{С_gamma}
	\vspace{-1em}
	\caption{Linear fit of obtained $\ln(\alpha)$-$\ln(\eta)$ data.}
	\label{fig:final_fit}
\end{figure}

The determined relation between PMMA viscosity and $\alpha$ (which represents the ratio of $scale$ to $t$) enables mobility-based thermal reflow simulation by SE.
The point is that SE allows to monitor $scale$ value during the simulation, but originally $scale$ is not equal to reflow time.
The most convenient relation between $scale$ and time would be an equality ($scale \equiv t$) and it could be achieved by setting mobility equal to $\alpha$.
Indeed, if one simultaneously set
\begin{equation} \label{eq:mu_scale}
	\cases{\mu \equiv \alpha = \frac{scale}{t}, & \\
		scale = t &}
\end{equation}
the equation~\ref{eq:SE_v} doesn't change:
\begin{equation}
	\vec{\delta} = \frac{\vec{F}}{A/3} \cdot \frac{scale}{t} \cdot t \equiv \frac{\vec{F}}{A/3} \cdot scale.
\end{equation}

The equations~\ref{eq:mu_scale} is the final piece in mobility-based thermal reflow simulation of e-beam exposed PMMA.
Viscosity profile of PMMA, obtained at previous step could be easily converted into mobility profile using formula (Fig.~\ref{fig:mob_arr})
\begin{equation} \label{eq:mu_eta}
	\mu = \frac{26.142}{\eta}.
\end{equation}
Thus, one can reconstruct the surface of PMMA structure in SE, set proper mobilities of surface vertices and then run SE simulation with tracking the $scale$ factor, which will be exactly equal to reflow time.

\begin{figure}[h]
	\centering
	\includegraphics[width=0.6\linewidth]{mob_arr_1_nC_cm_120C_LOG}
	\vspace{-1em}
	\caption{
		Simulation of averaged (along $z$ axis) vertex mobility profile of e-beam exposed PMMA layer for 120 $^\circ$C.
		Line dose is 1 nC/cm, e-beam energy -- 20 keV, PMMA layer thickness -- 500 nm.
		Initial PMMA number average molecular weight is 271000.
	}
	\label{fig:mob_arr}
\end{figure}

\section{Discussion}

The developed method of PMMA vertex mobility determination has several remarkable features.
First, it demonstrates the agreement between reflowed profiles, simulated by analytical spectral method and ones obtained by numerical SE simulation in area normalization mode.
This emphasizes the applicability of soapfilm modeling for the simulation of polymer reflow and confirms the linear relation between $scale$ and time, proposed by Kirchner~\cite{Kirchner_SE_1}.
Moreover, the relation of inverse mobility of PMMA surface vertices and PMMA viscosity turns out to be linear too.
It is also noteworthy that developed method brings clarity into SE reflow simulation process -- in case of setting the mobilities obtained by equation~\ref{eq:mu_eta} the $scale$ factor denotes exactly the actual reflow time.

Second, this method could be applied for reflow simulation of any structure obtained in PMMA by lithographic methods.
In case profile, obtained in PMMA by grayscale e-beam lithography, simulation of PMMA main-chain scission distribution could be directly converted to PMMA viscosity profile and, finally, to PMMA vertex mobility profile.
In case of structures obtained by nanoimprint lithography the PMMA viscosity dependence on temperature and molecular weight only should be taken into account.
If sample size is much greater than structure period one can neglect the edge effects (contact angle etc.) for the reflow simulation far from sample edges.
Then structure geometry and vertex mobility distribution become the only parameters required for simulation.

Third, the described algorithm allows to expand the boundaries of thermal reflow application. At present, thermal reflow is used predominantly for profile smoothing, which not causes dramatic profile transformation~\cite{Kirchner_GL_review,Kirchner_2017}.
This results in a need for sophisticated grayscale e-beam lithography process for achievement of staircase profile close enough to required one.
The complex but predictable thermal reflow process in turn could simplify the whole fabrication method provided that greater part of profile is being formed by reflow.

\section{Conclusion} \label{conclusion}
This paper presents simulation technique for viscoelastic thermal reflow of PMMA with non-uniform viscosity profile caused by e-beam exposure with dose modulation.
Thermal reflow simulation bases on numerical search of minimal surface by finite elements method, processed by free software ``Surface Evolver'' (SE).
PMMA non-uniform viscosity profile is described by specific mobilities of PMMA surface vertices, which are embedded in surface model in SE simulation processed in area normalization mode.
PMMA surface vertex mobilities determination bases on PMMA viscosity profile which is calculated in three steps.
First, Monte-Carlo algorithm is used for the simulation e-beam induced PMMA main-chain scissions.
Then, statistical model is used to convert PMMA main-chain scission distribution to PMMA number-average molecular weight distribution.
Finally, empirical equations are used for determination of PMMA viscosity profile relying in simulated PMMA number-average molecular weight distribution
The relation between PMMA viscosity and vertex mobility of its surface was determined by simulation of rectangular grating reflow by two approaches -- analytical, based on decay simulation of profile spatial harmonics, and numerical one, processed by SE.
The obtained agreement between reflowed profiles reveal direct proportionality between SE scale factor and reflow time.
Reflow simulation in wide viscosity range resulted in turn in almost inverse proportionality between PMMA surface vertex mobility and PMMA viscosity.
Using this relation, one can calculate proper vertex mobility of PMMA structure surface vertices and process clear SE reflow simulation with scale factor denoting exactly the actual reflow time.
The developed approach could simplify complex 3D structure fabrication by deeper insight into thermal reflow processes.

\ack
The investigation was supported by the Program no. FFNN-2022-0021 of the Ministry of Science and Higher Education of Russia for Valiev Institute of Physics and Technology of RAS.


\section*{References}
\bibliography{cite_papers}

\end{document}
